\newpage
\section*{Митрич и Циркуль}
\addtocontents{toc}{\protect\setcounter{tocdepth}{0}}

%Они были учителями труда и черчения соответственно. Оба\mdash мужчины. Был ещё физрук, но про эту мразь и вспоминать не хочется
$\ldots$Циркуль был весьма интересным человеком\mdash худой, как щепка, высокий, как баскетболист, с залысиной, высоким лбом, окладистой бородой и густым голосом. Неизменные брюки, рубашка, безрукавка дополняли образ интеллигента, коим он, вне всяких сомнений, являлся. Он был учителем рисования и черчения, а также преподавал дисциплину под~названием <<Мировая художественная культура>>, которая предполагала знакомство неокрепших умов с шедеврами мировой архитектуры, живописи, скульптуры. 

Класс Циркуля был сродни музею\mdash там была парочка репродукций, куча цветов на подоконниках, на шкафах были таблички с пошаговым процессом выполнения портретов и натюрмортов, имелись чертежи деталей в проекциях и~всевозможные другие наглядные пособия. К оформлению своего кабинета он подошёл, что называется, со всей душой\mdash находиться там было одно удовольствие; пожалуй, ни один другой кабинет не мог таким похвастаться. 

В классе имелся диапроектор с дистанционным пультом\mdash слайды переключались автоматически при~нажатии кнопки на пульте. Это воспринималось не~иначе как небольшое волшебство (стоит напомнить, что в то время у~большинства из нас ещё не было мобильных телефонов). У~Циркуля была огромная коллекция слайдов с шедеврами мирового искусства, и когда он начинал закрывать окна в~классе плотными светонепроницаемыми шторами\mdash было сразу понятно, сейчас будет таинство диапроектора и~знакомства с чем\sdash то интересным. Огорчало конечно то, что иногда мы сдавали зачёты по~просмотренному материалу\mdash нужно было запоминать, собственно, названия и авторов этих шедевров, что было значительно сложнее, нежели чем просто любование красотой архитектурных и~живописных форм. Но,~тем не менее, мы как\sdash то справлялись. Особенное впечатление, естественно, производила архитектура и~скульптура древней Греции и~древнего Рима, а также эпохи ренессанса. Также, у~Циркуля был телевизор, по~которому он показывал нам различные учебные фильмы и материалы$\ldots$ словом, его уроки были одними из самых интересных за~школьные года. На них мы погружались в какой\sdash то другой мир\mdash без синусов и косинусов, в~мир красоты и изящества мирового искусства, который был некой отдушиной в нашей, вообще говоря, не очень приятной школьной жизни. Кроме того, на этих уроках существовала прекрасная и такая заманчивая, чего греха таить, возможность немного$\ldots$ подремать в темноте под~жужжание проектора\mdash часто его уроки ставили нам после омерзительной физкультуры или ещё чего\sdash нибудь столь же нелюбимого.

Также он вёл и рисование, рассказывая о приёмах живописи, различных техниках рисунка и прочем. Рисовали мы, в основном, всегда акварелью и простыми карандашами. Это были натюрморты, пейзажи, различные сюжетные зарисовки, простенькие эскизы сценок. Чуть позже он учил нас рисовать людей и обучал построению перспективы на рисунке. Понимание пространства и перспективы очень пригодилось нам в дальнейшем\mdash на уроках черчения, которые начались в старших классах. Тут уже Циркуль проявлял себя максимально жёстко, но и корректно при этом\mdash черчение строгая дисциплина, и он старался привить нам эту строгость и опрятность при выполнении чертежей. А чертили мы, понятное дело, простыми карандашами на~ватмане. Ни о каких компьютерных программах для~черчения никто и в помине не знал тогда, и хотя век компьютеризации неотвратимо шагал по планете не первый год, до нас, до школьников, это, понятно, не доходило тогда ещё\mdash редко у кого из нашего класса был дома персональный компьютер. Даже уже позже, в институте, я тоже, кстати, чертил руками, к своему стыду, так и не освоив какую\sdash нибудь нормальную чертёжную программу$\ldots$

\vspace{1.0cm}
$\ldots$Митрич. Он был учителем труда. И хотя в духе идиотических экспериментов и издевательств над школьной программой его предмет назывался то <<Технологией>> (Технологией чего?!), то <<Технологией и ещё чем\sdash то>>, он всегда неизменно говорил нам, что будет называть его кратко и по\sdash старинке: <<Труд>>. И нас, должен сказать, это вполне устраивало. На его уроках мы выпиливали лобзиком, строгали, делали разные поделки из дерева и металла, работали на сверлильных станках, а позже добрались и~до~главной мечты и сокровенного таинства, до чего максимально можно было добраться на его уроках\mdash до токарного станка. Это была фантастика$\ldots$

Стены его класса украшали плакаты по выполнению простейших операций в сверлильном и токарном деле, по технике безопасности. Посередине класса стояли металлические верстаки с тисками. Это был класс металлообработки, как я позже понял. Потому что было ещё и соседнее помещение, напрочь заваленное школьным хламом\mdash партами, стульями и прочим барахлом. Этот заваленный хламом и оттого неиспользуемый класс раньше был классом деревообработки\mdash там стояли специально приспособленные для этого верстаки иной конструкции, со столярными зажимами. Увы, в реалиях всеобщего бардака в то время, всё было так, как было\mdash кто\sdash то решил, что и деревообработкой можно заниматься в классе металлообработки, и больно жирно одному учителю иметь 2~помещения. Ну да ладно$\ldots$ нам хватало и того, что было, просто порой некая внутренняя обида накатывала, как цунами\mdash ты ведь понимал, что \textit{когда\sdash то тут было всё совсем по\sdash другому! Причём совсем надавно!} 

У Митрича были Жигули 2104, универсал, и он часто подвозил к школе, прям к своему кабинету (кабинет труда имел отдельную дверь на улицу!) всевозможные деревяшки и железные прутки\mdash для наших поделок. Мы много раз становились свидетелями такой заготовки материала впрок$\ldots$ очевидно, всё это <<добывалось>>, поскольку денег на~покупку материалов, скорее всего, не выделялось.

Итак, в~классе металлообработки были верстаки по~центру, у~стены стояли сверлильные станки, а~вдоль окон\mdash вожделенная мечта\mdash токарно\sdash винторезные, ТВ\sdash 3. Работать на них мы начали через год от~того момента как начались уроки труда, и,~конечно же, всем и~каждому безумно хотелось освоить это дело. Всё~то, что мы делали до~токарного дела, было для меня не сильно в~новинку, поскольку летом в~деревне шла вечная стройка дома и~молоток, ножницы по металлу, рубанок, дрель\mdash уже побывали в моих любознательных руках. Токарка же была высшим пилотажем в нашей понимании. Вдобавок, станков было 6 штук, но в рабочем состоянии лишь половина, отчего вокруг них всегда толпилась очередь выточить очередную деталь, подогревая интерес. Начинали мы, конечно, с обработки древесины на этом станке, поскольку по неопытности легко и резец затупить, и по лбу заготовкой получить. Потом, после освоения приёмов продольной и~поперечной подачи, автоматической продольной подачи, мы выточили первое стоящее изделие\mdash деревянный подсвечник. Для того, чтобы выточить подставку большого диаметра для подсвечника, нужно было поставить в шпиндель станка обратные кулачки и развернуть резцедержатель на 90\degree$\ldots$ словом, дело было безумно интересное. Потом точили уже много чего, перейдя и к металлообработке. Помню кто\sdash то притащил откуда\sdash то стреляных гильз от автомата Калашникова калибра 5.45\thinspace мм, и~мы выточили <<пули>> из~проволоки и загнали их в~стреляные гильзы\mdash выглядели полученные <<патроны>> классно. 

Вспоминается ещё один интересный факт. Рядом с <<детского>> размера ТВ-3 стоял полноразмерный, <<взрослый>>, настоящий токарный станок, огромный просто\mdash он стоял вдоль окон и занимал два пролёта. Это был трофейный немецкий токарный станок\mdash все прикрученные на него чёрные таблички с белыми надписями были на немецком. Был и немецкий орёл с закрашенной свастикой. Жалко одно\mdash он был в нерабочем состоянии. По крайней мере, нам так говорил Митрич. Но и за все школьные года я ни разу не видел его в работе. Эх,~мощь! Он был реально огромен и потрясал воображение. Но одновременно в душу закрадывалось какое\sdash то такое нехорошее чувство\mdash а зачем он тут стоит? А у нас, что, своих таких станков нет? Ведь уже не конец 40\sdash х годов, а~вообще новое тысячелетие?! Но~и~понятно\mdash какие токарные станки, когда всё производство в стране в 90\sdash е шло под~откос$\ldots$ А нам, по крайней мере мне точно, так хотелось после ТВ\sdash 3 попробовать и эту мощу\mdash хоть разочек! 

Помню в период увлечения стрельбой из лука у меня сорвало резьбу на зажиме упругого плеча. Я, конечно же, обратился к Митричу с просьбой дать выточить этот <<барашек>> на токарнике, и он согласился, но потом как\sdash то всё завертелось, закрутилось, стало не до этого$\ldots$ вместо того зажима я просто загнал подходящего размер болт, что было неудобно, поскольку надо было делать слишком много оборотов для зажима плеча, но вполне решало задачу$\ldots$ Поскольку я ушёл из этой школы после 9\sdash го класса, на~токарнике больше поработать мне не удалось, но в глубине души засела такая мысль\mdash на старости лет точно приобрету токарник, поставлю его в деревне и буду самозабвенно что\sdash нибудь точить$\ldots$ эх, мечты!

Одним словом\mdash токарный станок это любовь на~всю жизнь$\ldots$ Как мёдом ложатся на сердце строчки пролетарско\sdash романтического народного стихотворения:

\vspace{1.0cm}
\noindent\textit{%
\hspace*{2.7cm}Я увидел тебя у станка,\\
\hspace*{2.7cm}Нежный взор твой меня озадачил.\\
\hspace*{2.7cm}Ты стояла, вращая слегка\\
\hspace*{2.7cm}Рукоятку продольной подачи.\\
\\
\hspace*{2.7cm}Я сказал тебе: <<Ваши черты\\
\hspace*{2.7cm}Несомненно полны благородства>>.\\
\hspace*{2.7cm}Мне в ответ улыбаешься ты\\
\hspace*{2.7cm}Без отрыва от производства.\\
\\
\hspace*{2.7cm}Я теперь без ума от тебя,\\
\hspace*{2.7cm}Быть не может с тобою ошибки.\\
\hspace*{2.7cm}Ты стояла в руках теребя\\
\hspace*{2.7cm}Радиально-упорный подшипник.\\
\\
\hspace*{2.7cm}А под вечер к тебе я приду,\\
\hspace*{2.7cm}Ты расскажешь мне голосом звонким\\
\hspace*{2.7cm}О подаче на левом ходу,\\
\hspace*{2.7cm}О гидравлике и шестерёнках.\\
\\
\\
\hspace*{2.7cm}И не будет ни ветра, ни туч\mdash\\
\hspace*{2.7cm}Только ты и луна в целом мире.\\
\hspace*{2.7cm}Хочешь, я подарю тебе ключ\\
\hspace*{2.7cm}19\footnote{Гаечные ключи 19 на 24 в СССР не изготавливались и, следовательно, могли быть только импортными, а посему факт дарения девушке такого ключа свидетельствует об очень сильных чувствах.} на 24?\\
\\
\hspace*{2.7cm}И вернувшись из ЗАГСа домой,\\
\hspace*{2.7cm}Этой ночью пленительно жаркой\\
\hspace*{2.7cm}Будем мы наслаждаться с тобой\\
\hspace*{2.7cm}Руководством по газовой сварке\\
\\
\hspace*{2.7cm}Жизнь пройдет, словно миг.\\
\hspace*{2.7cm}Я у милого тела заплачу\\
\hspace*{2.7cm}И к могилке твоей принесу маховик$\ldots$\\
\hspace*{2.7cm}С рукояткой продольной подачи!\\ 
}
\vspace{0.1cm}

Уроки труда, особенно токарка, вспоминаются с~такой~же теплотой, как и география, и уроки Циркуля. Вместе это\mdash три отдушины, три кита школьных, не~дававших совсем упасть духом и поддерживавших баланс добра и~зла, так скажем. 

По поводу вышесказанного, что \textit{когда\sdash то тут было всё совсем по\sdash другому}\mdash под окнами кабинета труда была небольшая будочка, где должны были храниться газовые баллоны для подачи газа в кабинеты химии и~биологии. Осознаёте масштаб? Не спиртовки или сухое горючее, а~настоящие газовые горелки при проведении опытов на~уроках! Парты в классах химии и~биологии были жёстко прикреплены к полу, и к ним были подведены газовые трубы. Потом на парте был тройник и~два газовых краника\mdash на каждого ученика за~партой. Когда мы учились, это всё уже не функционировало, отчего также становилось обидно и~горько\mdash воистину следы древней, более высокоразвитой цивилизации$\ldots$

$\ldots$Конец года после 9\sdash го класса, спортзал залит солнечным светом. Ощущение неотвратимо наступающих перемен, наступления момента подведения какой\sdash то невидимой черты, после которой начнётся другая жизнь. Счастливее ли? Лучше ли? Никто не знает заранее. Что\sdash то, естественно, станет лучше, что\sdash то хуже. Впрочем, как и~всегда и~во всём, в любом деле. В целом, счастливыми свои школьные года я назвать не могу точно, школа вызывала отвращение. Лишь эти три отдушины, о которых рассказано, были каким\sdash то уголком спокойствия и душевного равновесия. Остальное вызывало животное отторжение и~неприятие. Плюс конечно время было такое, переходное для нашей страны$\ldots$ всё это понимается уже позже, сейчас. Но~в~прошлое не попасть и ничего не исправить, что было\mdash то~было$\ldots$  

\begin{center}
	\psvectorian[scale=0.4]{88} % Красивый вензелёк :)
\end{center}