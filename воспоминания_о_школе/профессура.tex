\newpage
\section*{Профессура}
\addtocontents{toc}{\protect\setcounter{tocdepth}{0}}

$\ldots$Математичка Р. влетала в собственный класс стремительно, как спецназ врывается в штурмуемое помещение. Стены класса были выкрашены в синий цвет, отчего это мрачное скорбное место становилось ещё холоднее и неуютнее. Ёжился на стуле при её появлении, наверное, каждый. На голове вечная химия, осветляющая тёмные волосы наверху, очки тёмные и полностью закрывающие глаза так, что практически невозможно было понять, куда она смотрит. Завершала образ полностью отсутствующая талия. Мужа у Р. не было, но было двое детей\mdash взрослая дочь и старшеклассник сын, учившийся на года 3\thinspace\nobreakdash--\thinspace 4 старше.

\diagdash Садитесь!\mdash раздавалось её прокуренным, ненавистным голосом. Наверное, не было никакого другого учителя, которого ненавидели больше, чем её, не только в нашем классе, но и в школе в целом! В арке 9\sdash этажого дома, сквозь которую вела дорога к школе, кто\sdash то из самых отчаянных голов в отместку за её ор, за унижения, за гнетущую атмосферу на уроках, за полную безжалостность, за всю ту боль, психологически уродовавшую малолетние сознания и ежедневно причиняемую нам, написал из баллончика синей краской шикарным курсивом шедевр, целое художество, под которым, я уверен, мысленно был готов подписаться каждый ученик Р.:
\begin{center}
%<<\textit{\fontspec{Propisi}{Р.\mdash СУКА}>>.} 
\LARGE\textit{{\fontspec{Propisi}{Р}.}\mdash \fontspec{Propisi}{С У К А}.} 
\end{center}

Естественно, фамилия была написана полностью. Второе слово было написано смачно\sdash размашисто. Надпись была огромна\mdash буквы по полметра. Все знали, что она ходит на работу через эту арку каждый день. И другой дороги по асфальту нет\mdash только через парк по грунтовке, но Р. там не ходила. Художество висело примерно полгода. Это низко и ужасно, но когда мы, ученики, шли через эту арку с утра и вечером, синяя надпись придавала сил\mdash как от души отлегало. Ощущалось хоть некоторое отмщение. Словом, примерно полгода эта надпись была отдушиной. После того, как её закрасили, повторить подвиг никто не решился, хотя, вроде бы, автор так и не был найден.

\diagdash Света! К доске!\mdash нарочито с наслаждением предстоящей казни вызывала она несчастную на эшафот. В~тот день Р. была особенно не в духе, парочка свежевыставленных двоек уже красовались в журнале$\ldots$ Через несколько минут, когда Свету, особо не шарящую в алгебре, доводили чуть не до слёз, брань достигала апогея:

\diagdash Света, звезда$\ldots$\mdash тут Р. заминалась, подбирая приличное слово,\mdash $\ldots$балета! Садись, 2!

Следующий несчастный выбирался по журналу, и его ждали аналогичные эпитеты в случае провала.	

Но стоит признать, при всём при этом она часто, почти всегда, была справедлива. Если кто\sdash то чего\sdash то не знал\mdash получал заслуженно свои двойки и тройки. Знал\mdash без проблем ставила отлично. Она не снижала отметки из\sdash за личного отношения$\ldots$ такое впечатление, что мы все были ей одинаково противны. 

Подача же материала у нее была институтская, как я понял позднее. Она требовала в расписании себе неуклонно спаренных уроков. На первом устраивалась лекция\mdash мы писали конспекты. По сути, она, именно она, научила нас конспектировать, причём делала это профессионально\mdash давала время записать, всё разжёвывала. Эти конспекты были, без иронии, шедевром изложения материала, как я понял позднее\mdash когда появилось с чем сравнивать. На~втором уроке был, на институтский манер, семинар\mdash практическое занятие. Тут она оттягивалась по полной. Через призму лет абсолютно понятно за что\mdash ведь материал разжёвывался ею досконально, а мы же часто много чего не понимали (но боялись спросить!), много тупили, не всегда всё усваивали с  первого раза. И даже со 2\sdash го. И с 3\sdash го. Кроме того, что мы были детьми, которым хочется гулять, прыгать, бегать, дурачиться и ещё чёрте чем заниматься, многие из нас реально не понимали на кой нам сдалась эта математика на таком жёстком уровне. Многие откровенно не тянули и это нормально\mdash не всем же быть лапласами, бесселями, колмогоровыми и т.д. и т.п. Но Р. считала иначе, она требовала с каждого, а кто не тянул\mdash становился объектом едких подколок, унижений, ругательств.

В определенные дни казалось, что всё\mdash если завтра школа не сгорит, на неё не упадет метеорит или не смоет цунами, то придётся каким\sdash то самым нечеловеческим и ужасным образом расправиться с Р., чтобы избавиться от психологического гнёта. Но дальше синей надписи в арке ни у кого дело не зашло: Р. умела держать аудиторию\mdash не хуже капо.

Нас, <<ашников>> она называла <<профессурой>> в пылу гнева и брани за очередной невыученный материал. Ругалась она виртуозно. Как ей удавалось не переходить при этом на мат\mdash до сих пор не понимаю. Годы спустя, прочитав <<Географ глобус пропил>> \cite{ГеографГлобусПропил}, я был поражён\mdash ведь роман был написан в 1995 году, а полностью издан только в 2003, а Р. называла нас так же, как и герой\sdash учитель в~книге\mdash профессурой. Совпадение?! Звучало это очень ёмко, хлёстко, издевательски обидно, с чуть протяжным <<c>>.

Май, конец года. Профессуре было поручено вымыть класс Р. Девочкам\mdash отдраить окна, пацанам\mdash вымести пол и помыть его. Хуже наказания и представить трудно. Каждый, я подчеркиваю, \textit{каждый} был бы не прочь разбить эти ненавистные окна в этом классе, учинить расправу над каждой партой, каждым стулом, который был в часы уроков раскалён как геенна огненная под нашими ученическими задами. Но нас заставили мыть. Не было в мире б\'{о}льшего наслаждения, чем, улучив момент, незаметно помочиться в ведро с водой, которой потом <<помыли>> пол$\ldots$

Ко мне Р. относилась как и ко всем\mdash безразлично жестоко. Один раз я никак не мог понять куда в уравнении вида ${y=x^2}$ девается, собственно, функция? ${y=f(x)}$\mdash тут всё понятно: функция это $f$, а в уравнении ${y=x^2 + 2\times x}$? В голове, тем временем, совсем другое\mdash не эти скучные функции, не области определения, не интервалы, отрезки, неравенства и многочлены. Всё это пригодится позднее$\ldots$ но тогда\mdash тогда мы, никто из нас, я уверен, не понимал за что нам такая кара, где мы так провинились? Сейчас разбуди среди ночи и спроси формулу нахождения корней квадратного уравнения\mdash отвечу. Но зачем, есть же справочники? Кто\sdash то скажет, что всё обучение\mdash ради нейронных связей в мозгу. Да, возможно, но не таким способом. 

Уже не помню по каким причинам, но, кажется, мы мыли её класс часто\mdash то ли в какой\sdash то год у нее не было классного руководства и, соответственно, собственных несчастных кто бы это делал, то ли ещё почему\sdash то. Скорее потому, что у нашей классной было два класса\mdash 1\sdash й мыл наш <<родной>> класс, а 2\sdash й, то есть мы, профессура, отдувались на других фронтах, куда нас сдавали <<в аренду>>. Конечно, эффективность такого подневольного труда была крайне низкой. 

Говорят, что школьные годы вспоминаются с теплотой\mdash ничего подобного. Как по мне, невозможно вспоминать какое\sdash то время только с теплотой или только с ненавистью\mdash всегда речь идёт о конкретных моментах. Моменты у профессуры были большей частью негативные, даже, я бы сказал, жуткие. Может быть это моменты лишь моего восприятия? Но нет, синяя надпись говорила сама за себя\mdash не только моего. С каким наслаждением я вздохнул после того, как покинул эти стены навсегда! И хотя впереди была полная неизвестность, <<профессор>> был если не счастлив, то по крайней мере рад. 

И хотя я много раз, уже в институте, впоследствии сказал мысленно спасибо Р. за вдолбленные в голову знания, за системный подход к преподаванию, чего многим учителям, как школьным, так и институтским, попросту никогда не удаётся добиться и близко, тогда кроме животного страха, ненависти, желания жестокой расправы она не вызывала ни\sdash че\sdash го, увы. Если так доставалось <<профессуре>>, я просто не хотел и не хочу знать, как доставалось <<отцам>> и <<зондеркомманде>>~\cite{ГеографГлобусПропил}\mdash иногда, сидя на других уроках даже на другом этаже, даже за закрытой дверью, нам было слышно, как Р. орёт на очередных несчастных\mdash слушала вся школа\mdash дверь в свой класс она закрывала только на контрольных.

Считается, что плохое забывается со временем. Не~всегда! Часто забывается как раз хорошее, поскольку воспринимается как должное. А вот плохое, особенно очень плохое\mdash порой въедается так, что никакими средствами не отмоешь, не выскребешь, не отчистишь. Любой из <<профессуры>> это подтвердит$\ldots$

$\ldots$Покинув стены этого <<прекрасного>> заведения (к~слову, построенного в 70\sdash ые по какой\sdash то экспериментальной программе и существующего в~единственном, по~моим сведениям, экземпляре) и оказавшись на пляже после 9\sdash го класса в компании \textit{N}, который был младше меня на несколько лет и тоже несчастно попал к Р., я с грустью наблюдал, как он выводил ту самую синюю надпись на камнях и швырял их с горя в воды Чёрного моря\mdash ни яркому солнцу, ни нежному бризу, ни убаюкивающему шуму прибоя, ни ласковой воде не под силу было заглушить эту боль$\ldots$