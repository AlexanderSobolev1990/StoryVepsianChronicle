\newpage
\section*{Джомолунгма}
\addtocontents{toc}{\protect\setcounter{tocdepth}{0}}

$\ldots$У неё были густые длинные, чуть суховатые, натурально(?) рыжие волосы. С высоким ростом, нехудощавой комплекцией, широкими бёдрами, волнительной грудью, чуть узковато посаженными глазами она, молодая географичка, казалась нам, 6\sdash классникам, образцом совершенства. Словно для подчёркивания рыжины волос, она предпочитала в одежде зелёный цвет. 

На уроках царил порядок\mdash самые отъявленные негодяи стихали, откровенно любуясь её красотой. Конечно же она видела (не могла не видеть!) это и всё понимала, но вела себя в высшей степени достойно, как и подобает настоящему Учителю. Я не совру, если скажу, что почти все пацаны были в неё влюблены\mdash несмотря на то, что она была дочерью математички Р.

В противовес своей матери, она никогда не кричала, вела уроки тихо и спокойно, рассказывая нам о физической географии. Стоило каким\sdash нибудь раздолбаям в классе вдруг зашуметь, она лишь замолкала и пристально откровенно смотрела на них. Это работало ну просто безотказно — парни тут же смолкали под её взглядом, смущаясь, казалось бы, такого желанного внимания своего идола, и сидели внимали. 

Порядок, как мне кажется сквозь года, держался исключительно на её внешних данных, спокойствии и личном интересе к рассказываемому\mdash страха не было\mdash во время уроков географии никто не вспоминал про её мать, и уж точно никакая мысль, что Р. может прийти и наорать на нас\mdash что географичка может нажаловаться на нас матери, не была причиной той чарующей атмосферы географических открытий, топокарт, гор и долин, морей и океанов, царившей на её уроках$\ldots$ Удивительно, но, будучи разительно другой, нежели чем своя мать, она не вызывала у нас c той никаких ассоциаций\mdash никогда ни у кого не возникало желания выместить на ней свою злобу к математичке, отыграться как\sdash то, наоборот, она воспринималась нами как нечто кристально чистое и незамаранное в общей картине учительского болота, повсеместно окружавшего нас.
 
Она учила нас пользоваться топокартой: читать легенду карты — условные знаки, определять азимут на местности, обходить препятствия по азимуту, вычислять расстояния. Обучала масштабу и прочей науке$\ldots$ Так моя любовь к топокартам и без того подпитываемая настоящим советским географическим атласом (с надписью <<Главное управление геодезии и картографии при совете министров СССР>>), заиграла новыми красками\mdash я осознал насколько география и картография интересные и полезные науки. Дополнительную пищу для восхищения давал том номер 1 Детской Советской Энциклопедии, где тоже был раздел про картографию и походы. 

Она объясняла нам как вести дневники погоды и составлять розу ветров, как вести путевые наблюдения и различать типы облаков$\ldots$ Словом, физическая география была одним из интереснейших школьных предметов, если не самым интересным\mdash 
лишь немногим учителям удавалось по\sdash настоящему разжечь в нас какую\sdash то тягу к знаниям, представляя собой личный пример некой одухотворённости\mdash такое впечатление, что ей самой и вправду очень нравилась вся эта географическая кухня.

Всё вместе это\mdash да, именно это\mdash зародило в~самых потаённых уголках моей души чувство, что я должен когда\sdash нибудь сам пойти в поход. Чтобы у меня была топокарта, компас и чувство того единения, того отчуждения от всех, которое испытывают топографы, географы, работая в <<полях>>.
 
Как помнишь всегда ${x_{1,2}= \frac{b\pm \sqrt{b^2-4\times a\times c}}{2a}}$, так помнишь в~любое время дня и ночи и высоту самой высокой горы\mdash Джомолунгмы\mdash 8848 метров над уровнем моря. Но,~как нетрудно догадаться, совсем по другим причинам$\ldots$ Как~поразительно разными методами, способами, можно добиться от~детей знаний, и~как ужасно прочувствовать на~себе весь спектр средств, применяемых иными <<учителями>> в~работе$\ldots$ 

На географию, ясное дело, мы ходили как на праздник. Ни один школьный предмет не вспоминается с такой теплотой, как этот. Сквозь года очень сложно сказать, сколько географичке было лет, тем более в детстве все кажутся таким взрослыми$\ldots$ Вероятно ей было от 25 до 30. Она была самым молодым учителем в нашей школе. Вскоре на её руке появилось кольцо, и она уволилась. 

Курс экономической географии вместо неё читал нам другой учитель, чей <<авторитет>>, манера подачи, внешние данные\mdash короче всё было против неё. Естественно, вся пацанва была жутко расстроена уходом нашей географички, но сейчас, через года, ничего не остаётся как искренне за неё порадоваться\mdash работать в том коллективе учителей я бы не пожелал абсолютно никому. 

От экономической географии в голове не осталось ничего, а физическая до сих пор вспоминается не то чтобы даже с теплотой, а с каким\sdash то просветом, ярком лучом$\ldots$ поистине <<лучом света в тёмном царстве>> остальных предметов. А очарование топографической карты всегда вызывает в душе самые что ни на есть приятные, походные чувства\mdash вот уже ощущаешь звуки леса и шум реки, запах полей и разнотравья лугов, проносишься по лучевым просекам и петляющим ниточкам лесных дорог, уносишься в глушь нашей огромной территории, паришь действительно над <<зеленым морем тайги>>, где нежданно на лесной опушке тебя ждёт избушка охотника$\ldots$

Я мечтал стать Географом. Я им не стал$\ldots$

\begin{center}
	\psvectorian[scale=0.4]{88} % Красивый вензелёк :)
\end{center}
