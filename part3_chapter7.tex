\chapter{Дозаправка и первые шиверы} 
\corner{64}

С утра долго копошились с завтраком. Вдалеке, как и вчера, шли поезда. Гулким эхом их гудки отдавались в лесах этого заповедного края… утренняя леность никак не проходила. Я впал в состояние какой-то медитации. Но надо было готовить кашу и жена тоже вскоре к этому подключилась. Паша же, видя, что готовка завтрака затягивается, удалился пока на берег порыбачить и наловил-таки парочку мелких рыбёшек, а Олег решил искупаться и помыться. 
В этом году мы решили сделать походную баню, но места под реализацию своей идеи никак найти не могли, как, впрочем, и камней для печи. Плёнка же ехала в корме байдарки и ждала своего часа. На четвертый день похода мне тоже уже хотелось бы сделать баню, но увы. Олег же не стал ждать гипотетической бани и отправился к реке. Я тоже рассудил, что это верное решение и, пока каша закипала, быстро спустился к реке и устроил помывку в стиле «Спа-а-арта!!!», потому что утренняя водичка очень и очень бодрила. Жена также последовала нашему примеру и искупалась. 
Затем продолжили варить завтрак. После вчерашнего глазомер был немного сбит, и я налил слишком много воды в кашу. Ждали пока выкипит. Жена нарезала кураги, чернослива, орехов, а Олег, тем временем, наворачивал остатки вчерашнего ужина. Потом все остальные подкрепились традиционным завтраком, свернули лагерь и встали на воду. Даже как-то немного жаль было покидать такое неплохое место… Замечено, что практически любая стоянка по приходу кажется непригодной для обитания – высокая трава, ничего не обустроено и тэ дэ и тэ пэ, а с утра с неё не хочется уходить, потому что всё стало почти родным… и так каждый день в сплаве. Ну, кроме днёвок. Там изначально место должно «выстрелить», завоевать расположение туриста-водника. И вроде, казалось бы, совершенно неблагородное занятие – каждый день обустраивать в новом месте себе стоянку, а с утра сниматься с неё и идти дальше… но в этом тоже есть некая философия, некий смысл – смысл неостановимого движения вперед, к новым красотам реки, к новым трудностям на маршруте, к новым сосновым борам и упоительно красивым песчаным обрывам.
Пока долго с утра возились с завтраком, жена простирала штаны и теперь сушила их у костра, но тщетно – надо больше времени. В итоге пришлось их запаковывать сырыми – досушим на следующей стоянке. 
Отчалили поздновато. Но я был расслаблен – дальше начнётся ранее пройденная мной часть маршрута. Спустя совсем немного времени по левому берегу начались домики Заборья, а потом показался и железнодорожный мост. Тот самый, у которого я два года назад начал свой первый дикий сплав. 
Погода с самого утра стояла пасмурная, а на подходе к Заборью начался дождь, и пришлось немедленно, памятуя о внезапно начинавшихся ранее ливнях, облачиться в дождевики. Дойдя до моста, пристали к левому берегу и вылезли. Место как-то изменилось – я примерно помнил, где собирал байдарку тогда, 2 года назад, но сейчас всё поросло высокой травой.
Сразу как мы причалили, сверху с грохотом стал проноситься грузовой состав. Я успел отснять небольшую видеозарисовку, пока С.Ю. доставал деньги из гермопакета и потом мы все вместе, кроме Паши, отправились искать магазин. Паша же остался под мостом сторожить байдарки и вещи. 
На станции было оживление – путейцы ремонтировали полотно, грузовые составы проносились в обе стороны, громыхая на мосту, по громкой связи звучали разные объявления. Потом, после прохода очередного состава, всё стихало, и лишь брань путейцев нарушала идиллию. 
Мы шли знакомой мне дорогой вдоль железнодорожного полотна к станции мимо старой кирпичной водонапорной башни. Потом поднялись к платформе и решили спросить на станции – где, собственно, тут ближайший продуктовый магазин. Здание старой станции оказалось мертвым – глушь и пустота внутри, хоть и открыто. С.Ю. заглянул внутрь и, никого не найдя, пошел дальше. Мы с женой задержались на платформе, пока передовой отряд – Олег и С.Ю. искали магазин. Вскоре мы их нагнали у небольшого кирпичного домика белого цвета позади станции – это и был магазин. 
К нашему огромному счастью мы сумели приобрести четыре или даже пять баллончиков «Рафтамид»-а. У-у-у-х, берегитесь, комары! Пощады не будет! Также обзавелись двумя упаковками спиралей от насекомых. Помимо этого, прикупили хлеба, огурцов, куриных яиц и ещё чего-то по мелочи. После, запечатлели себя на фото у этого магазинчика, и пошли обратно к Паше, который, наверно, уже заждался.
Олег и С.Ю. Пошли обратно к берегу, а мы с женой решили пройтись до здания впереди, где тоже висела надпись «Заборье» – видимо техническое здание станции. Так и оказалось – внутри был пункт управления стрелками и сидели две женщины-железнодорожницы. Я поинтересовался у них, где тут в Заборье памятник «Дорога жизни». Ведь та самая знаменитая «Дорога жизни» начиналась тут, в Заборье в отдельный период Великой Отечественной… Железнодорожницы напряглись от такого неожиданного вопроса от людей в штормовках, но вспомнили, что до недавнего ремонта стояла тут подальше в сторону Вологды стела памятная, но после ремонта она… пропала. Вот такие дела.
По пути обратно мы с женой заглянули в здание старой станции, где никого не нашли. Окошечко кассы было заколочено и весела надпись, что бумажные билеты больше не продаются – только через интернет покупать. А нет интернета – и нефиг отсюда уезжать из этой Ж! Ишь, разъездились тут, понимаешь! Внутри здания было подобие зала ожидания со старыми стульями и печкой, приютившейся у стены. Неужели до сих пор зимой тут отапливаются дровами? Вот так да. А что, вполне себе может быть. Не найдя больше ничего интересного, кроме белых стен и печки, мы вышли наружу. На стене станции висело расписание пригородных поездов Бабаево-Тихвин… Ещё ходят. Ещё держатся… Где только на них билет брать?
По пути обратно, к мосту, встретили дикую утку. Она ловко приземлилась перед нами и стала улепётывать вперёд на своих перепончатых лапках. Мы ускорили шаг и начали её нагонять. Утка ускорилась. Ускорились и мы. Утке быстро надоел этот забег, и она улетела… Мы же подошли к остальной части команды, дожидавшейся нас под мостом, погрузились и отчалили. Вскоре после Заборья, как я помнил, должны были начаться перекаты.
Очень не хотелось получить повреждения байдарки, поэтому я зорко глядел с кормы, высоко сидя на своем гермомешке, чтобы спереди по курсу не было никаких камней и коряг, стараясь при этом выбирать место поглубже в русле. Явственно из глубин сознания вставали картины двухгодичной давности – те же узости русла, камни, торчащие из воды… и пробоины… Но в этом году значительно больше воды – и я не особо боялся получить пробоину ниже ватерлинии… Олег же достаточно быстро научился читать воду и мы довольно удачно избегали встреч с подводными валунами и острыми камнями, обходя их то справа, то слева. Жена наблюдала это прохождение и, недоумевая, спрашивала, как я мог два года назад получить тут столько пробоин? Все дело в уровне воды, отвечал я. Сейчас он выше как минимум на полметра в данном месте. И то, при таком уровне, приходится часто резко лавировать в русле. 
Так мы и шли в постоянном напряжении, обходя валуны, изредка слегка цепляя днищем подводные камни. Погода продолжала хмуриться. В самом Заборье прошли шиверу в сужении русла, а сразу за селом начался порожистый участок, после которого дошли без приключений до Гришкино. На подходе к Гришкино начал накрапывать дождик. Из-за поворота реки стал виден капитальнейший завал и подобие охотничьей вышки. Это оказалось небольшим сюрпризом, поскольку я ждал уже железобетонного моста около Гришкино, а это появилось непонятно что такое. Вспоминал по 2015 году, что это могло быть – не вспомнил. А проходить пришлось. У правого берега прижались и преодолели это место без каких-либо проблем – никто даже не зацепил днищем. А буквально за следующим поворотом нашему взору открылся вид на очередной изгиб реки и тот самый железобетонный автомобильный мост. 
Пройдя его в среднем пролёте, я вскоре увидал на левом берегу место своей стоянки двухгодичной давности. Всё поросло травой, и лишь остатки костровища, да высокая тонкая сосна на берегу, напоминали о той самой моей первой стоянке на этой реке… сейчас же мы шли дальше – километраж на сегодня пройден ещё незначительный и было не время останавливаться. Однако, как я помнил, дальше по реке практически нет места для хорошей стоянки. Но это было два года назад, к тому же, я мог и не приметить стояночного места, только встав на воду с ночёвки. Поэтому, больше оптимизма! Мы шли вперёд и байдарка С.Ю. то и дело обгоняла нас. 
Я помнил, что сегодня предстоит преодолеть ещё одну плотину бывшей местной ГЭС. Но, в отличие от Тресновской ГЭС, эту я уже проходил в 2015 году по малой воде безо всяких проблем. Так что, когда перед нами вдруг возникло сужение русла и течение стало ускоряться как будто в воронке, я сразу понял – это та самая плотина ГЭС, сомнений быть не может. Координаты N59°25'57.18" E35°09'53.29". Вот и остатки тела плотины показались… Мой экипаж пришел в ужас, как мне показалось, от открывшегося впереди слива и от того, насколько уверенно наша байдарка шла прямо в самый центр этого водного препятствия. Я же был уверен в себе и мы, выровняв байдарку перед грядущим испытанием, нырнули в сливчик. Нос слегка зарылся в воду и сразу же вынырнул. Прошли на ура! Я обернулся и стал наблюдать, как наш второй экипаж начинает заходить к сливу. Вовремя сообразил и попросил жену сфотографировать их в момент прохождения плотины. Кадр вышел отличным!
После этого испытания мы немного подрейфовали, передохнули и продолжили наше путешествие. Мне все хотелось уже где-нибудь встать на ночь, но тщетно – берега кажутся непригодными для стоянки! Поэтому идём дальше.
Камни в русле по-прежнему не редкость. Приходится иногда резко менять курс, борясь с течением, которое усиливается при подходе к перекатам… Изучая после похода спутниковые снимки маршрута и сопоставляя их с пережитым, удостоверился, что дело было в точке N59°25'27.51" E35°09'28.25"… Река там огибает островок, разделяясь на две протоки. Причём левая протока визуально хоть и шире, но кажется мельче – из воды торчали камни, были видны белые барашки… поэтому наша эскадра пошла в правую протоку, которая круто огибала небольшой, густо заросший кустарником, островок. Течение на перекате, как и всегда в таких случаях, ускорилось. Экипаж С.Ю. первым вошел в протоку. Мы же старались повторять их траекторию, входя в левый «вираж»… я на доли секунды замешкался и нас начало сносить вправо на камень, который мгновенно образовался, казалось, из ниоткуда. Я сразу понял, что глубина осадки не даст нам пройти его… времени на работу веслами совсем не осталось и тут, уже ожидаемо, случилось то, чего так не любят все туристы-водники без исключения… мы сели днищем на камень. Причем нас снесло на него боком и крепко посадило. Я немедленно стал пробовать оттолкнуться веслом от дна, но ни справа, ни слева по борту мне это не удалось – было слишком глубоко. Ситуация начала приобретать нешуточный характер – сильное течение начало заламывать корму прямо за моей спиной и разворачивать байдарку. Попытки Олега оттолкнуться от дна также ничего не дали. Я быстро понял, что наш единственный шанс избежать купания в холодной воде сейчас – это поддаться течению, позволить ему развернуть байдарку поперек русла, чтобы нас стащило с камня. Поняв, что корму не тащит дальше под воду, я стал помогать течению разворачивать нас и вскоре, сила давления набегающей водной массы на борт перевесила силу тяжести байдарки, засевшей на камне, и мы вновь оказались на плаву. Облегчение, что ещё тут скажешь! Но река не дремлет – впереди крутой поворот и надо его пройти, а мы почти поперёк русла… быстро отработали с Олегом обратным гребками по левому борту и выправили курс. Жена занималась непрерывным наблюдением за дном байдарки – не поступает ли вода. Выйдя из этого коварного раздвоения русла и встретившись с передовым отрядом, мы стали дрейфовать по течению и наблюдать за водой под ногами. Не прибивает! Невероятно, но, кажется, мы не пробились! 
– Я выпью сегодня за эту шкуру, товарищи! – огласил я команде. 
Действительно, байдарочная шкура выдержала! Это просто невероятно! Но все-таки, надо на стоянке удостовериться, что пробоины нет. Может задиры или фильтрующая пробоина, но точно не сквозная – воды не поступало! Факт прохождения такого препятствия добавил уверенности, и мы ринулись дальше.
А дальше нас ждал небольшой перекатик на месте брода и резкий S-образный поворот русла, проходя по которому, я заметил справа в прижиме остатки фонарного столба и прожектора – видимо это с размытой плотины, что мы прошли выше по течению. Сразу по выходу из этого поворота все заметили справа отличный подъем на берег и, судя по всему, стояночное место. Немедленно скомандовав причалить, мы стали интенсивно разворачиваться, поскольку сильное течение, вырвавшееся из своеобразного «бутылочного горлышка» поворота, гнало нас дальше по реке. Наконец, удалось подойти ближе к берегу, где было обратное течение, о чем мы судили по завихрениям пены на поверхности воды. Причалили, вылезли, поднялись по песчаному подъему наверх… место было отличным! Ровная широкая поляна, кое-какие дровишки валяются, вид с обрыва на реку был очень хорош. У меня в голове промелькнуло, что это, кажется, да-да, постойте-ка! Ведь это место на Днёвку! 
Но разведка продолжалась – за поляной обнаружили дорогу, впрочем, давно не езженную, уходящую в редкий сосновый лес и там обрывающуюся. Видимо, кто-то давно заготавливал тут древесину. Вспомнил карту – тут близко деревня Забелье… ну как близко… километра два по прямой. Несмотря на такое соседство с деревней и наличие дороги, мы решили остаться, а уж днёвка или не днёвка – решим позже.
Как обычно разгрузились, перетаскали вещи, выбрали места для костра, палаток, тента. Словом, всё как всегда. Место очень нравилось. С.Ю. разведал дальше по дороге редкий лес, немного сухостоя и, самое главное, отличный ракурс для фотоохоты. Но Олег и Паша расстроились – стоянка «не рыбная». Впрочем, это не помешало им попытать счастья с удочками чуть ниже по течению, предварительно напилив дров для готовки ужина. Пока Олег и Паша рыбачили, остальная часть команды приготовила зажарку и суп.
И тут Олег извлёк из гермы подарок от Миши специально для нас – крымский трёхзвездочный. У-у-у-у-у! Ай да Мишка, ай да! Вот молодец, подумал о товарищах! Знал как порадовать продрогших туристов-водников! Мы, не медля, откопали в гермах лимон и приступили к уничтожению и ужина и крымского… гитара скромно стояла в сторонке, запакованная в полиэтилен. 
Сегодня прошли около шестнадцати километров, что совсем скромно. Координаты стоянки N59°25'21.40" E35°09'03.50". Вечер выдался холодным, так что пришлось нацепить шапки и держаться поближе к костру. Сидели у огня долго, сыто и разгульно. С каждым поднятием железных кружек я убеждался в правильности решения – завтра устроим тут днёвку. Пора. Душа требует. 

\begin{center}
	\psvectorian[scale=0.4]{88} % Красивый вензелёк :)
\end{center}
