\chapter{В Чагоде год спустя. 03.08.16} 
\corner{64}

Ранним утром Адмирал внезапно обнаружил себя спящим в шортах и без тельняшки у костра. Носок на одной ноге отсутствовал. Полная перезагрузка. Сознание Адмирала было чистым\sdash пречистым, а разум светлым\sdash пресветлым и надо было вести эскадру вперёд. 

Утро было тяжёлым для всех, а для Адмирала особенно. Сборы лагеря и снаряжения прошли как\sdash то мучительно долго. Один только Ваня счастливо скакал вокруг нас, поминутно спрашивая что\sdash нибудь. Получая ответ, он успокаивался на пару минут и потом задавал новый вопрос. Кто\sdash то ругнулся в сердцах, а Ваня сказал что мы, мол, можем ругаться, потому что он <<все\sdash все слова знает!>>. Бойкий мальчик, подумалось мне. 

<<Ваня!>>,\mdash повысил голос Дима,\mdash <<Прекрати!>>. А Ваня поймал кураж и не унимался: <<А что, папа? Я все\sdash все слова выучил, я всё\sdash всё знаю!>>. И, видимо, желая добить папу и показаться нам крутым, продолжил: <<Ты думаешь, я ничего не знаю, а я всё\sdash ё\sdash ё знаю! А ещё я видел в интернете, как дядя тёте\ldots >>. <<ВАНЯ!!!>>,\mdash закричал его папа, прерывая поток малолетнего сознания,\mdash <<Замолчи немедленно!!!>>. Мы не сдержались и  дружно заржали\ldots Ваня явно нарвался на отлучение от интернета. Хотя, в общем\sdash то, ограничивать всё же нужно. Хотя бы от преждевременного знакомства с <<дядя\sdash тёте>>. Мы, тем временем, продолжали сворачивание лагеря\ldots 

Красивая стоянка провожала нас в лучах утреннего солнца, и мы с Саней опять отчалили первыми. Грести мы особо не гребли, да и состояние было не слишком боевым. Течение само несло по реке Чагоде к городу Чагоде, где, как и в прошлом году, мы планировали дозаправиться в магазине продуктами. Я лишь изредка подруливал веслом, как и Саня. По GPS определил, что скорость течения примерно 5 км/ч! Вот это да! Мы совсем расслабились и не гребли, отдыхали. Я достал трубочку и расслабился ещё больше. Саня, решивший перед походом бросить курить, жестоко разочаровался в своих намерениях и уже успел запастись вместе с Пашей куревом в Смердомском.

Команда отстала, потому что снова отчаливали по\sdash очереди. Мы же с Саней, лениво подруливая вёслами, добрались до железнодорожного моста в Чагоде. Решили причалить и подняться на него, пока команда догоняет нас. Тем более, я ещё в прошлом году хотел забраться на этот мост, но тогда мы почему\sdash то прошли мимо\mdash то ли рвались быстрее в магазин, то ли я поздно заметил место, где можно было причалить. Сказано\mdash сделано, и вот мы, совершив манёвр разворота носом против течения, пристали к низкому левому берегу после моста и пошли подниматься на железнодорожную насыпь.

С моста открывался отличный вид на устье Чагоды. Тут происходит слияние рек Песи и Чагоды в Чагодощу. Стоял и думал\mdash хочется ведь и по Песи сходить. Названия рек <<Песь>> и <<Лидь>> означают <<песчаная>>, только на разных языках. Правда, на каких именно, уже не припомню. Но то, что все это неразрывно связано с песком, я хорошо усвоил в прошлом году, лихо сев тут в устье на песчаную мель прямо посреди реки. 

Песь\ldots  С заброской будет не всё так просто, но жутко хочется попасть на неё. Как хотелось в прошлом году на Лидь, так сейчас хочется на Песь. Хотя, что значит непросто, если брать машину? Да элементарно. Посетить Хвойную хочется не только из\sdash за благозвучного названия и её истории. Там, как я теперь знаю, расположен местный пивзавод и просто обязан быть при нём магазин. (примечание от 2017 года\mdash опрошенные местные о Хвойнинском пиве знать не знают, что странно) Кроме того, Песь обещает быть отличной неширокой камерной речкой на всём своём протяжении вплоть до Сазоново! Потому что, откровенно говоря, Чагодоща для байдарки уже несколько великовата\mdash ширина 50\sdash 100 м, простор огромный, ощущения камерности нет, зато есть чувство покорения водного простора. Песь иная. Она не такая широкая и больше похожа на Лидь, недаром их названия означают одно и то же. Сходим, обязательно сходим, покорим!

Вид же самого железнодорожного моста был очень запущенным. Облезлая краска, старые рельсы и шпалы, заросшие травой. Железнодорожное хозяйство тут в упадке. Сделав пару фотографий, заметили нашу бригаду, показавшуюся из\sdash за поворота. Пора к байдарке. Дальше наш путь лежал до автомобильного моста, где по старой схеме хотели десантироваться на левый берег и сходить в магазин. Прошли мимо железнодорожного семафора. Координаты N~59\degree~09.2314\textprime~E~35\degree~18.2805\textprime. Я его уже видел раньше, и причаливать не хотел; команда тоже не выразила желания и мы пошли дальше. 

Добравшись до автомобильного моста и места на левом берегу, где я причаливал в прошлом году, обнаружили там двух личностей, судя по виду, без определенного места жительства, спящих под открытым небом. Оставаться в таком соседстве, пока кто\sdash нибудь пойдёт в магазин, не захотели и поэтому переправились на противоположный берег, где никого не было. Координаты N~59\degree~09.5099\textprime~E~35\degree~19.7887\textprime. Мы с Саней снова остались караулить байдарки, а команда выдвинулась в магазин, предварительно проинструктированная мной как его найти. Они пересекли Чагоду по автомобильному мосту и сфотографировали нас, а мы их. На мосту же свершалось историческое событие в масштабах города, области, страны, планеты, галактики\mdash  перекладывали асфальт\ldots 

К байдаркам приплыли утки, причём много! И они не очень\sdash то нас с Саней боялись, из чего мы подумали, что они домашние. Саня всё пытался одну из них поймать, чтобы зажарить на ужин, но тщетно\mdash осторожные! Но это и к лучшему\mdash никто из нас не охотник и освежевать дичь не сумеет. Хотя, не сложнее же курицы, но\ldots  утки были осторожны, словно читая наши хищные гастрономические мысли.

Саня всё караулил бдительных уток, а я вытащил из байдарки плащ\sdash палатку, расстелил её на траве и улёгся отдыхать. Погода стояла облачная, лёгкий ветерок обдувал нас, в\'{о}ды Чагодощи степенно и плавно проходили мимо нас дальше, к Мологе\ldots  я лежал в полудрёме и думал о порогах\mdash как они там, ждут нас? Или поджидают?

Команда вскоре вернулась с сумками продуктов. Ваня кушал пиццу, а остальные несли позвякивающие пакеты с яствами. Распихали всё прикупленное по грузовым отсекам и скоро отчалили, взяв курс на мою прошлогоднюю стоянку за селом Мегрино. Ориентир\mdash мегринская церковь. После неё еще километра два и мы должны быть на месте.

К церкви подошли около пяти часов вечера. В этот раз при низкой серой облачности место воспринималось совершенно по\sdash другому, не так торжественно что ли, но церковь всё равно смотрелась очень красиво\mdash белая, на левом берегу, среди редких сосен.
 
Трос через реку для парома в Мегрино был на месте, только мне показалось, что красных флажков на нем несколько поубавилось. Парома по\sdash прежнему было не видно. Итак, оставалось около двух километров до стоянки. Я помнил, что место там появлялось как\sdash то внезапно среди неказистых берегов. Так, естественно, и произошло. После поля на левом берегу и какого\sdash то разваленного сарая последовал перелесок и вот\mdash шикарный подъём и большая ровная поляна, окружённая соснами и берёзами. Причаливаем! 

Я снова нашел свои старые костровые палки, подумать только! Мы их использовали повторно. Дров было завались\mdash кто\sdash то распилил прошлогоднюю огромную поваленную берёзу. Из той же берёзы, судя по всему, сделали новые брёвна\sdash скамейки около стола. Словом\mdash место класс! Можно было бы и устроить тут днёвку, но команда пожелала найти место более безлюдное\mdash всё таки рядом село и пилорама в Горках. Что ж, с этим я согласен, будем искать днёвку завтра. Только, насколько я помню, вплоть до Званца места будут не очень\mdash заросшие берега и лиственный лес. А значит, днёвки завтра нам не видать. Но команда этого не знает и надеется. Текущая же стоянка всем понравилась (ещё бы!), мы остались на ночёвку и стали разгружаться. Координаты места N~59\degree~08.1418\textprime~E~35\degree~37.1077\textprime.

Тем временем, лагерь уже оборудован, костер пылает, вовсю идёт заготовка дров. Я успеваю искупаться и вымыться в реке перед вечерней трапезой. Солнце садится в тучу красным цветом. Не очень хороший знак, как бы не испортилась погода!  

У нас, как всегда, шикарнейший ужин\mdash в этот раз даже на столике и со скамьями из брёвен. Откуда ни возьмись, Дима достаёт забугорное снадобье из агавы. В этом сплаве он и Саня\mdash главные виночерпии, наполняющие наши железные кубки. Под вечерний супец, да с прибаутками, всё улетает просто на ура. 

Ночь спускалась незвёздная, небо заволокло низкими тучами. Жаль! В этот раз я тащу с собой уже не просто фотострубцину, как в прошлогоднем сплаве, а полноценный фотоштатив\mdash хочу поснимать всё те же астропейзажи, но каждый раз к полуночи небо заволакивает облаками\ldots  не везёт.

Укрыв палатку полиэтиленовой пленкой от дождя, я заполз в спальный мешок и моментально заснул. Назавтра, как я помнил с прошлого сплава, должен был быть долгий ходовой день\ldots 

\begin{center}
	\psvectorian[scale=0.4]{88} % Красивый вензелёк :)
\end{center}
