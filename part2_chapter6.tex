\chapter{Проходим устье Лиди. 02.08.16} 
\corner{64}

С утра мы немного проспали, зато выспались, что так необходимо было после ночной заброски. Насчёт километража я был спокоен\mdash  мы слегка перевыполнили норму вчера, да и течение! Течение помогает нам очень сильно! При таком течении всё пойдёт гладко, был уверен я. Приготовили на завтрак мою традиционную походную кашу\sdash овсянку с орехами и сухофруктами. Подкрепились. Потом собрали вещи, свернули лагерь и стали грузиться в байдарки по\sdash очереди, потому что выход к воде был очень узким. 

На этот раз мы с Саней отчалили первыми. Я рвался вперед\mdash  сегодня по плану проходим устье моей старой знакомой\mdash  Лиди, и останавливаемся в районе моей прошлогодней стоянки. Саня уже вчера приноровился к гребле, и мы почти не петляли в русле. Надо будет в следующем сплаве попробовать поставить руль на байдарку\ldots  конечно, сделать это будет трудновато, потому что я люблю на корму привязывать вещи, но\ldots  можно постараться. Хотя С.Ю. и говорит, что руль для байдарки это довольно бесполезная вещь, и я в целом с ним согласен, попробовать всё же хочется\ldots 

Буквально через 200\sdash 300 метров после выхода на воду, мы с Саней замечаем на левом берегу огромную отличнейшую ровную стоянку\ldots  работа 2\sdash го Закона Лучшей Стоянки: <<Лучшая Стоянка чуть ниже по течению>>. Причаливаем, выходим на берег. Да что ж такое! Места много, есть приличный стол, скамьи, причём довольно свежие. Но зато есть наезженный подъезд\mdash  место, видимо, популярно. В сторонке стоит стенд (!) с плакатами, посвященными Тихвинской водной системе. Координаты N~59\degree~12.9085\textprime~E~34\degrees~57.3281\textprime. Ничего не поделаешь\mdash  не дошли сюда, так не дошли. На нашем месте тоже было неплохо. Остальная часть флотилии, тем временем, уже подплывает к нам. Занимаем места в своей байдарке, и наша эскадра выдвигается дальше вниз по Чагоде\ldots  

Берега в этот день не радуют\mdash  сплошные заросли. Перед Смердомским на левом берегу встретилось одно хорошее место для стоянки с навесом и столом. Но, во\sdash первых, очень близко к населённому пункту, а во\sdash вторых, километраж на сегодня ещё не прошли. Координаты места N~59\degree~11.2177\textprime~E~35\degree~02.4071\textprime. Вскоре приближаемся к Смердомскому, названному так, вероятно, из\sdash за стекольного заводика, который находится тут с незапамятных времён и <<смердит>>. С воды видна еле\sdash еле дымящая труба, склад завода и готовая продукция\mdash  бутылки в прозрачных контейнерах. Причаливаем, команда желает сходить в магазин. Координаты N~59\degree~10.8221\textprime~E~35\degree~04.0067\textprime. Мы с Саней остаёмся караулить байдарки и маяться бездельем. Немного общаемся с местным, который косит траву на берегу. Потом приходят парень с девушкой, и отплывают на моторке рыбачить. Скоро возвращаются и наши гонцы из магазина, позвякивая пакетами\ldots 

В этот день прошли под мостом через трассу А\sdash 114 в Первомайском, по которому проезжали при заброске. Потом прошли и деревню Анисимово, где, согласно старым картам ГШ, должен был быть водомерный пункт. Разыгравшееся воображение рисовало водомерный пункт в виде покосившейся будки, откуда суровый мужик в ватнике, сапогах и шапке ушанке выходит раз в день ровно в полдень и идёт к реке снимать показания с огромной закопанной там на берегу линейки, а потом аккуратно записывает их в толстенную амбарную книгу наклонным ГОСТ\sdash овским шрифтом. Однако, ничего этого, конечно же, там не оказалось. Водомерного пункта не было или мы не заметили его. Собственно, это нас нисколько не огорчило, уровень воды был очень высок, о чем я судил по своим замерам веслом, по прибрежной растительности, а также по следам от на опорах мостов. 

Мы шли дальше\mdash  навстречу перекату перед устьем Лиди. О его существовании я узнал из прочитанных при подготовке к сплаву отчётов, но не был уверен, что по такой высокой воде мы его ощутим. Так и вышло. Слева впала Лидь, переката никакого на Чагоде не было. Более того, в устье Лиди тоже не было видно переката, хотя в прошлом году по низкой воде я очень хорошо на нём протрясся. Так же, как и в прошлом году, причалили на левом берегу около турбазы тут же в устье и расположенной. Вылезли походить и размяться. 

С базы доносился собачий лай, но смотритель, как в прошлом году, не вышел. Глядя на увеличившуюся ширину реки, С.Ю. слегка раздосадовано сказал, что самый интересный участок мы прошли в первые же 2 дня. Да, река стала гораздо шире, но течение! Течение продолжало радовать. Идти по узкой речке здорово\mdash  есть атмосфера камерности и миниатюрности. Тут же ощущаешь размах и простор.

Маршрут дальше по Чагоде я уже проходил в прошлом году, так что почувствовал своеобразную уверенность, и мы понеслись дальше\mdash  до моей прошлогодней стоянки. Здорово и интересно идти в неизвестность по новой реке, но в уверенности на пройденном ранее маршруте тоже что\sdash то есть. Потому что каждый год всё по\sdash разному, ничего не повторяется точь\sdash в\sdash точь и снова можно открыть для себя что\sdash то новое. Одновременно чувствуешь себя бывалым, опытным проводником среди дикой природы, сталкером (по Стругацким), можно сказать.
 
Перекатов на Чагоде после впадения Лиди мы также не заметили, хотя в прошлом году они были и добавили немного экстрима в маршрут. Остаток нашего сплавного дня прошёл достаточно спокойно и размеренно, но к концу я уже просто ёрзал, сидя в байдарке на своем гермомешке\mdash  так хотелось быстрее увидеть свою прошлогоднюю стоянку и что там на её месте стало. 

Стоянка в прошлом году тут под Чагодой конечно была не особо хорошая\mdash  мало ровного места и донимали полчища муравьёв. Но мы всё равно причалили для осмотра. Я нашел своё костровище и даже мои костровые палки продолжали стоять на том же месте! Чуть ниже по течению была ровная площадка, я её заприметил ещё с прошлого года. Решили дойти до неё и причалить. Там оказалось слишком много мусора и, что особенно плохо\mdash  много битого стекла. Настолько много, что я не могу поверить в то, что к этому причастны немногочисленные сплавщики. Скорее местные из Чагоды приезжают сюда <<культурно>> отдохнуть на алкорыбалку. 

Экипаж С.Ю. прошел ещё метров 200\sdash 300 ниже по течению, разведал место и вернулся. Как они утверждали, там было лучше, но снова крутой подъем. С прошлого года я такого не помнил, но отчего же не попытать счастья? Мы с Пашей вдвоём пошли пешком в разведку. Разведанное место пришлось мне по душе, решили оставаться. Ну а крутой спуск к воде\mdash  нам не привыкать, зато поляна просто загляденье\mdash  вокруг высокие сосны и нет адских муравьев. Координаты стоянки N~59\degree~09.4813\textprime~E~35\degrees~14.5561\textprime.

Причаливали и перетаскивали вещи снова по\sdash очереди. Бригаду смущало, что в двадцати метрах от стоянки проходит грунтовая дорога, но я уверил всех, что тут вообще никто не ездит. Забегая вперед, скажу, что пока мы стояли, ни одна машина не прошла мимо нас, потому что был вторник, а не выходные. Эта грунтовка соединяет ту базу отдыха, что стоит в устье Лиди, с городом и ездят по ней просто ну крайне редко. Эта же грунтовка, надо полагать, тянется и дальше\mdash вдоль берега Лиди в сторону Тургоши.

Вырубили костровые палки в молодой поросли сосен и берёз на другой стороне грунтовки, обустроили костровище на холмике, организовали полевую кухню. Место радовало\mdash красивый пригорок, высокие янтарные сосны на крутом склоне к воде. К закату у нас уже был отменный ужин, и Дима снова с блеском в глазах извлек Jameson, который был немедленно уничтожен. 

В прошлом году я дневал тут чуть выше по течению. Но тогда это была вынужденная днёвка\mdash  надо было заклеить байдарку после <<блестящего>> прохождения шивер. И тогда это был уже третий сплавной день, а в этот раз только второй\mdash  рановато дневать. Поэтому на совете Капитанов решили остаться тут только на ночёвку, хоть место, в принципе, и подходило для днёвки.

Солнце уже не освещало стоянку, зато верхушки сосен на противоположном берегу были великолепны в его розово\sdash оранжевых закатных лучах. Ночь спускалась не звёздная, небо снова заволокло серыми тучами. Горючее закончилось, народ загрустил, стал расползаться по палаткам. Но тут я достал десантную флягу с 96\sdash ым, передал её Сане для разведения в пропорции <<два к трём>> и мы с Саней и Пашей продолжили посиделки. Я вспомнил песню Дольского <<Выходной без любви>> и немедленно приступил к исполнению\ldots  <<А мы привратнику кричим: <<Откройте двери\ldots>>, на нас там занято, но нам швейцар не верит\ldots Открыл и пузом отодвинул он Алёшу\ldots>>

\begin{center}
	\psvectorian[scale=0.4]{88} % Красивый вензелёк :)
\end{center}
