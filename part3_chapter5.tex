\chapter{Тресновская ГЭС} 
\corner{64}

С утра я традиционно приготовил свой фирменный походный завтрак – овсяная каша на сухом молоке и сгущёнке, с добавлением кураги, чернослива и орехов. Вышло, как всегда, отлично. Вдобавок, в этом году не удалось купить нормального сухого молока, поэтому взяли старые запасы, а также приобрели сухие сливки, которые по составу обещали быть натуральными и не обманули – это было нечто! Жирность была что надо. Каша стала сразу похожей на кашу, а не просто так, лёгкая закуска! 
Снова обработались заканчивающимися репеллентами, но это как-то не возымело никакого эффекта на полчища комаров и мошки… надо брать репелленты с содержанием средства ДЭТА как можно больше. А в наших этой ДЭТЫ оказалось так, «для галочки». До Заборья около двух ходовых дней, понимали мы – раньше магазинов нет. Вопрос с комарами стал просто животрепещущим! Одеколон «гвоздика», взятый С.Ю. в качестве проверенного средства от комаров, тоже помогал лишь на несколько минут. Оставалась только одна надежда – что в селе Лидь, а это в пределах одного ходового дня, есть хоть какой-нибудь магазин.
Свернули лагерь, погрузились и отчалили нормально, не поздно, в районе 10 часов утра. Олег был новичком в нашей команде, но всё схватывал на лету, что очень радовало – ничего не надо было разъяснять или показывать – чувствовался человек, много проводящий в околопоходных условиях на природе – на рыбалке… а это очень важно – когда идёшь таким микроколлективом, то люди должны понимать что надо делать даже не с полуслова, а на уровне телепатии. И в нашей команде, к счастью, это было так или почти так. Слаженность действий и их следование единому замыслу. Но довольно философии – пора на воду!
На воде ждало спасение – комары и мошка отставали. Лишь слепни временами догоняли, но вскоре быстро отставали, отгоняемые нами. Так, мы неспешно шли дальше, увлекаемые тёмными водами Лиди. Временами, как и вчера, встречались подобия перекатиков, но о-о-очень, очень отдалённые – воды в этом году было много и все камушки глубоко под водой, что не может не радовать – очень не хочется добавить себе заплаток на днище. 
Вскоре дошли до развалин тресновской ГЭС. Отчего никто не достал фотоаппарат – вообще не пойму. Зрелище было завораживающим – большая разрушенная бетонная плотина, сплошь заросшая лесом и густым подлеском, а посередине ревущий слив примерно на метр. Естественно, сходу проходить не стали – звук клокочущей воды как-то не способствовал. Причалили для осмотра к левому берегу. Берега крутые, вылезти нормально можно только в одном месте. Я осторожно оказался на берегу и стал пробираться сквозь заросли оценить ситуацию. Олег держался за прибрежную траву, чтобы байдарку не сносило ниже. Экипаж С.Ю. тоже причалил позади нас и стал потихоньку вылезать. Аккуратно взобрались на плотину и прошли ниже по течению оценить характер слива. С таким препятствием я столкнулся впервые. Вышел, подошел к краю обрыва и, цепляясь за ломкую ольху, которая не преминула сломаться, спустился к воде, к самому ревущему водопаду. Ничего непонятно. С.Ю. попытался пройти хотя бы метр по воде на слив, чтобы понять – есть там брёвна, препятствия или что-то подобное – иными словами – можно ли пройти слив на байдарке или надо обноситься? Перспектива обноса не радовала – подъём на плотину весьма крутой, сплошные заросли… короче та ещё прелесть. По итогам осмотра ничего толком не поняли и решили всё же попробовать пройти слив ближе к левому берегу – рискнуть. Я продолжал стоять на небольшом островке перед сливом. Экипаж С.Ю. занял места в байдарке, и они пошли первыми – я страховал на подходе к сливу и наблюдал за происходящим…
Байдарка С.Ю. немного криво вошла в слив… секунды казались вечностью… и вот… Падение! Волны! Рёв воды! Белая пена… уф-ф-ф! Прошли! Нормально! Лихо заложили разворот и встали у берега против течения, ожидая нас. Прокричали нам сквозь шум слива и показали жестами, что всё в норме. Слив зловеще клокочет! Прямо как настоящий Ниагарский водопад! Но довольно, теперь наша очередь! Я вернулся к своему экипажу, занял место в корме, сел повыше, чтобы лучше видеть ситуацию впереди и мы предельно аккуратно, у самого левого берега, слегка цепляя камни днищем, стали ювелирно подходить к ревущему сливу… 
Секунды тянулись безумно медленно! Казалось, будто время остановилось в сознании! Олегу крикнул не подруливать веслом, а сам обратными гребками выравнивал курс, придавая нам небольшую отрицательную скорость... мы медленно, но верно приближались к зловещей пропасти…. время как бы мгновенно разморозилось и полетело стремглав в неизвестность… впереди был слив! Нос байдарки нырнул в пену… и сразу же вынырнул – вода как бы не поняла, что могла перелиться через борт, и отступила. Байдарка будто бы изогнулась легкой дугой, и вот уже корма, спустя доли секунды после носа, прошла сквозь белую, кричащую, словно в ужасе, пену и перепад высот. В действительности прошли какие-то мгновения, доли секунды, но в моём сознании подход к сливу и падение были вечностью… Мы прошли! Невероятно! А казалось страшнее, чем было на самом деле!  
Экипаж С.Ю. подождал нас, и мы вместе пошли дальше кильватерной колонной. Жена выразила восторг от прохождения слива! Ещё бы! Я и сам был на мощнейшем подъёме – такое препятствие! Сначала, конечно, всем было боязно, и неизвестность пугала, но сейчас мы счастливо обсуждали, как классно нырнули в эту пену. Осознание преодоления этой плотины придало нам сил и уверенности. Небольшой экстрим – те самые ощущения, ради которых мы здесь. 
Судя по старому описанию нашего маршрута, при высокой воде он относится ко второй категории сложности. Так что мы можем, почти не моргнув глазом, рассказывать теперь, что прошли по маршруту второй категории сложности – впереди нас ждала ещё одна плотина… но это потом, не сегодня… сегодня нам бы подобраться вплотную к селу Лидь. Олег и Паша впервые проходили такое препятствие и держались молодцами. Мы уже с женой проходили перекаты на Зилиме, потом знаменитый сливчик на Киржаче… а я ещё и плотину, которая дальше тут на Лиди после Заборья. Опыт С.Ю. тоже чего-нибудь, да значил. Так что тресновская ГЭС стала «боевым крещением» для наших двух матросов.
Плыли дальше и думали – отчего мы все так растерялись и забыли про фото и видеосъемку? Невероятно! И я-то тоже хорош! Мог бы додуматься достать фотоаппарат! Такое красивое интересное место и так бездарно прохлопать возможность запечатлеть его! Придётся возвращаться против течения – шутили мы. На самом деле, маршрут по Лиди достоин того, чтобы проходить его вновь и вновь. Места удивительно дикие, красивые… хоть и затопленные в этом году – все песчаные пляжи скрыты под водой, подходы сплошь поросли густой травой, о стоянках байдарочников и речи нет – за год пройдет максимум 1-2 группы, судя по стоянкам, а точнее по их отсутствию.… 
Мда-а-а, с фотосъемкой разрушенной плотины явно сплоховали. Но, с другой стороны, не станешь же напрасно рисковать и доставать фототехнику, когда вокруг ситуация каждое мгновение грозит бедою? Поэтому-то и ставят страховки ниже по течению, оператора со штативом и камерой у слива, имеют рации… так проходят препятствия, пороги и сливы настоящие профессионалы, всякие клубные сплавщики с супер-пупер снаряжением, касками, спасжилетами и всё вот это вот, ну вы понимаете… а кто мы? Мы так, любители этого дела, забредшие путники, потерянные души в этом забытом всеми крае, приехавшие сюда вовсе не за фотокарточками (хотя, кого мы обманываем, за ними тоже), не за этим каким-то показным тщеславием от прохождения маршрута… мы приехали сюда, нет, даже не приехали, а сбежали сюда (!) для отчуждения – для полного отчуждения – в попытках уйти от реальности и отстранится от привычного… и это удаётся нам сполна – красавица Лидь не жалея окунает в эту атмосферу…
В этот день прошли урочище Тресно, по имени которого и была названа та плотина выше по течению. Когда-то это урочище, видимо, было деревней, а сейчас – один полузаброшенный домик с хорошим выходом к воде. Мы, повинуясь сильному течению, проплываем мимо – причаливать нет желания… да и не ясно – жилой ли дом или просто это временное пристанище для рабочих, лесорубов или лесников.
Мы рвёмся вперёд и вскоре нашему взору открывается преграда… деревянный мост, положенный на бетонные трубы большого диаметра. Течение ускоряется и грозит бедою – как проходить этот мост – не ясно. Видны завалы в трубах. Мы аккуратно зачаливаемся у правого берега, в зарослях какой-то прибрежной травы. Вылезаем все без исключения на берег и идём в разведку. Мост широкий, деревянный сверху, в приличном состоянии, способном выдержать прохождение лесовозов – для чего он, собственно, скорее всего, тут и сделан – видны следы протекторов шин грузовых автомобилей. Обносить байдарки по берегу дико не охота, лень. В двух бетонных трубах моста нет завалов из плавника и можно попытаться пройти там, но вход перегораживает прибитое течением бревно. Пока Паша ведет разведку у трубы близ правого берега, мы с Олегом спускаемся к трубе у левого берега. Если подвинуть прибитое течением бревно – можно пройти! Но бревно прижимает сильным течением – сделать это будет очень нелегко… да и неизвестно, вдруг там что-то на дне лежит в этой трубе? Чтобы развеять сомнения, я спускаюсь с моста вниз, ниже по течениию, к трубе у левого берега и аккуратно, против приличного течения, захожу в неё, держась сверху за деревянные перекрытия. Прохожу практически до середины пролёта – всё нормально – камней нет, острых деревяшек тоже. Решаю, что точно надо идти тут, только избавившись от того бревна на входе.
Паша возвращается от соседней трубы и настаивает идти там, а С.Ю. колеблется. Олег ждёт, пока мы с С.Ю. определимся. Я тоже иду оценить вторую трубу и понимаю, что там идти вообще не вариант – в первой глубже, по моему мнению. Остаётся единственно верное решение – оттолкнуть то бревно, как я и хотел, и проходить в трубе у левого берега. Так и поступаем. Находим с Олегом оглоблю прямо тут, на настиле моста, и, спустя несколько минут, используя её как огромный рычаг, с трудом вдвоём отодвигаем бревно от прохода в сторону. Теперь можно идти! 
Первыми снова пускаем экипаж С.Ю. Они у нас в роли разведчиков. Заходят нормально в створ трубы только со второй попытки – сначала их относит к опоре моста и приходится резко уходить назад обратными гребками, выравниваться, а затем пробовать снова. И вот – заход… выравнивание… уф!!! Прошли нормально, слегка пригнув головы в трубе. Теперь наша очередь. Мы тоже сперва заходим против течения подальше, а потом, выравнивая байдарку обратными гребками, предельно аккуратно заходим в эту трубу. На выходе нас уже ждут С.Ю. и Паша… и опять отчего-то не засняли на фото ни мост, ни трубы, ни прохождение… что такое? Эмоциональный перегруз, не иначе. Жаль, очень жаль! Место очень было интересное. Но мы идём дальше.
В этот день больше никаких препятствий не встретилось, и мы размеренно плыли в поисках стоянки. Крайняя точка на сегодня – это село Лидь, поэтому надо было найти стоянку до него. Но найти хорошее место как раз не выходило – все берега чересчур лесистые и заболоченные или травы по пояс… наконец, когда уже из-за очередного поворота реки показался дом села Лидь (а другого ничего тут по карте и нет), мы заприметили выход к воде на правом берегу. Естественно, причалили и пошли на разведку. Место было так себе – и близко к деревне, и следы старой стоянки рыбаков с мусором… но, при этом и поляна большая по соседству и сход к воде более менее приличный. Да и хотелось уже встать лагерем, потому что если идти дальше – то километров 5 точно – надо пройти село тогда и вставать за ним…
В итоге решили оставаться. Перетаскали вещи, вытащили байдарки, разбили лагерь. Костер развели около поваленного дерева, чтобы использовать его как лавочку. Комары не дремали и жадно принялись за кровопускания свежей партии человечинки… репелленты закончились. Стало ясно – или завтра мы купим в этом селе репелленты, или ещё двое суток терпеть эти отнюдь не повышающие боевой дух укусы кровососущих.
Над быстро вытоптанной поляной возник наш тент – продукты и снаряжение, как всегда, сложили под ним. Между часто росшими соснами на окраине поляны растянули веревку и развесили сушить вещи. Олег с Пашей быстро поставили палатки и удалились на берег реки рыбачить. Мы с женой и С.Ю. потихоньку кашеварили и сделали чай. Этот день принёс немало экстрима – одна плотина чего только стоила, а разведка и прохождение моста заняли не менее часа наверно. 
Комары снова просто одолевают. Репелленты закончились окончательно и бесповоротно… Я вспоминаю, что у меня есть в аптечке вьетнамская «звёздочка»… она же пахучая, ведь так? А что, если этот адский запах отпугнёт насекомых? Надо проверить – быстро к палатке, лезу в герму. Вот и мой подсумок с аптечкой – нахожу эта крохотную баночку, мучительно открываю её ногтями и немедленно смазываю свои кисти рук и лоб, а также шею… и тут понимаю, что если шее и рукам – нормально такие издевательства, то напрочь искусанный лоб был к этому явно не готов – ощущения те ещё! Просто бодрящая свежесть! Но это не главное. Как ведут себя комары? Наблюдаем… держатся на почтительном расстоянии в несколько сантиметров и садиться не хотят явно! Ура! Кратковременная победа! Все желающие немедленно пользуются возможностью хоть немного облегчить участь своих открытых участков тела…
Наши рыбаки потихоньку приносили с берега добычу – некрупных рыбёшек, которых мы решили назавтра зажарить на чугунной сковородке, запасливо взятой с собой. Провиант и горючее планомерно уничтожались, и комары как-то снова отходили от этого на второй план… Олег, как и все, притомился и гитару даже не расчехляли.
Сегодня прошли 16.5 километров – немного, казалось бы, зато впечатлений – хоть отбавляй. Завтра нас ждало прохождение села Лидь и бросок до окрестностей Заборья, а пока – сплавное братство гуляло, щедро разливая из канистры. Песен не пели, но долго уменьшали запасы горючего, запивая его душистым чаем. Походный быт к этому дню был окончательно налажен, и всё было как надо. Спать завалились поздно в надежде, что хоть завтра комары умерят свои таёжные аппетиты.

\begin{center}
	\psvectorian[scale=0.4]{88} % Красивый вензелёк :)
\end{center}
