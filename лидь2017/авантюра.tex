\chapter{Авантюра} 
\corner{64}

Работы на работе нет. После успешного «бумажного» закрытия очередного этапа\mdash полгода тишины и урезания зарплаты. Загнивающий «ящик» во всей красе. Атмосфера беспросветной унылости. Всё тот же облезлый кожзам кресел у окна и наши с С.Ю. обсуждения маршрута. Выходило странно. Вроде бы утвердили на этот год реку Песь. Причём очень заранее утвердили\mdash ещё зимой, кажется. Я распланировал все переходы, примерные места стоянок$\ldots$ запланировал посещение песчаного карьера и заброшенного паравозного депо Горны, что на железнодорожной ветке, соединяющей Подборовье и Кабожу. И даже уже заложил путевые точки и маршрут сплава в GPS-навигатор$\ldots$ но что-то витало в воздухе такое неуловимое$\ldots$
 
\mdash «Ли-и-идь!»,\mdash напевал северный ветер странствий и жажда реванша над этой коварной красавицей,\mdash «Ли-и-и-и-идь!»,\mdash вторило ему эхом желание С.Ю. двухгодичной давности пройти по этой реке. И вроде всё ровно и гладко распланировано по Песи$\ldots$ что же такого особенного зовёт на Лидь? Ощущение полного, абсолютнейшего отрыва от всего. Там не встретишь байдарочников. Не будет множества стоянок. Почти нет деревень. Нет сёл. Ничего нет. Только река разбивает то низкие, то высокие песчаные берега своей чистейшей тёмной водой$\ldots$ Когда с утра вылезаешь из палатки\mdash вокруг рассветный туман. Стоишь так один посреди лагеря в невольном оцепенении, ощущая себя песчинкой в масштабах природы, планеты, космоса$\ldots$

Эта необъяснимая тяга к отчуждению сделала своё дело. Мы стали колебаться. Я заложил в навигатор ещё и маршрут по Лиди. Так, на всякий случай. Точное решение пока не пришло$\ldots$ шла внутренняя борьба запланированного с авантюрным.

А пока собирали снаряжение. Идти на речку не впервой, поэтому сборы прошли гладко\mdash привез всё необходимое из деревни в город, разложили, как обычно, на полу в квартире, перепаковали. Докупили с женой ещё один надувной матрас и пару армейских вещмешков для снаряжения. Всё. Остальное есть с прошлых сплавов. И самое главное и радостное\mdash в этом году нам с женой удалось совместить отпуска и взять по две недели на конец июля! Ура-а-а! Будем вдвоём! А то уже два года подряд не получалось провести оба отпуска по две недели вместе.

Обычно в конце июля стоит тёплая погода, и мы надеялись, что в этом году будет так же, как, например, в 2016-ом. Ан нет! Лето 2017-го выдалось непривычно холодным! И дождливым. И вообще непохожим на лето. С ужасом следил я за прогнозами погоды, тщетно ожидая улучшения. Улучшение не наступало. Зато дожди лили с завидной регулярностью. Уровень воды в Мологе рос не по дням, а по часам, на полтора (!!!) метра превышая среднее значение для июля, о чем я узнавал в интернете на ГИС портале регистра и кадастра, где есть возможность посмотреть уровень на водомерном посту в Устюжне. 

Что ж, если в Мологе столько воды, то в Лиди и Песи тоже должен быть высокий уровень. Это придавало некоторую уверенность, что если всё-таки пойдем на Лидь, то вероятность прибиться там на камнях, как в 2015-ом, будет гораздо ниже. Но клей и заплатки, естественно, бережно проверил и взял с собой.

По команде в этот раз получалось совсем как-то уж грустно. Зимой я познакомился с молодой семейной парой, не боящейся туристических вылазок на природу. С ними потекла длительная бурная переписка о сплавах и всей этой кухне, и я уже было вот-вот понадеялся, что у нас +2 человека в команде$\ldots$ но, как всегда, непредвиденные обстоятельства и они оба не смогли составить нам компанию. Грустно. 

Хотели пойти тремя байдарками, потому что уж очень понравилось это в 2016-ом. Первый экипаж\mdash мы с женой, второй это Паша и С.Ю.\mdash товарищи с работы, а с третьим проблема\mdash Дима с Ваней не смогли по датам. Геннадьич в этом году, к сожалению, отказался. Я как всегда кинул без особого энтузиазма клич по ближайшим знакомым\mdash ожидаемо никто не согласился. Вообще стало как-то грустновато. Жена тоже считала, что надо ещё народу на один экипаж или хотя бы одного человека. И тут я нашёл у себя в контактах Олега, с которым мы были знакомы через моего одногруппника по институту\mdash Мишу. Олег скептически отнёсся к перспективе выбить отпуск на требуемые даты, но одновременно, внезапно почти сразу, выразил желание идти с нами. Я же, в свою очередь, не давил и ждал окончательного ответа. 

Параллельно мы с С.Ю. думали, что делать с заброской. Олег пока не давал точного ответа, так как не было написано заявление на отпуск, и нас в команде официально было четыре человека. Я искал нам машину типа «Газели», чтобы заброситься сразу от Москвы$\ldots$ но по мере поисков приходило осознание того, что это мероприятие выльется в солидную копеечку, по сравнению даже с прошлым годом, потому что нас идёт всего четверо, да и мы с женой\mdash один кошелёк на двоих$\ldots$ вообще и грусть, и печаль, и всё вот это вот. И с командой ещё к тому же непонятно$\ldots$ то ли будут ещё люди, то ли нет. Пришлось привезти в город оба комплекта байдарки\mdash на «Таймень-2» и «Таймень-3». Таким образом, я был готов к любому развитию событий по численности команды$\ldots$ и ещё пришла мысль\mdash уж если пойдём вчетвером\mdash то не плацкартом ли ехать до Ефимовского?

Олег, тем временем, уладив вопрос с отпуском, подтвердил своё решение и согласие идти с нами. Поиски людей в команду прекратили и окончательно решили идти двумя байдарками\mdash КНБ двушкой «Гарпун» и моей трёшкой «Таймень». И как-то незаметно выкристаллизовалось решение ехать в путешествие на своих машинах$\ldots$ да-да! Представьте, эти прожжённые любители плацкарта выкинули такой фортель! Не так давно став с женой счастливыми обладателями первого в своей жизни авто, пусть и не нового, нам, естественно, хотелось впервые в жизни проехаться на такое дальнее расстояние\mdash около 750 км, тем более, что я уже более менее освоился на дороге и привык к сцеплению. Ещё раз всё взвесив несколько раз, мы с женой, С.Ю. и Олегом решили, что однозначно надо ехать на своих авто\mdash это и экономия, и удобство, и вообще первое, второе и компот\mdash полная свобода действий! Паша тоже поддержал такое наше решение, тем более, что это означало существенную экономию денежных средств. 

Но машины надо где-то оставить на время, собственно, сплава. Стали искать гостиницы в Чагоде\mdash получалось, что эта географическая точка стала для нас чем-то вроде центра притяжения\mdash от неё мы бы сравнительно легко попали дальше хоть на Песь к Ракитинскому озеру, как первоначально планировали, хоть на Лидь в Радогощь, хоть на тот же Горюн к Вожанскому озеру как в прошлом году. В результате переписки с гостиницей в Чагоде, мы решили оставить машины у них на стоянке. До стапеля же дальше решили добираться на уазике «буханке», которую любезно пообещали достать для нас в гостинице. 

Вопрос с гостиницей и стоянкой уладили, вещи все собраны, остаётся самое неприятное\mdash дождаться, наконец, отпуска и отпускных. А потом\mdash заправить полный бак и вперёд, на вольные луга\mdash полный Отрыв, полная Свобода! 

Традиционно заранее организованно закупились продуктами. На этот раз мы с женой вдвоём поехали на машине и купили всю провизию в сетевом магазине, а за тушёнкой отправились вместе с С.Ю.\mdash на базу в Перово, как и в прошлом году. Взяли, как обычно, белорусскую\mdash говяжью и свиную, а также белорусское сгущённое молоко в банках и пластиковых тубах. Как хорошо, что батька их там ещё держит за причинное место\mdash продукты что надо! Продуктовая эпопея на этом почти закончилась\mdash надо было ещё купить еду в дорогу, а С.Ю. должен был достать сала на всех.

По техническому оснащению в этом году у нас было небольшое новшество\mdash весной я обзавелся двумя радиостанциями и теперь решил непременно взять их с собой, чтобы держать связь между машинами и, возможно, потом между байдарками, потому что нередки случаи, когда один из экипажей отстаёт. Поэтому, помимо штатного аккумулятора, к рациям были взяты дополнительные батарейные отсеки и упаковка элементов питания. Конечно, для полного счастья неплохо бы и ещё тангенты выносные и антенну на крышу авто, но это, видимо, уже в следующий раз. Пока что и этого было достаточно. Продукция китайского радиопрома непривычно удивила качеством сборки и работы\mdash ничего не люфтило и качество звука было отменным. Посмотрим, как рации покажут себя в деле!

Итак, сборы были окончены, команда обещала быть отличной, потому что 4/5 её уже были испробованы в предыдущих сплавах, а Олег внушал доверие. Последняя рабочая пятница, естественно, неотвратимо настала и мы с С.Ю. и Пашей разошлись с работы по домам, чтобы назавтра, в субботу, снова встретиться в Новокосино на точке сбора и начать нашу очередную авантюру, наше погружение в мир путешествий, открытий и первобытности.

\begin{center}
	\psvectorian[scale=0.4]{88} % Красивый вензелёк :)
\end{center}
