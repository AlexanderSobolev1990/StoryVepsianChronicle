\chapter{Тресновская ГЭС} 
%\corner{64}
\vepsianrose

С утра я традиционно приготовил свой фирменный походный завтрак\mdash овсяная каша на сухом молоке и~сгущёнке, с добавлением кураги, чернослива и орехов. Вышло, как всегда, отлично. Вдобавок, в этом году не~удалось купить нормального сухого молока, поэтому взяли старые запасы, а также приобрели сухие сливки, которые по~составу обещали быть натуральными и не обманули\mdash это было нечто! Жирность была что надо\mdash каша на сливках стала сразу похожей на кашу, а не просто так, лёгкая закуска! 

Снова обработались заканчивающимися репеллентами, но это как-то не возымело никакого эффекта на полчища комаров и мошки$\ldots$ надо брать репелленты с содержанием средства ДЭТА как можно больше. А в наших этой ДЭТА оказалось так, <<для галочки>>. До Заборья около двух ходовых дней и раньше никаких магазинов нет. Вопрос с комарами стал просто животрепещущим! Одеколон <<гвоздика>>, взятый С.Ю. в качестве проверенного средства от комаров, тоже помогал лишь на несколько минут. Оставалась только одна слабая надежда\mdash что в селе Лидь, а~это в пределах одного ходового дня, есть хоть какой\sdash нибудь магазинчик.

Свернули лагерь, погрузились и отчалили нормально, не~поздно, в районе 10 часов утра. Олег был новичком в нашей команде, но всё схватывал на лету, что очень радовало\mdash ничего не надо было разъяснять или показывать\mdash чувствовался человек, много проводящий в околопоходных условиях на природе\mdash на рыбалке$\ldots$ а~это очень важно\mdash когда идёшь таким микроколлективом, то~люди должны понимать что надо делать даже не~с~полуслова, а на уровне телепатии. И в нашей команде, к счастью, это было так или почти так. Слаженность действий и их следование единому замыслу. Но довольно философии\mdash пора на воду!

На воде ждало спасение\mdash комары и мошка отставали. Лишь слепни временами догоняли, но вскоре быстро отставали, отгоняемые нами. Так, мы и шли дальше, увлекаемые тёмными водами Лиди. Временами, как и~вчера, встречались подобия перекатиков, но о\sdash о\sdash очень, очень отдалённые\mdash воды в этом году много и все камушки было глубоко под водой, что не может не радовать\mdash очень не~хотелось добавить себе заплаток на днище. 

Вскоре дошли до развалин Тресновской ГЭС. Отчего никто не достал фотоаппарат\mdash вообще не пойму, снимков этого места не осталось. Зрелище было завораживающим\mdash большая разрушенная бетонная плотина, сплошь заросшая лесом и густым подлеском, а посередине ревущий слив примерно на метр. Естественно, сходу проходить не стали\mdash звук клокочущей воды как\sdash то не способствовал. Причалили для осмотра к левому берегу. Берега крутые, вылезти нормально можно только в одном месте. 

Я осторожно оказался на берегу и стал пробираться сквозь заросли оценить ситуацию. Олег держался за~прибрежную траву, чтобы байдарку не сносило ниже. Экипаж С.Ю. тоже причалил позади нас и стал потихоньку вылезать. Аккуратно взобрались на плотину и прошли ниже по течению оценить характер слива. С таким препятствием я столкнулся впервые. Вышел, подошел к краю обрыва и, цепляясь за ломкую ольху, которая не преминула сломаться, спустился к воде, к самому ревущему водопаду. Ничего непонятно. С.Ю. попытался пройти хотя бы метр по воде на слив, чтобы понять\mdash есть там брёвна, препятствия или что-то подобное\mdash иными словами\mdash можно ли пройти слив на байдарке или надо обноситься? Перспектива обноса не~радовала\mdash подъём на плотину весьма крутой, сплошные заросли$\ldots$ короче та ещё прелесть. По итогам осмотра ничего толком не поняли и решили всё же попробовать пройти слив ближе к левому берегу\mdash рискнуть. Я продолжал стоять на~небольшом островке перед сливом. Экипаж С.Ю. занял места в~байдарке, и они пошли первыми\mdash я страховал на~подходе к~сливу и наблюдал за происходящим.

Байдарка С.Ю. немного криво вошла в слив$\ldots$ секунды казались вечностью$\ldots$ и вот$\ldots$\\
Падение!\\
{\hspace*{19mm} \large Волны!\\
\hspace*{40mm} \Large Рёв воды!\\
\hspace*{70mm} \LARGE Белая пена$\ldots$}\\
\noindent Уф\sdash ф\sdash ф! Прошли! Нормально! Лихо заложили разворот и~встали у берега против течения, ожидая нас. Прокричали нам сквозь шум слива и показали жестами, что всё в норме! Слив зловеще клокочет! Прямо как настоящий Ниагарский водопад! Но довольно, теперь наша очередь! Я вернулся к~своему экипажу, занял место в корме, сел повыше, чтобы лучше видеть ситуацию впереди, и мы предельно аккуратно, у самого левого берега, слегка цепляя камни днищем, стали ювелирно подходить к ревущему сливу$\ldots$
 
Секунды тянулись безумно медленно! Казалось, будто время остановилось в сознании! Олегу крикнул не подруливать веслом, а сам, табаня, выравнивал курс, придавая нам небольшую отрицательную скорость$\ldots$ мы~медленно, но верно приближались к зловещей пропасти$\ldots$ время как бы мгновенно разморозилось и, ускорившись, полетело стремглав в неизвестность$\ldots$ впереди был слив!\\
{\large У-у-у-у-ух!!!\\
\hspace*{29mm} \Large Пш-ш-ш-ш-ш!!!!!!\\
\hspace*{80mm} \LARGE Плюх!!!!!!!!!}\\
\noindent Нос байдарки нырнул в пену$\ldots$ и сразу же вынырнул\mdash вода как бы не поняла, что могла перелиться через борт, и~отступила. Байдарка будто бы изогнулась легкой дугой, и~вот уже корма, спустя доли секунды после носа, прошла сквозь белую, кричащую, словно в ужасе, пену и перепад высот. В действительности прошли какие\sdash то мгновения, доли секунды, но в моём сознании подход к сливу и падение были вечностью$\ldots$ Мы прошли! Невероятно! А казалось страшнее, чем было на самом деле! 
 
Экипаж С.Ю. подождал нас, и мы вместе пошли дальше кильватерной колонной. Жена выразила восторг от~прохождения слива! Ещё бы! Я и сам был на мощнейшем подъёме\mdash такое препятствие! Сначала, конечно, всем было боязно, и неизвестность пугала, но сейчас мы счастливо обсуждали, как классно нырнули в эту пену. Осознание преодоления этой плотины придало нам сил и уверенности. Небольшой экстрим\mdash те самые ощущения, ради которых мы и оказались здесь, на реке. 

Судя по старому описанию нашего маршрута, при высокой воде он относится ко второй категории сложности. Так что мы можем, почти не моргнув глазом, рассказывать теперь всем желающим за рюмкой чая, что прошли по~маршруту второй категории сложности\mdash впереди нас ждала ещё одна плотина$\ldots$ но это потом, не сегодня$\ldots$ сегодня нам бы подобраться вплотную к селу Лидь. Олег~и~Паша впервые проходили такое препятствие и держались молодцами. Мы уже с женой проходили перекаты на Зилиме, потом знаменитый сливчик на Киржаче$\ldots$ а я ещё и плотину, которая дальше тут на Лиди после Заборья. Опыт С.Ю. тоже чего\sdash нибудь, да значил. Так что Тресновская ГЭС стала <<боевым крещением>> для наших двух матросов.

Плыли дальше и думали\mdash отчего мы все так растерялись и забыли про фото и видеосъемку? Невероятно! И я\sdash то тоже хорош! Мог бы додуматься достать фотоаппарат! Такое красивое интересное место и так бездарно прохлопать возможность запечатлеть его! Придётся возвращаться против течения и повторять на бис\mdash шутили мы. На самом деле, маршрут по Лиди достоин того, чтобы проходить его вновь и вновь. Места удивительно дикие, красивые$\ldots$ хоть и~затопленные в этом году\mdash все песчаные пляжи скрыты под водой, подходы сплошь поросли густой травой, о стоянках байдарочников и~речи нет\mdash за год пройдет максимум 1\thinspace\nobreakdash--\thinspace 2~группы, судя по~стоянкам, а точнее по их отсутствию вдали от населённых пунктов$\ldots$

Мда\sdash а\sdash а, с фотосъемкой разрушенной плотины явно сплоховали. Но, с другой стороны, не станешь же напрасно рисковать и доставать фототехнику, когда вокруг ситуация каждое мгновение грозит бедою? Искупать зеркальный фотоаппарат было бы слишком, хватит и утопленного GPS. Поэтому\sdash то и ставят страховки ниже по течению, оператора со штативом и камерой у слива, имеют рации$\ldots$ так проходят препятствия, пороги и сливы настоящие профессионалы, всякие клубные сплавщики с супер-пупер снаряжением, касками, спасжилетами и всё вот это вот, ну вы понимаете$\ldots$ а кто мы? Мы так, любители этого дела, забредшие путники, потерянные души в этом забытом всеми крае, приехавшие сюда вовсе не за фотокарточками (хотя, кого мы обманываем, за ними тоже), не за этим каким\sdash то показным тщеславием и хвастовством от прохождения маршрута$\ldots$ мы приехали сюда, нет, даже не приехали, а сбежали (!) сюда для отчуждения\mdash для полного отчуждения\mdash в попытках уйти от реальности и отстранится от привычного$\ldots$ и это удаётся нам сполна\mdash красавица Лидь не жалея окунает нас в~эту~атмосферу$\ldots$

В этот день прошли урочище Тресно, по имени которого и была названа та плотина выше по течению. Когда\sdash то это урочище, видимо, было деревней, а сейчас\mdash один полузаброшенный домик с хорошим выходом к воде. Мы, повинуясь сильному течению, проплываем мимо\mdash причаливать нет желания$\ldots$ да и не ясно\mdash жилой ли дом или просто это временное пристанище для рабочих, лесорубов или лесников.

Мы рвёмся вперёд и вскоре нашему взору открывается преграда$\ldots$ деревянный мост, положенный на бетонные трубы большого диаметра. Течение ускоряется и грозит бедою! Как проходить этот мост\mdash не ясно. Видны завалы в трубах. Мы аккуратно зачаливаемся у правого берега, в зарослях какой-то прибрежной травы. Вылезаем все без исключения на берег и идём в разведку. Мост широкий, деревянный сверху, в приличном состоянии, способном выдержать прохождение лесовозов\mdash для чего он, скорее всего, тут и сделан\mdash видны следы протекторов шин грузовых автомобилей и тракторов. Обносить байдарки по~берегу дико не охота, ведь придётся разгружаться. В двух бетонных трубах моста нет завалов из плавника и можно попытаться пройти там, но вход перегораживает прибитое течением бревно. 

Пока Паша ведёт разведку у трубы близ правого берега, мы с Олегом спускаемся к трубе у левого берега. Если подвинуть прибитое течением бревно\mdash можно пройти! Но бревно прижимает сильным течением\mdash сделать это будет очень нелегко$\ldots$ да и неизвестно, вдруг там что\sdash то на дне лежит в этой трубе? Чтобы развеять сомнения, я~спускаюсь с моста вниз, ниже по течению, к трубе у левого берега и аккуратно, против приличного течения, захожу в~неё, держась сверху за деревянные перекрытия. Прохожу практически до середины пролёта\mdash всё нормально\mdash камней нет, острых деревяшек тоже. Решаю, что точно надо идти тут, только избавившись от того бревна на входе.

Паша возвращается от соседней трубы и настаивает идти там, а С.Ю. колеблется. Олег ждёт, пока мы с С.Ю. как Капитаны байдарок, определимся. Я тоже иду оценить вторую трубу и понимаю, что там идти вообще не вариант\mdash в первой глубже, по моему мнению. Остаётся единственно верное решение\mdash оттолкнуть то бревно, как я и хотел, и~проходить в трубе у левого берега. Так и поступаем. Находим с Олегом оглоблю прямо тут, на настиле моста, и,~спустя несколько минут, используя её как огромный рычаг, с трудом вдвоём отодвигаем бревно от прохода в сторону. Теперь можно идти! 

Первыми снова пускаем экипаж С.Ю. Они у нас в~роли разведчиков. Заходят нормально в створ трубы только со~второй попытки\mdash сначала их относит к~опоре моста и приходится резко уходить назад обратными гребками, выравниваться, а затем пробовать снова. И~вот\mdash новый заход$\ldots$ выравнивание$\ldots$ уф!!! Прошли нормально, слегка пригнув головы в трубе. Теперь наша очередь\mdash мой экипаж занимает места в байдарке. Мы тоже сперва заходим против течения подальше, а~потом, выравнивая байдарку обратными гребками, предельно аккуратно заходим в эту трубу. На~выходе нас уже ждут С.Ю. и Паша$\ldots$ и~опять отчего\sdash то не засняли на фото ни мост, ни трубы, ни~прохождение$\ldots$ что~такое? Эмоциональный перегруз, не~иначе. Всё это останется только в нашей памяти. Жаль,~очень~жаль! Но мы идём~дальше.

В этот день больше никаких препятствий не~встретилось, и мы размеренно плыли в поисках стоянки. Крайняя точка на сегодня\mdash это село Лидь, поэтому надо было найти стоянку до него. Но найти хорошее место как раз не выходило\mdash все берега чересчур лесистые и~заболоченные или травы по пояс$\ldots$ наконец, когда уже из\sdash за очередного поворота реки показался дом села Лидь (а~другого ничего тут по карте и нет), мы заприметили выход к воде на правом берегу. Естественно, причалили и пошли на разведку. Место~было так себе\mdash и близко к деревне, и~следы старой стоянки рыбаков с мусором$\ldots$ но, при этом и поляна большая по~соседству и сход к воде более менее приличный. Да~и~хотелось уже встать лагерем, потому что если идти дальше, то~километров 5 точно\mdash надо пройти село тогда и вставать за~ним$\ldots$

В итоге решили оставаться. Перетаскали вещи, вытащили байдарки, разбили лагерь. Костёр развели около поваленного дерева, чтобы использовать его как лавочку. Комары не дремали и жадно принялись за кровопускания свежей партии человечинки$\ldots$ репелленты закончились. Стало ясно\mdash или завтра мы купим в этом селе репелленты, или нам ещё двое суток терпеть эти отнюдь не повышающие боевой дух укусы кровососущих\mdash до магазина в Заборье.

Над быстро вытоптанной поляной возник наш тент\mdash продукты и снаряжение, как всегда, сложили под ним. Между часто росшими соснами на окраине поляны растянули веревку и развесили сушить промокшие вещи. Олег с~Пашей быстро поставили палатки и удалились на берег реки рыбачить. Мы~с~женой и~С.Ю. потихоньку начали кашеварить и сделали сперва чаёк. Этот день принёс немало экстрима\mdash одна плотина чего только стоила, а разведка и~прохождение моста заняли не менее часа наверно. 

Комары снова просто одолевают. Репелленты закончились окончательно и бесповоротно$\ldots$ Я вспоминаю, что у меня есть в аптечке вьетнамская <<звёздочка>>$\ldots$ она же пахучая, ведь так? А что, если этот адский запах отпугнёт насекомых? Надо проверить\mdash быстро к палатке, лезу в герму. Вот и мой подсумок с аптечкой\mdash нахожу эту крохотную баночку, мучительно открываю её ногтями и~немедленно смазываю свои кисти рук и лоб, а также шею$\ldots$ и тут понимаю, что если шее и рукам\mdash нормально такие издевательства, то напрочь искусанный лоб был к этому явно не готов\mdash ощущения те ещё! Просто адски бодрящая свежесть! Но это не главное. Как поведут себя комары? Наблюдаем$\ldots$ держатся на почтительном расстоянии в~несколько сантиметров и садиться не хотят явно!!! Ура! Кратковременная победа! Все желающие немедленно пользуются возможностью хоть немного облегчить участь своих открытых участков тела$\ldots$

Наши рыбаки потихоньку приносили с берега добычу\mdash некрупных рыбёшек, которых мы решили назавтра зажарить на чугунной сковородке, запасливо взятой с собой. У нашей стоянки было мелковато\mdash дно просматривалось, из воды торчали несколько камней, но удача была на стороне наших рыбаков\mdash то и дело они срывались на звук колокольчиков, вытаскивали рыбёшек. Провиант и горючее планомерно уничтожались у костра, и комары как\sdash то снова отходили от этого на второй план$\ldots$ Олег, как и~все, притомился за~сегодняшний день, поэтому гитару даже не расчехляли\mdash они с Пашей не приминули воспользоваться возможностью оттянуться\mdash порыбачить. 

Сегодня прошли около 16 километров, как я примерно прикинул по бумажной карте\mdash немного, казалось бы, зато впечатлений\mdash хоть отбавляй, одна плотина ГЭС чего стоит. Завтра нас ждало прохождение села Лидь и бросок до~окрестностей Заборья, а пока\mdash сплавное братство гуляло, щедро разливая из~канистры. Походный быт к~этому дню был окончательно налажен, и~всё было как надо. Спать все завалились не очень поздно в надежде, что хоть завтра комары умерят свои таёжные аппетиты.

\begin{center}
	\psvectorian[scale=0.4]{88} % Красивый вензелёк :)
\end{center}
