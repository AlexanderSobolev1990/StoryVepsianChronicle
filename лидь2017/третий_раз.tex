\chapter{Третий раз под Чагодой} 
\corner{64}

С утра все проснулись в отличнейшем расположении духа\mdash погода способствовала. Иссиня синее высокое северное небо без единого облачка и яркое солнце! Вот они, контрасты природы этого края! Ещё вчера нам казалось, что мерзопакостнее погоды быть не может, а сегодня нам светит яркое солнышко и тёплым ветерком тянет с~реки. Становится даже жарко. Завтрак готовлю, скинув штормовку и тельняшку\mdash солнце печёт что надо! Нужно воспользоваться моментом и хоть немного подзагореть, а то как будто и на реке не был, честное слово. И тут-то понимаю\mdash руки, оголенные закатанными рукавами штормовки, уже прилично загорели до состояния коричневатости, а остальное тело нет. Что ж, готовлю завтрак с голым торсом\mdash надо хоть немного напитаться витамином D.

Сегодня цель наша ясна\mdash стоянка, на которой мы хотим остановиться, хорошо известна мне по двум прошлым сплавам. Так что будем стоять там уже в третий раз. Километраж до неё тоже прикидываю за завтраком\mdash около десяти километров. Это примерно два с половиной часа хода. Сознание этого расхолаживает\mdash всё делаем медленно и не торопясь. Размеренно готовим завтрак\mdash пока сварим кашу, пока чай подойдёт$\ldots$ я успеваю ещё записать несколько видеодневниковых записей. Потом неспешно сворачиваем лагерь, заливаем костёр и собираем высушенные вещи с веревок. Не всё, кстати, высыхает полностью\mdash придётся досушивать вечером.

Встаем на воду в этот день попозже. Нормально. 10~километров можно пройти и со скоростью течения. Поэтому в этот день совершенно не налегаем на вёсла и плывём более чем расслабленно.  Характер берегов, по мере приближения к устью Лиди, как и положено, становится более пологим, крутые откосы пропадают. Вскоре впереди становится виден скат большой крыши\mdash это ориентир\mdash база отдыха, стоящая в месте впадения Лиди в Чагоду. Так же, как в двух предыдущих сплавах, хотим причалить на мысу и размять ноги. Плывём, переговариваемся и готовимся зачалиться, но$\ldots$ с базы нас неодобрительно прогоняют, мол вставайте не стоянку дальше, а тут база! Ну и ну! Не ожидал такого отношения, тем более, что в прошлые года мы нормально причаливали и даже общались со смотрителем. Но что делать\mdash придётся разминать ноги в другом месте. 

Переката в устье Лиди не было\mdash слишком высок уровень воды. Дальше по Чагоде тоже от перекатов осталось лишь подобие\mdash ускорение течения, да и всё. Не то, что в 2015-ом. Мы без каких-либо опасений прошли эти участки и не спеша приблизились к первому островку на Чагоде. Встать на нём лагерем снова не представлялось возможным, как и в прошлые года. Не знаю почему это упорно советуют в отчетах с этого маршрута$\ldots$ Острова на Чагоде и Чагодоще сплошь заросшие и не слишком пригодные для стоянок, даже для такой маленькой эскадры, как у нас.

Ширина реки стала больше, течение замедлилось. По Чагоде я шёл в третий раз$\ldots$ как хочется же пройти и в этот раз дальше, ниже по течению, чтобы не думать об антистапеле вообще$\ldots$ чтобы до Весьегонска или Рыбинска$\ldots$ чтобы отпуск был на 4 недели, чтобы вообще ни о чём не думать, а предаться созерцанию природы, фотоохоте и астрофотографии$\ldots$ всласть дымить самодельной трубкой и грязно ругаться$\ldots$ Но не в этот раз, снова не в этот$\ldots$ в груди отчего\sdash то появилось щемящее чувство конца похода.

\begin{center}
	\psvectorian[scale=0.4]{88} % Красивый вензелёк :)
\end{center}
