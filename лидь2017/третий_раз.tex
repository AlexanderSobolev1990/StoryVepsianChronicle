\chapter{Третий раз под Чагодой} 
\corner{64}

С утра все проснулись в отличнейшем расположении духа\ndash погода способствовала. Иссиня синее высокое северное небо без единого облачка и яркое солнце! Вот они, контрасты природы этого края! Ещё вчера нам казалось, что мерзопакостнее погоды быть не может, а сегодня нам светит яркое солнышко и тёплым ветерком тянет с~реки. Становится даже жарко. Завтрак готовлю, скинув штормовку и тельняшку – солнце печёт что надо! Нужно воспользоваться моментом и хоть немного подзагореть, а то как будто и на реке не был, честное слово. И тут-то понимаю – руки, оголенные закатанными рукавами штормовки, уже прилично загорели до состояния коричневатости, а остальное тело нет. Что ж, готовлю завтрак с голым торсом – надо хоть немного напитаться витамином D.

Сегодня цель наша ясна – стоянка, на которой мы хотим остановиться, хорошо известна мне по двум прошлым сплавам. Так что будем стоять там уже в третий раз. Километраж до неё тоже прикидываю за завтраком – около десяти километров. Это примерно два с половиной часа хода. Сознание этого расхолаживает – всё делаем медленно и не торопясь. Размеренно готовим завтрак – пока сварим кашу, пока чай подойдёт… я успеваю ещё записать несколько видеодневниковых записей. Потом неспешно сворачиваем лагерь, заливаем костёр и собираем высушенные вещи с веревок. Не всё, кстати, высыхает полностью – придётся досушивать вечером.

Встаем на воду в этот день попозже. Нормально. 10 километров можно пройти и со скоростью течения. Поэтому в этот день совершенно не налегаем на вёсла и плывём более чем расслабленно.  Характер берегов, по мере приближения к устью Лиди, как и положено, становится более пологим, крутые откосы пропадают. Вскоре впереди становится виден скат большой крыши – это ориентир – база отдыха, стоящая в месте впадения Лиди в Чагоду. Так же, как в двух предыдущих сплавах, хотим причалить на мысу и размять ноги. Плывём, переговариваемся и готовимся зачалиться, но… с базы нас неодобрительно прогоняют, мол вставайте не стоянку дальше, а тут база! Не ожидал такого отношения, тем более, что в прошлые года мы нормально причаливали и даже общались со смотрителем. Но что делать – придётся разминать ноги в другом месте. 
Переката в устье Лиди не было – слишком высок уровень воды. Дальше по Чагоде тоже от перекатов осталось лишь подобие – ускорение течения, да и всё. Не то, что в 2015-ом. Мы без каких-либо опасений прошли порожистые участки и не спеша приблизились к первому островку на Чагоде. Встать на нём лагерем снова не представлялось возможным, как и в прошлые года. Не знаю почему это упорно советуют в отчетах с этого маршрута… Острова на Чагоде и Чагодоще сплошь заросшие и не слишком пригодные для стоянок, даже для такой маленькой эскадры, как у нас.
Ширина реки стала больше, течение замедлилось. По Чагоде я шёл в третий раз… как хочется же пройти и в этот раз дальше, ниже по течению, чтобы не думать об антистапеле вообще... чтобы до Весьегонска или Рыбинска… чтобы отпуск был на 4 недели, чтобы вообще ни о чём не думать, а предаться созерцанию природы, фотоохоте и астрофотографии… всласть дымить самодельной трубкой и грязно ругаться… Но не в этот раз, снова не в этот… в груди отчего-то появилось щемящее чувство конца похода.

\begin{center}
	\psvectorian[scale=0.4]{88} % Красивый вензелёк :)
\end{center}
