\chapter{Третий раз под Чагодой} 
\corner{64}

С утра все проснулись в отличнейшем расположении духа\mdash погода способствовала. Иссиня синее высокое северное небо без единого облачка и яркое солнце! Вот они, контрасты природы этого края! Ещё вчера нам казалось, что мерзопакостнее погоды быть не может, а~сегодня нам светит яркое солнышко и тёплым ветерком тянет с~реки. Становится даже жарко. Завтрак готовлю, скинув штормовку и тельняшку\mdash солнце печёт что надо! Нужно воспользоваться моментом и хоть немного подзагореть, а то как будто и на реке не был, честное слово. И тут\sdash то понимаю\mdash руки, оголенные закатанными рукавами штормовки, уже прилично загорели до состояния коричневатости, а остальное тело нет. Что ж, готовлю завтрак с голым торсом\mdash надо хоть немного напитаться витамином D.

Сегодня цель наша ясна\mdash стоянка, на которой мы хотим остановиться, хорошо известна мне по двум прошлым сплавам. Так что будем стоять там уже в третий раз. Километраж до неё тоже прикидываю за завтраком\mdash около десяти километров. Это примерно два с половиной часа хода. Сознание этого расхолаживает\mdash всё делаем медленно и~не~торопясь. Размеренно готовим завтрак\mdash пока сварим кашу, пока чай подойдёт$\ldots$ я успеваю ещё записать несколько видеодневниковых записей. Потом неспешно сворачиваем лагерь, заливаем костёр и собираем высушенные вещи с веревок. Не всё, кстати, высыхает полностью\mdash придётся досушивать вечером.

Встаем на воду в этот день попозже. Нормально. 10~километров можно пройти и со скоростью течения. Поэтому в этот день совершенно не налегаем на вёсла и плывём более чем расслабленно. Характер берегов, по мере приближения к устью Лиди, как и положено, становится более пологим, крутые откосы пропадают. Вскоре впереди становится виден скат большой крыши\mdash это ориентир\mdash база отдыха, стоящая в месте впадения Лиди в Чагоду. Так же, как в двух предыдущих сплавах, хотим причалить на мысу и размять ноги. Плывём, переговариваемся и готовимся зачалиться, но$\ldots$ с базы нас неодобрительно прогоняют, мол вставайте не стоянку дальше, а тут база! Ну и ну! Не~ожидал такого отношения, тем более, что в прошлые года мы нормально причаливали и даже общались со смотрителем. Но что делать\mdash придётся разминать ноги в другом месте. 

Переката в устье Лиди не было\mdash слишком высок уровень воды. Дальше по Чагоде тоже от перекатов осталось лишь подобие\mdash ускорение течения, да и всё. Мы без каких-либо опасений прошли эти участки и не спеша приблизились к первому островку на Чагоде. Встать на нём лагерем снова не представлялось возможным, как и в прошлые года. Не знаю почему это упорно советуют в отчетах с этого маршрута$\ldots$ Острова на Чагоде и Чагодоще сплошь заросшие и не слишком пригодные для стоянок, даже для такой маленькой эскадры, как у нас.

Ширина реки стала больше, течение замедлилось. По~Чагоде я шёл уже в третий раз\mdash ощутил себя почти что дома. И хотя рельеф местности был слабо узнаваем из\sdash за высокого уровня воды, всё же отдельные участки реки ярко вставали перед глазами из воспоминаний по прошлому опыту. Мне уже в который раз подумалось\mdash как же хочется пройти дальше\mdash ниже по течению, чтобы не думать об антистапеле вообще! Пройти ниже по Мологе, добраться до Волги, чтобы отпуск был на 4 недели, чтобы вообще ни о чём не думать, чтобы предаться созерцанию природы, фотоохоте и астрофотографии$\ldots$ всласть дымить самодельной трубочкой и купаться в освежающей воде$\ldots$ но~не в этот раз, снова не в этот. Должна же быть у человека какая\sdash то немного нереальная мечта, иначе как жить без мечты? Пусть такой поход\sdash отрыв длиной в месяц будет для меня такой мечтой. Но я верю, что рано или поздно, мне удастся её осуществить. Пусть через 5\thinspace\nobreakdash--\thinspace 10 лет, но осуществить. Главное, чтобы огонь странствий не погас в душе, а глаза загорались диким пламенем от одной только мысли о таком мероприятии. Эх$\ldots$ в груди отчего\sdash то появилось такое знакомое щемящее чувство конца похода. 

Вскоре впереди показался будто бы хорошо знакомый участок реки с высокими соснами на левом берегу$\ldots$ так и есть\mdash это та самая стоянка! Именно тут мы зализывали раны после шивер в 2015 году и хорошенько оттянулись чуть ниже по течению в 2016\sdash м. Мы причалили сразу за левым поворотом реки, у хорошего пляжика. Сходили на мою стоянку 2015 года\mdash я показывал жене где мы стояли, как был разбит лагерь и всё в таком духе. Моё костровище и костровые палки всё ещё были на месте\mdash поразительно! 

Экипаж С.Ю. и Паши, тем временем, решил пройти ниже по реке и разведать прошлогоднее место. К берегу подъехали местные на машинах\mdash мы дружелюбно поболтали, выяснив, что они хотят порыбачить. Потом наш экипаж занял свои места в байдарке и пошли догонять наш передовой отряд, который к этому времени уже высадился у стоянки, разгрузил байдарку, и теперь втаскивал её на крутой подъём. 

Времени было около 13\thinspace\nobreakdash--\thinspace 14 часов. Мы, как и хотели, устроили себе полуднёвку. Привычно и быстро разбили лагерь\mdash поставили палатки, добыли дров, развели костёр. Надо было поставить тент. Я решил сделать это по новой технологии\mdash натягивать тент не к земле оттяжками, закрепляя их колышками, а делать оттяжки к деревьям, растущим вокруг поляны. Я с самого приобретения этого тента хотел попробовать так сделать, но всё не подворачивался случай. И вот теперь всё сошлось. Оттяжки тента, и без того не короткие, были изначально мною сделаны так, чтобы их можно было удлинить в 2 раза\mdash ведь до ближайших деревьев может быть далеко. Далее\mdash самое интересное. Способ крепежа\mdash никаких узлов! Я заказал себе шикарные алюминиевые карабины, позволяющие крепить веревку хитрым способом без узлов. Утверждалось, что такая конструкция выдерживает до 150 кг. Проверим! Немного поколдовав над расположением тента относительно деревьев, мы с ребятами стали оттягивать его углы к ближайшим соснам. Оборачивая оттяжки вокруг стволов деревьев, туго натянули их при помощи карабинов\mdash вышло на отлично!

Когда покончили с тентом, С.Ю. уже поставил кипятиться воду на чай, а ребята приготовились к рыбалке. Олег колдовал с приманками, пробуя разные изыски в попытках выловить под конец похода что\sdash то достойное. Они с Пашей облюбовали бережок под обрывом прямо у нашей стоянки и вскоре уже закинули там удочки. Пока они ходили туда\sdash сюда по лагерю к удочкам и обратно, Олег нашел куст чёрной смородины. Вот так да! Это сказка какая\sdash то просто! Нарвав немного веток смородины, он пришел к костру и добавил все это в кипяток на манер заварки:

\diagdash Будет обалденно!\mdash заявил Олег.

\diagdash Но там же листва прямо с веточками,\mdash парировал было я,\mdash Давай только листья добавим?

\diagdash Нормально!

Мы проварили смородиновые ветки с листьями в~котелке и стали пробовать. Что сказать$\ldots$ вышло кисло, даже очень. Пить это было совершенно невозможно. Точнее, в~лекарственных целях может и можно по чайной ложечке, но точно не вместо чая. Разочаровавшись в таком способе, вылили варево и решили просто сделать чай с добавлением одних только листьев смородины, без веточек. Спустя ещё минут 20 вкус нового напитка удовлетворил, надо полагать, всех\mdash приятная терпкость смородины добавила немного кислинки и травянистости в душистый чай на и без того тёмной торфяной воде. 

Меня охватила невозможная ленность. Видимо на~подсознательном уровне я уже ощущал скорое окончание нашего мероприятия, нашего отчуждения. И оттого\sdash то и~хотелось как\sdash то растянуть что ли остаток похода.

\begin{center}
	\psvectorian[scale=0.4]{88} % Красивый вензелёк :)
\end{center}
