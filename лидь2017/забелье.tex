%\chapter{Днёвка близ Забелья} 
\chapter{Борщ} 
\corner{64}

Утро выдалось почти безоблачным\mdash разгорался хороший денёк! Утреннюю кашу готовили как\sdash то лениво\sdash размеренно, однако мы с этим справились, а~после окончательно утвердили решение\mdash остаёмся тут на днёвку. В честь этого решения гордо поднимается на~флагштоке из~берёзы флаг Советского Союза над нашей стоянкой\mdash непременный атрибут нашего днёвочного быта$\ldots$ 

Олег и Паша, расстроенные вчера отсутствием под берегом <<рыбного места>>, решили сплавать на байдарке выше по течению, к перекату. С.Ю., скрепя сердце, доверил им свою ласточку и они, набрав удочек и разных снастей, стали готовиться к~отплытию. Остальная часть команды с~интересом наблюдала за ними и~вела видеосъемку процесса спуска байдарки на~воду$\ldots$
 
Мне было интересно как они пройдут тут в <<бутылочное горлышко>> против течения, притом что удочки, торчащие из байдарки в разные стороны, мешают грести как следует\mdash с широким размахом. Как все и предполагали, сначала у них ничего не получилось и течение сносило назад. Потом Паша сел поудобнее, поправил удочки и подналёг на весло\mdash дело стало спорится. Так, потихоньку выправляя курс и упираясь веслами, наши рыболовы успешно преодолели самую узкую часть русла и скрылись за поворотом.

Денёк разгорался жаркий. Частые белые кучевые облачка ползли по северному небу, дождя не предвиделось. Урвали днёвку, не прогадали, подумал я. Может, место и не самое удачное в плане рыбалки и вообще, но это лучше, чем вовсе идти без днёвок.

Остальная часть команды занялась лесозаготовкой. Мы~с~С.Ю. сходили дальше по заросшей дороге и нарубили там сухостоя. Потом притащили дрова в лагерь и свалили у костра. Пилить будем потом\mdash когда наши рыбаки вернутся, а пока пьем чаёк, болтаем о~том, о~сём, отдыхаем. С.Ю. просушивает и ремонтирует свой надувной матрас, а мы с женой сушим постиранные вещи\mdash полотенца и штаны. Также я решаю, что неплохо бы просушить бахилы химзащиты. Сказано\mdash сделано. Промываю бахилы в реке от песка и надеваю их сушиться на колья, вбитые тут же неподалёку. 

С.Ю. внезапно цапнул клещ в ногу, и он немедленно занялся его извлечением. <<Везёт>> нам что\sdash то в этом сплаве и с погодой и с насекомыми$\ldots$ но что поделать, год на год не приходится, однако. Место укуса на ноге покраснело и припухло. Прижигаем спиртом. После проводим обильную санобработку штормовок и прочей верхней одежды средством от клещей$\ldots$
 
Потом я снимаю утреннюю видеозарисовку и тут, прямо во время съемки, замечаю, что по рукаву моей штормовки ползёт клещ! Абсолютно машинально и инстинктивно смахиваю его свободной от камеры рукой в сторону! И сразу же осознаю, какую я глупость сделал\mdash ведь теперь он тут где\sdash то в траве!!! И всё это около нашей с женой палатки. Бр\sdash р\sdash р! Беру флакон средства от клещей и поливаю обильно вход нашей палатки, особенно нижнюю часть. На этом первый баллончик заканчивается\mdash у нас есть ещё три. С.Ю. аккуратно пробивает закончившийся баллон топором, и мы сжигаем его в костре. 

Настроение\mdash какое\sdash то упадническое. Сходили, понимаешь, в лес за дровами$\ldots$ притащили двух клещей! С.Ю., похоже, не сильно расстроился от укуса, по крайней мере, не подал виду и удалился на берег мыть котелки после завтрака$\ldots$ А мы с женой у костра подводим итоги наших четырёх дней похода\mdash сегодня пятый. Кипятим воды в котелке, приходит С.Ю. и мы чаёвничаем.

Потом ещё раз предпринимаем вылазку в лес\mdash я хочу отснять панораму залитого лучами солнца редкого леса, сплошь устланного черничником и брусничником. Идём аккуратно, регулярно проверяясь на клещей\mdash вроде бы всё чисто. Сделав исключительной красоты кадры, возвращаемся назад, в лагерь, снимая параллельно прокопанную кем\sdash то канаву для осушения верхового болотца, заполненного темной торфяной водой.

С.Ю. находит неподалёку нестрелянный охотничий патрон двенадцатого калибра. На дне гильзы выбиты цифры 12 и 73. Он что, 73 года?! Невероятно! Выглядит вполне себе целым и неповреждённым$\ldots$ проверять, правда, в костре его боеготовность никто не стремится, откладываем в сторону сию находку. 

Тут мне приходит в голову\mdash как жаль, что нет с~собой хотя бы радиоприёмника\mdash сейчас бы сводки погоды послушали. Сделали бы из проволоки антенну, закинули бы на ближайшую берёзку и приняли бы позывные <<Маяк>>\sdash а$\ldots$ но, во\sdash первых, приёмника у нас нет, а во\sdash вторых, на УКВ тут в глуши ловить нечего, а СВ вещание давно почило в нашей стране, как, впрочем, и КВ$\ldots$

Возвращаются наши рыбаки с переката. Добыча их невелика\mdash крупная рыба не ловится. Они решают сделать приманку из пшёнки или перловки и пойти попытать счастья уже с берега, чуть ниже по течению. С байдарки не особо вышло наловить$\ldots$

Днёвка проходит в ленности, как и положено, если нет никаких задач для <<дивизиона борьбы за живучесть>>\mdash ремонт байдарки, поиск дров и т.д. Я подымливаю трубочкой и весь день глушу ароматнейший чаёк с травами. Вернувшиеся рыбаки выражают желание немного согреться канистрой, и мы немедленно нарезаем сало и реализуем сие полноправное веление души.

Поскольку сегодня не сплавной день, то ужин решаем приготовить раньше\mdash начинаем где\sdash то в половину четвёртого. Напиливаем с Пашей и Олегом дров побольше и разводим хороший костёр. Итак, днёвка же! Нас ждут гастрономические выкрутасы! У нас с женой припасена свекла, морковь, лук, картошка, капуста. А также томатная паста, чеснок и специи. Ну\sdash у\sdash у? Все понимают??? Мы будем варить настоящий походный бо\sdash о\sdash орщ! Ну, не на свежем мясе, а~на тушёнке, но~всё~же! На второе изволим готовить кус\sdash кус с поджаркой из тушёнки и лука. Гвоздём программы будет салат с тунцом, свежими огурцами, варёными яйцами, кукурузой и жареным лучком! Кто после этого скажет, что в походах бедные несчастные туристы едят одну тушенку и макароны? Ммм? Так\sdash то!

Кус\sdash кус и салат пока откладываем, надо заняться борщом\mdash чистим все составляющие, режем на байдарочном сиденье, а потом ещё и обжариваем на сковородке в масле. Всё как положено, никаких тут <<левых>> технологий! Вскоре зажарка и бульон готовы, соединяем их вместе, продолжаем варить, добавляем нашинкованную капусту. Народ восхищается нашими кулинарными ухищрениями и помогает по мере сил\mdash Олег начинает готовить зажарку для второго блюда и параллельно варим кус\sdash кус. Эта крупа по виду вроде пшёнки, но имеет немного другой вкус. Паша возвращается с берега с небольшим уловом\mdash маленькая рыбёшка бьётся в руках. Поступающие шутливые предложения усилить готовящийся борщ рыбой отвергаем. 

Варим яйца для салата, я достаю консервированного тунца, а также кукурузу и всё это, наряду с уже готовым лучком и огурцами, <<ассемблируется>> в салатнике в обалденное лакомство. Вскоре, все три блюда готовы и чаёк тут как тут! Садимся трапезничать. Под кружки, полные яблочной, произносятся речи и мы, не торопясь, вкушаем приготовленные яства, не забывая потчевать себя солидными кусками сала с хлебом. 

Что это? Впадение в какое\sdash то излишество, варварство или первобытность? Пробуждение чего\sdash то животного, давно забытого? Никто не скажет. Всё проще\mdash вкусно поесть, да сладко поспать на свежем воздухе. Не для этого ли есть отпуск? Наедаемся от пуза и валяемся у костра на туристических ковриках, а под берегом течёт наша Лидь. Разговоры и пересуды увлекают нас, и мы сидим под звёздным северным небом, которое начинает затягиваться серыми облаками, и болтаем о разном; о сплавах и случаях из жизни, о рыбалке, о каких\sdash то моментах, объединяющих наши общие интересы$\ldots$ словом, мы отдыхаем душой и телом, наслаждаемся моментом, восстанавливаемся от города\mdash северная природа смывает городскую шелуху, открывает глаза на красоты, окружающие нас, и принимает такими, какие мы есть\mdash туристы\sdash водники, забравшиеся в этот безразмерный скит отчуждения от городской суеты$\ldots$ 

\begin{center}
	\psvectorian[scale=0.4]{88} % Красивый вензелёк :)
\end{center}
