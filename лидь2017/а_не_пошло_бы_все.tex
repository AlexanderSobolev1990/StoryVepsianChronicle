\chapter{Отчаяние} 
%\corner{64}
\vepsianrose

\setlength{\epigraphwidth}{0.7\textwidth}

\epigraph{%	
	Считай по-нашему, мы выпили не много,\mdash\\
	Не вру, ей-бога,\mdash скажи, Серега!\\
	И если б водку гнать не из опилок,\\
	То что б нам было с пяти бутылок!}
	{
	\begin{flushright}
		\small{Владимир~Высоцкий,\\Милицейский протокол, 1971}
	\end{flushright}
	}

%Дома. Что\sdash то такое плещется на дне классического гранёного стакана$\ldots$ а хотя$\ldots$ какого классического? Наличие 250 граммов объёма и отдалённо ребристых краёв не делают его <<тем самым стаканом>> из детства, из школьной столовой. Тут и грани скругленные, не цепкие, не острые, и вообще что за$\ldots$ ерунда? Стакан сделать советский уже слабо? Вот тебе на! Брюзжу, брюзжу как всегда. Словно немой упрёк потомкам, потерявшим технологию изготовления гранёного стакана с нормальными гранями, мой экземпляр стоит на столе и преломляет падающий на стол боковой свет от лампы$\ldots$.

\diagdash А не пошло бы всё к чёрту?\mdash говорю я себе в очередной раз,\mdash Что\sdash то всё безумно тошно, безумно утомительно и безумно пошло, противоестественно, невозможно и вообще$\ldots$\mdash с почти такими мыслями я аккурат каждый день вываливаюсь из электрички ровнёхонько в 19:17 и иду на чёртов автобус, ибо добавить себе ещё двухкилометровую прогулку от станции до дома вечером после работы\mdash почему-то выше моих сил.
%<<А не пошло бы всё к чёрту?>>\mdash говорю я себе в очередной раз! Что\sdash то всё безумно тошно, безумно утомительно и безумно пошло; противоестественно, невозможно и вообще$\ldots$\mdash с почти такими мыслями я аккурат каждый день вываливаюсь из электрички ровнёхонько в 19:17 и иду на чёртов автобус, ибо добавить себе ещё двухкилометровую прогулку от станции до дома вечером после работы\mdash почему\sdash то выше моих сил. 
Не потому что не могу пройти эти 2.5 километра, а потому что лень. Впрочем, нет, обманываю\mdash иногда иду пешком\mdash если есть насущная и животрепещущая необходимость заглянуть в магазин$\ldots$ но это редко\mdash маршрутка до дома и вперёд. %Тем более что климат у нас, вообще говоря, не сахар\mdash «то дождь, то снег, то мошкара над нами$\ldots$», хоть мы и не в тайге, а всего в каких\sdash тто 24 километрах от московского кремля, оплота и символа уже новой власти, победившей таких ненавистных молоткастых\sdash тсерпастых и всё вот это вот. 
А дома что? Дома принять горизонтальное положение, растянуть свои кости и усталые мышцы. Это старость? Увы,~нет\mdash было бы простительно! Это всего-навсего только чёртовых 27 лет$\ldots$

О чём это я? Ах да\mdash не пошло бы всё к~чёрту? Встал в 6:15 утра, в 7 c копейками прибежал на~электричку чтобы к 9 быть на работе$\ldots$ к 8 вечера оказался дома, при условии, что улизнул пораньше с~работы. Йеx! А что, у нас так всё подмосковье живёт и хорошо ещё, если так. Тьфу! Электричка вся занюханная, что, впрочем, не смущает никого регулярно поднимать на~неё цены, и давка в ней приличная такая$\ldots$ ну, словом, вы понимаете. Как~в~такой ситуации сохранить здоровье и не только физическое? А~очень просто\mdash почитайте бессмертное <<Москва\thinspace\nobreakdash---\thinspace Петушки>>\cite{МоскваПетушки}, а~после проедьтесь самолично на этом маршруте\mdash сразу поймёте, прочувствуете, пронюхаете, тэк скэзэть. Контингент начинает пить уже на вокзале, совершенно не стесняясь проходящих мимо патрулей, которые, вообще говоря, в~петушинскую электричку стараются и не заходить от греха подальше, видимо.

Словом, каждое утро как селёдочка в бочке, будь любезен, и марш с улыбкой на лице зарабатывать, как можешь, на своё якобы светлое и якобы счастливое капиталистическое будущее. Зато типа <<честно>> и~<<свободно>>, говорят нам. Заработал\mdash вот оно, не заработал\mdash свободен$\ldots$ И никакой пенсии по старости, что\sdash то мне кажется. А что, это теперь не Советский Союз, не социально\sdash ориентированное государство\mdash никто тебе ничего не обещал. А кто говорит, что обещали\mdash гони того поганой метлой.

Но какое оно, это будущее, к чёрту светлое? Тут~на настоящее, порой, не сильно хватает. 
%Порой молишь свои ботинки: «Родимые, ещё сезончик, ну давайте же! Не время рассыпаться!» И это притом, что, в общем-то, ничего такого себе не позволяешь\mdash и это не преувеличение, не взмах руки «зажравшегося москаля», не огульная жалоба на «плохую жизнь», а воистину печальное положение дел. А что, вся страна примерно так и живёт, да чего уж там\mdash хорошо, если так. А ты что, уникальный типа? Ха-ха, как бы не так! 
При этом мысли о выживании от зарплаты до зарплаты все чащё начинают походить на навязчивые идеи околокриминального характера.

\diagdash Езжай на машине, делов-то!\mdash скажут вам прожжённые оптимисты! \mdash Не мучайся в своей зачуханной электричке, раз так не нравится, и вперёд, в <<светлое капиталистическое будущее>> на личном авто$\ldots$ 

Ха! Три раза ХА! Таких советчиков сразу посылайте по известному адресу! Действительно, это занятие для мазохистов\mdash езда по Москве и Подмосковью на машине в~часы пик на работу и с работы\mdash поверьте, я знаю, о чём говорю. А цены на бензин, а~почему\sdash то вдруг ставшие платными парковки$\ldots$ печальное положение дел$\ldots$ на машине, ха-ха! Словом, чем дальше в~лес, тем толще партизаны и всё вот это вот$\ldots$%\mdash хочется просто выть.

\diagdash Право, заврался! Разбухтелся! Пойди и заработай на~нормальную жизнь,\mdash кричат
тут.
%те, кто, как правило, либо сравнительно легко всё получил, либо ослеплён столичной и околостоличной жизнью на контрасте со своим родным условным Гадюкино, где всё такое родное и милое, но совсем нет работы,\mdash ноет он, понимаешь, что денег нет и ботинки дырявые!\mdash слышатся недовольные высказывания! Марш работать! 

Да-да, уже иду. Вот только пойму, куда пойтить-то, куда податься, и пойду, полечу просто. Как фанера над~Пари$\ldots$ над Москвой. 

\diagdash Москва! Как много в этом звуке!$\ldots$. 

Да ничего в этом звуке! Ничего в нём не осталось. Ни\sdash че\sdash го. Город трансформировался во что\sdash то не просто уже не родное, а во что\sdash то местами целиком и полностью гадкое, и разросся совершенно неприлично, зачем\sdash то построив вставные челюсти вроде безвкусных небоскрёбов «Сити» и поглотив, вдобавок, троицкий район. 

Места какие\sdash то из детства\mdash не узнать. Взгляду не за что зацепиться. Да, конечно, многое приведено в~порядок, многое улучшено\mdash этого не отнять, но$\ldots$ нет уже того, что было раньше, в детстве, в юношестве, что оставило какие\sdash то приятные воспоминания, ощущения$\ldots$ все это застроено чем\sdash то новым и не всегда нормально воспринимающимся.
%Так легко же поднять условный уровень жизни в Троицке, а заодно и цены на землю (предварительно скупив её), на недвижимость (предварительно тоже скупив её и понастроив новой), на всё остальное (ну, вы понимаете), просто объявив этот район Москвой. Так-то. Ловкость рук и никакого тут вам понимаешь! Не забалуешь! Кто прав, тот и лев! 
А~ты~давай, сопли свои вытри о рукав и~вперёд\mdash в~занюханную электричку! Не нравится? Плати в~2 раза больше и шуруй за счастьем на электричке\sdash экспрессе. И~чебурек в ней купи с~пивасом у проходящих торговцев. В~пути, теперь, понимаешь, и кормят и п\'{о}ят. Чтоб совсем не~подох, электорат:

\diagdash Ма\sdash а\sdash ароженое, пи\sdash и\sdash иво, семачки!

Кто\sdash то берёт, естественно. Добавить\sdash то надо! Эх,~любители! Ибо профессионалы берут только с собой, втридорога не берут у этих торгашей.

\diagdash Ха\sdash а\sdash алодное ма\sdash а\sdash ароженое, пи\sdash и\sdash иво, семачки!\mdash проходит, удаляясь, дальше по составу.

Бр-р-р! Хорошо, что запретили это непотребство\mdash как вспомнишь, так вздрогнешь$\ldots$

%«Москва$\ldots$» Пустота в этом звуке! В старых усадьбах, особняках\mdash либо запустение, либо какие-то непонятные конторы и ночные клубы со всякими сами знаете кем «с пониженной социальной ответственностью»; исторические здания тоже никому, в общем-то, не нужны, а вся их вот эта вот табличка «охраняется государством»\mdash ложь и фикция\mdash государство непременно извлечёт из этого выгоду, уж поверьте, ибо нечего тут. Выжмет всё как из лимона и бросит. Или продаст. Или выжмет и продаст. Нужное\mdash подчеркнуть. И «привет»! Это уже не «всенародное достояние», которое «охраняется государством», а чёрт знает что\mdash  частная собственность. И мгновенно выясняется, что это всё можно не реставрировать, а снести и построить что-то новое, дорогущее, безумно, дичайше (!) безвкусное, стеклянно-бетонное, и снимать с этого сливки$\ldots$ 

%У власти непойми кто вообще. Не то чекисты, не то легализованные бандиты. Или вообще и то и то в одном флаконе. Мафия, одним словом, хоть всё культурно и на бумаге «законно». И всё выходит на манер «дикого помещика», который извести мужика захотел. Да только тут страшнее\mdash 

C виду <<всё культурненько, всё пристойненько>>, а~держат тебя <<невидимой рукой рынка>> за~яй$\ldots$ за~горло. Ослабят чуть\mdash счастлив! Рученьки целовать готов. 
%Что\sdash то <<ура\sdash патриотичное>> покажут по ящику\mdash и кто\sdash то даже, страшно сказать, подумает, что в великой стране живёт! 
И всё тему норовят сразу перевести, мол, а~страна\sdash то у нас какая! У\sdash у\sdash у! Природа, погляди какая! Ы\sdash ы\sdash ы! Дух захватывает! Народ\sdash то у нас какой! У\sdash у\sdash у! Великий! Люди, правда, отдельно взятые, почему\sdash то часто на поверку\mdash Г редчайшее. И чем ближе хоть к какой\sdash то власти и деньгам, тем$\ldots$ сами понимаете. Так и живём$\ldots$

И среди всего этого, отягощенного непростым прошлым, попробуй\sdash ка найди своё место$\ldots$ место посреди дикого посткоммунистического, недо\sdash или пере\sdash  капиталистического, вообще не пойми какого мира. И можно бы, кстати, уже в этом месте попросить <<парабеллум>> с одним патроном. Потому как не понятно вообще ничего. Дикое совершенно соединение коммунизма, социализма, национализма, религиозности и капитализма, царизма, непременно имперских амбиций и чудовищных, ужасных контрастов, а~также, традиционно, раздолбайства и воровства. 

\diagdash Совсем бред! Что не нравится-то?\mdash слышим мы снова возмущение тех, у кого всё хорошо или тех, кто думает, что у него всё хорошо. Тот, кто не пользуется общественным транспортом каждый день, не спускается в~метро, не ездит нашими поездами и электричками, кто чувствует себя если не <<хозяином жизни>>, то, по крайней мере, <<удавшимся человеком>>. У которого всё <<типа>> удалось. Или тот, кто в <<розовых очках>>, что многократно гнуснее. Отъедь, камрад, за 200 километров от Москвы на~какую\sdash нибудь периферию непопулярного направления\mdash и~всё\sdash ё\sdash ё! Кончилась цивилизация!

Грустно всё это, товарищи. Тут хочется выйти на~балкон, забить туго трубку, дымить и долго смотреть в~закат. Но это всё, конечно, не то. И сбегаю я$\ldots$ сбегаю в~очередной раз ото всего этого. Не могу и не хочу видеть я этой действительности. Не\sdash хо\sdash чу! Изменить что\sdash то\mdash не в~силах, терпеть\mdash терплю, но тошно. Зомбоящик смотреть\mdash одно расстройство, радио
за редким исключением\mdash сплошной шлак, а на интернет пока смотрят\sdash смотрят, понять до конца не могут что это такое, но прикрыть грозятся$\ldots$
%передачи скатились с упадком аудитории, СВ и КВ вещание благополучно закрылись, УКВ
%\mdash уже и торренты прикрывали и тэ дэ и тэ пэ. Проблем, чёрт подери, нет других. Дороги сделайте нормальные в стране! Нет, они обеспокоены, что народ музыку и фильмы в интернете «нелегально» скачает$\ldots$ да наши фильмы современные и смотреть-то за деньги не станешь$\ldots$ не все конечно, но всё же. Тьфу!

Где же Эра Мирового Воссоединения?! Нельзя? Нельзя$\ldots$ человеческая сущность, увы, не такая$\ldots$ сделав ставку на создание нового человека, нового общества, коммунисты и проиграли. Да, царя конечно в расход и~старые порядки туда же, но люди\sdash то, общество\sdash то\mdash всё с треском не переключить мгновенно, не перестроить за~один миг; это всё надо ломать и строить заново\mdash нудно, мучительно долго и, вообще говоря, непонятно как. Неужели этого никто не понимал? Странно, очень странно\mdash вроде неглупые люди были, по крайней мере, не те, кто с~винтовками. Но вся эта шумиха, коммуны и прочее, быстро относительно закончилось\mdash получилось совсем не так, как мечтали эти наивные мечтатели с наганами. Их самих слегка проредили и стали строить что\sdash то другое, что было более реально, что могло защищаться, как\sdash то развиваться и существовать, повергая в страх остальной мир кажущейся дикостью и~традиционно имперскими амбициями$\ldots$

И всё бы ничего, как мне кажется. Не могло так быть вечно\mdash всё равно бы встали на околокапиталистический путь, как это с успехом делает поднебесная сейчас. Но нет, терпеливо работать, менять всё потихоньку, учитывая недостатки\mdash не наш стиль. Наш стиль\mdash здесь и сейчас, наш стиль\mdash 300\% <<навара>> сегодня, а завтра трава не~расти. Всё или ничего! И опять великие потрясения. Снова <<до основанья, а затем$\ldots$>>. И что затем? Затем натовские <<наблюдатели>> ногами открывали все двери в стране. Всё производство, каким бы отстающим оно ни было\mdash к чёрту. Зато баксы, шмаксы, джинсы, кока\sdash кола, <<заграница нам поможет>> и прочее, прочее, прочее, так ярко пришедшееся на~пору <<детей 90\sdash х>>, к которым себя причисляю$\ldots$ А теперь, спустя ещё четверть века, вроде как Запад нам снова если не кровный, то точно какой\sdash то враг и всё такое прочее$\ldots$ Где <<свои>>, где <<чужие>>? Где политрук, который честно и~открыто скажет об этом? Всё шиворот\sdash навыворот.

Но ведь и что\sdash то хорошее было в Союзе? Не могло ведь не быть, иначе совсем петля или <<парабеллум>>. Да, было. Почему было, оно и есть, продолжает быть. Только не для всех и не даром. Да и раньше не для всех и не даром, подсказывают тут всякие$\ldots$
%<<либерасты>> со всех сторон\mdash того и гляди укусят, растерзают за свои сучьи свободы. Да, тоже не для всех и не даром, но по сравнению с теперешним <<не для всех и не даром>>, тогдашнее\mdash просто детский лепет. 
И опять не во всём! Нет универсального критерия, чтобы рассудить. На машину раньше фигушки накопишь, а накопишь\mdash ещё фигушки купишь. Сейчас\mdash работай, копи, покупай, бери в кредит$\ldots$ Лучше? Кажется, что да. По крайней мере обилие авто на улицах явно говорит само за себя\mdash лучше. В сытые перерывы между кризисами накопить на загранпоездку и поехать\mdash вот просто так, на недельку\mdash реально? Реально. Привет, такого не было точно! 
%А ещё эти чекистские штучки перед выездом, ох-ё, в страны, страшно подумать, даже соцлагеря. 
Вот с жильём было проще, подсказывают заклятые сталинисты. Потолки под три метра и всё в таком духе. Но тут тоже как посмотреть$\ldots$ а ещё молоко по 36 копеек и мороженое <<лучшее в мире>> и~тэ~дэ~и~тэ~пэ$\ldots$ 

Если так всё прелестно было, что ж так быстро всё свергли? Надоело всем, накопился огромный вакуум, стало быть. А что надоело? Чтобы понять\mdash надо жить тогда$\ldots$ а всё остальное, что сейчас говорят по этому поводу\mdash полуправда в лучшем случае. Нам уже не понять, не осознать\mdash мы не прочувствовали. Вакуум$\ldots$ надо было чем\sdash то его заполнить. Чем? Ясное дело, всем западным. Слепо, просто слепо всем западным, без разбору. Просто потому что оно другое. Красивое, другое$\ldots$ западное, одним словом. И это и есть наша самая большая трагедия\mdash ломать, не реформируя в здоровой конкуренции, старое. Ломать и не глядя заимствовать. А вакуум, надо полагать, был в самом, казалось бы, простом\mdash продовольствии и товарах народного потребления. В одежде, обуви, каких\sdash то предметах быта$\ldots$ чтобы и красиво было и не стоять за этим унизительно в очереди с талоном в руке$\ldots$ но это опять же\mdash взгляд на прошлое от непрочувствовавшего\mdash и поэтому никак не~претендует ни на истину, ни на полноту охвата.

Тут многие снова начнут возмущаться\mdash как же так, поднадоел уже со своим нытьём! Ты определись-ка, за кого ты\mdash за красных или за белых? А то ни то и ни то, похоже, не нравится! И вообще, это уже о сплавах или где? Нет\sdash нет, постойте, я ещё пожужжу! Чтобы понятнее было от чего я убегаю на реки. %От окружающей реальности, какой бы замечательной ни хотели её представить нам!

%А ещё и религиозный вопрос! Вот где последний гвоздь! За 70 лет вроде как все перекрасились, избавились от предрассудков, научный коммунизм, исторический материализм и всё вот это вот$\ldots$ ан не-е-е-е-ет!\mdash снова в мгновение ока все перекрестились и впали в религиозность! И что самое плохое\mdash с поддержкой на государственном уровне. Злость закипает внутри! Это полный провал, господа-товарищи коммунисты. А что делать, скажите вы, не поддерживать же всяких мерзотных лгбт-шников?\mdash вот мы типа в противовес и поддерживаем духовность, религиозность\mdash скажут вам! Чтобы всё чинно и благородно и вроде бы с благими намерениями. Врут конечно, твари. Там «скрепы духовные» и бронированные машины для «духовенства» и всё вот это вот. Тьфу! Нет никакой возможности смотреть на этот цирк!

Изменить ситуацию сверху\mdash никто не хочет, потому что, похоже, не знает как. Да и не надо им этого\mdash у них и~так всё хорошо. Сверху знают лишь как удержаться на этом верху и продолжать всех доить. И доить как можно дольше. Снизу изменить это\mdash нет больше такой возможности\mdash как в 1917-м\mdash да и не переживёт страна такого в третий раз. Как пить дать не переживёт. И печаль-то момента в том, что изменить снизу\mdash ха-ха!\mdash уже нельзя\mdash всех купили! Да-да$\ldots$ с потрохами просто! Купили капиталистическим укладом жизни, мнимой вседозволенностью, кредитами купили. Купили незаметной и заметной пропагандой, <<лишь бы не было войны>>, купили <<счастливым>> будущим, купили круглогодичными бананами в магазинах (как дикарей, право), круглосуточными услугами и всё такое прочее. И~сработало! Ну а кто бы отказался? Мало денег\mdash заработай, говорят. Что\sdash то не устраивает\mdash в твоих силах изменить! Надорвёшься, правда, но что\sdash то конечно изменишь. Своё~положение относительно уровня земли, к~примеру.

К чему я это всё? Больно оно надо рассуждать о всяком непотребстве вокруг нас? Достало. Сильно достало. Но ведь есть же что\sdash то хорошее, иначе совсем петля и <<парабеллум>>? И~вообще, уже опять кричат тут, определяйся, за кого ты? А всё дело в том, что я за то, чтобы сохранять всё лучшее, что у нас есть, при этом изобретая новое, нужное, перспективное, чтобы когда\sdash нибудь жить хотя бы чуточку похоже на то, как написано в\cite{ТуманностьАндромеды}$\ldots$ делать все рационально, летать к далёким галактикам, исследовать космос$\ldots$ но, это, похоже, так же несбыточно, как и Вторая Великая Октябрьская. И надо бы просто, как советуют некоторые, <<забить>> на эту всю несправедливость и попытаться улучшить что\sdash то для своей семьи, для себя лично. Звучит просто, да бывает трудновато сделать$\ldots$ да~и~правильно ли это? Кто\sdash то говорит, что да, правильно, естественно. Что\sdash то внутри говорит, что не совсем\mdash а как же ефремовская Эра$\ldots$ и всё вот это вот? Неужели проклятый капитализм в крови у человечества? И~сбегаю я в леса~Вологодчины$\ldots$ 
%но правы они, эти некоторые,\mdash чем понуро и угрюмо смотреть на весь этот ужас, на пустой кошелёк, ничерта не делать и охать, таки лучше начать с себя и благосостояния своей семьи. И никогда не получится так, как в романах Ефремова и статьях Советской Детской Энциклопедии. Но, чтобы начать что-то делать, улучшать\mdash должно что-то случиться\mdash <<гром не грянет\mdash мужик не перекрестится>>$\ldots$ 

%И случилось\mdash порезали зарплату. Работа, и так не пыльная, стала вообще незаметной. Тематика стала загнивать с катастрофической скоростью. И что самое печальное, все, конечно, понимают, что происходит, но сделать ничего нельзя, как обычно. А почему? «Да потому что мы\mdash м**ки»,\mdash говорит один достопочтенный человек. В галактическом масштабе\mdash перспектива потери Родиной технологии того, что мы делаем на работе. Ну и хорошо\mdash значит не надо! А все сказки про «затянуть потуже пояса»\mdash мимо. Увы, ребята, мы не в 20 веке\mdash хватит. Это мы уже проходили. Не надо\mdash значит не надо. Мухляж с трудовыми договорами, непонятные выкрутасы кадровиков, урезание зарплаты и отсутствие реальной работы\mdash это край. Осталось отгулять отпуск, чтобы при итоговом расчёте его не «забыли» учесть и$\ldots$ будь что будет. В любом случае, после протухающего НИИ должно быть только лучше$\ldots$ ибо что может быть хуже постсоветского «почтового ящика» со всеми отягчающими? Очень возможно, что я ошибаюсь, но искренне надеюсь, что нет.

%А пока отпуска не отгуляны\mdash 
Единственная альтернатива\mdash забыться$\ldots$ капитально забыться. Отстраниться. Сбежать. Освободиться. Сорваться. Уйти в параллельные миры. Миры путешествий по~диким, дичайшим просто закоулкам нашей Страны. Где~первозданная Природа и полная Свобода$\ldots$ где мы плывём на своих байдарках. В этой параллельной реальности ощущаешь\mdash насколько мало надо человеку, чтобы радоваться жизни$\ldots$ эти бескрайние леса Вологодчины\mdash светлые сосновые боры, верховые болота, буйство кустарников и мхов никого не оставят равнодушным; они очистят душу, откроют прекрасное в каждом сердце, однажды соприкоснувшимся со всем этим. 

Почему так полюбился мне этот край\mdash бассейн Верхней Волги, а конкретно район рек Лидь, Чагода, Чагодоща, Кобожа, Песь, Молога? С.Ю. открыл мне этот заброшенный край с природой, очень похожей на~нашу, родную Владимирскую, района реки Киржач. Только реки там, на Севере, интереснее\mdash полноводные, местами порожистые, и населённых пунктов почти нет. Зато есть чувство бесконечной Свободы и глубокой отрешенности от остального мира. Это всё овладевает сознанием настолько, что становится навязчивой идеей оказаться там и~прочувствовать это вновь и вновь. И что удивительно\mdash не~надоедает! Из года в год всё по-разному и в каждый год открываешь для себя что\sdash то новое в этом забытом всеми~крае$\ldots$
 
Так, сидя перед монитором, разглядывая топографическую карту и думая обо всём этом, хлопнув злосчастные 150, нажал клавишу Enter и маршрутные точки вместе с путевой нитью сплава стали перекачиваться в~GPS-навигатор. Идём на реку!% Alea jacta est!

\begin{center}
	\psvectorian[scale=0.4]{88} % Красивый вензелёк :)
\end{center}
