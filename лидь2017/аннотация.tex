\chapter*{}

В повести рассказывается об очередном и не менее авантюрном, чем прежние, сплаве по рекам бассейна Верхней Волги\mdash Лиди, Чагоде, Чагодоще. Автор спустя 2 года вновь оказывается на том же маршруте и вновь проносится по~самым глухим закоулкам на так полюбившемся ему стыке Ленинградской и Вологодской областей, на стыке двух миров$\ldots$ Лидь, \textit{вепсская река} (вепс. Veps{\"a}n jogi), снова проявляет свой коварный нрав и окунает путешественников в желанную атмосферу отчуждения от мирской суеты. На~этот раз путешествие начинается в деревне Радогощь (вепс.~Arskaht'), бывшем административном центре Радогощинского сельского поселения, и заканчивается в~посёлке Чагода, ставшем для нашей группы своеобразным местом притяжения. 
%
%\vspace{\fill}
%\begin{flushright}
%	\copyright~Соболев~А.А.~2022
%\end{flushright}