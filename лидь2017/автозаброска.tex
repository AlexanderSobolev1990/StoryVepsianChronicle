\chapter{Час быка} 
%\corner{64}
\vepsianrose

В день отъезда мы с женой очень долго собирались. Времени было вагон\mdash в путь нам только к вечеру. И это расхолаживало. Поспать подольше не вышло\mdash сон не шёл. Как ни старался я заставить себя уснуть днём после завтрака, чтобы отоспаться впрок, никак у меня это не получалось. План дальнейших действий был таков\mdash встреча в 17 часов на точке сбора, потом часов 10 дороги с учётом остановок, и к утру мы должны были быть в Чагоде. Так я хотел самое опасное время суток\mdash <<час быка>>, 1\thinspace\nobreakdash--\thinspace 3 часа ночи\mdash проехать уже ближе к Чагоде, где движение не такое оживлённое. 

Наконец, жена почти закончила свои сборы, я уже был полностью экипирован и готов к этому времени\mdash надо таскать вещи к машине. Тут звонит Олег и говорит, что он подъехал на точку сбора команды. На полчаса раньше. Вот оно как! Придётся ему у С.Ю подождать нас. А мы пока с женой резко активизируемся\mdash надо распихать всё снаряжение по просторам багажника и заднего ряда сидений. Делаем два-три похода вниз к машине и обратно в квартиру за новой партией тюков и вещмешков. Поскольку мы поедем с женой вдвоём в машине, то всё свободное пространство заднего ряда сидений заваливаем вещами. Байдарку вмещаем в багажник, откинув задние сидения и затащив с горем пополам этот тюк с костями\mdash шкура помещается на этот раз отдельно\mdash иначе упаковка не входит в невысокий багажник седана. Вещмешки с тушенкой, топором, прочим групповым снаряжением кладём в багажник сбоку от байдарочных костей, а малые гермомешки с провизией\mdash распихиваем в~салоне между рядами сидений. Наши большие гермомешки с личными вещами располагаем в салоне сзади поверх всего остального снаряжения. Уф! Всё влезло! Не с первой попытки, как всегда, но влезло. 

В итоге, стартуем от подъезда уже около 17:00. Пока доезжаем до точки сбора в Новокосино, проходит ещё минут 40. Ну, нормально. Адмиралу можно и задержаться! Въезжаем во двор, паркуемся. Созваниваюсь с С.Ю., а~тут и~Паша показывается\mdash традиционно с удочкой. И~вещей заметно меньше он набрал в этом году\mdash становится опытным сплавщиком. Итак, все в сборе! Олег почти не изменился с тех пор, как мы виделись у Миши на Оке. Постояли, пообщались, обсудили план. А план один\mdash плана~нет! Вперёд на~просторы! 

Но нет, план у Адмирала, конечно же, есть небольшой\mdash едем как в 2016 году по МКАД до~Ленинградского шоссе, далее по Ленинградскому шоссе, избегая всяких платных участков, до Твери. Въезжаем в Тверь, уходим на Бежецк, затем до Весьегонска, после которого на Устюжну, а потом направо до трассы А\sdash 114 и уже по ней до Чагоды. Ну, как то так вот безоблачно и~легко это выглядело на словах и на карте. Выдал Олегу рацию\mdash он с энтузиазмом воспринял такую возможность поддерживать связь между машинами, Паша тоже оживился. 

Несмотря на то, что у Олега машина на газу и багажник уменьшен на объём, соответственно, газового баллона, вещи на троих человек вместились отлично, потому как всё групповое снаряжение и продукты ехали у меня. Так что проблем с размещением вещей, байдарок, снаряжения и продовольствия не было никаких.

Итак, историческое событие свершилось\mdash успешно стартовали примерно в шесть вечера, 15 июля. Нормально, хоть и на час позже, чем хотели. Без проблем вышли на МКАД и Олег резво рванул в левый ряд и полетел дальше. Я ещё не очень уверенно чувствовал себя в таком потоке машин, но решил не отставать и, держа связь по рациям, догнал Олега. Дальше мы пошли более менее не отставая друг от друга. Начались вечерние пробки, и пришлось немного потолкаться на выезде из Москвы, хоть это была суббота, а не будни. Олег предупреждал, что не любит ездить колонной, потому что у каждого свой стиль вождения и т.д. и т.п., но в итоге мы, всё равно почему-то поехали колонной и меня это нисколько не напрягало.

До Твери добрались отлично\mdash Олег, как более опытный водитель, шёл первым и держал хороший скоростной режим. Притормаживали мы только перед населёнными пунктами.%, да перед камерами на скорость, местоположение которых было указано у меня в навигаторе. О камерах я обычно заранее говорил Олегу по рации, если шёл позади. Однако, пару раз я замешкался и не успел достать рацию сказать Олегу о камере, и мы оба слегка попали$\ldots$
Я достаточно быстро освоился в новой ситуации\mdash ещё никогда прежде мне не доводилось ездить в колонне. Мы несколько раз ездили семьёй на юг на машине, но тогда я был не за рулём. И хотя общие ощущения были теми же\mdash дорога, скоростной режим и всё вот это вот, всё же некоторое было в новинку. Концентрация внимания кардинально отличается, когда идёшь в колонне. Взгляд в основном прикован к впередиидущему и его действиям, нежели к другим участникам движения и обстановке, хоть это, наверняка, и не совсем правильно. 

Несколько раз менялись местами в нашей маленькой колонне\mdash то Олег шёл впереди, то я. Перед Тверью едва не пропустили поворот\mdash Олег шёл впереди, а я слегка отстал и в последний момент прокричал по рации, что пора поворачивать направо, иначе бы мы оказались на объездной Твери, что в наши планы не~входило. Свернув в Тверь, мысленно порадовался, что у нас есть рации.

Пока что мы в точности повторяли наш маршрут заброски 2016 года. В Твери пересекли Волгу и ушли на Бежецк, остановившись ненадолго\mdash Олег по рации запросил остановку для того, чтобы пополнить запасы походной аптечки в придорожном магазине. 

Романтика путешествия увлекала нас всё больше и~больше. После выезда из Твери машин стало исчезающее мало, а по мере удаления от неё, так вообще. Погода стояла весьма влажная\mdash выпал обильный туман, видимость снизилась примерно до 400\thinspace\nobreakdash--\thinspace 500 метров. Спустя около получаса после выезда из Твери, мы решили остановиться и перекусить, что и сделали прямо на обочине дороги\mdash движение на трассе было не слишком оживленным. Олег разложил запасы провизии прямо на багажнике своего видавшего виды седана, и мы подкрепились бутербродами, сыром, закатанным в лаваш и чаем из термосов. После продолжили наш путь по разбитым тверским дорогам$\ldots$

Дороги\mdash действительно наша беда, и неизвестно какая из двух хуже. Я понимаю, конечно, что расстояния у нас не как в европах, но всё же\mdash почему не сделать один раз нормально дорогу по нормальной технологии?! Ничего подобного! Зато заплатки\mdash это пожалуйста. Время от~времени латать ямы\mdash сколько угодно! Порой мы снижали скорость до 60\thinspace\nobreakdash--\thinspace 70 км/ч или и того ниже при разрешённых 90 на этих участках. Разрешённые 90 тут как издевательство скорее. 

Так, с переменным успехом, добрались до Бежецка, а потом и до Красного Холма, где в столь поздний час было неожиданно много народу на улицах, и стояла полиция. Как мы поняли\mdash местные отмечают с большим размахом чью\sdash то свадьбу. Мы же аккуратно объезжали гуляющих по дороге подвыпивших местных жителей и продолжали дальше наше путешествие. 

Через десяток километров после Красного Холма, в~Хабоцком, проехали тот злосчастный левый поворот на Молоково, где неслабо протряслись в прошлом году на ужасно разбитой дороге, и поехали прямо\mdash на Весьегонск через Кесьму. Там, как я надеялся, дорога будет лучше! Во всяком случае, в интернете утверждалось, что покрытие там\mdash асфальтобетон. Смущало только отсутствие панорамных снимков с дорог в этом крае$\ldots$ но, это всё от его заброшенности, думалось мне, и мы с каждой секундой приближались всё ближе и ближе к городу Весьегонску, порту на Рыбинском водохранилище. И я, конечно, понимал, что со смертью Тихвинской водной системы порт этот, как и город, конечно же, будет в полном упадке, но масштаб разрухи края, в~который мы погружались с каждым мгновением всё глубже и глубже, просто поражал и не переставал удивлять$\ldots$ а~ты ещё там смеешь жаловаться что плохо в своей Москве живешь! Тут вообще полный край. 

Явственно начал я ощущать с наступлением темноты и~резким уменьшением количества машин на дороге атмосферу полного отчуждения, нереальности происходящего; та самая неуловимая тонкая грань стыка двух миров вновь подкралась как-то незаметно и~быстро овладела сознанием, увлекая нас, путников с тюками и байдарками, всё дальше и дальше в~бескрайние просторы нашей Родины$\ldots$ я ощущал себя уже и~не дома, но ещё и не на реке; уже не в привычной атмосфере города, но ещё и не в привычном ритме сплава\mdash мне казалось, что сейчас я, как канатоходец, иду вперед по этой тонкой грани, балансируя над пропастью, боясь сорваться в~обыденное, привычное, опостылевшее$\ldots$

Погода стояла очень и очень влажная\mdash мы ехали, взрезая своими авто обильный туман. Видимость резко ухудшилась. Пробовал разные комбинации\mdash ближний свет и противотуманки или только противотуманки$\ldots$ с дальним ехать вообще невозможно\mdash слепит. Приходилось снижать скорость$\ldots$ пришёл к выводу, что лучше просто пониже опустить фары с включенным ближним светом и всё, а~противотуманки почему-то больше слепят, чем помогают$\ldots$ так что их выключил. Как-то вот так и ехали. 

До Весьегонска добрались раньше расчётного времени, что радовало. Остановились в самом его центре у~бывшей железнодорожной станции, чтобы перекусить, походить, размять ноги. На часах было около часу ночи. Мы преодолели почти 2/3 пути или около 450 километров. Оставалось ещё примерно 170 километров до Чагоды, и я был относительно спокоен\mdash б\'{о}льшую часть мы прошли, в сон не сильно клонит, концентрация внимания не снижается, реакция в норме. Хотя, некоторая усталость, естественно, начала ощущаться. Жена временами подрёмывала, откинувшись в кресле. Ей не мешала даже постоянно играющая в салоне музыка. 

Не могли упустить возможность сходить на~железнодорожную станцию в Весьегонске\mdash пошли вместе с С.Ю. исследовать сей заброшенный объект посреди ночи, подсвечивая себе фонариком. А это оказалась и не станция вовсе, а просто пассажирская платформа. Вокзал сгорел несколько лет назад, как я потом выяснил, а новый строить никому не надо. Табличка новая на платформе висит\mdash красивая, красочная, в новом стиле\mdash красная с~серым, и на английском ещё продублированная. Расписания нет. Поезда не ходят от слова вообще. Пассажирское сообщение тут отсутствует с начала 2000-х годов\mdash раньше ходило несколько прицепных вагонов$\ldots$ а сейчас, наверно, и «кукушек» не осталось. И что это\mdash беспросветный упадок или наоборот, все такие богатые и на машинах теперь рассекают?! Ну, сами знаете ответ$\ldots$ станция Семичеловечья\cite{ГеографГлобусПропил}, так и раз эдак, с красочной табличкой посередине.

Рядом находилось какое\sdash то закрытое здание\mdash видимо автобусная станция. Ещё дальше виднелись огоньки и громко играла музыка\mdash местная дискотека или клуб. И всегда так\mdash вспомнить даже музыку из лагеря под Тургошью в~2015 году\mdash чем хуже дыра, тем круче там <<отдыхают>>$\ldots$ до~беспамятства, с размахом и громкой музыкой$\ldots$ иначе совсем петля и <<парабеллум>>, видимо$\ldots$ Мы же стояли и~попивали чаёк из термосов, разминали ноги. Потом умылись с Олегом водой, освежились. У Олега для этой цели была припасена пятилитровая канистра. Снова пора в путь!

После остановки в Весьегонске нам оставалось не~так уж и много\mdash всего\sdash то добраться до Устюжны, а~там уже по хорошо знакомой трассе А\sdash 114 до Чагоды. Но на выезде из Весьегонска я почуял неладное\mdash дорога стала просто кошмарной!!! Я поздно заметил капитальную яму и применил экстренное торможение до~срабатывания антиблокировочной системы, параллельно выжимая сцепление в пол$\ldots$ реакция\sdash то в норме, а внимание притупилось по\sdash любому, вот и пропустил$\ldots$ инстинктивно глянул в зеркало заднего вида и обнаружил, что Олег, уходя от столкновения со мной, выехал на встречку. Благо и~скорость была не так высока, и на встречке никого не~было$\ldots$ 

\diagdash Не де-е-елай так больше,\mdash недовольно и вполне справедливо послышалось по рации от Олега. 

\diagdash Извиняй$\ldots$\mdash поёжился я в кресле,\mdash Проморгал! \mdash ответил я и отпустил тангенту.

Впредь поехали аккуратнее и теперь я, давя на тормоз, старался все же бросить взгляд в зеркало заднего вида$\ldots$ Дорога, тем временем, становилась только хуже. Просто Дрезден после бомбардировки союзной авиацией! 

Началась грунтовка! Чёртовы карты\mdash где обещанный асфальтобетон?! А ямы какие?! Временами даже 30 км/ч\mdash непозволительно много! Полный атас. Мне стало безумно жалко подвеску автомобиля! Я вспомнил карту и понял, что весь участок Весьегонск\thinspace\nobreakdash---\thinspace Устюжна будет таким, поскольку больше крупных населенных пунктов тут нет$\ldots$ выходит, в~прошлом году через Сандово было ещё очень даже неплохо! Там хотя бы был асфальт, а тут просто грунтовка! Хоть и~широкая и местами укатанная, но грунтовка. С ямами, колдобинами и лужами. Движения, естественно, никакого. В~час ночи тут практически никто не ездит. И вот\mdash мы одни, в такой откровеннейшей глуши, пробираемся на этом, уже вполне осязаемом, стыке миров\mdash нашего и параллельного, к конечной цели\mdash Чагоде$\ldots$ и это кажется нам просто вечностью при такой дороге. Как караван в сердце пустыни Кара-Кум\mdash с такой же, временами, скоростью.

Если дорога до Весьегонска пролетела ну почти как один миг, хоть и составила 2/3 по расстоянию, то после Весьегонска, как мне показалось, мы тащились просто как черепахи. Временами удавалось разогнаться до 50~км/ч, но тут же приходилось сбрасывать, завидев колоссальные выбоины и колдобины. Один раз опять слишком поздно заметил выбоину и влетел левым колесом до пробоя подвески$\ldots$ остановились, взял фонарик, пошёл осматривать, изрекая неблагозвучные послания тем, кто делал эту дорогу, тем, кто за ней должен следить и вообще проклятия до~седьмого колена на вcех наших дорожников. Визуальный осмотр ничего не дал\mdash вроде бы всё было на месте, и даже диск с виду не погнулся. Навигатор язвительно показывал ограничение скорости 90~км/ч$\ldots$ Аккуратно продолжили движение, впредь стараясь всё же не разгоняться выше 40~км/ч. Так и ехали. Оживлённее потекли радиопереговоры по рациям\mdash это хоть как\sdash то снимало тоску от перспективы сотни километров разбитой грунтовой дороги$\ldots$

А туман, тем временем, стал ещё гуще. С дальним светом ехать опять стало невозможно, поэтому шли с~ближним. Встречного и попутного транспорта не было и в~помине. Мы одни. В такой глуши, что временами становится даже немного страшновато, жутковато. Но ветер странствий, задувающий в открытое окно, зовёт вперёд и некогда грустить\mdash надо объезжать ямы! Встречались редкие населённые пункты, а точнее таблички с их названиями\mdash сами деревни в стороне от дороги. И ни души$\ldots$

Ближе к Устюжне остановились ещё разок\mdash умыться и~освежиться. От влажности и тумана у меня стало сильно запотевать лобовое стекло. Решил испытать, как работает обогрев лобового стекла\mdash раньше я не встречал, чтобы прямо в него были вмонтированы тончайшие нити для обогрева. Обдув в комплексе с обогревом стекла очень выручили, а Олег ещё подсказал включать в такие моменты кондиционер\mdash он осушает воздух и эффект запотевания пропадает$\ldots$ так и ехали\mdash временами закрывали окна и включали обогрев стекла и кондиционер. Снаружи же стала спускаться приятная ночная прохлада, вечерняя духота давно пропала. Я часто открывал окна, чтобы заглотнуть свежего воздуха, а потом и вовсе выключил кондиционер и полностью открыл окно водительской двери. Приятный прохладный ночной воздух растекался по салону, окутывая нас и ободряя от~мимолётно набегающей усталости.

Въехали в Соболево. Большая деревня или даже село. Укреплённые избы, ухоженные участки, но самое главное\mdash появился асфальт! Это было просто спасением! Мы капитально приуныли от этой грунтовки, на которой нам встретилось на всём протяжении лишь одинокие остановки автобуса (!) «времён Очаковских и покоренья Крыма» около забытых всеми деревень, да парочка лесовозов$\ldots$ Туман пропал, но погода не улучшалась. Мы вплотную подобрались к Устюжне, и Олег по рации сообщил, что недурно было бы заправиться. Возвращение к какой\sdash никакой, но цивилизации. Решили заправиться бензином сразу по выезду из Устюжны. Газовую заправку для Олега решили специально не искать, поскольку справедливо рассудили, что ночью они не работают.

В Устюжне так же, как и в 2015 году, постояли на~одиноком светофоре в самом её центре и поехали в сторону Лентьево, повернув не налево на адскую грунтовку через деревню Мочала, как тогда в 2015, а направо. На выезде из Устюжны заправились бензином, выяснив, что газовая заправка ночью у них в городе действительно не работает. Вот так! Заправили полные баки под горловину и полетели дальше\mdash на трассу А\sdash 114, до которой оставались считанные километры. Асфальт стал заметно лучше! Асфа\sdash а\sdash альт! Выехав на трассу А\sdash 114, мы вообще воспрянули духом\mdash трасса мечты после грунтовки Весьегонск\thinspace\nobreakdash---\thinspace Устюжна! Снова пошли с хорошей скоростью, не забывая про камеры. Однако вскоре пошёл легкий дождь, покрытие стало мокрым. Пришлось умерить скоростной аппетит, да и усталость стала накатывать волной от единообразия дороги\mdash хороший асфальт, прямая дорога, ямы объезжать не надо. Останавливались ещё раз умыться водой и освежиться. Потом, примерно на середине пути от Устюжны до Чагоды, ливень стал очень сильным, вокруг вновь образовался туман, снова стало запотевать стекло. Резко снизили скоростной режим и предельно аккуратно выполняли редкие обгоны$\ldots$

И вот, наконец, указатель поворота на Чагоду. Ура! Теперь остаётся совсем немного. Усталость уже сильно чувствуется. Но мы включаем музычку погромче, жена уже не дремлет и мы болтаем, разгоняя усталость. Вскоре впереди видится что\sdash то вроде полицейской заставы\mdash машины, сине\sdash красные мигалки$\ldots$ что такое\mdash непонятно, поэтому снижаем скорость. Оказалось, что это совершенно жуткая авария. 3\thinspace\nobreakdash--\thinspace 4 развороченных до состояния металлолома машины, куча, видимо, родственников, пластиковые чёрные мешки, содержимое которых не оставляет сомнений, полицейские, разруливающие движение$\ldots$ короче картина ужасная. Видимо под утро кто\sdash то уснул за рулём и свернул на встречку, скорее всего\footnote{Позже, через год я узнал от местных, что водитель был пьян, увы.}. Мне было не впервой наблюдать такую картину, но, тем не менее, пробрало до мурашек. Действительно, <<час быка>>$\ldots$ Дальше поехали предельно аккуратно.

Чагода встречала нас хмурой погодой и редким дождичком. Пересекли реку Чагодощу по тому самому мосту, на котором в прошлом году дорожники перекладывали асфальт, и, проехав ещё пару поворотов, оказались на~Кооперативной\mdash центральной улице этого рабочего посёлка. Притормозили около нового магазина\mdash в прошлом году его не было\mdash чуть передохнули и сделали последний рывок до гостиницы, что на окраине посёлка. Пока ехали туда, мимо промелькал чагодощенский стеклозавод, производящий бутылки и прочую тару. Площади большие, производство не~простаивает, судя по всему. На этом заводе, надо полагать, и~работает подавляющее большинство местного населения$\ldots$

Наша цель, гостиница, оказалась на месте, отмеченном на старых топокартах как <<лесобиржа>>, и~представляла собой двухэтажное здание из силикатного кирпича. Со стороны это строение больше напоминало частный дом, нежели гостиницу. Но, это было то, что надо. Припарковались рядом и вышли из машин. Мы~добрались сюда! Мы сделали это! Первый этап заброски пройден! Но состояние какое\sdash то вообще не боевое. Достал трубочку, набил потуже. Ценой неимоверных усилий не~стали доставать заветные канистры прямо сейчас\mdash вопрос с~дальнейшей заброской\sdash то ещё не решен. И хотя я сомневался, что нам придётся снова садиться сегодня за руль своих авто, подумали, что лучше не рисковать. 

Решили перекусить и немного поспать прямо в~машинах, поскольку было ещё слишком рано\mdash часа 4 утра. Устали. Дорога вышла утомительная не только для водителей, но и для пассажиров\mdash грунтовка после Весьегонска нас просто вымотала. В гостиницу решили пойти попозже, чтобы зря не будить дежурную. Всё равно заселяться нам не надо, а просто ходить туда\sdash сюда смысла нет. Поэтому мы перекусили, разложив уже привычно провизию и термосы на олеговом багажнике, и расползлись по машинам поспать или хотя бы сделать вид. 

Подремать не очень\sdash то получалось. Откинуть кресла назад нельзя\mdash там сзади вещи. Поэтому довольствовались тем, что было, кое\sdash как устроившись поудобнее. Снова начался дождь. Небо свинцового цвета. Настроение вновь подавленное какое\sdash то. Впереди полная неизвестность, потому как вопрос с дальнейшей заброской окончательно решён ещё не был\mdash надо было дождаться утра и хозяйки гостиницы. Кроме того, и речку\sdash то тоже могли запросто переназначить прямо тут и сейчас, если с заброской всё окажется печально$\ldots$ в любом случае, надо было ждать~утра. 

Быть может не самое разумное распределение времени вышло, но я считаю, что расчёт был верен\mdash самый плохой участок ночи, когда клонит в сон, 3\thinspace\nobreakdash--\thinspace 4 часа утра, как раз пришёлся на конец пути и пустынную хорошую трассу. Если бы мы выехали из Москвы позже, то б\'{о}льший остаток пути до Чагоды пришлось бы ехать в таком квёлом и разбитом состоянии, что не есть хорошо. И словно в подтверждение моих дум об опасности этого рассветного времени, мы встретили ту страшную аварию на въезде в Чагоду$\ldots$

Часов в 7 утра решил, что хватит ждать\mdash набрал хозяйке гостиницы$\ldots$ и был неприятно удивлён, что о~нас забыли. А вся переписка по электронной почте воспринималась ею, оказывается, лишь как предварительная договорённость, и водитель уазика о нас знать не знает$\ldots$ Но, как говорится, клиенты на месте и почему не заработать? Поэтому я и не очень переживал по этому поводу\mdash почему\sdash то была уверенность, что транспорт образуется. Так~и~вышло. Сама хозяйка подъехала через час, повздыхала, что мы такие\sdash сякие безответственные и могли бы позвонить при выезде из Москвы. Да, это было бы наверно не лишним, она права\mdash учтём на всякий случай в следующий раз. 

О стоянке договорились\mdash загнали машины на задний двор гостиницы, где параллельно стояли ремонтировались несколько дальнобойщиков. Потом хозяйка пригласила нас на кухню, напоила горячим чаем с печеньем и сказала подождать, пока она попробует договориться об уазике. Я~почему\sdash то был относительно спокоен. Это выльется ну в 5\thinspace\nobreakdash--\thinspace 7 тысяч и навряд ли кто откажется за несколько часов немного заработать. Так и вышло\mdash спустя буквально полчаса пришел Юрий\mdash водитель уазика <<буханки>>. Мы~быстренько пошли перетаскали вещи из наших машин в <<буханку>>. Там внутри были скамейки по бортам, а~посередине пустота, куда и было свалено всё снаряжение. Бригада разместилась вдоль бортов на лавочках, а я сел спереди и достал GPS. В путь! Водитель согласился ехать до Радогощи. Это означало\mdash Лидь!

Попрощались с хозяйкой и отъехали. Пока проезжали по Чагоде, вовремя вспомнили, что забыли докупить хлеб и~остановились на выезде около небольшого магазинчика, где я взял три батона белого и буханку черного. Дальше дорога ясна\mdash снова до трассы А\sdash 114, поворот направо\mdash на Ефимовский. Потом после Ефимовского по гравийно\sdash грунтовой дороге до Радогощи и там$\ldots$ нас ждёт коварная красавица Лидь!

\begin{center}
	\psvectorian[scale=0.4]{88} % Красивый вензелёк :)
\end{center}