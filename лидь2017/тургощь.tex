%\chapter{Тургощь или рекорд шести сплавов} 
\chapter{Дубль два} 
%\corner{64}
\vepsianrose

Глаза с утра никак не хотели раскрываться и, хотя по~времени я должен был уже выспаться, организм требовал ещё покемарить полчасика. Я попытался, насколько это возможно, дистанцироваться от утренней готовки на костре и~налегал на подоспевший чай. Бригада неспешно приготовила завтрак и свернула лагерь. Сегодня нас ждал длинный переход, думалось мне\mdash до Тургощи совершенно точно. На~воду мы вышли не поздно, хорошо.  

Минула половина сплава$\ldots$ стало слегка грустно, как и~обычно в~середине похода$\ldots$ так и крутилось в голове дать на работу телеграмму в стиле: <<нахожусь глухой тайге зпт причине невозможности взять обратный билет зпт прошу продлить отпуск тчк>> и пусть думают что хотят. Какие телеграммы, XXI век$\ldots$ Эх, но ведь как хочется плюнуть вообще на всё и пойти дальше\mdash не до Чагоды, а до Лентьево, как уже ходили, и дальше\mdash до Весьегонска по~Мологе. А~может и вовсе до Рыбинска. Но это всё мечты, мечты$\ldots$ надо месяц отпуска разом при таком раскладе по~километражу, или по крайней мере недели три.

Итак, мы отчалили и пошли дальше по этой петляющей речушке бассейна Верхней Волги. Погодка выдалась снова пасмурной. Как хорошо, что вчера был солнечный теплый день, и мы сделали днёвку, подумалось мне. Шли без особых приключений, но реку я вообще не узнавал. Уровень воды обладает каким\sdash то совершенно удивительным свойством менять облик реки\mdash всего полметра\sdash метр разницы и это уже совершенно другая река! Так и сейчас\mdash я не узнавал вообще ничего. Мне помнилось, что уже тут начались поворотики с песчаными пляжами, но нет\mdash берега, проходимые нами, были сплошь заросшими и затопленными. Лишь временами виднелись желтеющие сквозь прозрачную, с коричневым оттенком, воду затопленные песчаные косы.

Почти сразу как снялись со стоянки, прошли устье Белой. И я мысленно передал ей привет от Башкирской тёзки. В одном из лучших описаний маршрута по Лиди есть вариант с заходом по Белой против течения до места бывшей мукомольной мельницы, обозначенной на картах как сарай в километре южнее деревни Забелье\cite{КатераиЯхтыЛидь}. Как~утверждается, место там очень красивое и тихое (и,~скорее всего, открытое по ошибке, поскольку о GPS можно тогда было только мечтать) Проверить нам этого, к сожалению, не~удалось, поскольку против течения в Белую заходить как\sdash то не хотелось. Но я искренне надеюсь, что эти красоты Лидь ещё откроет мне в~будущем$\ldots$

Спустя примерно час прошли устье небольшой речушки Межницы. Теоретически, можно подняться по ней против тихого течения сначала до озера Межнинского, затем пересечь железнодорожную ветку Подборовье\thinspace\nobreakdash---\thinspace Кабожа и~оказаться там в озере Карасинском, а потом и~в~озере Великом, на западном берегу которого, как также утверждают в\cite{КатераиЯхтыЛидь}, можно найти неземных красот стоянку. Проверять это мне почему\sdash то не захотелось\mdash во\sdash первых, против течения идти как\sdash то совсем грустно, а во\sdash вторых, устье было также слегка заросшим и~не~внушающим доверия. Поэтому мы пошли дальше, хотя может быть и зря.

Так, неспешно гребя ещё около часа, мы добрались до весьма интересного места\mdash на левом берегу открылся вид на большой воронкообразный песчаный подъём. У меня эта стоянка никак не отложилась в памяти со сплава 2015 года. Естественно, мы причалили. Экипаж С.Ю. вылезать не захотел. Наш же экипаж, привязав байдарку чалкой за корни растущих на берегу сосен, поднялся на высокий берег. Картина наверху была шикарна! Стоянка что надо! Есть незамусоренное костровище, брёвна вокруг него на манер лавочек, чисто, много места для палаток\mdash хорошая ровная лужайка! Отчего я её не приметил 2 года назад\mdash непонятно. Ну, ничего, зато теперь буду знать. Координаты места\mdash \CoordsLidSuperPlace. Оставаться тут не входило в наши планы, хотя и было заманчиво, во\sdash первых потому что стоянок на данном участке практически нет, а во\sdash вторых рядом, в нескольких километрах на восток, в глухом лесу, раньше находилась деревня Тургощь, а потом она плавно перекочевала километров на восемь южнее, к железной дороге. На её месте позже якобы располагалась какая-то радиолокационная станция. И хотя понятно, что всё там давно брошено и растащено, такие места почему-то обладают каким-то особым притяжением для меня. Как бы там ни было, мы передохнули немного, отчалили и продолжили наш путь дальше, заметив себе, что место было очень даже стояночным.

Река начала приближаться к железнодорожной ветке Подборовье\thinspace---\thinspace Кабожа. Берега стали ниже и по\sdash прежнему были покрыты сплошными зарослями растительности. Скоро должны были начаться перекаты$\ldots$ так и произошло. Стали аккуратно проходить препятствия, вовремя лавируя в русле. Каких-то особых проблем это не вызывало\mdash нам удавалось вовремя наметить более глубокое место в русле и избежать встреч с большими подводными камнями. Вообще, надо сказать, прохождение было намного, намного проще, чем 2 года назад. Тогда нам несколько раз приходилось вылезать в холодную воду из байдарки, севшей на камни, вытаскиваться$\ldots$ тут такого не было\mdash всего несколько раз камни неприятно прошелестели по днищу и всё обошлось. 

Но длилось такое счастье недолго. Я помнил, что где\sdash то на этом участке, к которому мы сейчас подходили, был суровый перекат, на котором мы по неопытности заработали тогда, в 2015\sdash м, десять пробоин$\ldots$

Поскольку постоянно сверяться с навигатором в этом сплаве у меня возможности не было\mdash доставать телефон из гермокоробки во время прохождения поворотов, постоянно требующих от капитана байдарки активных действий\mdash тот ещё фокус, то я плыл практически вслепую, полагаясь на опыт и воспоминания двухгодичной давности. Так, при подходе к очередному перекату, при просмотре с воды я не смог наметить более\sdash менее приемлемый путь прохождения. В голове успело мелькнуть\mdash это он, тот перекат$\ldots$ быть пробоинам.

Так и вышло$\ldots$ ускорившийся поток начал сносить нас на камни. Передовой отряд прошёл нормально, и мы силились повторить их траекторию$\ldots$ Олег старался выправить курс, я вовсю лопатил веслом, но поздно\mdash момент упущен! Мы не погасили скорость заранее, и теперь что-либо исправить было поздно\mdash сильный толчок в днище! Паша и С.Ю. проскочили это место буквально в полуметре правее\mdash их каркасно\sdash надувная байдарка плавно проскользила по камню и не получила повреждений. Мы же здорово получили хороший удар в районе первой трети кильсона$\ldots$ Вода вроде бы стала прибывать, но не сильно. Ладно, впереди небольшой поворотик и$\ldots$ песчаная коса. Да, история имеет свойство повторяться\mdash теперь я окончательно узнал это место. дубль два\mdash nот самый перекат и та же самая песчаная коса, на которой в 2015 году новоиспечённые сплавщики, рискнувшие покорить эту речку, терпеливо заклеивали свою байдарку$\ldots$ 

Наша эскадра причалила к песчаному островку, который, естественно, был в этом году меньше из-за высокого уровня воды. Вылезли, начали осмотр днища байдарки. Настроение как\sdash то резко упало. Пробоину найти смог не сразу\mdash оказалось что это ма\sdash а\sdash аленькая дырочка сразу около леи. Подумав, решил ничего с ней не делать и просто положить поверх кусочек туристического коврика, потому что сейчас клеиться\mdash не вариант. Погода не располагает, берега тоже. Да и ради одной маленькой дырочки разгружать байдарку, доставать клей, разводить костёр$\ldots$ что\sdash то как\sdash то лениво. Поэтому мы просто передохунли, попили чая из термосов, перекусили бутербродами и снова встали на воду.

Дальше прошли крутой поворот и завал по правому берегу в старице. Перекаты стали проще и вскоре совсем прекратились. Зато речка сузилась, а лес как бы нависал над ней\mdash красота сумасшедшая! И берега вновь стали выше\mdash мы приближались с каждым поворотом русла к Тургощи\mdash станции на железной дороге. Там я планировал пройти чуть ниже по течению и встать на стоянку сразу после железнодорожного моста. 

Начался дождик. Уже привычно мы облачились в дождевики, а Паша в свой плащ химзащиты$\ldots$ погода испортилась. Шли и искали место, чтобы встать. По километражу получалось, что район Тургощи как раз подходит под дневную норму$\ldots$

Вскоре показалась и Тургощь. Та же турбаза на берегу и пляжик. Завал разрушенного моста разметало как щепки и снесло еще ниже по течению. Место моей стоянки было обжито и замусорено, что огорчало. На берегу были немногочисленные отдыхающие$\ldots$ Я силился вспомнить, какой же сегодня день недели\mdash почему так много народу на берегу$\ldots$ вспомнить не смог. 

Мы плыли дальше\mdash места не нравились, и даже на разведку вылезать не хотелось$\ldots$ так потихонку и подошли к железнодорожному мосту. С небес лились капитальные потоки воды, вылезать и фотографировать табличку <<р.\thinspace Лидь>> у железнодорожного полотна совершенно расхотелось, хотя я и планировал. Настроение было отвратительным, команда тоже была подавлена отвратной погодой и неизвестно какими перспективами стоянки$\ldots$

Небо густо затянуло тучами. Дождь то прекращался, то снова налетал неистовыми порывами. Ветер тоже доставлял массу впечатлений. Жена стала замерзать и чтобы согреться, стала время от времени брать у Олега весло и тоже грести. Так мы прошли довольно долго. Вода, скапливаемая в складках дождевика, временами переливалась в байдарку, а временами мне на штаны, отчего спина и штаны в районе пояса слегка промокли. Ноги тоже уже были замочены к этому времени$\ldots$ словом, спустя час проливного дождя не осталось ни единого сухого места в байдарке$\ldots$ Жена стала переживать или даже паниковать по поводу полного отсутствия стояночных мест и даже намёков на них. То, что все мы основательно промокли, несмотря на предпринятые меры в виде дождевиков, тоже не добавляло радости.

И вот, дрейфуя на повороте реки, вы увидели впереди на высоком левом берегу основательное сооружение\mdash нечто вроде беседки. Естественно, причалили. Место было так себе. Навес со столом\mdash да, были, но вокруг многовато мусора, крыша навеса из разодранной плёнки$\ldots$ тот ещё видок, да и вокруг высоченная неуютная трава. Также, разведка показала наличие дороги, хоть и практически не проходимой, ну, если только на <<Урал>>\sdash е. Наш экипаж, забравшийся на берег, стоял и осматривался. Место не нравилось, хоть и очень хотелось уже встать лагерем, обогреться и сварить какую-нибудь похлёбку$\ldots$ сели в байдарку, пошли дальше.

Далее ещё в течение нескольких часов мы часто останавливались, вылезали на разведку, но места категорически не подходили под наше понимание о нормальной стоянке. Отчаяние нашло на наши экипажи. Мы плыли и жадно высматривали место для причаливания. Казалось, вот-вот очередной поворот принесет нам хорошее стояночное место, но нет. Заросший берег, отсутствие схода к воде и большая глубина сразу под берегом. Я~чувствовал по начавшим ныть мышцам, что километраж мы опережаем значительно$\ldots$ Идея пристать к берегу, где попало, и провести широкомасштабные шанцевые работы по выравниванию местного ландшафта, уже не казалась такой бредовой.

Стало холодно. Все подмерзли. Паша кутался в зеленый плащ химзащиты, но даже такая суровая защита не слишком воодушевляла. С.Ю. поглубже нацепил свою коричневую фетровую шляпу. Жена тихонько проклинала всё и вся и погоду в особенности. Олег дымил сигаретой и стоически переносил погодные условия. Я же был относительно спокоен\mdash до темноты еще много времени, мы где-нибудь да найдём место. Однако промокшие штаны и мокрая штормовка не грели меня совершенно.

И вот, на выдавшемся прямом участке реки мы заприметили место на правом берегу и немедленно жадно причалили для осмотра. Старое стояночное место. Есть небольшой полусгнивший навес и разваливающийся столик из досок. Местность не слишком ровная\mdash бывшая колея, но что поделать. Надо вставать лагерем\mdash лучше мест может и не быть\mdash Законы Лучших стоянок тоже, бывает, не срабатывают.

Разгрузились, перетаскали вещи, поставили тент первым делом. Сразу же натянули веревку и развесили на ней по максимуму мокрые вещи. Поставили палатки и сразу же переоделись в сухое. Ка\sdash а\sdash айф. Нет не так. К\sdash а\sdash а\sdash а\sdash а\sdash а\sdash а\sdash айф!!! Теперь костёр. Всё сырое и мокрое. Разгоралось очень долго и мучительно, даже при помощи газовой горелки. Но мы бы не были туристами\sdash водниками, если бы не смогли совладать с костром. И вот вскоре уже поспевает чаёк и все единогласно поддерживают идею сварить макароны с тушёнкой$\ldots$

Поразительно, как теплая одежда, сухость, горячая еда и чай мгновенно ободряют человека\mdash понимаешь, что большего тебе сейчас, вообщем\sdash то и не надо. Ан нет, надо! Где там наша канистра? Только при её помощи и уже прилично распалённого костра удаётся нормально согреться. Вот тебе и лето, так и раз эдак! 
 
Процесс приготовления ужина запущен, и я немного отлучаюсь со стоянки провести доразведку местности\mdash Паша сказал, что близко дорога. Так и есть\mdash причём слегка оживленная, что неприятно. При мне по ней проехала пара машин. Ладно, мы тут не на днёвку, а только переночевать. Лес редкий, видно всё очень далеко. Все устлано мхами и опавшей хвоей. Запах стоит\mdash сногсшибательный. Аромат влажного настоящего хвойного леса$\ldots$ Красота!

Ужин, тем временем, уже поспел и мы, приютившись у старого стола, быстро и жадно его поглощаем, обильно заедая салом, чтобы согреться\mdash вечер выдаётся далеко не самым теплым. У костровища лежало несколько брёвен на манер сидений и мы подвинули их ближе к огню, расселись. Хорошо! Я полез за телефоном, включил GPS и не поверил своим глазам! Мы проплыли сегодня по грубым оценкам не меньше 40 километров! Позднее, уже в Москве, я точно высчитал киломераж\mdash в этот дождливый и пасмурный день с пробоиной в борту мы прошли 41 километр. Вот это да! Этот показатель стал рекордом моих шести сплавов. Столько я за один дневной переход ещё ни разу не проходил. Что там про пробоину? Точно! 

\vspace{0.1cm}
\noindent\textit{%
	\hspace*{1.0cm}В Кейптаунском порту, с пробоиной в борту$\ldots$
}
\vspace{0.1cm}

Настало время подумать об оставшемся километраже и прикинуть дальнейшие перспективы. Фактически, нам остаётся один дневной переход до Чагоды, нашего конечного пункта. Днёвку на этой стоянке мы устраивать не хотим\mdash место не глухое и вообще не днёвочное, никакое. Значит, решаю я, план таков\mdash идём завтра до уже разведанной в прошлых годах стоянки под Чагодой, там ночуем, а послезавтра символических пару километров и антистапель. Обильно закрепляем решение яблочной, и я блаженно набиваю трубочку, привалившись к бревну у костра$\ldots$ хорошо! 

Дождь прекратился, и чуть попозже мы натянули ещё одну веревку для просушки вещей, потому что первой не хватило\mdash промокло очень много всего. Гитару опять не расчехляли\mdash что\sdash то было не до того, да и как\sdash то настрой был не тот после тяжелого дня. Позастольничали и завалились спать не слишком поздно\mdash сорокакилометровый переход нас сильно вымотал.

\begin{center}
	\psvectorian[scale=0.4]{88} % Красивый вензелёк :)
\end{center}
