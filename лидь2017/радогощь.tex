\chapter{Радогощь, стапель и Лидское озеро} 
\corner{64}

<<Буханка>> медленно ползла по трассе. 80 км/ч для неё был просто предел. Стрелка на спидометре прыгала с разбросом $\pm$20 км/ч из\sdash за дикой тряски. Всё\sdash таки предназначение этого пепелаца\mdash далеко не шоссе$\ldots$ якобы его разрабатывали как катафалк для ядерной войны. Так ли это, точно не знаю, но суровость и простота\mdash убийственные. Вдобавок, почему\sdash то не выключалась печка. Из воздуховода дуло горяченным воздухом прямо мне в ноги и закрыть или как\sdash то отключить это было нельзя. Водитель тоже ехал и, видимо, терпел, ведя машину в сапогах. Я же максимально вжался в дверь, отстранив ноги от патрубка, откуда дул горячий воздух, и мысленно завидовал бригаде, разместившейся в салоне. Открыл стекло двери ручкой, устроился поудобнее и начал писать, преодолевая тряску, путевые заметки, чтобы отвлечься. 

До Ефимовского добрались нормально, а дальше поехали по заметно ухудшившийся, но пока ещё асфальтовой дороге. Вскоре пересекли Тихвинский канал и дальше началась гравийная дорога, кстати, весьма и весьма приличная. Но скорость по ней держать высокую невозможно, да и на <<буханке>> это нереально. Поэтому ехали не торопясь. Один раз остановились сходить освежиться. Дорога как\sdash то затягивалась. Путевые заметки стало писать невозможно\mdash очень сильная тряска. Бригада сзади тряслась на лавочках. Паша откинулся и лёг на скамейку, получая, несомненно, отличный массаж спины.

Проехали поворот на Пудрино\mdash там оказалась разбитейшая грунтовка с огромными лужами и грязищей. Даже на <<буханке>> там было не проехать. Но мы об этом знали из Google-панорам и ехали дальше\mdash на Радогощь. Пересекли мост через маленькую речушку Ундрегу, которая впадает в Лидское озеро. Речушка была очень узкой и заваленной, так что начинать сплав с неё мы, ожидаемо, не стали. Гравийная дорога при подъезде к Радогоще закончилась и началась укатанная грунтовка. Вскоре пересекли ещё один мосток через ручей, также впадающий в Лидское озеро, а потом показалась и Радогощь. 

Радогощь\mdash это одно из самых южных поселений вепсов, если не самое южное. И названия местные все сплошь из вепсского языка. К слову сказать, вепсы относятся к финно-угорским народностям, и их алфавит состоит из латинских букв. В настоящее время больше всего вепсов проживает в Ленинградской области и в Карелии. Так что, побывав в этих местах, мы смогли легонько прикоснуться к истории этого древнего народа, корни у которого, наверняка скандинавские. 

В Радогоще при советской власти, по моим предположениям, располагался крупный совхоз и посреди современной деревни стоят два обшарпанных трехэтажных панельных многоквартирных дома (!!!). Упадок русского севера просто колоссальный. Тут наверняка вовсю кипела жизнь, а малые гидроэлектростанции на реке Лидь давали электричество окрестным деревням и весям. Теперь-то электричество есть и так, без этих давно разрушенных гидроэлектростанций времен плана ГОЭЛРО, от которых остались только разрушенные плотины, а уровень жизни и обжитости края падает из года в год$\ldots$ все бегут в города\mdash оно и понятно, кому охота жить в таком всеми забытом захолустье, куда ведет лишь гравийно-грунтовая дорога? А самое главное\mdash как и где тут можно заработать на жизнь$\ldots$? Ладно, брюзжу как всегда. 

Мы повернули направо на центральном перекрёстке, проехали мимо этих самых трёхэтажных панелек, которые в контексте места своего расположения смотрелись просто сюрреалистично. Дальше начиналось самое интересное\mdash надо было найти место для стапеля. Я достал навигатор и рассудил, что лучше вставать на воду около разрушенной ГЭС, но реальность оказалась иначе\mdash грунтовая дорога становилась всё хуже и хуже. В итоге водитель отказался ехать дальше и съехал на ответвление вправо, где уже стояли две машины. Делать нечего, нужна разведка. Мы вылезли, прошлись. Действительно, к ГЭС не проехать, хоть до неё и оставалось метров 300\thinspace\nobreakdash--\thinspace 400. Пешком туда не пошли\mdash жидкая грязища и лужи\mdash а жаль. Увидели б своими глазами, что там осталось от плотины$\ldots$ но не пошли так не пошли. Должно оставаться что-то непокорённое\mdash а то если всё покорить, что дальше делать? Отвлеклись на диких, нет, просто дичайших комаров! Они атаковали нешуточно! Такое обилие насекомых очень не радовало. Надо было что-то решить с местом выгрузки и поскорее залезть в вещмешки за репеллентами!

Юрий ушёл вправо по разбитому ответвлению дороги и вскоре вернулся. Там, по его словам, до реки близко. Решено. Подключив передний мост <<буханки>>, он развернулся и задним ходом подъехал максимально близко к началу топкого разбитого участка, за что ему большое спасибо. Координаты\mdash \CoordsLidSeventeenStapel. Там мы выгрузили вещи из салона, и надели бахилы химзащиты ОЗК, а С.Ю.\mdash полукомбез от костюма химзащиты Л\sdash 1, подготовившись, таким образом, к грязище. Заброска автотранспортом пройдена! Теперь надо перетаскать вещи на берег реки. Я расплатился с водителем\mdash заброска от Чагоды до Радогощи обошлась нам в 5 тысяч. Юрий пожелал нам удачи и, немного погодя, уехал, а мы стали потихоньку таскать вещи к берегу, обильно набрызгавшись репеллентами.

Бахилы химзащиты нереально выручили\mdash грязь была мокрой, топкой и ужасной. Преодолев самое топкое место, мы пробрались сквозь заросли и вышли на берег Лиди. Это было то, чего я ждал целый год\mdash начинается этот дзен, это отчуждение, это приключение, этот отрыв!!! По\sdash о\sdash олный отрыв! Но времени размышлять о высоком не было\mdash комары дико атаковали и надо было быстро перетаскать все наши тюки и начинать собирать байдарки. 

Снаряжение перенесли довольно быстро\mdash каждый сходил к берегу и обратно к месту выгрузки всего раза по два-три. На берегу около нашего стапеля стояла на привязи чуть подтопленная деревянная самодельная лодка. Жена моментально облюбовала место в ней и сидела копалась зачем-то в телефоне. Все приходили в себя после гравийной дороги и начинали распаковку вещей и байдарок, параллельно сделав несколько снимков на фотоаппараты. 

Места на берегу было маловато и везде грязь, однако удалось всё же найти более менее чистый участок с травой, где мы и начали сборку своей трёшки вместе с женой и Олегом. С.Ю. и Паша собирались тут же, в пяти метрах, тоже на траве, чтобы зря не пачкать снаряжение на жирной влажной почве. Примерные координаты \CoordsLidSeventeenStapel.

Выяснили, что рацию, которую я отдал Олегу, он оставил в машине в Чагоде. Связи между лодками не будет$\ldots$ обидно, знаете ли, но что делать\mdash всегда без этого ходили и сейчас обойдёмся. Свою рацию, ставшую бесполезной, выключил и убрал в гермомешок.

На сборку байдарки ушло около получаса, а С.Ю. справился немного раньше. Олег на лету схватывал принцип сборки <<Таймени>> и работа спорилась. Трудности, как всегда доставила лишь натяжка фальшбортов, но тут уж, традиционно, с помощью пассатижей и какой\sdash то матери, удалось, спустя минут 10, закрыть все застёжки. Привязали чалку, спустили байдарку на воду. Стали таскать и грузить вещи. Комары при этом не дремали и атаковали в полную силу\mdash от такого я уже давно отвык! Лоб и руки начали мгновенно распухать от укусов. Репелленты не помогали практически никак, лишь на несколько минут ослабляя натиск неутомимых насекомых, жаждущих свежей кровушки$\ldots$ С.Ю. с Пашей быстрее нас погрузились в байдарку, отчалили, не дожидаясь на то команды, и скрылись за поворотом, желая оторваться поскорее от берега в надежде спастись на воде от дичайшего полчища летающих кровососов.

Мы же возились с погрузкой\mdash компактно укладывали вещмешки. Как всегда, это дело техники и спустя пару пять мне удалось более менее приблизить компоновку вещей в трюмах к идеальной, и мы стали усаживаться в байдарку. 

И тут случилось$\ldots$ непоправимое. Правда, понял это я уже потом, вечером на стоянке. В запарке, в желании догнать ушедший экипаж и тоже оторваться от комаров, усаживаясь в лодку, я на доли секунды потерял свой GPS навигатор из виду. Усевшись, я решил, что он упал на дно байдарки и откатился назад, сбоку от моего сиденья. Отчалив, мы стали держать курс на левый поворот реки, стараясь увернуться от веток, нависающих над водой с правого берега, а я шарил рукой по днищу в тщетных попытках найти прибор. Но нет. Отплывая всё дальше и дальше, я понял, что, скорее всего, выронил навигатор за борт$\ldots$ но надежда ещё теплилась\mdash надо было на стоянке основательно обшарить корму\mdash мне казалось, что он непременно должен быть там. Тщетно. На следующей стоянке навигатор найти не удалось. Так коварная Лидь недовольно встречала путешественников, дерзнувших покорить её$\ldots$

Пошли без навигации, а куда деваться. Маршрут и карту я помнил достаточно хорошо, а уж ниже Заборья я и вовсе уже ходил. Кроме того, как справедливо заметил С.Ю., с реки\sdash то всё равно никуда не денемся. Так-то оно так, но вскоре Лидь впала в озеро Глубокое, из которого надо было найти выход. Припомнив карту, относительно без проблем нашли выход. Не так\sdash то уж и нужен навигатор, хе\sdash хе! Тем более, что где\sdash то в герме был резервный комплект бумажных карт, да и GPS в телефонах никто не отменял$\ldots$ Но вскоре попали во второе озеро. Его я по карте не помнил и сначала принял уже за Лидское, но размеры явно были меньше. Оказалось, что это озеро Малое. И вот в нём-то мы изрядно попетляли в поисках выхода. Тут бы, конечно GPS пригодился. И трек похода теперь не запишешь$\ldots$ у\sdash у\sdash у\sdash у\sdash у!!! горе\sdash горюшко!!! Это ж надо было так утопить навигатор прямо в первый день! В первую же минуту сплава!!! Но что произошло\mdash то произошло, и уже ничего не изменишь. Даже если бы вернулись и стали обшаривать дно у стапеля\mdash оно очень топкое. Илистое. И мутное. Шансы найти маленький утопленный прибор стремились к нулю. Подмывало достать флягу$\ldots$ но мы шли дальше.

Вскоре вышли в Лидское озеро, встретив несколько рыбаков на лодках$\ldots$ рыба, стало быть, есть, раз местные ловят. Наши рыбаки встрепенулись и тоже хотели поскорее закинуть удочки на стоянке, о которой уже пошли некоторые толки среди экипажей\mdash народ хотел пораньше встать на стоянку после тяжелой дороги. 

По Лидскому озеру стало трудно идти. Течения нет, вода стоячая. Хорошо, что хотя бы ветра нет. Однако пошел дождь. Пришлось нацепить дождевики. В мои планы не входило геройствовать в первый же день\mdash я намеревался встать лагерем на первом острове на этом широченном озере. Но уровень воды в этом году был высок, как никогда, и остров оказался подтоплен. Пришлось идти дальше. По мере продвижения стало понятно, что характер берегов тут везде одинаков\mdash они все подтоплены и поэтому встать лагерем на озере вряд ли получится\mdash придётся искать выход в реку и попытать счастье там. 

Погодка не радовала\mdash лил дождь, с носа капало. Бригада была не в самом лучшем расположении духа после бессонной ночи, а погодные условия как будто решили испытать нас на прочность. Экипаж С.Ю. и Паши то и дело подходил к правому берегу на разведку\mdash нет ли там выхода из озера. Но тщетно. Я не разделял их энтузиазма, поскольку помнил по карте, что до выхода ещё лопатить и лопатить\mdash почти до конца озера, а мы только-только проплыли почти опустевшую деревню Пудрино, что в самом его начале. Вскоре прошли и заброшенную деревню Буржаиху за лесистым мысом. Выход был явно дальше. С.Ю. и Паша вновь предприняли попытку найти исток реки из озера и отклонились от фарватера к правому берегу, а наш экипаж сбавил ход передохнуть. Мне наскучило слепое скитание по водному простору, и я достал телефон из гермокоробки, включил GPS. Привязавшись к местности, сориентировался\mdash выход из озера должен был быть перед следующим мысом, куда мы и стали держать курс после небольшого перерыва в гребле. GPS в телефоне выключил и убрал аппарат в гермокоробку\mdash утопить ещё и телефон было бы слишком.

У всех было желание поскорее встать на стоянку, обогреться и основательно поесть\mdash уже около суток мы держимся на перекусах и из горячего только чай в термосах, который, кстати, уже закончился. Почти сразу на выходе из озера в Лидь, встретили стоянку с дымящимся костровищем. Вспомнил, что пока мы шли по озеру, мимо нас пролетела моторная лодка\mdash видимо это те рыбаки тут стояли. Бригаде стоянка не понравилась, пошли дальше. В принципе, времени было ещё мало\mdash можно плыть и плыть. Преодолели что-то вроде остатков моста или небольшой плотинки. Позже посмотрел по карте, что это остатки моста дороги к Платанихе. Попытались найти место сразу за бывшим мостом у правого берега, но неудачно. Попадались лишь рыбацкие стоянки, да все почему-то замусоренные. И ведь местные сами мусорят, не сплавщики, коих тут раз, два и обчёлся. Эх$\ldots$ культура! После нас, я клянусь, остаётся всегда только потушенный костёр, остатки напиленных дров и обожжённые банки из-под тушёнки, не более! Тот мусор, что горит, мы обязательно сжигаем. А тут$\ldots$ стеклотара, полиэтилен, обёртки какие-то, ПЭТ бутылки$\ldots$ срамота! 

Отчалили от места очередного просмотра берега. Начала чувствоваться усталость. Сколько точно мы прошли, я без навигатора сказать не мог, но по моим прикидкам выходило около 15 километров, что подтвердилось позже. На стоянку встали после заброшенной деревни Платанихи, от которой ничего не осталось. Координаты места\mdash \CoordsLidSeventeenFirst. Имелись напиленные дрова, поляна, много места для палаток и вполне приличный для такого уровня воды спуск к реке. Стали выгружаться. 

Дальше действовали как обычно\mdash разгрузили байдарки и затащили их на берег, развернули тент от дождя. Сложив всё снаряжение и провизию под тентом, отправились за дровами и костровыми палками. Лес мокрый после дождей и сухостой промок, но мы упёртые! Нам удалось найти парочку брёвен посуше. 

Пока ставили лагерь и разводили костёр\mdash комары просто заедали. Выяснилось, что Олег забыл репелленты в машине$\ldots$ осталось только наших с женой два баллончика. При таких кровососах это катастрофа! Немедленно провели санобработку верхней одежды всей команды. Правда, спасало это ненадолго. Больше помогал дым костра$\ldots$ комары как будто с цепи сорвались\mdash полчища некормленых комаров атаковали нас непрерывно!

Над костром уже приветливо пошумливал котелок кипятка на макароны с тушёнкой, когда Паша вдруг так кстати предложил, поглаживая бороду: 

\diagdash А где, собственно, канистра?$\ldots$

\diagdash У-у-у-у-у!~Ы-ы-ы-ы-ы!~Да-а-а-а-а!\mdash завопила походная братия, ибо, возможно, только горючее сейчас было способно отвлечь нас от полчищ комаров.
 
\diagdash Иди сюда, родимая!\mdash Олег немедленно извлёк из пакета пятилитровую яблочного снадобья. Открыл, вдохнул ароматы целительного нектара и принялся наполнять наши кубки$\ldots$

Это то, что нам сейчас нужно! Надо расслабиться и снять лишнее напряжение в мышцах, которые пока ещё не слишком понимают, что весло\mdash это дней на десять, а не просто так! Переход по озеру тяжело нам дался. Отсутствие постоянной спутниковой навигации и плохая погода тоже радости не добавляли, поэтому яблочная уходила исключительно хорошо.

\diagdash Хы-ы-ы-ы-ы! Ключница делала!\mdash зелье было превосходно.

\diagdash А то! Не в 1 раз беру, проверенное качество!\mdash со знанием дела изрёк наш виночерпий.

\diagdash Что-то и комары подотстали? Или нет, а?

\diagdash Чёт пока нет!!! Ну-ка, плесни-ка!

Паша, тем временем, уже забросил удочку и выловил ма\sdash а\sdash аленькую рыбёшку. Клевало не очень, поэтому он оставил удочку заброшенной и тоже присоединился к нашей яблочной вакханалии, достав припасённый лимон из гермы с продуктами. Временами, когда колокольчики, привязанные к удочке, начинали позвякивать, Паша стремглав срывался к берегу в напрасной надежде выловить достойный экземпляр$\ldots$ Дождик временами то начинался, то прекращался, и макароны с тушёнкой мы вкушали сидя под тентом на деревянных чурбачках, напиленных кем-то тут бензопилой. 

Девизом этого похода можно назвать фразу: <<Хрен знает, в какой-то герме!>>:

\diagdash Са-а-ань, а где то\sdash то? (Имеется в виду что то из еды или из снаряжения). 

\diagdash Хрен знает, в какой\sdash то герме!\mdash неизменно был мой ответ. 

Я что-то настолько расслабился и пустил всю кухню на самотёк, видя, как народ и так нормально справляется со всем, что справедливо рассудил\mdash рано или поздно сами освоятся. Но этого не происходило\mdash народ тоже расслабился и снова и снова спрашивал на каждой стоянке: 

\diagdash Са\sdash а\sdash а\sdash а\sdash ань, имей совесть, где то\sdash то? 

\diagdash Хрен знает, в какой\sdash то герме!\mdash и взрыв моего подогретого хохота$\ldots$ хотя я был явно не прав. Если гермы с продуктами ещё отличались по цвету, то всё остальное лежало в четырёх одинаковых армейских вещмешках, и понять где что было невозможно\mdash надо обязательно залезать внутрь$\ldots$ но, тем не менее, выглядело это всё забавно. Пришло решение впредь повязывать разноцветные ленточки на вещмешки для их опознавания. Впрочем, спустя несколько дней народ освоился, и, не стесняясь, преспокойно рылся в гермах и мешках в поисках нужных вещей, чего я и добивался.

Олег расчехлил гитару. О\sdash о\sdash о, это было нечто! <<Таймень>>\mdash это, пожалуй, единственная байдарка на постсоветском пространстве, а может быть и в мире, в которую без проблем сверху можно положить гитару. А уж если <<Таймень\sdash 3>>, как у нас сейчас, то это вообще крейсер! Чтобы был понятен масштаб\mdash в нашей байдарке ехала вся провизия на пятерых (!) человек на 10 (десять!!!) дней, включая тушёнку, а также групповое снаряжение\mdash пила, топоры, котелки\mdash пехотный и десантные, газовая горелка для разведения огня, складная сапёрная лопатка, тент от дождя, плёнка полиэтиленовая для бани, складные стульчики. И, само собой, личные вещи экипажа, а также палатки. И, при всём при этом, сверху в грузовом отсеке ещё оставалось место для гитары$\ldots$

Кроссовки Олега промокли и он повесил их сушиться у костра, а поскольку в шлепанцах находится было совершенно невозможно из\sdash за комаров, то ноги он поставил в полиэтиленовый пакет, замотал верхушку поплотнее, усевшись на походном стульчике, и взял в руки свою видавшую виды гитару:

\vspace{0.1cm}
\noindent\textit{%
	\hspace*{3.5cm}Вот! Новый поворот!\\
	\hspace*{3.5cm}И мотор\sdash р\sdash р  р\sdash р\sdash ревё\sdash ё\sdash ёт,\\
    \hspace*{3.5cm}Что он нам несёт\mdash\\
	\hspace*{3.5cm}Пропасть или взлёт$\ldots$
}
\vspace{0.1cm}

%\diagdash Вот!!! Новый поворот! И мотор\sdash р\sdash р  р\sdash р\sdash ревё\sdash ё\sdash ёт$\ldots$
\noindent\mdash понеслось над поворотами Лиди. Сидя у костра и отбиваясь от комаров, полистывая сборник песен, Олег выбирал что\sdash нибудь из него или из памяти, и исполнял, а мы подпевали, как умели. С уменьшением содержимого канистры песнопения наши становились всё веселее и веселее$\ldots$ гитара в походе\mdash это однозначно не просто огромный плюс, а то, что однозначно должно быть. Вместе~с~гитаристом, разумеется. Сразу атмосфера становится другой и даже на комаров сразу наплевать$\ldots$ ну, почти. Так мы сидели на берегу этой тёмноводной, красивой и местами коварной речушки и самозабвенно пели любимые песни$\ldots$

Солнце уже давно зашло, красивого заката не было\mdash мы стояли в окружении высокого леса и густого подлеска, да и облачно. Настали сумерки. Яблочная уходила на ура, притупляя комариные укусы и бессонную ночь за рулём. Меня просто рубила усталость и мы с женой залегли спать в этот день рано, около 10 часов вечера, потому как очень хотелось нормально выспаться, а бригада утихомирилась чуть позже.

\begin{center}
	\psvectorian[scale=0.4]{88} % Красивый вензелёк :)
\end{center}
