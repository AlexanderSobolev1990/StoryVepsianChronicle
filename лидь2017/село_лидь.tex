\chapter{Лидь икс-одовая} 
%\chapter{Село Лидь и бросок до Заборья} 
%\corner{64}
\vepsianrose

С утра сворачивались не торопясь. Олег пожарил вчерашнюю рыбу на сковородке\mdash вышел прелестный завтрак. Параллельно, как всегда, наварили походной овсянки и изволили позавтракать уже наверно часу в~десятом. Лениво собирались, тщательно осматривая поляну на предмет забытых вещей\mdash примятая трава и мох моментально скрывали в себе что\sdash то невзначай уроненное.

Как только отчалили, началось село по обоим берегам. Но народу\mdash никого не видно, никто не купается, белье не полощет, и всё в таком духе. Ни души, а домов много. Причалили на правом берегу недалеко от моста, что посередине села. Рядом имелся деревянный трамплин, видимо для прыжков местной ребятни в воду. Подойдя к берегу, вылезали долго и аккуратно\mdash глубина была существенной, а дно топким с вязкой грязью и илом. Все перепачкали ноги и чуть не потеряли шлёпанцы. Пришлось присесть на трамплинчик и с него ополаскиваться. Когда с этим было покончено, Паша остался караулить байдарки, а остальные пошли вести разведку местности в поисках магазина. С.Ю. нашел местную жительницу, которая косила траву, и спросил у неё про магазин. Так мы узнали, что магазина в селе как такового нет, но лавка, конечно же, имеется. Однако сейчас она, скорее всего, закрыта, потому что хозяева уехали за товаром. Расспросили также про церковь и узнали, что её ремонтирует энтузиаст из Заборья$\ldots$
 
Пока стояли и получали весьма ценную развединформацию, мимо нас проехала <<буханка>> защитного цвета\mdash это оказались хозяева лавки. Мы~скорее пошли к сараю, указанному местной жительницей. Сарай оказался импровизированным магазином. Оказалось, что репеллентов у них в продаже нет! Катастрофа!!! Пока мы немного помогали хозяевам лавки разгрузить привезенный товар, они сходили в дом и нашли\sdash таки для нас один баллончик <<Рафтамид>>\sdash а. Это было спасением! Он~оказался действенным репеллентом! Стало гораздо веселее\mdash отпугивающие свойства сохранялись надолго\mdash комары и~мошка держались на почтительном расстоянии. 

Пока стояли ждали около лавки, стали подтягиваться местные жители, также отследившие, видимо, приезд <<буханки>>. У одних была собака и её укусил клещ в нос. Мы~были неприятно обескуражены тем фактом, что клещи таки есть в этих краях и даже не <<где\sdash то там далеко>>, а прямо вот тут рядом!

Ободрённые долгожданным приобретением репеллента, мы пошли обратно через луг около церкви, заправив штаны в носки, и постоянно осматривая друг друга во время движения, а также по прибытию на берег\mdash очень не хотелось получить укус такого неприятнейшего насекомого, как клещ. На берегу немедленно устроили санобработку нашей верхней одежды из прикупленного баллончика:

\diagdash Готов?!?!

\diagdash Ага!!!

\diagdash Команда ГАЗЫ!!!\mdash жена неистово поливала всех из~баллончика, средство имело стойкий запах ванили.

Погода стояла хорошая\mdash солнечная. По небу плыли частые кучевые облачка. Жизнь, кажется, начала налаживаться\mdash средство от комаров купили, погода нормализовалась! Впервые за мои походы я так прогадал с погодой$\ldots$ но что делать, отпуска были уже получены, всё было приготовлено$\ldots$ в конце концов, у природы нет плохой погоды, просто люди могут оказаться к ней не вполне готовыми. Сейчас же мы несколько привыкли, что за день дождь может внезапно пойти раз пять и так же внезапно прекратиться, а может и зарядить на полдня. 

Погрузились и отчалили, аккуратно усаживаясь в~байдарки с того деревянного трамплинчика. Медленно прошли село дальше\mdash богатых старых домов практически не заметили\mdash может их уж и не осталось, а может просто где\sdash то в глубине села. Церковь на берегу реки относится, вероятно, к середине XIX века. Состояние строения плачевное, требуется капитальная реконструкция для его сохранения хотя бы как памятника архитектуры. Но,~кажется, все моменты упущены\mdash косметические ремонты больше без толку, а делать что\sdash то капитальное навряд ли кто будет. Так что осталось ей лет 5\thinspace\nobreakdash--\thinspace 10 как максимум$\ldots$ На наш век хватило увидеть остатки этого северного деревянного зодчества. Купол у церкви обвалился, колонны сбоку\mdash доживают свой век, их каменный фундамент тоже нуждается в капитальном ремонте. И~всё это никому не нужно, потому что потому. Потому что вся реальная, живая жизнь ушла куда\sdash то бесконечно вперёд$\ldots$ А что же осталось тут? Эта разваленная церковь? Похоже, что так. Так и живут тут\mdash пользуются мобильной связью, косят траву бензиновой газонокосилкой, а церковь стоит как некий символ давно ушедшей эпохи, зачем\sdash то оставленный потомкам на растерзание. Но находятся энтузиасты, которые где\sdash то достают силы и средства всё это пытаться восстанавливать, ремонтировать$\ldots$ И не ругаю я их, не осуждаю\mdash риторически спрашиваю я\mdash доколе будем мы цепляться за прошлое или делать вид, что цепляемся, когда в реальности все давно от него отошли? Где же исторический материализм и всё вот это вот? Где Эра Великого Кольца \cite{ТуманностьАндромеды}? Где галактические экспедиции к далёким мирам?

Впрочем, я отвлёкся\mdash посещение старых заброшенных мест всегда навевает какую\sdash то тоску глубоко в~закоулках души$\ldots$ Оставляя за спинами это старое село, превратившееся в деревню, мы проносимся дальше на своих байдарках\mdash нас ждет река и километраж, а также странные думы по поводу бренности бытия. По плану сегодня грести около 25 километров, если, конечно, не встретим никаких препятствий, как вчера. Но, судя по карте, которую я тихонько рассматриваю в командирском планшете, их не предвидится, поэтому рвёмся вперёд!

Олег просит посмотреть, что там такое чешется на~спине. Оказывается, это впившийся клещ. Мда\sdash а\sdash а. Я~в~замешательстве, а~Олег, напротив, не выражает никаких опасений и~говорит, что, мол, вечером вытащим, на~стоянке. Район\sdash то, говорит, не энцефалитный, да и кусали его много раз на рыбалке, так что он человек привычный. У~меня в голове такое спокойствие слабо укладывается, ибо район всё\sdash таки энцефалитный, но я отчего\sdash то соглашаюсь потянуть с процедурой до вечера, хотя, по\sdash хорошему, надо было вытаскивать немедленно$\ldots$

Проходим несколько урочищ с бывшими переправами\mdash остатки мостов, опор. Но всё это давно разрушенное и~никакой опасности для нас не представляет. Лишь рельеф местности не перестает удивлять\mdash попадаются холмы, обрывы, резкие перепады высот обоих берегов. Встречается много хороших мест для рыбалки\mdash <<рыбные места>>, как говорит Олег. Они, Олег с Пашей, уже все извертелись, сидя на носах байдарок, в поисках <<хорошего рыбного места>>, да так, чтобы рядом была стоянка. В действительности никогда не получается так, чтобы и место для палаток хорошее, и~подход к воде не крутой, и~песчаный пляжик, и~дрова чтобы, и~сосны вокруг, а~ещё чтобы рыба ловилась$\ldots$ Если~хотя бы несколько пунктов из этого списка выполняются\mdash мы думаем оставаться. А пока гребём дальше\mdash мест для стоянок категорически нет.

Одно хорошо\mdash погодка радует, светит яркое приветливое солнышко! Дождь сегодня пока не собирается, и мы идём и жадно ищем глазами стоянку. Километров за 10 до Заборья начинаются сосновые боры. Но всё почему\sdash то на высоких берегах, да таких, что забраться крайне затруднительно, а уж байдарку затащить\mdash вообще дело практически невозможное. Тут~в~полной мере осознаешь почему произносится <<З\'{а}борье>>, а~не~<<Заб\'{о}рье>>\mdash потому что не от~слова <<заб\'{о}р>>, а~<<за б\'{о}ром>>, то~есть <<за~сосновым бором>>$\ldots$ сразу вспоминается насчёт этого <<там за~туманами$\ldots$>> и~гитарные аккорды$\ldots$ Так~и~плывём, периодически вылезая на~берега этого бора для~разведки.

По высокому левому берегу идёт наезженная грунтовая дорога и вставать в таком соседстве совсем не хочется. А~правый берег\mdash ещё выше левого. Ничего не попишешь, идём дальше. Погода не портится, запас времени ещё есть. Около 17 часов мы, порядком притомлённые переходом и в надежде найти хорошее место, замечаем на левом берегу остатки стоянки. Командую немедленно причаливать! Взбираемся на обрыв\mdash место неоднозначное$\ldots$ С одной стороны, помним про дорогу по левому берегу\mdash значит тут тоже она где\sdash то рядом, с другой стороны места маловато\mdash плоская поверхность, пригодная для постановки палаток, вытянута вдоль берега, а дальше от обрыва идёт какая\sdash то канава. Надо принимать решение$\ldots$ Но сначала экипажи расходятся на разведку. Дороги близко не находим. Решаем, что места для палаток нам хватит. Да и столик со скамейками подкупает. Подъезда или подхода к стоянке нет\mdash всё вокруг заросло высокой травой. Получается, что это как будто островок, а сбоку идёт канава, протока, которую по весне наверняка затапливает. Решаю оставаться на ночёвку, тем более начинается дождь, и мы устало идём разгружаться$\ldots$ сегодня мы прошли около 24 километров как я примерно прикинул.

К 18 часам лагерь окончательно развёрнут\mdash над скамейками и столом натянули тент, рядом поставили палатки, притащили дров, развели костёр$\ldots$ усталость почему\sdash то накатывает волной. А ещё дождь идёт мелкий. Но не время расслабляться! У нас медицинская операция по~плану\mdash надо вытащить клеща из Олега, поскольку спина\mdash не самое удачное место, чтобы делать это самостоятельно. 

Проведя консилиум, склонившись над спиной Олега, мы решаем, что вытаскивать клеща буду я:

\diagdash Паш, давай ты?

\diagdash Не, товарищ Адмирал, не могу возложить на себя такую ответственность!\mdash отзывается тот.

\diagdash Сергей Юрьевич?

\diagdash Шутишь? Я не вижу так мелко,\mdash говорит он.

%Хоть мне никогда и не доводилось этого делать, 
Жена смотрит на меня красноречивым взглядом\mdash не возьмётся. Что ж, я примерно представляю, как всё должно быть. А там уже по ходу дела посмотрим, главное не торопиться. Достаю пинцет из швейцарского ножа, спирт, ватный тампон. Проводим дезинфекцию, и я принимаюсь выкручивать мерзкое насекомое. А клещ, походу, того, помер. Не шевелится совсем. Это облегчает задачу. Аккуратно выкручиваю его, стараясь не раздавить и не оставить ничего в ране. Но одна ножка или клешня или что там у этого гада есть, всё же остаётся впившейся. 

\diagdash Потерпи, я сделаю надрез,\mdash говорю Олегу.\mdash Надо вытащить остатки.

Тот хладнокровно терпит:

\diagdash Давай, я нормально.
 
Аккуратно делаю небольшой надрез острым ножом и вытаскиваю пинцетом небольшой чёрный кусочек. Уф!!! Ещё раз промываю место спиртом. Готово! Первая в~моей жизни операция по удалению клеща завершена$\ldots$ Сверху накладываю небольшой тампон и заклеиваю рану пластырем. Теперь можно и ужином заняться$\ldots$

Вдалеке слышна железная дорога\mdash значит мы совсем немного не дошли до Заборья. Олег расчехляет гитару, С.Ю. очень кстати вспоминает про сало и горючее, Паша уже возится с удочками, а мы с женой занимаемся ужином на команду. Вечер выдаётся шикарным как по~заказу\mdash словно вознаграждение за наше терпение\mdash оранжевое солнышко закатными лучами нежно освещает стоянку. Сосны становятся ярко янтарными, лес залит закатным светом$\ldots$ и подо всем этим великолепием стоит наш тент и славно гуляет сплавная братия под звон металлических кружек и переливы гитары$\ldots$ атмосфера этого пробирает до самой глубины естества. Атмосфера спокойствия, отчуждённости, свободы\mdash ничто не заботит, не тяготит, не давлеет. Звуки природы ласкают слух, за сосновыми борами стучат поезда, добавляя романтики, а~гитарные аккорды прочищают душу$\ldots$ 

\vspace{0.3cm}
\noindent\textit{%
	\hspace*{1.9cm}Там, за туманами, вечными пьяными,\\
	\hspace*{1.9cm}Там за туманами берег наш родной$\ldots$ 
}
\vspace{0.3cm}

До моста в Заборье отсюда всего 3 километра по~реке, узнаю я, включив GPS в телефоне. Координаты\mdash \CoordsLidSeventeenNearZaborie. Завтра я снова окажусь в Заборье спустя два года\mdash когда новоиспечённый Адмирал с~двумя зелёными матросами высыпался из поезда на низкую платформу посреди этого всеми, казалось, забытого места и шагнул в неизвестность, преодолевая какие\sdash то страхи, какую\sdash то неуверенность, будучи в твёрдом намерении совершить дерзкое путешествие по этой реке, покорить водную гладь на стыке двух областей, на стыке двух периодов своей жизни\mdash До и После Первого Самостоятельного и по\sdash настоящему Дикого Сплава; приблизить ту детскую мечту\mdash сходить в поход, чтобы всё было так, как изложено в томе номер 1 Советской Детской Энциклопедии$\ldots$ 

От нахлынувших воспоминаний становится безумно хорошо, а яблочная помогает достать из глубин сознания буквально все ощущения, все картинки и даже запахи травы и дёгтя от старых шпал, когда только выпорхнули из поезда, принёсшего нас в 2015 году на станцию Заборье, в эти заповедные и почти нетронутые человеком места$\ldots$ Олег играл и играл на гитаре, Паша разливал и разливал, а~вокруг нас наступала тихая северная ночь. Настроение у~всех было преотличным\mdash мы много пели, отводя душу в~этом самозабвенном занятии. Даже С.Ю. подключился к~нам и~спел из~Высоцкого:%$\ldots$

\vspace{0.1cm}
\noindent\textit{%
	\hspace*{2.9cm}Парня в горы тяни\mdash рискни,\\
	\hspace*{2.9cm}Не бросай одного его$\ldots$ 
}
\vspace{0.1cm}

Спать залегли в этот день все очень поздно\mdash уже в~кромешной темноте. Ничего, это нужно было нам. Отдохнуть душой и стряхнуть с себя лишнее. А сейчас северная природа и чистейший влажный полночный воздух, полный запахов трав и хвои, успокоят нас, усыпят и придадут сил на~завтрашний день, на~дальнейшую жизнь$\ldots$ 

В палатке холодно. У нас с женой простенькая, но~проверенная временем двускатная палатка. И хоть она номинально трёхместная, нам вдвоём с вещами тесновато. Чтобы хоть немного это исправить, я растягиваю также и~бока внутренней палатки, но этого всё равно недостаточно\mdash тесновато на уровне плеч и головы, если сидеть внутри. Из года в год всё хочу приобрести нормальную двускатную палатку, побольше размером$\ldots$ но~их ассортимент тает с~каждым годом\mdash народ предпочитает новомодные дуговые$\ldots$ а мне милее старого образца\mdash домиком, типа <<памирки>>. От этой формы веет тем же, чем веет от стакана в подстаканнике в поездах, от стола с зелёным сукном, от старых ламп с зелёным абажуром в библиотеке$\ldots$ да$\ldots$ но пока мечты о лучшей палатке остаются мечтами, пользуемся тем, что есть. На~ночь накрываю задний торец палатки плёнкой, чтобы там не~гулял воздух и не дуло в голову. И каждый год с началом сезона снова из безумной искры разгорается мысль, что, быть может, в этом году я обзаведусь хорошей палаткой, с~разделкой под печку$\ldots$ и тогда в горы? На Алтай? Мечты$\ldots$

Хотел вести судовой журнал в этом году. Но~<<хотели как лучше, а получилось как всегда>>\mdash только в~первый день одна запись и всё. Как\sdash то не~ложится на~душу ведение журнала. А все впечатления с помощью фотографий не~передать. Поэтому в этом году решаю вести видеозарисовки. То есть снимаю видео, в которых расказываю, что делали сегодня, сколько прошли, излагаю свои впечатления и т.д. Получается сносно. Большей частью перед сном, уже в палатке, собираюсь с мыслями и~записываю материал до 10 минут\mdash больше не позволяет камера. Вот и сейчас, улегшись на матрасе, подвожу итоги трёх дней сплава, а потом мгновенно погружаюсь в глубокий богатырский сон.

\begin{center}
	\psvectorian[scale=0.4]{88} % Красивый вензелёк :)
\end{center}
