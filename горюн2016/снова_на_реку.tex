%\chapter{Снова на реку!} 
\chapter{Тихвинская водная система} 
%\corner{64}
\vspace{-9mm}
\vepsianrose

\setlength{\epigraphwidth}{0.75\textwidth}

\epigraph{%	
	А мы привратнику кричим: <<Откройте двери,\\
	На нас там занято!>> Но нам швейцар не верит.\\
	Открыл и пузом отодвинул он Алёшу:\\
	<<А ты, мол, вовсе, ты уже хороший!>>}
	{
	\begin{flushright}
		\small{Александр~Дольский,\\<<Воскресный выходной>>}
	\end{flushright}
	}

Странное явление или состояние. Сродни какой\sdash то мании. По~моему убеждению\mdash это неизлечимое. Можно, конечно, на какой\sdash то период заглушить в себе тягу к~водным путешествиям, но стоит только не удержаться\mdash взять и открыть топографическую карту$\ldots$~всё, пропал для~окружающего мира. Снова из глубин сознания достаются воспоминания о пройденных маршрутах, дым костра, крутые берега с сосновыми борами, зов чего\sdash то первобытного манит на реку, как Маугли в джунгли. 

Хочется выйти на улицу, оглянуться, чтобы никто не видел, и бежать, бежать, бежать, убежать от всей обыденности окружающей действительности, от всего привычного убежать\mdash на реку, в бескрайние леса Вологодчины, где только ты, река и байдарка. Ощутить себя шкипером, командором, Адмиралом флотилии, представить, что чувствовали Френсис Дрейк и Америго Веспуччи, если бы они ходили по рекам$\ldots$ 

Так и случилось в очередной раз в 2016 году. Ещё~с~прошлого года меня гложил тот факт, что я не~прошёл маршрут по Чагодоще до конца. А ведь там, в конце маршрута, должны были быть пороги, аж 3~штуки. Пороги я ни разу ещё не проходил в своём сплавном опыте. И~не~важно, что в найденных отчётах об этих порогах ни~слова, на~старой\sdash то карте они есть! А~очарование старых советских топокарт манит меня необъяснимо. Решено было всё\sdash таки пройти их, эти 3 порога\mdash Горынь, Буг и Вяльскую Гряду. Стало быть, снова собираемся на Чагодощу!

Опять же, в 2015 году, Главный Идеолог сплава\mdash Сергей Юрьевич (далее по тексту\mdash С.Ю.), мой начальник, наш 4\sdash й член экипажа, не смог пойти. Хотя именно он и стал первопричиной нового слова в моих водных путешествиях, рассказав в подробностях о байдарочных сплавах. До~этого я ходил только на катамаранах в~Башкирии. Именно по~совету С.Ю. была приобретена старая байдарка <<Таймень>>, двускатная палатка, изготовлены костровые крючки и~приспособления, спецчехол для топора, а~также разные другие полезные мелочи. Его рассказы внесли, безусловно, неоценимый вклад в само становление идеи <<дикого>> байдарочного сплава. Сколько раз мы обсуждали реки, заброску и выброску, снаряжение, его и мои прошлые походы\mdash просто не счесть. Не торопясь, во время перерывов, попивая чай из термосов, устроившись в креслах поудобнее, мы обговаривали маршруты, способы заброски, и вообще всё то, что касается темы сплавов.

Итак, на этот раз всё должно было сложиться. Маршрут, в~общих чертах обговоренный ещё год назад, Два Капитана\mdash С.Ю.~и~я, а~также три порога. Заноза странствий, засевшая в мозгу, начала давать о себе знать уже поздней осенью, а весной 2016 года стали детально прорабатывать наше будущее путешествие. Хотели сразу начать с реки Чагоды\mdash от села Усадище. Но добраться туда не так легко, как на Лидь\mdash железной дороги рядом нет. А~забрасываться хотели, естественно, поездом\mdash как же родимый плацкарт без нас? 

Попутно читали отчеты о Тихвинской водной системе, проходившей на стыке Ленинградской и Вологодской областей по рекам Соминка, Горюн, Чагода, Чагодоща и~дальше по Мологе. Маршрут этот сейчас представляется достаточно сложным по причине падения уровня воды, огромного количества разрушенных шлюзов и, как следствие, обносов. 

Мне хотелось не просто мысленно поставить себе галочку\mdash <<Я прошел по древней Тихвинской водной системе>>, но и посмотреть, что осталось от шлюзов. Незаметно для себя, мы с С.Ю. свыклись с мыслью, что стартовать надо от берега Вожанского озера и идти дальше по Горюну, минуя 3 шлюза этой самой Тихвинской водной системы, потом впасть в Чагоду, далее в Чагодощу, а~закончить как мы и хотели год назад при нашей подготовке к первому совместному сплаву\mdash в селе Лентьево, под мостом трассы А\sdash 114. И если в 2015\sdash м С.Ю. не смог пойти с нами, то в этом году мы надеялись, что всё сложится как надо.

\newpage
Тихвинская система, запущенная в 1811 году, к 1916 году насчитывала 62 шлюза. Шестьдесят. Два. Шлюза. Что\sdash то невообразимое! Конечно, не все из них были чем\sdash то выдающимся в инженерном плане, поскольку поначалу все плотины и шлюзы были деревянными, но надо отдать должное нашим предкам\mdash пока не была построена Николаевская железная дорога, именно Тихвинская водная система была кратчайшем путём товарного сообщения Санкт\sdash Петербурга с Мологой и Рыбинском. Более того, двусторонним путём. Потому что, скажем, Вышневолоцкая водная система этого не обеспечивала, поскольку пороги на Мсте против течения было не пройти. Сейчас, конечно, это всё уже не~актуально\mdash есть и железные дороги, есть автотрассы, авиасообщение наконец. Но тогда, в~XIX~веке, Тихвинский водный путь можно было считать настоящим прорывом. 

Найденные нами фотокарточки XIX~века наглядно показывали масштаб и размах, с которым эта водная система была построена. Шутка ли, создать не просто такое количество шлюзов, но и обеспечить их всей сопутствующей инфраструктурой. Черно\sdash белые с коричневатым оттенком ретро фотографии позволяли очень хорошо прочувствовать атмосферу этого строительства, перенестись во времени и мысленно там побывать$\ldots$ оставалось только всё спланировать и оказаться в тех краях лично, чтобы вдохнуть полной грудью воздух родных просторов и ощутить себя немного причастным к истории.

Читая о Тихвинской водной системе, находили несколько отчётов и фото с весенних сплавов. Даже по~весенней высокой воде некоторые шлюзы выше села Сомино требовали обноса. Обноситься в первые же дни мне очень не хотелось, поэтому\sdash то мы и решили компромиссно начать с~Вожанского озера, чтобы избежать многочисленных обносов. Может быть, конечно, и зря\mdash интересно было бы пройти маршрут, скажем, от Ефимовского. И тогда можно было бы заброситься поездом$\ldots$~но это еще плюс 60\thinspace\nobreakdash--\thinspace 70 километров к маршруту. У Ефимовского река должна быть \'{у}же и~камернее, но и мельче при этом. Причём узнать уровень воды мы заранее никак не могли\mdash все водомерные посты канули в лету, а играть в такую своеобразную рулетку не было желания. Поэтому окончательно решили идти от~Вожанского озера. 

\renewcommand*{\thefootnote}{\arabic{footnote}}
На этот раз я заполучил отличный туристический GPS\footnote[1]{Global Positioning System --- спутниковая система навигации.} навигатор с цветным дисплеем и полноценными картами. В~прошлом сплаве у меня был довольно простенький навигатор без карты, а только с маршрутными точками, соединенными прямыми отрезками. Новый же аппарат был замечателен\mdash цветной дисплей, питание от двух батарей АА, поддержка не только системы GPS, но и отечественной ГЛОНАСС\footnote[2]{ГЛОбальная НАвигационная Спутниковая Система\mdash отечественная спутниковая система навигации.}. Я~умудрился записать в этот навигатор свои собственные многослойные карты, подготовленные на основе старых топокарт и спутниковых снимков. Прибор, что ни~говори, был замечателен. С.Ю., правда, подшучивал\mdash мол, зачем он и~куда мы денемся с реки в сторону? Но я, понятно, прибор всё равно взял с собой вместе с запасом батареек, предварительно записав в~него наш маршрут.

Таким образом, с маршрутом было всё понятно\mdash идём покорять Тихвинскую водную систему! Оставалось собрать команду$\ldots$

\begin{center}
	\psvectorian[scale=0.4]{88} % Красивый вензелёк :)
\end{center}
