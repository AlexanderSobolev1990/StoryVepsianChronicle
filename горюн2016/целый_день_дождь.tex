%\chapter{Целый день дождь. 08.08.16} 
%\chapter{Пороги и Салынь. 08.08.16} 
\chapter{Пороги и Салынь} 
%\corner{64}
\vepsianrose

Проснулся. Жив. Уже хорошо. Выполз из палатки и~стал разжигать костёр. Дров практически не осталось\mdash спалили почти все запасы вчера. Утро команда провела в~опаске, димины рогатины лежали рядом. Медведь, однако, и не думал появляться снова, и мы более\sdash менее спокойно приготовили завтрак и быстро свернули лагерь. 

Вышли на воду не поздно. Погода испортилась. Нас~нагнали тучи, и стал накрапывать дождик. По плану у меня оставалось 3 ходовых дня, из них последний чисто символический\mdash 5 километров. Так что, поход, кажется, сокращается. Берега после Бывальцево стали неприветливыми, как и писали в отчетах, и заросшими сплошной буйной смешанной растительностью. Характер реки резко изменился\mdash песчаные пляжи и повороты пропали\mdash русло спрямилось и уширилось. Течение, правда, почему\sdash то усилилось. Судя по GPS, мы вплотную подобрались к~порогу~Горынь.

Дождь зарядил пуще прежнего. Мы надели дождевики и укутали ноги полиэтиленовой плёнкой от брызг с весла. Налетел встречный ветер. Похолодало. Руке, опущенной в~воду, было теплее, чем в воздухе! Стали активнее грести, чтобы не мёрзнуть. Вскоре запищал GPS\mdash подошли к~первому порогу$\ldots$ а его нет. Изгиб русла как на карте, а~порога нет\mdash слишком высок уровень воды! Но мы всё равно причалили передохнуть. Единственное, что хоть как\sdash то намекало на порог\mdash у противоположного берега виднелся большой~камень$\ldots$ 

В GPS очень некстати сели батарейки, пришлось аккуратно заменять, прикрыв прибор своим телом от лившихся с небес потоков воды. Потом прошёлся по местности\mdash сплошной уклон и~холмы. Ровного места нет. Дров нет. Лес голый. Место, видимо пользуется популярностью у рыбаков\mdash на правом берегу заметили что\sdash то вроде засидки. Прикинул, что лагерем вставать не только рано, но и пока что негде\mdash места все попадались не очень, да и тут, у <<порога>>, не лучше. Поплыли дальше. 

У деревни Мережа прошли точку с координатами порога Буг. Тоже без каких либо проблем по высокой воде. Ничего не~заметили вообще. Ну островок в русле. Всё. Разочарование. Я не ожидал, конечно, какого\sdash то категорийного препятствия с белой бурлящей водой и~острыми камнями, но всё таки хоть что\sdash то в этом месте должно было быть, иначе зачем на топокарте обозначать это место как порог? Непонятно! В устье Лиди и то был перекат посущественнее, ну да ладно. 

Перед деревней Мережа прошли интересное место на~реке\mdash в русле было два плотно заросших лесом острова, стоявшие не друг за другом последовательно, а как бы параллельно в русле. Река тут протекала по трём протокам между этими островами. Течение в протоках было не просто приличным, а сильным. Мы с Геннадичем пошли в левой протоке и не прогадали\mdash течение вынесло нас вперёд перед теми, кто пошёл посередине.

Мы шли дальше и вскоре на правом берегу показались домики деревни Мережа. Причаливать не~было никакого желания, тем более, что дождь явно не~собирался заканчиваться. За левым поворотом реки после Мережи должен был быть последний порог\mdash Вяльская Гряда. Но,~вы уже догадались, его тоже не оказалось там, как и~двух предыдущих. Не было ни малейшего намека на~какое\sdash то водное препятствие. Я был, конечно, раздосадован. Уж~подобие быстрины или переката должно было быть, думал~я. С~другой стороны, если бы пороги действительно были на своих местах, да причем серьёзные, возможность получить повреждения байдарок в такой чертовски хмурый, пасмурный и дождливый день с неприветливыми берегами в~мои планы явно не входило.

Дождь лил и лил, а силы наши таяли. В расслабленном режиме плыть не получалось, потому что замерзали. Надо~было искать стоянку среди заросших берегов. Пройдя урочище Гвёрстно на левом берегу, заметили на правом берегу стоянку с одинокой байдаркой. Тоже уже хотелось где\sdash нибудь причалить. Я смотрел на GPS и не верил своим глазам\mdash мы перевыполняем норму километража, а времени до вечера ещё очень далеко! Вот что дождь и течение делают! 

В устье реки Любосивец попробовали найти место для стоянки, но берега показались нам непригодными. Что~ж,~последняя надежда\mdash конечная точка на сегодня\mdash урочище Салынь. Добрались да него довольно быстро, просто в каком\sdash то едином порыве, налегая на вёсла. Место~вокруг было необычайной красоты! Река делает изгиб влево, огибая два острова подряд, и снова круто берёт вправо. На мысу сосновый лес, маленькая бухточка как раз для байдарок, плавный подъём, наверху ровная площадка и~дальше холмистый рельеф с~сосновым редколесьем. 

Я находил информацию, что раньше тут располагался склад сплавного леса. Мы же обнаруживаем кучу необрезных досок, стол и скамейки. А также кучи мусора повсюду$\ldots$ это удручает. Чем ближе мы к цивилизации, тем более загаженными оказываются обжитые стоянки. Люди, зачем вы это делаете? Остался мусор\mdash сожги его в~костре! Железные банки от тушёнки, обожжённые в~костре, достаточно быстро ржавеют и разлагаются! Бумага, пластик\mdash всё это сгорает. Не хочешь жечь\mdash увези с~собой мусор!!! Не можешь увезти\mdash закопай, в конце концов, поглубже, сделай милость! Но поделать в данной ситуации ничего было нельзя и мы, вооружившись сапёрной лопатой и перчатками, приводим стоянку в более\sdash менее нормальный и пристойный вид$\ldots$ 

Дождик постепенно прекращается, но вокруг всё мокрое! Расходимся по лесу в поисках сухих дров. Местность, где мы причалили, очень впечатляет\mdash холмы и впадины, резкие перепады высот. Под ногами мягкий мох белёсого цвета, лес в основном сосновый. Одним словом, красота! Вскоре каждый притаскивает по одной\sdash двум оглоблям, которые мы распиливаем, и я приступаю к разведению костра из этих всё равно сырых дров. 

Продрогли. Дал задание притащить ещё сухих, по~возможности, дров, а сам всё вожусь с костром. Никак не~разгорается\mdash дрова насквозь промокшие за целый день дождя. Дима соображает вскипятить воды на чай на его газовой плитке. Так и поступаем\mdash пока делается чай, я всё колдую над костром, разжигая его газовой горелкой. В~итоге, уже почти с кружкой свежего чая в руке, мне удаётся поддерживать нормальное горение костра, и мы немедленно приступаем к готовке ужина, а также ставим тент от дождя. Хочется поскорее откушать горячей еды. Чистим картошку, достаём очередные банки с тушёнкой\mdash всё как обычно. Вскоре наш ужин почти готов. 

Я с наслаждением откидываюсь в походном стульчике, сидя под тентом, попыхивая трубочкой и ожидая вечернюю похлебку. Паша и Геннадич крутят <<коровки>>, стреляя у~меня табачок. И тут меня осеняет, что завтра же выброска и~маршрут мы прошли на~1~день быстрее и~все договоренности о микроавтобусе$\ldots$ так, надо бы позвонить водителю. Бригада доделывает ужин, а~я~включаю телефон$\ldots$ связи мобильной нет. Хожу, ищу место, вдруг появится$\ldots$ нахожу. Стою на пеньке с~телефоном на~вытянутой руке$\ldots$ показывает одну <<палочку>> связи. Но~прозвониться не~удалось. Тучи сгущаются над адмиральской головой. 

Диме удаётся прозвониться Владимиру в Москву и~зачем\sdash то напрячь его поисками нам микроавтобуса. Я~же~пока безуспешно ловлю связь в попытках прозвониться в~Москву. Тут С.Ю. и говорит:

\diagdash Все эти ваши смартфоны\mdash гуано, это понятно же!\mdash и достаёт из гермопакета кнопочное чудо китайской радиоэлектронной промышленности с sim\sdash картой другого оператора связи и полным заполнением индикатора уровня сигнала.

\diagdash Спасены!\mdash я звоню по этому чуду в контору, где заказывал заброску и спустя полчаса договариваюсь о выброске, а Владимиру отсылаем отбой. 

\diagdash Ну, Адмирал, это был просчёт! 

\diagdash Кто ж знал!!! <<Иридиум>>\footnote[1]{Всемирный оператор спутниковой телефонной связи.} дороговато с собой брать, знаете ли! Итак, ура! Завтра вечером нас заберут на микроавтобусе от моста в~Лентьево, довезут до Москвы и развезут прямо по домам, вопрос решён\mdash объявил я бригаде.

Бригада поглаживает отросшие щетины и~вопросительно смотрит на меня. Оста\sdash а\sdash ался, конечно остался ещё 96\sdash й, как раз на этот вечер. Старпом, традиционно приговаривая <<два к трём>>, вновь разводит алхимию с кружечкой. Ужин, тем временем, поспел, а С.Ю. нарезает остатки сала. Сил этот день у нас отнял прилично\mdash мы прошли около 35 километров под проливным дождём. Фляги мои окончательно опустели, всё приготовленное мы скушали. Сидели под тентом за столиком и попивали чаёк, уплетая печенье с мёдом и доедая остатки сала. 

Погода к вечеру снова испортилась, небо нахмурилось, опять стал накрапывать дождик. От ощущения того, что наше мероприятие подходит к концу, стало грустно. Я,~можно сказать, только\sdash только недавно ощутил этот дзен отчуждения и вот скоро снова город. Всё хорошее в этой жизни, увы, скоротечно. Будем радоваться, что это было с~нами. По палаткам расползлись рано, около 10 часов вечера. 

В этом сплаве я отчего\sdash то не вёл хронометраж и~не~указывал время в судовом журнале, а вообще зря. В~основном я пользовался GPS и метками в нём, благо мой прибор обладал хорошей чувствительностью и ловил спутники даже в густом лесу, а~уж на воде под открытым небом вообще никаких проблем с этим не было.

\begin{center}
	\psvectorian[scale=0.4]{88} % Красивый вензелёк :)
\end{center}
