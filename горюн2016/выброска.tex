\chapter{Выброска. 09.08.16} 
\corner{64}

Вылезать из палатки совершенно не хотелось. Сознание того, что нам осталось пройти 5 километров до антистапеля, а водитель на микроавтобусе будет только вечером, расхолаживало окончательно. Поэтому утренние сборы я вообще пустил на самотёк. Абсолютно неспешно приготовили традиционную утреннюю кашу, ещё более неспешно позавтракали, ещё более лениво свернули лагерь. Погода в этот день радовала\mdash светило солнце, частые низкие облака скользили по небу. Полностью подготовились к отплытию только около 12\sdash ти часов. Я символично оставил в углублении на дереве коробок спичек как дар богам леса в знак того, что в этом походе мы ничего не забыли из личных вещей на стоянках, кроме моей ложки.

Итак, в путь! 5 километров\mdash это можно и пешком пройти, поэтому не гребём совершенно, а только подруливаем, течение само нас потихонечку несёт. Облака рассеиваются, погода налаживается, скидываю штормовку. Вскоре после урочища Салынь, где мы ночевали, проходим любопытный островок в русле. Течение резко и, я бы сказал, борзо ускоряется тут при сужении реки и мы с Геннадичем, налегая на весла, ради хохмы замеряем скорость по GPS\mdash выходит 10\sdash 11 км/ч! Вот это да! Дальше после этого уже абсолютно неспешно плывем по течению. 

Проходим деревню Шелохачь на левом берегу и вскоре после неё слышим шум машин. Показывается из\sdash за поворота мост трассы А\sdash 114. Это и есть наша конечная точка маршрута. Я достаю фотоаппарат и снимаю мост с воды, а Геннадич, достав губную гармошку, начинает наигрывать <<Плыли, плыли, вдруг остановка$\ldots$>>. Сие действо мне удаётся даже заснять на видео! Проезжающие по мосту машины грохотом отдаются в его железобетонной конструкции. Пройдя под мостом, я запримечиваю хороший спуск к воде на низком левом берегу и мы, по всей науке, развернувшись на 180, причаливаем. Всё. Сплав окончен, да здравствует Новый Сплав! Бригада подтягивается, тоже причаливает. Антистапелимся.

Времени до выброски микроавтобусом у нас часов 6\sdash 7. Запас огромный. Отдыхаем немного, разведываем место, где причалили. Берег укреплён бетонными плитами около моста, под мостом проходит грунтовка. Местность идёт как бы двумя террасами\mdash на верхней располагается грунтовка, потом укреплённый бетонными плитами откос и снова площадка внизу, выходящая уже непосредственно к воде. Погода в этот день очень ветреная  и переменчивая\mdash может в любой момент пойти дождик. Хоть и светит яркое солнце, облака плывут вполне себе дождевые. Поэтому решаем все вещи перетаскать под мост, предварительно расчистив там площадку от битого стекла. 

Байдарки разгружены, вещи перетасканы, настал черёд разборки и мытья наших байдарок. Разборкой занимаются Капитаны\mdash С.Ю., Дима и я, а Геннадич и Паша, взяв сумку, уходят искать магазин и вести разведку местности. Ваня счастливо бегает между нами и помогает разбирать лодки. Какие же лёгкие КНБ\sdash шки! Дима и С.Ю. быстро сдувают их, моют в реке и раскладывают шкуры сушиться. Я тоже достаточно быстро разбираю свою ласточку <<Таймень>>. Неудобство, как всегда, доставляют лишь замки стреляющих стрингеров, которые заклинивает после недельного сплава. Потом промываю кости в реке и сматываю их любимой синей изолентой. Далее промываю шкуру от песка и тоже раскладываю её на бетонных плитах сушиться. Вскоре Дима и С.Ю. уже запаковали свои лодки, а я всё ещё вожусь, пытаясь впихнуть кости покомпактнее. Через минут 15 мне это удаётся и мы сваливаем наши упаковки с байдарками около опоры моста, а сами располагаемся на стульчиках тут же рядом. Видим, как возвращаются наши гонцы из магазина. На сборку байдарок ушло где\sdash то полтора часа.

Геннадич и Паша притаскивают ящик пива(!!!) и закуску, мы садимся трапезничать. Тушёнки у нас еще осталось очень много\mdash я же брал из расчета на восьмерых человек, а нас всего шестеро. Поэтому каждый, кто хочет, берёт и вскрывает себе отдельную банку и уминает её с хлебом. Тут погода моментально портится, с неба опять льёт, как из ведра! А нам всё нипочём, мы сидим под мостом, кушаем. Геннадич достает гармошку$\ldots$ <<Сидим под мосто\sdash о\sdash ом$\ldots$ Нам всё нипочё\sdash ё\sdash ё\sdash ём$\ldots$ мы плыли сюда\sdash а\sdash а\sdash а, промокла нога\sdash а\sdash а\sdash а>> и т.д.\mdash что вижу, о том и пою, походный блюз! Потом он выдает на гармошке гимн Советского Союза и саундтрек из <<Титаника>>, вышибая скупую мужскую слезу. 

Погода\sdash погода, ты такая переменчивая! Снова светит яркое солнце, дождик заканчивается, как и ящик пива. Под мостом против течения проходит моторная лодка с рыбаками. До выброски ещё очень много времени, мы с Геннадичем снова идём в магазин.  Взбираемся на мост, я фотографирую дорожный указатель <<р. Чагодоща>>. На нашем пути пост ДПС, где два <<товарища>> странно косятся на нас, облачённых в одинаковые штормовки. Режем угол через какие\sdash то заросли и остатки какого\sdash то завода или автобазы, пробираемся к магазину. Наверно в этот день мы сделали им недельную выручку. Приобретаю магнитик на холодильник с фотографией и надписью <<Устюжна>>\mdash это город тут неподалеку на реке Мологе. Такого экспоната у меня ещё нет\mdash все знакомые, разглядывая на холодильнике коллекцию магнитиков, непременно спрашивают теперь, мол, а где ж эта Устюжна? И тут я начинаю свой рассказ$\ldots$

Идём обратно, позвякивая пакетами. Горланим на все окрестности: <<Круговая порука мажет, как копоть\mdash я беру чью\sdash то руку, а чувствую локоть$\ldots$>>, а потом и <<Одинокая пти\sdash и\sdash ица, ты лета\sdash а\sdash аешь высоко\sdash о\sdash о>> и прочее из Наутилосов. Возвращаемся к мосту и аккуратно спускаемся, шатаясь, по крутой лестнице вниз, к грунтовке, а потом и к нашей стоянке у опоры моста. 

Народ уже очумел сидеть на одном месте и ждать выброску, поэтому мы снова принимаемся за тушёнку, потом в ход идут разные истории из жизни и прочий трёп. Геннадич решает сделать из пустых бутылок ксилофон$\ldots$ Ставим в ряд бутылки, наливаем в каждую своё количество воды, чтобы $\ldots$ разная тональность, и он начинает палочкой по ним стучать, подбирая мелодию. Кстати, получается довольно интересно и даже похоже\mdash наигрывает мелодию из <<Крёстного отца>>. 

От скуки решаю исследовать мост. Я где\sdash то давно читал, что во всех мостах должны быть минные камеры, чтобы в случае войны можно было легко подорвать мост и не дать противнику переправиться через реку. Поднимаюсь наверх по насыпи, потом, ухватившись за край металлоконструкции и подтянувшись, оказываюсь под мостом, где он лежит на крайней опоре и переходит в насыпь. Тут, под полотном моста, идёт своеобразный служебный мостик из металлоконструкций, позволяющий попасть на опоры моста и вообще перейти реку не по полотну, то есть не сверху как положено\mdash где машины ездят, а под ним. Правда, страшновато\mdash мостик узкий, перила низкие, и все это из арматуры сварено. Под ногами щели и метров 10\sdash 15 высоты. Но во мне взыгрывает авантюрная нотка и я ставлю ногу на этот мосток и медленно начинаю продвигаться к опоре. 

С.Ю. внизу кричит, мол, а не дурак ли я? Отвечаю, что всё под контролем и аккуратно продолжаю продвигаться вперед. Вскоре я, переступив через перила, оказываюсь на опоре моста под дорожным полотном и начинаю осматриваться. Конечно, непонятно неподготовленному моему взгляду, что тут может быть минной камерой\mdash всяких закоулков, образованных металлоконструкциями, тут множество. Потом ещё больший дух авантюризма заставляет меня пойти дальше, на середину реки, по этому маленькому мостику. А он шатается и трясется каждый раз, когда сверху проезжает машина и ещё сильнее, если грузовик. 

Дойдя до середины, посмотрев вниз на реку и многозначительно решив, что миссия выполнена и дальше я не пойду, поворачиваю обратно к опоре. У опоры бригада фотографирует меня, запечатлев этот безрассудный поступок, и я благополучно добираюсь до конца моста и вылезаю из\sdash под него. Ух! Было не по себе на высоте, но сейчас всё позади. Спускаюсь вниз к бригаде. Почему иногда тянет на такие авантюры? Глубинное сознание того, что хочется это сделать, а так же чувство, что если не сделаешь, потом будешь долго корить себя за непреодоление своих страхов.

Я снова блаженно располагаюсь на стульчике, укутавшись в шемаг и штормовку\mdash похолодало. Набиваю последним табачком трубочку и тихонько попыхиваю. Связь тут отлично ловит\mdash рядом Лентьево. Все позвонили родным, поболтали чуток, сказали, что всё хорошо и скоро будем дома. Эх, своя сладость есть в этом чувстве возвращения домой. Домой, где тебя ждут! Может быть, надо иногда ощутить себя в отрыве от привычного, чтобы ярче и сильнее заиграла красками прежняя жизнь в городе? Определено так! Уже представляю, как жена выскочит из подъезда и бросится ко мне, заключая в свои жаркие объятия!

В голову, как и всегда в таких ситуациях, лезет еще всякая философия, мысли об окружающей природе, о далекой туманности Андромеды и прочих галактиках, как вдруг раздаётся звонок от водителя. Он свернул у поста ДПС налево, как мы ему и говорили. Теперь надо идти в село встречать его и показывать дорогу. Оглядев команду и увидев мутные взгляды Паши и Геннадича, я понял, что идти придётся мне. Взбодрился припасённым чаем из термоса, взял телефон и бодро потопал по грунтовке в Лентьево. 
 
По пути перепрыгнул с разбега через ручей, разлившийся, видимо, после недавно прошедшего дождя, прямо на дорогу. Вскоре, минут через 5\sdash 10, я вышел в край села Лентьево и был поражён красотой отделки изб! Такая красота уже исчезла из ближайшего подмосковья. Резные наличники, разнообразные вариации на тему резьбы по дереву и всё это сдобрено разноцветной красочной раскраской! Красота, да и только! Иду по улице и каждый дом такой! Висят тарелки спутникового ТВ, палисадники ухоженные, ворота и заборчики тоже. Не умерло село, пережило разруху 90\sdash ых, а теперь потихоньку возрождается, что не может не радовать. Может быть, это только дачи, а постоянно тут мало кто живет, но нет, какие же дачи? Тут в округе Лентьево и есть самое крупное село, не считая города Устюжны.

Вдалеке замечаю белый микроавтобус на перекрестке, звоню водителю\mdash так и есть, это он. Развернувшись, он подбирает меня, и мы катим к мосту за бригадой. Быстро добравшись и переехав через ручей, мы оказываемся у моста. Удобно и быстро грузим вещи\mdash у этого микроавтобуса сзади нет рядов кресел, так что места для багажа предостаточно\mdash сваливаем все наши тюки и немного переводим дух\mdash рюкзаки с байдарками таки тяжёлые. 

Команда начинает рассаживаться, я делаю пару фото, а потом по традиции бегу на прощание умыться водой из реки. Вернувшись, сажусь вперёд, рядом с водителем. Оборачиваюсь назад в салон, оглядываю с прищуром бригаду и, изрекая залихватское юрино <<Поехали$\ldots$>>, мы начинаем выбираться из села на трассу А\sdash 114. Предстоит выброска в ночь. У нас есть припасённый <<Старый Кёнингсберг>> и он медленно начинает исчезать под закуску лимоном, который удалось достать в магазине. Жируем, однако.

Сворачиваем с трассы А\sdash 114 на Устюжну и почти сразу же нам дорогу перебегает огромный лось! Но скорость наша низка, да и лось далеко, так что никакого вреда друг другу мы не причиняем. Я даже успел вскинуть руки и сделать несколько снимков\mdash вышло смазано, но очертания лося проглядываются вполне узнаваемо. Вот так да! Лис, зайцев, сов, ёжиков, даже медведя видели, а теперь ещё и лось! 

Продолжаем наш путь\mdash перед самым поворотом непосредственно в Устюжну, заезжаем на заправку. Бригада лениво тянется в магазин за минералочкой. Через ещё час всё, что крепче воды, оказывается уничтожено и с задних рядов слышится чей\sdash то богатырский храп. Странно, но у меня сна ни в одном глазу, я болтаю с водителем. Он оказывается чрезвычайно интересным человеком, причем такой невыговариваемой национальности народов Кавказа, что как я ни повторял её, запомнить к своему стыду, не смог. Причем православной национальности, что меня ещё больше удивило. Захотелось побольше узнать об этих людях. Водитель удивлялся обилию заброшенных церквей в тех краях, что мы проезжаем. Я рассказал ему краткую историю этого края, что некогда богатый Тихвинский водный путь пришел в упадок, а затопление купеческого города Молога для организации Рыбинского водохранилища окончательно поставило крест на былом расцвете региона$\ldots$ но жертва Мологи не была напрасной. В тяжелые годы Великой Отечественной войны Рыбинская ГЭС снабжала осаждённую Москву электричеством. Заводы Москвы и области продолжали работать и снабжать фронт, спасая нашу Родину, в том числе и благодаря этой электроэнергии\mdash первая ЛЭП на Москву была проложена в 1941 году$\ldots$ 

За разговорами мы незаметно подобрались к Ленинградскому шоссе и дальше полетели уже на одном дыхании, ещё раз заехав на заправку уже при подъезде к Москве. Я несколько раз всё\sdash таки проваливался ненадолго в сон. Потом снова ободрялся чаем из термоса. Бригада мирно спала, убаюканная под тихие звуки радио. Вскоре мы уже пересекли МКАД и понеслись на Электрозаводскую выгружать Ваню и Диму, что и сделали, когда на часах было часа 4 ночи. Потом в Новокосино выгрузили Пашу и С.Ю., а потом и меня в Железке. Геннадич помог перекидать мои тюки и байдарку из машины, и они с водителем поехали дальше\mdash в Любрецы. 

Так я снова оказался дома. Сваленные вещи лежали прямо на тротуаре. Вокруг тихо и только начинает светать. Позвонил жене\mdash она уже ждала меня. Как я и ожидал, она радостно выпорхнула из подъезда, и мы страстно обнялись, радуясь встрече! Что может быть прекраснее возвращения домой, где тебя любят и ждут? Стало быть, прав я? Освежает недолгая разлука чувства? Ещё как! 

\begin{center}
	\psvectorian[scale=0.4]{88} % Красивый вензелёк :)
\end{center}
