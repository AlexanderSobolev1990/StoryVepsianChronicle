\section*{Профессура}
\addtocontents{toc}{\protect\setcounter{tocdepth}{0}}

Математичка Р. влетала в собственный класс стремительно, как спецназ залетает в штурмуемое помещение. Стены класса были выкрашены в синий цвет, отчего это мрачное скорбное место становилось ещё холоднее и неуютнее. Ёжился на стуле при её появлении, наверное, каждый. На голове вечная химия, осветляющая тёмные волосы наверху, очки тёмные и полностью закрывающие глаза так, что практически невозможно было понять, куда она смотрит, полностью отсутствующая талия. Мужа у Р. не было, но было двое детей — взрослая дочь и старшеклассник сын.

- Садитесь!! - раздавалось её прокуренным, ненавистным голосом. Наверное, не было никакого другого учителя, которого ненавидели больше, чем её не только в нашем классе, но и в школе в целом! В арке 9-этажого дома, сквозь которую вела дорога к школе, кто-то из самых отчаянных голов в отместку за её ор, за унижения, за гнетущую атмосферу на уроках, за полную безжалостность, за всю ту боль, психологически уродовавшую малолетние сознания и ежедневно причиняемую нам, написал из баллончика синей краской шикарным курсивом шедевр, целое художество, под которым, я уверен, мысленно был готов подписаться каждый ученик Р. :

- «Р. - СУКА». Естественно, фамилия была написана полностью. Второе слово было написано смачно-размашисто.

Все знали, что она ходит на работу через эту арку каждый день. И другой дороги по асфальту нет. Надпись висела примерно полгода. Это низко и ужасно, но когда мы, ученики, шли через эту арку с утра и вечером, синяя надпись придавала сил - как от души отлегало. Ощущалось хоть некоторое отмщение. Словом, примерно полгода эта надпись была отдушиной. После того, как её закрасили, повторить подвиг никто не решился, хотя, вроде бы, автор так и не был найден.

- Света! К доске! - нарочито с наслаждением предстоящей казни вызывала она несчастную на эшафот. В тот день Р. Была особенно не в духе. Через несколько минут, когда Свету, особо не шарящую в алгебре, доводили чуть не до слёз, брань достигала апогея:

- Света, звезда… - тут Р. заминалась, подбирая приличное нематерное слово — звезда балета! Садись, 2!

Следующий несчастный выбирался по журналу и его ждали аналогичные эпитеты в случае провала.	

Но стоит признать, при всём при этом она часто, почти всегда, была справедлива. Если кто-то чего-то не знал — получал заслуженно свои двойки и тройки. Знал — без проблем ставила отлично. Она не снижала отметки из-за личного отношения… такое впечатление, что мы все были ей одинаково противны. 

Подача же материала у нее была институтская, как я понял позднее. Она требовала в расписании себе неуклонно спаренных уроков. На первом устраивалась лекция — мы писали конспекты. По сути, она, именно она, научила нас конспектировать, причём делала это профессионально — давала время записать, всё разжёвывала. Эти конспекты были, без иронии, шедевром изложения материала, как я понял позднее. На втором уроке был, на институтский манер, семинар — практическое занятие. Тут она оттягивалась по полной. Через призму лет абсолютно понятно за что — ведь материал разжёвывался ею досконально, а мы же часто много чего не понимали (но боялись спросить!), много тупили, не всегда всё усваивали с  первого раза. И даже со 2-го. И с 3-го. Кроме того, что мы были детьми, которым хочется гулять, прыгать, бегать, дурачиться и еще чёрте чем заниматься, многие из нас реально не понимали на кой нам сдалась эта математика на таком жёстком уровне. Многие откровенно не тянули и это нормально — не всем же быть лапласами, бесселями, колмогоровыми и т.д. и т. п. Но Р. считала иначе, она требовала с каждого, а кто не тянул — становился объектом едких подколок, унижений, ругательств.

В определенные дни казалось, что всё — если завтра школа не сгорит, на неё не упадет метеорит или не смоет цунами, то придётся каким-то самым нечеловеческим и ужасным образом расправиться с Р. чтобы избавиться от психологического гнёта. Но дальше синей надписи в арке ни у кого дело не зашло. Р. умела держать аудиторию — не хуже капо.

Нас, «ашников» она называла «профессурой» в пылу гнева и брани за очередной невыученный материал. Ругалась она виртуозно. Как ей удавалось не переходить при этом на мат — до сих пор не понимаю. Годы спустя, прочитав [географ глобус пропил], я был поражён – ведь роман был написан в 1995 году, а полностью издан только в 2003, а Р. называла нас так же, как и герой-учитель в книге — профессурой. Совпадение? Звучало это очень ёмко, хлёстко, издевательски обидно, с чуть протяжным <<c>>.

Май, конец года. Профессуре было поручено вымыть класс Р. Девочкам — отдраить окна, пацанам — вымести пол и помыть его. Хуже наказания и представить трудно. Каждый, я подчеркиваю, каждый был бы не прочь разбить эти ненавистные окна в этом классе, учинить расправу над каждой партой, каждым стулом, который был в часы уроков раскалён как геенна огненная под нашими ученическими задами. Но нас заставили мыть. Не было в мире бОльшего наслаждения, чем, улучив момент, незаметно помочиться в ведро с водой, которой потом помыли пол…

Ко мне Р. относилась как и ко всем — безразлично жестоко. Один раз я никак не мог понять куда в уравнении y=x2 девается, собственно, функция? y=f(x) — тут всё понятно, функция это f, а в уравнении y=x2 ? А что если уравнение сложнее, в несколько слагаемых? В голове, тем временем, совсем другое, не эти скучные функции, не области определения, не интервалы, отрезки, неравенства и многочлены. Всё это пригодится позднее… но тогда — тогда мы, никто из нас, я уверен, не понимал за что нам такая кара, где мы так провинились? Сейчас разбуди среди ночи и спроси формулу нахождения корней квадратного уравнения — отвечу. Но зачем, есть же справочники? Кто-то скажет, что всё обучение — ради нейронных связей в мозгу. Да, возможно, но не таким способом. 

Уже не помню по каким причинам, но, кажется, мы мыли её класс часто — то ли в какой-то год у нее не было классного руководства и, соответственно, собственных несчастных кто бы это делал, то ли ещё почему-то. Скорее потому что у нашей классной было 2 класса — 1-й мыл наш «родной» класс, а 2-й, то есть мы, профессура, отдувались на других фронтах. Конечно, эффективность такого подневольного труда была крайне низкой. 

Говорят, что школьные годы вспоминаются с теплотой — ничего подобного. Как по мне, невозможно вспоминать какое-то время только с теплотой или только с ненавистью — всегда речь идёт о конкретных моментах. Моменты у профессуры были большей частью негативные, даже, я бы сказал, жуткие. Может быть это моменты лишь моего восприятия? Но нет, синяя надпись говорила сама за себя — не только моего. С каким наслаждением я вздохнул после того, как покинул эти стены навсегда! И хотя впереди была полная неизвестность, «профессор» был если не счастлив, то по крайней мере рад. 

И хотя я много раз, уже в институте, впоследствии сказал мысленно спасибо Р. за вдолбленные в голову знания, за системный подход к преподаванию, чего многим учителям, как школьным, так и институтским, попросту никогда не удаётся добиться и близко, тогда кроме животного страха, ненависти, желания расправы она не вызывала ни-че-го, увы. Если так доставалось «профессуре», я просто не хотел и не хочу знать, как доставалось «отцам» и «зондеркомманде»[географ] — иногда, сидя на других уроках даже на другом этаже, даже за закрытой дверью, нам было слышно, как Р. орёт на очередных несчастных — слушала вся школа — дверь в свой класс она закрывала только на контрольных… 

Говорят, плохое забывается со временем. Нет уж! Забывается как раз хорошее, поскольку воспринимается как должное! А вот плохое, особенно очень плохое — въедается так, что никакими средствами не отмоешь, не выскребешь, не отчистишь. Любой из «профессуры» подтвердит...