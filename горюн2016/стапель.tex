%\chapter{Стапель и первый день на воде. 01.08.16} 
\chapter{Вожанское озеро и Горюн. 01.08.16} 
\corner{64}

Немного придя в себя после ночной заброски, потихоньку стали распаковывать тюки со снаряжением, собирать байдарки. Координаты стапеля N~59\degree~17.8370$^\prime$~E~34\degree~55.5540$^\prime$. С наслаждением  набил новую трубочку ирландским табачком, не спеша задымил, и мы с Геннадичем стали распаковывать рюкзак, раскладывать шкуру и кости байдарки на земле. 

Две КНБ\sdash шки нашей эскадры быстро обрели правильные очертания, когда их стали накачивать воздухом при помощи насосов, а вот нам с Геннадичем пришлось повозиться над моим <<Тайменем>>\mdash снова никак не хотели натягиваться фальшборта. Одолели замки фальшбортов только при помощи С.Ю., который ловко вправил их, поставив байдарку на борт\mdash опыт, как\sdash никак! Потом я подложил куски туристического коврика под точки крепления штевня к кильсонам, а также под лесенки кильсона. Ведь в прошлом сплаве именно в этих местах я заработал пробоины. Тележку для байдарки прикрепил сзади на корме, учтя горький опыт размещения её внутри грузового отсека. Делая эту <<антипробойную>> подготовку, я очень надеялся, что в этом сплаве клеиться не придётся. 

Ласточка моя, <<Таймень>>, была великолепна. Такое было возможно наверно только в Советском Союзе\mdash производство байдарки на заводе авиационных двигателей. Но всё хорошее рано или поздно заканчивается\mdash Союза уже нет, завод <<Салют>> пока ещё есть, но байдарки больше не выпускает. Да, они были тяжелы, да, в чём\sdash то конструкция была малость неудачна, но, чёрт возьми, каркасные байдарки великолепны по красоте, ходкости и, что самое главное, по вместительности и обитаемости.

Спустили собранные байдарки на воду, приступили к погрузке вещей. Вода с утра в озере показалась уж больно холодной, и мы с Геннадичем расхаживали в воду и обратно, таская снаряжение, в бахилах химзащиты, а экстремал Паша\mdash в обычных шлепанцах на босу ногу. Бр\sdash р\sdash р! Саня aka Геннадич был одет в такую же штормовку, как у меня, и камуфляжные брюки. Так что мы реально были похожи на слаженный экипаж, что меня воодушевляло. 

Я с ужасом смотрел на жизненное пространство в <<Шуе>> и <<Гарпуне>>. Как там вообще что\sdash то может поместиться? С.Ю.~и~Паша планировали крепить всё сверху, потому что юбка на <<Гарпуне>> несъёмная\ldots~Дима же решил, что просто покидает вещи горой сверху. Мы с Геннадичем, тем временем, разместили наше снаряжение под борта и по отсекам <<Таймени>>\mdash ничего наружу не торчало, так то! 

Настал черёд дележа ящика тушёнки. Её мы примерно поровну разделили на три экипажа. Класть тушёнку в КНБ\sdash шки оказалось практически некуда, и я понял, что на следующей же стоянке большинство банок перекочует обратно к  нам в <<Таймень>>\ldots

Немного перекусили перед отплытием. У меня была припасена чурчхела, привезённая женой с юга, и она ушла на ура. Кажется, у кого\sdash то оставался чай, и он был выпит. Последние приготовления, прошелся по берегу посмотреть ничего ли не оставили. Ну, вот и всё, экипажи готовы, можно отчаливать! Ваня~переоделся в сапоги и кофту и бегал туда\sdash сюда мимо нас. Я даже несколько ему завидовал\mdash попасть в байдарочное путешествие и в столь юном возрасте! Я таким похвастать не мог. Надеюсь, ему понравится и потом, когда\sdash нибудь, он тоже продолжит дело сплавов\ldots

Примерно в этот момент сплошная облачность слегка рассеялась, и солнце оранжево\sdash красными лучами пробилось к нам. Запечатлели в этот момент нашу команду на фотоаппарат. Время~8:03 утра\mdash отплытие! Стапель пройден, вперёд на фарватер Вожанского  озера! Облачность снова сгустилась, красивый цвет неба пропал, но всё же стало значительно светлее.

Мне удалось так расположить вещи в грузовом отсеке байдарки, чтобы оставить место для ног и сумки с фотоаппаратом. Геннадичу в этом повезло меньше\mdash передние фальшборта не дают нормально поставить ноги спереди при высоком росте\mdash упираются колени. Можно, правда, периодически вытягивать ноги в нос, если там не набито вещей\mdash я делал так на Киржаче при первом спуске моей ласточки на воду.  

Достал GPS навигатор, включил отслеживание по заранее загруженному маршруту, быстро сориентировался на местности. Надо было найти выход из Вожанского озера в реку Горюн. Наш экипаж отстал от команды при старте, усаживаясь поудобнее, а они тем временем сбавили ход и вопрошали куда идти дальше. Впереди были заросли камыша. Сверяясь с навигатором, я повёл эскадру вперёд. Через пару минут, огибая заросли, и выполнили крутой левый поворот и вышли на фарватер. Активно гребя и прогоняя сон, пошли искать исток Горюна. Встретили рыбака на деревянной лодке, перебросились с ним парой фраз и понеслись дальше. Вскоре озеро сузилось, и нам открылся вид на искомый исток.

Горюн мне сразу же понравился. Уровень воды был высоким, а берега достаточно узкими, поросшими сплошными джунглями растительности и сосновым лесом\mdash почти идеальная речка для байдарки. Но я не расслаблялся, потому что читал историю Тихвинской водной системы и возможное происхождение слова <<Горюн>>. По одной из версий, капитаны \textit{соминок} и \textit{тихвинок}, то есть судов, на которых перевозили тут грузы, \textit{горевали}, когда шли по этой речке. То ли от того, что она была слишком мелкой для их посудин, то ли от того, что очень узкой. Однако нам горевать не пришлось, и вскоре мы достигли остатков Остроленского шлюза. Точнее, перед нами предстала сильно разрушенная бетонная плотина, вся поросшая мхом, травой, а также деревьями и кустарником. Координаты N~59\degree~16.2406$^\prime$~E34\degree~55.1372$^\prime$. 

Все причалили к левому берегу, вскарабкались сквозь плотные заросли по крутому склону и пошли по сильно заросшей прибрежной дороге на разведку. Стало понятно, что справа от плотины Горюн капитально расширил своё русло и можно попытаться пройти там без обноса. Возможно, раньше там была шлюзовая камера, но всё настолько разрушилось и заросло, что сложно было сказать наверняка. Непосредственно перед плотиной в русле виднелись деревянные столбы\mdash то ли остатки старого деревянного шлюза, то ли защитных сооружений от весеннего льда. Признаться, я уже действительно подумывал об обносе, потому что перспектива пропороть в первый же сплавной день днище об торчащие из воды деревянные столбы и остаться тут на ночевку для ремонта совершенно не радовала. Паша первым забрался на плотину и жестами сквозь шум плотины показал, что справа точно можно пройти по воде. Я тоже сходил оценить ситуацию. 

Вода высокая, проход справа широченный. Вряд ли там что\sdash то коварное под водной гладью. Рискнём! Решено было идти по воде под самым правым берегом. Течение в этом месте значительно ускорялось. Дима, без тени сомнения, решил идти первым. Все остальные внимательно и напряжённо смотрели ему вслед\ldots~прошёл, ура! Мы с Геннадичем хотели вторыми, но замешкались при развороте байдарки носом по течению и в итоге пошли последними. Все преодолели эту плотину без повреждений, что радовало. Я почему\sdash то несколько по\sdash другому её себе представлял, но уж что есть. Не могу сказать, что я был сильно разочарован увиденным, но явно ожидал чего\sdash то большего. Природа достаточно быстро разрушает заброшенные объекты, как бы очищаясь от людской грязи\ldots

Почему\sdash то не задержались подольше у плотины. Место было какое\sdash то неуютное, неприветливое, поросшее высоченной травой. Пошли дальше. Первая плотина меня несколько удручила, одним словом. После бессонной ночи я весь квёлый, холодок утренний пробирает, погодка так себе\mdash сплошная облачность. Но я\mdash на реке, я хотел этого целый год, и я гребу веслом, упиваясь восторгом! Всё идет как надо\mdash команда отличная, мы встали на маршрут и уже преодолели треть пути, запланированного на первый переход. Подкрепляюсь ещё чурчхелой и думаю, что недурно было бы позавтракать\ldots 

Виляем немного с Геннадичем по руслу\mdash пока не приноровились грести согласованно. Вскоре впереди показывается подобие переката с островком посередине. Сверяюсь с GPS\mdash точно! Мы~подошли к Колевчинскому шлюзу. От него не осталось ничего вообще. Только небольшой перекат обозначает место, где когда\sdash то находилось тело плотины. Никаких брёвен и тем более бетона. Координаты места N~59\degree~15.3142$^\prime$~E~34\degree~55.1178$^\prime$. 

Дима снова вырывается вперёд и проходит в левой протоке около островка. Вот экстремал! Оборачивается и показывает большой палец вверх, хотя небольшие толчки от столкновения с подводными камнями не ускользнули от моего пристального взгляда, следившего за его прохождением. Мы с Геннадичем, погасив скорость, аккуратно повторяем его траекторию, увёртываясь от камней, и тоже преодолеваем этот небольшой перекатик. 

Этот <<шлюз>> разочаровал меня ещё больше, чем первый. Но~мы идём дальше. На первый день у меня было запланировано порядка двенадцати километров\mdash до устья Горюна. Берега становятся высокими и красивыми. И чем дальше, тем краше! Через ещё минут 15\sdash 20 мы причаливаем на левый берег передохнуть. Координаты остановки N~59\degree~14.6948$^\prime$~E~34\degree~55.7934$^\prime$. Я~взбираюсь на обрыв высотой наверно метров 5. Сажусь на его край и, не торопясь, набиваю трубочку. Вид с обрыва вниз по течению красив необычайно. Вокруг стоят высокие сосны, мягкий белёсый мох устилает землю\mdash красотища! Рядом полузаросшая грунтовая дорога\mdash надо полагать, до следующего шлюза\mdash Варшавского. Команда отдыхает от гребли, не вылезая из байдарок. Я ещё немного прохаживаюсь наверху по дороге и потом спускаюсь к реке. Ну что же, пора дальше! 

Мы отчаливаем и следующая точка нашего маршрута\mdash устье Горюна, потому что точные координаты Варшавского шлюза мне достать, представьте, не удалось, как ни искал. Судя по карте ГШ, там должен быть канал на левом берегу и шлюзовая камера\ldots~но неизвестно как всё это выглядит с реки в реальности. Плывем потихонечку, за рекордами сегодня гнаться смысла нет\mdash времени навалом, километраж на сегодня небольшой, а течение весьма приличное! 

Геннадич быстро втянулся в греблю и уже к концу Горюна освоил что такое <<табан>> (обратный гребок), <<наход>> (прямой гребок) и <<подтяг>> (боковое смещение), а также как вообще грести, чтобы плыть куда хочешь. Паша с С.Ю. тоже, кажется, приноровились грести более\sdash менее согласованно, а Дима потрясал тем, что с виду не отличаясь физической силой и выносливостью, гребя в одиночку, порой обгонял всех нас! Ваня, наш Юнга, маялся бездельем, изредка гребя, а больше подрёмывая в байдарке или поедая что\sdash нибудь. 

Так, незаметно для себя, мы не спеша приблизились к устью. Шлюза всё не было. А ведь как раз современные фото Варшавского шлюза в интернете я находил, и там было видно подобие канала из брёвен. Но\ldots~ни малейшего намёка. Странно. Видимо, начало канала мы пропустили в плотных прибрежных зарослях. Берега стали более пологими: низкий правый берег утопал в буйной растительности, а песчаные откосы на левом берегу стали ниже. 

И вот, примерно в 11:40, мы достигли устья Горюна. Перед нами предстала совершенно замечательная своей красотой картина\mdash широченный разлив в месте слияния Горюна и реки Чагоды. Решили сделать остановку на высоком левом берегу и перекусить. Координаты N~59\degree~12.9949$^\prime$~E~34\degree~55.6531$^\prime$. К этому времени распогодилось, в высоком иссиня\sdash синем августовском небе повисли кучевые облачка, вовсю припекало солнышко. Погода начала радовать, одним словом. 

Экипажи достали из сумок колбасу, сыр, хлеб и откуда\sdash то очень кстати взявшийся резервный термос с чаем. Глядя на то, как команда ведёт себя на воде и на берегу, я понял, что не прогадал\mdash с такими можно, что называется, в разведку. Состояние, по правде сказать, у всех было довольно квёлое, вялое после ночной заброски. Но перекус придал нам сил, я был доволен тем, что плановый километраж уже выполнен, а на часах ещё и полудня нет. Мы~разминали ноги, ходили по берегу, приминая высокую траву, и болтали о всяком разном.

Место же вокруг было необычайной красоты. При подготовке к сплаву я читал, что где\sdash то тут неподалеку была найдена стоянка древних охотников\mdash наконечники стрел, первобытные орудия, костяные скребки и ещё чего\sdash то там. И вот теперь, оказавшись в этих краях, я стоял с бутербродом в одной руке и кэпской кружкой чая в другой и мысленно переносился на многие тысячи лет назад, представляя, как доисторические люди охотятся на всякую живность\mdash саблезубых тигров и мамонтов. А реки как сливались вместе тут, так и до сих пор примерно в том же месте сливаются. И облака так же размеренно плывут по небу, отражаясь в водной глади\ldots

Как осознать человеку свою ничтожность в масштабах возраста нашей планеты? Чего\sdash то делаем, за что\sdash то боремся, как\sdash то живём. А если вдуматься, то вся наша цивилизация \mdash это просто микросекунда в галактическом масштабе\ldots~Интересно всё же, а мамонты тут проходили?

Настало время подумать о стоянке. Первоначально я хотел тут же в устье и остановиться на ночёвку, но вокруг росла трава прямо по грудь! Команде место было тоже не по душе, и я принял решение спокойно, не упираясь, идти дальше до первого попавшегося нормального места. Оно оказалось буквально в километре\sdash двух ниже по течению Чагоды. Разведка показала наличие хорошего места на склоне с расщелиной. Наверху склона было две ровных площадки посреди отличного соснового леса. Из минусов\mdash далеко таскать вещи от воды. Но времени у нас вагон, поэтому решили остаться. Координаты первой стоянки N~59\degree~12.9823$^\prime$~E~34\degree~56.9573$^\prime$. 

Не спеша поставили палатки на одной ровной площадке, а на другой организовали место под полевую кухню и перетаскали туда провизию. Паша достал удочки и отправился попытать счастья на берег. Добавка к рациону в виде свежей рыбы\mdash это всегда очень приятно! Остальные нарубили костровых палок, я обустроил костровище, а потом мы с Геннадичем напилили дров двуручной пилой. Всё шло как надо! С.Ю. организовал перекус со свежим чаем и зажаренной на костре курицей. И тут Дима с горящими глазами заявил: <<Господа, Jameson?>> <<У\sdash у\sdash у\sdash у\sdash у\sdash у\sdash у\sdash у\sdash у\sdash у!>>,\mdash загудела походная братия! <<Йо\sdash хо\sdash хо!!!>>,\mdash звон сдвигающихся кружек затмил всё на свете. В нашем распоряжении был весь день и вечер. Да чего уж там, теперь в нашем распоряжении 9\sdash 10 дней! 

Тем временем, Паша, кажется, успел сплавать на матрасе вдоль берега и поставить жерлицы или что\sdash то вроде того. Но желаемого результата это не дало. Рыба не клевала, потому что течение под нашим берегом было слишком сильным. Течение в этот раз вообще радовало! Уровень воды высок, течение быстрое\mdash мы урвали то время, когда вода ещё не начала активно спадать\mdash кра\sdash со\sdash та!
 
Сплавная братия гуляла. Jameson уже изволил закончиться, как вдруг Дима, словно фокусник, явил нам вторую и веселье продолжилось. Я соорудил около костровища флагшток из берёзы и водрузил на него флаг Советского Союза под аккомпанемент Сани, наигравшего гимн на губной гармошке\ldots отдых начался и начался отлично! Когда ты, как командир, только о чём\sdash то подумал, например, что надо бы напилить дров, а кто\sdash то уже ждет тебя с пилой или решил что\sdash то ещё сделать, а кто\sdash то из бригады уже тоже начинает этим заниматься, ты понимаешь, что не ошибся в людях. Толком не приходилось ни командовать, ни что\sdash то заставлять делать\mdash все как\sdash то нормально самоорганизовались и весь походный быт достаточно быстро наладился. Я просто с умилением радовался своей бригаде.

Стемнело. После ужина собрались у костра. Заброска нас утомила. Грунтовка и адский асфальт неслабо нас протрясли, но всё равно это гораздо лучше и удобнее, чем заброска поездом. Поэтому на первый день я и заложил малый переход, чтобы можно было пораньше встать лагерем и нормально отдохнуть с дороги. Километраж по плану будем наращивать плавно, с небольшим снижением в середине похода\mdash так легче переносится нагрузка.

Половина команды рано ушла спать, а мы с Саней и Пашей утихомирились около половины 12\sdash го. Очарование ночного костра и атмосферы похода захватило наши души и не отпустит до антистапеля. Да что там\mdash до следующего стапеля\ldots а если крепко задуматься\mdash отрава странствий уже в крови и это\mdash навсегда.

\begin{center}
	\psvectorian[scale=0.4]{88} % Красивый вензелёк :)
\end{center}
