%\chapter{<<Есть чё?>>. 02.08.16} 
\chapter{<<Есть чё?>>} 
%\corner{64}
\vepsianrose

С утра мы немного проспали, зато выспались, что так необходимо было после ночной заброски. Насчёт километража я был спокоен\mdash мы слегка перевыполнили норму вчера, да и течение! Течение помогает нам очень сильно! При таком течении всё пойдёт гладко, был уверен~я. Приготовили на завтрак мою традиционную походную кашу\mdash овсянку с орехами и сухофруктами. Подкрепились. Потом собрали вещи, свернули лагерь и стали грузиться в~байдарки по\sdash очереди, потому что выход к воде был очень узким на склоне с расщелиной. 

На этот раз мы с Геннадичем отчалили первыми. Я~рвался вперед\mdash  сегодня по плану проходим устье моей старой знакомой\mdash  Лиди, и останавливаемся в~районе моей прошлогодней стоянки. Геннадич уже вчера приноровился к~гребле, и~мы~почти не~петляли в~русле. Надо будет в следующем сплаве попробовать поставить руль на~байдарку$\ldots$ конечно, сделать это будет трудновато, потому что я люблю на корму привязывать вещи, но$\ldots$ можно постараться. Хотя С.Ю. и говорит, что руль для байдарки это довольно бесполезная вещь, и~я, в~целом, с ним согласен, попробовать всё же хочется$\ldots$ 

\diagdash А есть чё?\mdash делая характерный жест рукой, вопрошал Геннадич,\mdash Мои закончились!

\diagdash У меня только рассыпной,\mdash предлагал я,\mdash скоро Смердомский, сходите в магаз с Пашей!

Бросить сию пагубную привычку в походе\mdash дело гиблое\mdash не стоит пытаться, и Геннадич не стал исключением: он и Паша отчего\sdash то не запаслись куревом и~теперь страдали.

Буквально через 200\thinspace\nobreakdash--\thinspace 300 метров после выхода на~воду, мы с Геннадичем замечаем на левом берегу огромную отличнейшую ровную стоянку$\ldots$  работа Закона Лучшей Стоянки: <<Лучшая Стоянка чуть ниже по~течению>>. Причаливаем, выходим на берег. Да что ж такое! Места много, есть приличный стол, скамьи, причём довольно свежие. Но зато есть наезженный подъезд\mdash  место, видимо, популярно. В сторонке стоит стенд (!) с плакатами, посвященными Тихвинской водной системе. Координаты\mdash \CoordsChagodaGood. Ничего не поделаешь\mdash не дошли сюда, так не дошли\mdash на нашем месте тоже было неплохо. Остальная часть флотилии, тем временем, уже подплывает к нам. Занимаем места в своей байдарке, и наша эскадра выдвигается дальше вниз по Чагоде$\ldots$  

Берега в этот день не радуют\mdash  сплошные заросли. На~левом берегу встретилось одно хорошее место для стоянки с~навесом и~столом. Но, во\sdash первых, очень близко к населённому пункту, а во\sdash вторых, километраж на сегодня ещё не прошли. 

Вскоре приближаемся к Смердомскому, где с~незапамятных времён находится стекольный заводик и~<<смердит>>. С воды видна еле\sdash еле дымящая труба, склад завода и готовая продукция\mdash  бутылки в прозрачных контейнерах. Причаливаем, команда желает сходить в магазин. Я остаюсь караулить байдарки и маяться бездельем. Немного общаемся с местным, который косит траву на берегу. Потом приходят парень с~девушкой и~отплывают на моторке рыбачить. Скоро возвращаются и~наши гонцы из~магазина, позвякивая пакетами$\ldots$ 

В этот день прошли под мостом через трассу А\sdash 114 в Первомайском, по которому проезжали при заброске. Потом прошли и деревню Анисимово, где, согласно старым картам, должен был быть водомерный пункт. Разыгравшееся воображение рисовало водомерный пункт в~виде покосившейся будки, откуда суровый мужик в ватнике, сапогах и шапке\sdash ушанке выходит раз в день ровно в полдень и идёт к реке снимать показания с огромной закопанной там на берегу линейки, а потом аккуратно записывает всё в толстенную амбарную книгу {\fontspec{GOST type A Italic}{наклонным~ГОСТовским~шрифтом}}. Однако, ничего этого, конечно же, там не оказалось. Водомерного пункта не было, или мы не заметили его. Собственно, это нас нисколько не огорчило\mdash уровень воды был очень высок, о чем я судил по своим замерам веслом, по~прибрежной растительности, а также по следам от воды на~опорах~мостов. 

Мы шли дальше\mdash  навстречу перекату перед устьем Лиди. О его существовании я узнал из прочитанных при подготовке к сплаву отчётов, но не был уверен, что по такой высокой воде мы его ощутим. Так и вышло. Слева впала Лидь, переката никакого на Чагоде не было. Более того, в~устье Лиди тоже не было видно переката, хотя в прошлом году по низкой воде я очень хорошо на нём протрясся. Так~же, как и в прошлом году, причалили на левом берегу около турбазы тут же в устье и расположенной. Вылезли походить и~размяться. 

С базы доносился собачий лай, но смотритель, как в~прошлом году, не вышел. Глядя на увеличившуюся ширину реки, С.Ю. слегка раздосадовано сказал, что самый интересный участок мы прошли в первые же 2 дня. Да,~река стала гораздо шире, но течение! Течение продолжало радовать. Идти по узкой речке здорово\mdash  есть атмосфера камерности и миниатюрности. Тут же ощущаешь размах и~простор водной глади.

Маршрут дальше по Чагоде я уже проходил в прошлом году, так что почувствовал своеобразную уверенность, и~мы~понеслись дальше\mdash  до моей прошлогодней стоянки. Здорово и интересно идти в неизвестность по новой реке, но~в~уверенности на пройденном ранее маршруте тоже что\sdash то есть. Потому что каждый год всё по\sdash разному, ничего не~повторяется точь\sdash в\sdash точь и снова можно открыть для себя что\sdash то новое. 
 
Перекатов на Чагоде после впадения Лиди мы также не~заметили, хотя в прошлом году они были и добавили немного экстрима в маршрут. Остаток нашего сплавного дня прошёл достаточно спокойно и размеренно, но к концу я уже просто ёрзал, сидя в байдарке на своем гермомешке\mdash  так хотелось быстрее увидеть свою прошлогоднюю стоянку, и что там на~её~месте~стало. 

Стоянка в прошлом году тут под Чагодой конечно была не особо хорошая\mdash  мало ровного места и донимали полчища муравьёв. Но мы всё равно причалили для осмотра. Я нашёл своё костровище, и даже мои костровые палки продолжали стоять на том же месте! Чуть ниже по течению была ровная площадка, я её заприметил ещё с прошлого года. Решили дойти до неё и причалить. Там оказалось слишком много мусора и, что особенно плохо, много битого стекла. Настолько много, что я не могу поверить в то, что к этому причастны немногочисленные сплавщики. Скорее местные из Чагоды приезжают сюда <<культурно>> отдохнуть на алкорыбалку. 

Экипаж С.Ю. прошел ещё метров 200\thinspace\nobreakdash--\thinspace 300 ниже по~течению, разведал место и вернулся. Как они утверждали, там было лучше, но снова крутой подъем. С прошлого года я такого не помнил, но отчего же не попытать счастья? Мы~с~Пашей вдвоём пошли пешком в разведку. Разведанное место пришлось мне по душе, решили оставаться. Ну~а~крутой спуск к воде\mdash  нам не привыкать, зато поляна просто загляденье\mdash  вокруг высокие сосны и нет адских муравьев. Координаты стоянки\mdash \CoordsChagodaSixteenDrunk.

Причаливали и перетаскивали вещи снова по\sdash очереди. Бригаду смущало, что в тридцати метрах от стоянки проходит грунтовая дорога, но я уверил всех, что тут вообще никто не ездит. Забегая вперед, скажу, что пока мы стояли, ни одна машина не прошла мимо нас, потому что был вторник, а не выходные. Эта грунтовка соединяет ту базу отдыха, что стоит в устье Лиди, с городом и ездят по ней просто ну крайне редко. Эта же грунтовка, надо полагать, тянется и дальше\mdash вдоль берега Лиди в сторону Тургоши.

Вырубили костровые палки в молодой поросли сосен и~берёз на другой стороне грунтовки, обустроили костровище на холмике, организовали полевую кухню. Место радовало\mdash красивый пригорок, высокие янтарные сосны на крутом склоне к воде. К~закату у~нас уже был приготовлен отменный, как и всегда, ужин, и~все собрались у~костра. 

В прошлом году я дневал тут чуть выше по течению. Но~тогда это была вынужденная днёвка\mdash  надо было заклеить байдарку после <<блестящего>> прохождения шивер. И тогда это был уже третий сплавной день, а в этот раз только второй\mdash рановато дневать. Поэтому на совете Капитанов решили остаться тут только на ночёвку.

Солнце уже не освещало стоянку, зато верхушки сосен на противоположном берегу были великолепны в~его розово\sdash оранжевых закатных лучах. Ночь спускалась не звёздная, небо снова заволокло серыми тучами. Горючее закончилось, народ загрустил, стал расползаться по~палаткам. Но тут я достал десантную флягу с~96\sdash ым, передал её Старпому для разведения в пропорции <<два~к~трём>>, и мы продолжили посиделки. 

Я вспомнил песню Дольского <<Выходной без любви>> и~немедленно приступил к исполнению$\ldots$  

\vspace{0.5cm}
\noindent\textit{%
	\hspace*{1.5cm}А мы привратнику кричим: <<Откройте двери, \\
	\hspace*{1.5cm}На нас там занято!>> Но нам швейцар не верит. \\
	\hspace*{1.5cm}Открыл и пузом отодвинул он Алёшу: \\
	\hspace*{1.5cm}<<А ты, мол, вовсе, ты уже хороший!>>\\
	\\
	\hspace*{1.5cm}<<Да нас же трое! Это же не сто!>>\\
	\hspace*{1.5cm}А он нам отвечает: <<Нет местов!\\
	\hspace*{1.5cm}Я вас за пьянство не хочу обидеть,\\
	\hspace*{1.5cm}Но вы, ребята, все в абстрактном виде$\ldots$>> 
%	\hspace*{1.7cm}А мы привратнику кричим: Откройте двери, \\
%	\hspace*{1.7cm}На нас там занято! Но нам швейцар не верит. \\
%	\hspace*{1.7cm}Открыл и пузом отодвинул он Алёшу: \\
%	\hspace*{1.7cm}\diagdash А ты, мол, вовсе, ты уже хороший!\\
%	\\
%	\hspace*{1.7cm}\diagdash Да нас же трое! Это же не сто!\\
%	\hspace*{1.7cm}А он нам отвечает\mdash Нет местов!\\
%	\hspace*{1.7cm}Я вас за пьянство не хочу обидеть,\\
%	\hspace*{1.7cm}Но вы, ребята, все в абстрактном виде$\ldots$ 
}
\vspace{0.5cm}

$\ldots$земля вдруг перестала держать Адмирала, и он воспарил над ней; сознание было вдруг пронзено вспышкой молнии школьных воспоминаний$\ldots$

\newpage
\section*{Профессура}
\addtocontents{toc}{\protect\setcounter{tocdepth}{0}}

$\ldots$Математичка Р. влетала в собственный класс стремительно, как спецназ врывается в штурмуемое помещение. Стены класса были выкрашены в синий цвет, отчего это мрачное скорбное место становилось ещё холоднее и неуютнее. Ёжился на стуле при её появлении, наверное, каждый. На голове вечная химия, осветляющая тёмные волосы наверху, очки тёмные и полностью закрывающие глаза так, что практически невозможно было понять, куда она смотрит. Завершала образ полностью отсутствующая талия. Мужа у Р. не было, но было двое детей\mdash взрослая дочь и старшеклассник сын, учившийся на года 3\thinspace\nobreakdash--\thinspace 4 старше.

\diagdash Садитесь!\mdash раздавалось её прокуренным, ненавистным голосом. Наверное, не было никакого другого учителя, которого ненавидели больше, чем её, не только в нашем классе, но и в школе в целом! В~арке 9\sdash этажого дома, сквозь которую вела дорога к~школе, кто\sdash то из самых отчаянных голов в отместку за её ор, за унижения, за~гнетущую атмосферу на~уроках, за~полную безжалостность, за всю ту боль, психологически уродовавшую малолетние сознания и~ежедневно причиняемую нам, написал из~баллончика синей краской шикарным курсивом шедевр, целое художество, под~которым, я уверен, мысленно был готов подписаться каждый ученик~Р.:
\begin{center}
%<<\textit{\fontspec{Propisi}{Р.\mdash СУКА}>>.} 
\LARGE\textit{{\fontspec{Propisi}{Р}.}\mdash \fontspec{Propisi}{С У К А}.} 
\end{center}

Естественно, фамилия была написана полностью. Второе слово было написано смачно\sdash размашисто. Надпись была огромна\mdash буквы по полметра. Все знали, что она ходит на работу через эту арку каждый день. И другой дороги по~асфальту нет\mdash только через парк по грунтовке, но Р. там не ходила. Художество висело примерно полгода. Это низко и ужасно, но когда мы, ученики, шли через эту арку с утра и вечером, синяя надпись придавала сил\mdash как от души отлегало. Ощущалось хоть некоторое отмщение. Словом, примерно полгода эта надпись была отдушиной. После того, как её закрасили, повторить подвиг никто не~решился, хотя, вроде бы, автор так и не был найден.

\diagdash Света! К доске!\mdash нарочито с наслаждением предстоящей казни вызывала она несчастную на эшафот. В~тот день Р. была особенно не в духе, парочка свежевыставленных двоек уже красовались в журнале$\ldots$ Через несколько минут, когда Свету, особо не шарящую в~алгебре, доводили чуть не до слёз, брань достигала апогея:

\diagdash Света, звезда$\ldots$\mdash тут Р. заминалась, подбирая приличное слово,\mdash $\ldots$балета! Садись, 2!

Следующий несчастный выбирался по журналу, и его ждали аналогичные эпитеты в случае провала.	

Но стоит признать, при всём при этом она часто, почти всегда, была справедлива. Если кто\sdash то чего\sdash то не знал\mdash получал заслуженно свои двойки и тройки. Знал\mdash без проблем ставила отлично. Она не снижала отметки из\sdash за личного отношения$\ldots$ такое впечатление, что мы все были ей одинаково противны. 

Подача же материала у нее была институтская, как я понял позднее. Она требовала в расписании себе неуклонно спаренных уроков. На первом устраивалась лекция\mdash мы писали конспекты. По сути, она, именно она, научила нас конспектировать, причём делала это профессионально\mdash давала время записать, всё разжёвывала. Эти конспекты были, без иронии, шедевром изложения материала, как я понял позднее\mdash когда появилось с чем сравнивать. На~втором уроке был, на институтский манер, семинар\mdash практическое занятие. Тут она оттягивалась по полной. Через призму лет абсолютно понятно за что\mdash ведь материал разжёвывался ею досконально, а мы же часто много чего не~понимали (но боялись спросить!), много тупили, не всегда всё усваивали с  первого раза. И даже со 2\sdash го. И с 3\sdash го. Кроме того, что мы были детьми, которым хочется гулять, прыгать, бегать, дурачиться и ещё чёрте чем заниматься, многие из нас реально не понимали на кой нам сдалась эта математика на таком жёстком уровне. Многие откровенно не тянули и это нормально\mdash не всем же быть лапласами, бесселями, колмогоровыми и т.д. и т.п. Но Р. считала иначе, она требовала с каждого, а кто не тянул\mdash становился объектом едких подколок, унижений, ругательств.

В определенные дни казалось, что всё\mdash если завтра школа не сгорит, на неё не упадет метеорит или не~смоет цунами, то придётся каким\sdash то самым нечеловеческим и~ужасным образом расправиться с Р., чтобы избавиться от~психологического гнёта. Но дальше синей надписи в~арке ни~у~кого дело не~зашло: Р. умела держать аудиторию\mdash не~хуже~капо.

Нас, <<ашников>> она называла <<профессурой>> в пылу гнева и брани за очередной невыученный материал. Ругалась она виртуозно. Как ей удавалось не переходить при этом на мат\mdash до сих пор не понимаю. Годы спустя, прочитав <<Географ глобус пропил>> \cite{ГеографГлобусПропил}, я был поражён\mdash ведь роман был написан в 1995 году, а полностью издан только в 2003, а Р. называла нас так же, как и герой\sdash учитель в~книге\mdash профессурой. Совпадение?! Звучало это очень ёмко, хлёстко, издевательски обидно, с чуть протяжным <<c>>.

Май, конец года. Профессуре было поручено вымыть класс Р. Девочкам\mdash отдраить окна, пацанам\mdash вымести пол и помыть его. Хуже наказания и представить трудно. Каждый, я подчеркиваю, \textit{каждый} был бы не прочь разбить эти ненавистные окна в этом классе, учинить расправу над каждой партой, каждым стулом, который был в часы уроков раскалён как геенна огненная под нашими ученическими задами. Но нас заставили мыть. Не было в мире б\'{о}льшего наслаждения, чем, улучив момент, незаметно помочиться в~ведро с водой, которой потом <<помыли>> пол$\ldots$

Ко мне Р. относилась как и ко всем\mdash безразлично жестоко. Один раз я никак не мог понять куда в уравнении вида ${y=x^2}$ девается, собственно, функция? ${y=f(x)}$\mdash тут всё понятно: функция это $f$, а в уравнении ${y=x^2 + 2\times x}$? В голове, тем временем, совсем другое\mdash не эти скучные функции, не области определения, не интервалы, отрезки, неравенства и многочлены. Всё это пригодится позднее$\ldots$ но тогда\mdash тогда мы, никто из нас, я уверен, не понимал за что нам такая кара, где мы так провинились? Сейчас разбуди среди ночи и спроси формулу нахождения корней квадратного уравнения\mdash отвечу. Но зачем, есть же справочники? Кто\sdash то скажет, что всё обучение\mdash ради нейронных связей в мозгу. Да, возможно, но не таким жестоким способом. 

Уже не помню по каким причинам, но, кажется, мы мыли её класс часто\mdash то ли в какой\sdash то год у~неё не~было классного руководства и, соответственно, собственных несчастных кто бы это делал, то ли ещё почему\sdash то. Скорее потому, что у нашей классной было два класса\mdash 1\sdash й мыл наш <<родной>> класс, а 2\sdash й, то есть мы, профессура, отдувались на других фронтах работ, куда нас сдавали <<в аренду>>. Конечно, эффективность такого подневольного труда была крайне и крайне низкой. 

Говорят, что школьные годы вспоминаются с~теплотой\mdash ничего подобного. Как по мне, невозможно вспоминать какое\sdash то время только с теплотой или только с ненавистью\mdash всегда речь идёт о конкретных моментах. Моменты у профессуры были большей частью негативные, даже, я бы сказал, жуткие. Может быть это моменты лишь моего восприятия? Но нет, синяя надпись говорила сама за себя\mdash не только моего. С каким наслаждением я~вздохнул после того, как покинул эти стены навсегда! И~хотя впереди была полная неизвестность, <<профессор>> был если не~счастлив, то~по крайней мере~рад. 

И хотя я много раз, уже в институте, впоследствии сказал мысленно спасибо Р. за вдолбленные в голову знания, за системный подход к преподаванию, чего многим учителям, как школьным, так и институтским, попросту никогда не~удаётся добиться и~близко, тогда кроме животного страха, ненависти, желания жестокой расправы она не~вызывала ни\sdash че\sdash го, увы. Если так доставалось <<профессуре>>, я просто не хотел и не хочу знать, как доставалось <<отцам>> и~<<зондеркомманде>>~\cite{ГеографГлобусПропил}\mdash иногда, сидя на других уроках даже на другом этаже, даже за закрытой дверью, нам было слышно, как Р. орёт на очередных несчастных\mdash слушала вся школа\mdash дверь в свой класс она закрывала только на контрольных.

Считается, что плохое забывается со временем. Не~всегда! Часто забывается как раз хорошее, поскольку воспринимается как должное. А вот плохое, особенно очень плохое\mdash порой въедается так, что никакими средствами не отмоешь, не выскребешь, не отчистишь. Любой из <<профессуры>> это подтвердит$\ldots$

$\ldots$Покинув стены этого <<прекрасного>> заведения (к~слову, построенного в 70\sdash ые по какой\sdash то экспериментальной программе и существующего в~единственном, по~моим сведениям, экземпляре) и~оказавшись на пляже после 9\sdash го класса в компании \textit{N}, который был младше меня на несколько лет и тоже несчастно попал к Р., я с грустью наблюдал, как он выводил ту самую синюю надпись на камнях и швырял их с горя в~воды Чёрного моря\mdash ни яркому солнцу, ни нежному бризу, ни убаюкивающему шуму прибоя, ни ласковой воде не под силу было заглушить эту боль$\ldots$

\begin{center}
	\psvectorian[scale=0.4]{88} % Красивый вензелёк :)
\end{center}

\newpage
\section*{Джомолунгма}
\addtocontents{toc}{\protect\setcounter{tocdepth}{0}}

$\ldots$У неё были густые длинные, чуть суховатые, натурально(?) рыжие волосы. С высоким ростом, нехудощавой комплекцией, широкими бёдрами, волнительной грудью, чуть узковато посаженными глазами, она, молодая географичка, казалась нам, 6\sdash классникам, образцом совершенства. Словно для подчёркивания рыжины волос, она часто носила зелёный цвет. 

На уроках царил порядок\mdash самые отъявленные негодяи стихали, откровенно любуясь её красотой. Конечно же она видела (не могла не видеть!) это и всё понимала, но вела себя в высшей степени достойно, как и подобает настоящему Учителю. Я не совру, если скажу, что почти все пацаны были в неё влюблены. Несмотря на то, что она была дочерью математички Р.

В противовес своей матери, она никогда не кричала, вела уроки тихо и спокойно, рассказывая нам о физической географии. Стоило каким нибудь раздолбаям в классе вдруг зашуметь, он лишь замолкала и пристально откровенно смотрела на них. Это работало ну просто безотказно — парни тут же смолкали под её взглядом, смущаясь, казалось бы такого желанного внимания своего идола, и сидели внимали. 

Порядок, как мне кажется сквозь года, держался исключительно на её внешних данных, её спокойствии и личном интересе к рассказываемому\mdash страха не было\mdash во время её уроков никто не вспоминал про её мать и уж точно никакая мысль, что Р. может прийти и наорать на нас или что географичка может нажаловаться на нас матери, не была причиной той чарующей атмосферы географических открытий, топокарт, гор и долин, морей и океанов, царившей на её уроках$\ldots$ Поразительно, но, будучи разительно другой, нежели чем своя мать, она не вызывала у нас c той никаких ассоциаций, никогда ни у кого не возникало желания выместить на ней свою злобу к математичке, отыграться как\sdash то, наоборот, она воспринималась нами как нечто кристально чистое и незамаранное в общей картине учительского болота.
 
Она учила нас пользоваться топокартой: читать легенду карты — условные знаки, определять азимут на местности, обходить препятствия по азимуту, вычислять расстояния. Обучала масштабу и прочей науке$\ldots$ Так моя любовь к топокартам и без того подпитываемая настоящим советским географическим атласом для учителей средней школы (с надписями главное управление геодезии и картографии при совете министров СССР), заиграла новыми красками\mdash я осознал насколько география и картография интересные и полезные науки. Дополнительную пищу для восхищения давал том номер 1 Детской Советской Энциклопедии, где тоже был раздел про картографию и походы. Всё вместе это\mdash да, именно это\mdash зародило в самых потаённых уголках моей души чувство, что я должен когда-нибудь сам пойти в поход. Чтобы у меня была топокарта, компас и чувство того единения, того отчуждения от всех, которое испытывают топографы, географы, работая в <<полях>>.
 
Как помнишь всегда ${x_{1,2}= \frac{b\pm \sqrt{b^2-4\times a\times c}}{2a}}$, так помнишь в любое время дня и ночи и высоту самой высокой горы\mdash Джомолунгмы\mdash 8848 метров на уровнем моря. Но, как нетрудно догадаться, совсем по другим причинам$\ldots$ Как поразительно разными методами, способами, можно добиться от детей знаний и как ужасно прочувствовать на себе весь спектр средств, применяемых иными <<учителями>> в~работе$\ldots$ 

На географию, ясное дело, мы ходили как на праздник. Ни один школьный предмет не вспоминается с такой теплотой, как этот. Сквозь года очень сложно сказать, сколько географичке было лет, тем более в детстве все кажутся таким взрослыми$\ldots$ Вероятно ей было от 25 до 30. Она была самым молодым учителем в нашей школе. Вскоре на её руке появилось кольцо и она уволилась, а может быть ушла декрет\mdash мы не знали. Курс экономической географии вместо нее читал нам другой учитель, чей <<авторитет>>, манера подачи, внешние данные\mdash короче всё было против нее. Тогда конечно вся пацанва была жутко расстроена уходом нашей географички, но сейчас, через года, ничего не остаётся как искренне за неё порадоваться. Работать в том коллективе учителей я бы не пожелал абсолютно никому. 

От экономической географии в голове не осталось ничего, а физическая до сих пор вспоминается не то чтобы даже с теплотой, а с каким то просветом, ярком лучом$\ldots$ поистине <<лучом света в тёмном царстве>> остальных предметов. А очарование топографической карты всегда вызывает в душе самые что ни на есть приятные, походные чувства\mdash вот уже ощущаешь звуки леса и шум реки, запах полей и разнотравья лугов, проносишься по просекам и ниточкам лесных дорог, уносишься в глушь нашей территории, где нежданно в лесу тебя ждёт избушка охотника$\ldots$

Я мечтал стать Географом. Я им не стал$\ldots$


\newpage
\section*{Митрич и Циркуль}
\addtocontents{toc}{\protect\setcounter{tocdepth}{0}}

%Они были учителями труда и черчения соответственно. Оба\mdash мужчины. Был ещё физрук, но про эту мразь и вспоминать не хочется
$\ldots$Циркуль был весьма интересным человеком\mdash худой, как щепка, высокий, как баскетболист, с залысиной, высоким лбом, окладистой бородой и густым голосом. Короче говоря\mdash внешностью он чем\sdash то походил на Достоевского, по крайней мере мне так тогда казалось. Неизменные брюки, рубашка, безрукавка дополняли образ интеллигента, коим он, вне всяких сомнений, являлся. Он был учителем рисования и черчения, а также преподавал дисциплину под~названием <<Мировая художественная культура>>, которая предполагала знакомство неокрепших умов с шедеврами мировой архитектуры, живописи, скульптуры. 

Класс Циркуля был сродни музею\mdash там была парочка репродукций, куча цветов на подоконниках, на шкафах были таблички с пошаговым процессом выполнения портретов и натюрмортов, имелись чертежи деталей в проекциях и~всевозможные другие наглядные пособия. К оформлению своего кабинета он подошёл, что называется, со всей душой\mdash находиться там было одно удовольствие; пожалуй, ни один другой кабинет не мог таким похвастаться. 

В классе имелся диапроектор с дистанционным пультом\mdash слайды переключались автоматически при~нажатии кнопки на пульте. Это воспринималось не~иначе как небольшое волшебство (стоит напомнить, что в то время у~большинства из нас ещё не было мобильных телефонов). У~Циркуля была огромная коллекция слайдов с шедеврами мирового искусства, и когда он начинал закрывать окна в~классе плотными светонепроницаемыми шторами\mdash было сразу понятно, сейчас будет таинство диапроектора и~знакомства с чем\sdash то интересным. Огорчало конечно то, что иногда мы сдавали зачёты по~просмотренному материалу\mdash нужно было запоминать, собственно, названия и авторов этих шедевров, что было значительно сложнее, нежели чем просто любование красотой архитектурных и~живописных форм. Но,~тем не менее, мы как\sdash то справлялись. Особенное впечатление, естественно, производила архитектура и~скульптура древней Греции и~древнего Рима, а также эпохи ренессанса. Также, у~Циркуля был телевизор, по~которому он показывал нам различные учебные фильмы и материалы$\ldots$ словом, его уроки были одними из самых интересных за~школьные года. На них мы погружались в какой\sdash то другой мир\mdash без синусов и косинусов, в~мир красоты и изящества мирового искусства, который был некой отдушиной в нашей, вообще говоря, не очень приятной школьной жизни. Кроме того, на этих уроках существовала прекрасная и такая заманчивая, чего греха таить, возможность немного$\ldots$ подремать в темноте под~жужжание проектора\mdash часто его уроки ставили нам после омерзительной физкультуры или ещё чего\sdash нибудь столь же нелюбимого.

Также он вёл и рисование, рассказывая о приёмах живописи, различных техниках рисунка и прочем. Рисовали мы, в основном, всегда акварелью и простыми карандашами. Это были натюрморты, пейзажи, различные сюжетные зарисовки, простенькие эскизы сценок. Чуть позже он учил нас рисовать людей и обучал построению перспективы на рисунке. Понимание пространства и перспективы очень пригодилось нам в дальнейшем\mdash на уроках черчения, которые начались в старших классах. Тут уже Циркуль проявлял себя максимально жёстко, но и корректно при этом\mdash черчение строгая дисциплина, и он старался привить нам эту строгость и опрятность при выполнении чертежей. А чертили мы, понятное дело, простыми карандашами на~ватмане. Ни о каких компьютерных программах для~черчения никто и в помине не знал тогда, и хотя век компьютеризации неотвратимо шагал по планете не первый год, до нас, до школьников, это, понятно, не доходило тогда ещё\mdash редко у кого из нашего класса был дома персональный компьютер. Даже уже позже, в институте, я тоже, кстати, чертил руками, к своему стыду, так и не освоив какую\sdash нибудь нормальную чертёжную программу$\ldots$

Митрич оставил в сознании хорошие воспоминания\mdash наверно таким должен быть Настоящий учитель. С грустью думалось тогда\mdash почему он не может преподавать ещё и алгебру? Сколько бы нервов было сохранено?

При всей при этой возвышенности и высокопарности, царившей в его кабинете на уроках, ничто человеческое, естественно, ему было не чуждо. Митрич имел пагубную привычку курить и, судя по всему, немного этого стеснялся, поскольку не должно Учителю подавать такой пример Ученикам. Помню как\sdash то раз мы стояли со школьным товарищем и ждали урока у его двери. Циркуль мимо нас семимильными шагами проник в кабинет, попутно убирая пачку сигарет в карман брюк:

\diagdash Гляди, <<Kent>>! \mdash изрёк мой курящий одноклассник, гляди на быстро убираемую пачку.

Это резануло слух Циркуля по каким\sdash то причинам и, влетев в класс, он мигом вернулся спиной назад обратными шагами:

\diagdash Не кент, а графство Кент! \mdash то ли с насмешкой, то ли с укоризной и налётом таинственности изрёк он и многозначительно вновь просквозил в класс.

Циркуль вообще был мастером слова и остёр на выражения в хорошем, не обидном смысле слова. Отчего таких Учителей так мало в наших школах$\ldots$ 
%Помнится, как он водил меня и других отличившихся учеников на олимпиаду по черчению, по дороге рассказывая по то, как в армии бегал с карабином СКС

\vspace{1.0cm}
$\ldots$Митрич. Он был учителем труда. И хотя в духе идиотических экспериментов и издевательств над школьной программой его предмет назывался то <<Технологией>> (Технологией чего?!), то <<Технологией и ещё чем\sdash то>>, он всегда неизменно говорил нам, что будет называть его кратко и по\sdash старинке: <<Труд>>. И нас, должен сказать, это вполне устраивало. На его уроках мы выпиливали лобзиком, строгали, делали разные поделки из дерева и металла, работали на сверлильных станках, а позже добрались и~до~главной мечты и сокровенного таинства, до чего максимально можно было добраться на его уроках\mdash до токарного станка. Это была фантастика$\ldots$

Стены его класса украшали плакаты по выполнению простейших операций в сверлильном и токарном деле, по технике безопасности. Посередине класса стояли металлические верстаки с тисками. Это был класс металлообработки, как я позже понял. Потому что было ещё и соседнее помещение, напрочь заваленное школьным хламом\mdash партами, стульями и прочим барахлом. Этот заваленный хламом и оттого неиспользуемый класс раньше был классом деревообработки\mdash там стояли специально приспособленные для этого верстаки иной конструкции, со столярными зажимами. Увы, в реалиях всеобщего бардака в то время, всё было так, как было\mdash кто\sdash то решил, что и деревообработкой можно заниматься в классе металлообработки, и больно жирно одному учителю иметь 2~помещения. Ну да ладно$\ldots$ нам хватало и того, что было, просто порой некая внутренняя обида накатывала, как цунами\mdash ты ведь понимал, что \textit{когда\sdash то тут было всё совсем по\sdash другому! Причём совсем надавно!} 

У Митрича были Жигули 2104, универсал, и он часто подвозил к школе, прям к своему кабинету (кабинет труда имел отдельную дверь на улицу!) всевозможные деревяшки и железные прутки\mdash для наших поделок. Мы много раз становились свидетелями такой заготовки материала впрок$\ldots$ очевидно, всё это <<добывалось>>, поскольку денег на~покупку материалов, скорее всего, не выделялось.

Итак, в~классе металлообработки были верстаки по~центру, у~стены стояли сверлильные станки, а~вдоль окон\mdash вожделенная мечта\mdash токарно\sdash винторезные, ТВ\sdash 3. Работать на них мы начали через год от~того момента как начались уроки труда, и,~конечно же, всем и~каждому безумно хотелось освоить это дело. Всё~то, что мы делали до~токарного дела, было для меня не сильно в~новинку, поскольку летом в~деревне шла вечная стройка дома и~молоток, ножницы по металлу, рубанок, дрель\mdash уже побывали в моих любознательных руках. Токарка же была высшим пилотажем в нашей понимании. Вдобавок, станков было 6 штук, но в рабочем состоянии лишь половина, отчего вокруг них всегда толпилась очередь выточить очередную деталь, подогревая интерес. Начинали мы, конечно, с обработки древесины на этом станке, поскольку по неопытности легко и резец затупить, и по лбу заготовкой получить. Потом, после освоения приёмов продольной и~поперечной подачи, автоматической продольной подачи, мы выточили первое стоящее изделие\mdash деревянный подсвечник. Для того, чтобы выточить подставку большого диаметра для подсвечника, нужно было поставить в шпиндель станка обратные кулачки и развернуть резцедержатель на 90\degree$\ldots$ словом, дело было безумно интересное. Потом точили уже много чего, перейдя и к металлообработке. Помню кто\sdash то притащил откуда\sdash то стреляных гильз от автомата Калашникова калибра 5.45\thinspace мм, и~мы выточили <<пули>> из~проволоки и загнали их в~стреляные гильзы\mdash выглядели полученные <<патроны>> классно. 

Вспоминается ещё один интересный факт. Рядом с <<детского>> размера ТВ-3 стоял полноразмерный, <<взрослый>>, настоящий токарный станок, огромный просто\mdash он стоял вдоль окон и занимал два пролёта. Это был трофейный немецкий токарный станок\mdash все прикрученные на него чёрные таблички с белыми надписями были на немецком. Был и немецкий орёл с закрашенной свастикой. Жалко одно\mdash он был в нерабочем состоянии. По крайней мере, нам так говорил Митрич. Но и за все школьные года я ни разу не видел его в работе. Эх,~мощь! Он был реально огромен и потрясал воображение. Но одновременно в душу закрадывалось какое\sdash то такое нехорошее чувство\mdash а зачем он тут стоит? А у нас, что, своих таких станков нет? Ведь уже не конец 40\sdash х годов, а~вообще новое тысячелетие?! Но~и~понятно\mdash какие токарные станки, когда всё производство в стране в 90\sdash е шло под~откос$\ldots$ А нам, по крайней мере мне точно, так хотелось после ТВ\sdash 3 попробовать и эту мощу\mdash хоть разочек! 

Помню в период увлечения стрельбой из лука у меня сорвало резьбу на зажиме упругого плеча. Я, конечно же, обратился к Митричу с просьбой дать выточить этот <<барашек>> на токарнике, и он согласился, но потом как\sdash то всё завертелось, закрутилось, стало не до этого$\ldots$ вместо того зажима я просто загнал подходящего размер болт, что было неудобно, поскольку надо было делать слишком много оборотов для зажима плеча, но вполне решало задачу$\ldots$ Поскольку я ушёл из этой школы после 9\sdash го класса, на~токарнике больше поработать мне не удалось, но в глубине души засела такая мысль\mdash на старости лет точно приобрету токарник, поставлю его в деревне и буду самозабвенно что\sdash нибудь точить$\ldots$ эх, мечты!

Одним словом\mdash токарный станок это любовь на~всю жизнь$\ldots$ Как мёдом ложатся на сердце строчки пролетарско\sdash романтического народного стихотворения:

\vspace{1.0cm}
\noindent\textit{%
\hspace*{2.7cm}Я увидел тебя у станка,\\
\hspace*{2.7cm}Нежный взор твой меня озадачил.\\
\hspace*{2.7cm}Ты стояла, вращая слегка\\
\hspace*{2.7cm}Рукоятку продольной подачи.\\
\\
\hspace*{2.7cm}Я сказал тебе: <<Ваши черты\\
\hspace*{2.7cm}Несомненно полны благородства>>.\\
\hspace*{2.7cm}Мне в ответ улыбаешься ты\\
\hspace*{2.7cm}Без отрыва от производства.\\
\\
\hspace*{2.7cm}Я теперь без ума от тебя,\\
\hspace*{2.7cm}Быть не может с тобою ошибки.\\
\hspace*{2.7cm}Ты стояла в руках теребя\\
\hspace*{2.7cm}Радиально-упорный подшипник.\\
\\
\hspace*{2.7cm}А под вечер к тебе я приду,\\
\hspace*{2.7cm}Ты расскажешь мне голосом звонким\\
\hspace*{2.7cm}О подаче на левом ходу,\\
\hspace*{2.7cm}О гидравлике и шестерёнках.\\
\\
\\
\hspace*{2.7cm}И не будет ни ветра, ни туч\mdash\\
\hspace*{2.7cm}Только ты и Луна в целом мире.\\
\hspace*{2.7cm}Хочешь, я подарю тебе ключ\\
\hspace*{2.7cm}19\footnote{Гаечные ключи 19 на 24 в СССР не изготавливались и, следовательно, могли быть только импортными, а посему факт дарения девушке такого ключа свидетельствует об очень сильных чувствах.} на 24?\\
\\
\hspace*{2.7cm}И вернувшись из ЗАГСа домой,\\
\hspace*{2.7cm}Этой ночью пленительно жаркой\\
\hspace*{2.7cm}Будем мы наслаждаться с тобой\\
\hspace*{2.7cm}Руководством по газовой сварке$\ldots$\\
\\
\hspace*{2.7cm}Жизнь пройдет, словно миг.\\
\hspace*{2.7cm}Я у милого тела заплачу\\
\hspace*{2.7cm}И к могилке твоей принесу маховик$\ldots$\\
\hspace*{2.7cm}С рукояткой продольной подачи!\\ 
}
\vspace{0.1cm}

Уроки труда, особенно токарка, вспоминаются с~такой~же теплотой, как и география, и уроки Циркуля. Вместе это\mdash три отдушины, три кита школьных, не~дававших совсем упасть духом и поддерживавших баланс добра и~зла, так скажем. 

По поводу вышесказанного, что \textit{когда\sdash то тут было всё совсем по\sdash другому}\mdash под окнами кабинета труда была небольшая будочка, где должны были храниться газовые баллоны для подачи газа в кабинеты химии и~биологии. Осознаёте масштаб? Не спиртовки или сухое горючее, а~настоящие газовые горелки при проведении опытов на~уроках! Парты в классах химии и~биологии были жёстко прикреплены к полу, и к ним были подведены газовые трубы. Потом на парте был тройник и~два газовых краника\mdash на каждого ученика за~партой. Когда мы учились, это всё уже не функционировало, отчего также становилось обидно и~горько\mdash воистину следы древней, более высокоразвитой цивилизации$\ldots$

$\ldots$Конец года после 9\sdash го класса, школьный спортзал залит солнечным светом. Ощущение неотвратимо наступающих перемен, наступления момента подведения какой\sdash то невидимой черты, после которой начнётся другая жизнь. Счастливее ли? Лучше ли? Никто не знает заранее. Что\sdash то, естественно, станет лучше, что\sdash то хуже. Впрочем, как и~всегда и~во всём, в любом деле. В целом, счастливыми свои школьные года я назвать не могу точно, школа вызывала отвращение. Лишь эти три отдушины, о которых рассказано, были каким\sdash то уголком спокойствия и душевного равновесия. Остальное вызывало животное отторжение и~неприятие. Плюс, конечно, время было такое, переходное для нашей страны$\ldots$ всё это понимается уже позже, сейчас. Но~в~прошлое не попасть и ничего не исправить, что было\mdash то~было$\ldots$  

\begin{center}
	\psvectorian[scale=0.4]{88} % Красивый вензелёк :)
\end{center}

$\ldots$Ранним утром Адмирал внезапно обнаружил себя спящим в шортах и без тельняшки у костра на берегу Чагоды. Носок на одной ноге почему\sdash то отсутствовал. Полная перезагрузка. Сознание было чистым\sdash пречистым, а разум светлым\sdash пресветлым и~надо было вести эскадру вперёд.

\begin{center}
	\psvectorian[scale=0.4]{88} % Красивый вензелёк :)
\end{center}
