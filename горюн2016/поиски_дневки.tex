%\chapter{Поиски днёвки. 06.08.16} 
%\chapter{Лукобор. 06.08.16} 
\chapter{Лукобор} 
%\corner{64}
\vepsianrose

\vspace{5mm}
Ночка была не самая тёплая, но, тем не менее, я~не~замёрз. Новый надувной матрас надёжно изолировал от~прохладной земли\mdash всё\sdash таки уже август, хоть и~намного теплее, чем в прошлом году, когда я шёл без~матраса. Я~выполз из~палатки не слишком рано, народ тоже потихонечку начал раскачиваться. У меня отчего\sdash то возникло лёгкое ощущение неотвратимо приближающегося конца похода, хотя на самом деле, только\sdash только минула его половина. Видимо, это потому что в прошлом году я антистапелился в Загривье, мимо которого мы должны были сегодня проплыть. 
 
Итак, предстояло найти уникальное место для днёвки. Чтобы и песчаная коса была для купания, и лес красивый сосновый, и ровная полянка для палаток, и чтобы без травы высокой, и чтобы столик был желательно и чтобы, и чтобы, и~чтобы$\ldots$  

\newpage
После завтрака мы неторопливо и аккуратно спустили байдарки на воду с высокого обрыва. Потом привычно погрузились и отчалили. В скорректированном плане на~сегодня мы должны были пройти Загривье\mdash место моего прошлогоднего антистапеля\mdash и идти дальше, но~не~далее Бывальцево, поскольку после него характер берегов Чагодощи снова поменяется\mdash песчаные пляжи пропадут, о чём я вычитал в интернете и соотнёс с топокартой. Погода радовала\mdash небо снова расчистилось, ярко светило солнышко. Белые кучевые облака медленно бороздили высокую августовскую синь. Жарко. Я снова сидел в~байдарке, лишь прикрыв плечи шемагом, а голову панамой, постоянно смачивая её водой. 

Около деревни Загривье мы остановились на~широченном песчаном пляже\mdash там же, откуда я в~прошлом году пошёл на разведку выброски. Тут, судя по старой топокарте, должен был быть паром. Остатков парома я также не нашёл, как ни искал. Решили помыться в реке и передохнуть. Течение в этом месте было довольно приличным, так что купались аккуратно. Все намылились мочалками и с громким <<плюх>>, оставляя на поверхности воды мыльную пену и пузыри, заныривали на глубину промыть свои тела. Прохладная водичка приятно бодрила! Покончив с водными процедурами, немного перекусили. На~берег приехали местные на машинах. Я стал расспрашивать одного дедка о порогах перед Лентьево. Как оказалось, он слышал о них, но никогда сам лично там не бывал$\ldots$ что ж, посмотрим скоро сами, что это за пороги такие. 

Отчалив от пляжа, прошли протоками около Загривья и повернули вслед за руслом направо, оставляя слева высоченный песчаный обрыв с беседкой наверху. Фотографии этого места вышли просто умопомрачительными. Трепет охватил меня\mdash дальше снова первопроход\mdash если маршрут до этого места был мне знаком, то дальше мы снова идём в неизвестность! Ура\sdash а\sdash а!

Мы с Геннадичем замешкались на песчаной косе, делая фотографии, и сильно отстали от бригады. Вышли на~широченный плёс, увидели нашу флотилию далеко впереди, лопатящую веслами. Налетел встречный ветер, стало тяжело идти. Расстояние между нами и авангардом флотилии стремительно увеличивалось. Надо было исправить ситуацию, ибо негоже адмиральской байдарке идти замыкающей. Зарядил сигнал охотника зелёной ракетой и~пальнул вперёд под углом градусов 45, чтобы бригада услышала выстрел, заметила ракету и подождала нас. Они, к счастью, догадались и подождали. Дальше пошли более менее стройным кильватерным строем.

Песчаные пляжи и обрывы становились всё краше и~краше, но, увы, они скоро должны были закончиться близ деревни Бывальцево, о чём, как уже говорилось, я знал из отчётов и спутниковых снимков. Так что место на днёвку надо было в любом случае найти на этом промежутке до Бывальцево. Плыли медленно и даже, я бы сказал, лениво. Два раза причаливали оценить место\mdash нам всё казалось не очень. То не было пляжика, то~место для лагеря слишком далеко от воды, то~ещё что\sdash нибудь. Привередничали.

И вот, пройдя примерно половину пути от Загривья до Бывальцево, мне показалось, что на левом берегу сразу за прибрежными кустами должна быть хорошая поляна без~травы. Мы высадились на песчаный берег и~взобрались по~невысокому подъему наверх. Стоянка была отличная! Поляна ровная, стол, скамейки, куча места для палаток, кое\sdash какие дровишки валялись, мусора не было. И огромная песчаная коса есть, как и хотели. Решили, что, возможно, это лучшая стоянка из найденных нами в этом сплаве. Координаты\mdash \CoordsChagodoschaSixteenDnevka. На~правом берегу чуть поодаль в глубине леса находилось урочище Лукобор, как оно было обозначено на карте. Что~там раньше находилось, мне найти не удалось. Возможно какая\sdash то деревня, потому что место тут и вправду очень красивое\mdash река делает затяжной S\sdash образный поворот и имеются отличные песчаные косы.  Ясное дело, мы остались на днёвку.

Делали всё лениво, потому что времени было вагон\mdash на стоянку пришли часов в 14 или 15. Как обычно, перетаскали вещи, байдарки затащили и перевернули вверх днищем, развернули лагерь. Над столом натянули тент, памятуя о~внезапно настигшем нас вчера дожде. Костёр сделали чуть в стороне, чтобы не подпалить тент. Место под палатки расчищали от безумного количества шишек\mdash видимо тут кормятся б\'{е}лки, но самих пушистых непосед мы не видели. 

Прошел лёгкий дождичек и появилась радуга. Природа радовала нас, одним словом. Паша и С.Ю., тем временем, отправились рыбачить ниже по течению. Дима и Геннадич обсуждали рок\sdash группы и музыку вообще, а Ваня счастливо участвовал в ничегонеделании. Я тоже отдыхал, слоняясь по~лагерю с фотоаппаратом. К середине похода мышцы уже окончательно перестали ныть, привыкнув к физнагрузке, а~кожа на руках огрубела от весла, потому что я грёб по~привычке без перчаток. 

С днёвкой всё сложилось как надо\mdash отличнейшее место в полнейшей глуши$\ldots$ связи и той нет. Я включил телефон и попытался отправить жене смску, но сообщение не проходило. Встал на пенёк, связь лучше не стала. В итоге, походив по берегу и поискав связь, я бросил это бесперспективное дело, поскольку до ближайшего населенного пункта\mdash Бывальцево\mdash километров 8\thinspace\nobreakdash--\thinspace 10 по~прямой, да~и~не~факт, что там есть вышка сотовой связи.

Вечером заготовили дров\mdash притащили ещё брёвен и распилили их. Дима вызвался приготовить ужин, а~остальные стали ему помогать. Наш шеф\sdash повар порадовал нас кус\sdash кусом с тушёнкой, салатом с тунцом(!) и~традиционным супом. Вот это да! Настоящее походное пиршество! Приготовив ужин, сидели застольничали допоздна, благодарили нашего шеф\sdash повара. Стемнело и в ход даже уже пошли фонари. Геннадич наигрывал на губной гармошке. Я блаженно сидел на стульчике, прислонившись к дереву, и пускал вверх колечки дыма из~трубки. Пожалуй, в этот вечер я, возможно впервые за этот сплав, по\sdash настоящему прочувствовал атмосферу уединения и~отречённости от цивилизации. Эти сладкие минуты! Дорожу~ими. Собственно, ради этих мгновений и~затевается всё вот это вот. Они проносят, как будто тайком, нелегально, в душу спокойствие, удовлетворение, сознание дикости, первобытности, естества. Именно эти минуты очищают сознание и ради них мы идём в походы, сплавы$\ldots$ 

Спать расползлись очень поздно\mdash всё равно завтра выспимся! Ведь завтра долгожданная днёвка! Каждый из~команды наверно уже много раз тихонько проклинал меня и~С.Ю., как командиров, что мы никак не остаёмся на днёвку. Но от длительного ожидания место, выбранное сейчас нами, становилось только лучше в сознании притомленного туриста\sdash водника.

\begin{center}
	\psvectorian[scale=0.4]{88} % Красивый вензелёк :)
\end{center}
