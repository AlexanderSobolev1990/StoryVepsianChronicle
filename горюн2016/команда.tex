\chapter{Команда} 
\corner{64}

С командой всегда, увы, проблемы. Народ нынче очень любит комфорт. <<Где мой утренний кофе в постель?>>,\mdash это про них. А также простое непонимание, зачем всё это нужно, если есть <<нормальный>> отдых, например на курорте, лучше если заграничном. Действительно, зачем? Это вопрос из разряда\mdash если вы не понимаете, то объяснить это практически невозможно. Я ни в коем разе не отрицаю комфорта ленно\sdash пляжного отдыха и тоже не против так отдохнуть, что периодически и делаю на радость жене, но$\ldots$~сплав почему\sdash то оставляет в душе больше впечатлений и ярких красок. Почему? Может всё дело в той безграничной свободе, ощущение которой появляется вдали от людей, а может дело в самой удаленности от людей. А может в <<дикости>>? Человек\thinspace--\thinspace человеку, как известно, зверь. Как бы ни хотели привить нам обратного, но по природе своей человек\mdash индивидуалист. И может кооперироваться с другими только ради достижения определенных целей. Соответственно, чем сильнее развито общество, тем выше степень кооперации, выше потребность в сплоченных действиях, и выше, стало быть, плотность населения. Плотность, которая порой так утомляет нас. Забитый транспорт\mdash автобусы, электрички, метро, автотрассы. Осознаешь, что по\sdash другому жить уже, скорее всего, никогда не получится\mdash чёртова цивилизация всё равно возьмет своё. ­

Человек слишком отдалился от природы, причем настолько, что часто утратил всякую в ней потребность\mdash нас поглотили <<каменные джунгли>>. И сбежать от всего этого в дикую природу, пусть на неделю и полностью экипированным, для некоторых уже проблема, хотя, казалось бы\mdash чего в этом такого? Не~отрицаю, можно <<сбежать>> и на необитаемый остров посреди Индийского, например, океана и чтобы там было <<всё включено>>\mdash вопрос только в средствах. И людей там тоже не будет в диком количестве, зато шик, блеск, красота. Однако, вопрос стоимости этого мероприятия мы тут обсуждать не будем$\ldots$ а байдарка и палатка же\mdash это единственные дорогостоящие необходимые вещи для сплава. Остальное худо\sdash бедно можно собрать и наскрести. И~в~этом~что\sdash то есть. 

Сплав обходится значительно дешевле по средствам, если уже имеется хотя бы часть снаряжения$\ldots$~но только ли всё дело в отсутствии необходимого количества денег? Когда под рукой есть любое снаряжение, любая вещь и куча возможностей\mdash это, не спорю, хорошо и классно, но$\ldots$~немного скучно. А когда ты сам сделал оттяжки для тента или навеса, модернизировал или даже сам сшил палатку, сам изготовил чехлы или какое\sdash то другое снаряжение\mdash это и есть то самое состояние, когда ты понимаешь\mdash а вот так вот, хе\sdash хе!\mdash и никому за это не заплатил! Фиг вам всем с маслом, проклятые капиталисты! Горите в аду, дяди Сэмы, продающие палатки по цене, равной половине и выше зарплаты инженера!  Ни капли не заплачу лишнего\mdash хватит вам всяких НДС, налога на недвижимость и движимость и прочих идиотских правил вашего идиотского и проклятого капиталистического мира! Да здравствует свобода! К~чёрту все правила, к~чёрту идеологию общества потребления, к~чёрту цивилизацию, к~чёрту~всё! Есть только река, ты, байдарка и больше ничего не отделяет тебя от первобытности и естества! 

Да, скорее так\mdash всё дело в огромном количестве людей в больших городах и идиотских правилах нашей повседневной жизни, к которым мы привыкли и не замечаем их, а также в этой поганой атмосфере мирового капитализма и общества потребления. И когда ты можешь сбросить с себя оковы цивилизации, оковы большого города и прочей требухи, хотя бы на время\mdash это бесценно. Это сродни тем детским моим впечатлениям от беспечного нахождения летом в деревне у родителей за 100 км от Москвы\mdash там, где, казалось, время остановилось, где плохо с электричеством, а телевизор ловит пару программ. Но зато вокруг бескрайние леса, по которым носишься на велосипеде и открываешь всё новые и новые лесные тропы$\ldots$ 

Вообще, говоря о сплавах, не надо думать об этом действе, как о чём\sdash то безумном и из ряда вон выходящем в плане отсутствия обычных комфортных нам условий жизни. Кофе с утра сварить\sdash таки можно. Следует понимать, что группа берёт с собой всё необходимое не только для выживания в дикой природе, но и комфортного размещения. И чем больше группа, тем больше она может позволить себе взять бытовых предметов, создающих этот комфорт. Например, сюда можно отнести такие вещи как складные стулья или даже складной стол, большой тент от дождя, комфортный надувной матрас для сна на каждого члена группы и т.д. Словом, организовать себе комфортные условия\mdash дело достаточно простое и нехитрое, главное было бы желание. 

Люди в сплав конечно должны браться проверенные и лучше не первый год друг с другом знакомые. Это не те <<чужие>> из автобусов, электричек, метро, а <<свои>>. <<Своим>> можно многое простить, а они могут многое простить тебе. Но самое главное\mdash ты в них уверен. <<Своих>> в группе должно быть большинство и тогда проблем не будет. Но собрать, сколотить такую команду <<своих>>\mdash это всегда проблематично. Поэтому приходится идти на уступки и в ход идут <<знакомые>> и <<знакомые знакомых>> и <<знакомые знакомых знакомых>> и т.д. Впрочем, при должной подготовке всех и каждого, такой вариант, как показала практика, тоже оказывается приемлемым.

На этот раз команду стали собирать сильно заранее\mdash ранней весной. Однокурсник С.Ю., Владимир, выразил желание пойти вместе с женой. На работе откликнулся Паша, хотевший ещё с прошлого года пойти с нами. Итого получалось пятеро человек, но вербовка не останавливалась. К концу весны стало понятно, что Владимир прибавляет нам ещё двоих\mdash Дмитрия с сыном\sdash первоклассником Ваней. 

Я тоже долго агитировал всевозможных знакомых, но откликнулся только один товарищ по институту\mdash Саня aka Геннадич. Он, как и я, закончил радиотех и работает по специальности. Правда специализации у нас разные\mdash у него биомедицинские аппараты и системы. Кроме того, Геннадич играет на бас\sdash~гитаре, а музыкант в команде это всегда большу\sdash у\sdash ущий плюс. Кто, как не менестрель, скрасит погодные условия и тронет сердца туристов\sdash водников своими аккордами?

С Геннадичем мы встретились в пивной на Савёловской и обсудили предстоящее мероприятие. Мне показалось, что его душевный настрой в ту весну требовал чего\sdash то такого непонятного и неизведанного, чем для него стал предложенный мной сплав. Таким образом, сложилось полное совпадение интересов, что не могло не радовать. Ещё Геннадич обещал взять с собой губную гармошку\mdash и тут я понял, что всё идет как надо! Кроме того, за него я был спокоен\mdash хоть мы и не учились бок о бок в одной группе, но достаточно часто пересекались во время учебы и я знал, что Геннадич\mdash тот человек, который не подведёт в трудной ситуации. Я передал ему полный список необходимого снаряжения, и мы продолжили подготовку к нашему действу\mdash предстояло закупить продовольствие на всю команду.

Традиционно заранее была произведена контрольная закупка тушёнки на продовольственной базе в Перово. Меня постигло большое разочарование\mdash снововская белорусская тушёнка испортилась, а отечественные образцы вообще оказались кошмарными. В итоге, перепробовав сортов 6\sdash 7, остановились на белорусской тушёнке жлобинского комбината. В выходные Геннадич встретил нас с С.Ю. в Новокосино на машине и мы отправились в Перово\mdash закупили ящик говяжьей и банок десять свиной тушёнки, а также кое\sdash какие мелочи. Остальные продукты на всю команду решили организованно закупить в крупном магазине в Реутово, поскольку на продбазе отпускали только оптом, что многовато для нашей небольшой команды.  

Хронометрист и Фотограф в сплаве\mdash это может быть любой и каждый по своему желанию, но у Адмирала хронометраж должен быть под контролем обязательно, чтобы точно знать ответ на вопрос: <<Сколько нам ещё осталось до стоянки?>>, потому что шуточки в стиле <<пока грести можешь>> плохо воспринимаются экипажами после многочасовой гребли под некстати начавшимся проливным дождём. 

Медик конечно тоже желателен, аптечка обязательно должна присутствовать. Геннадича я как раз определил Медиком\mdash зря что ли <<Биомедицинские аппараты и системы>>? Должность Завхоза мы поделили с С.Ю. на двоих, составив список снаряжения и продовольствия. Остальные хозяйственные должности вроде Кострового и Повара, я считаю правильным распределять между всеми по\sdash очереди на добровольных началах, а также учитывая возможности и способности каждого. 

Помимо адмиральской и разных хозяйственных, есть ещё флотские должности на каждой байдарке\mdash Капитан и Матрос. Капитан в байдарке обычно более опытный и сидит на корме, а Матрос спереди. При прохождении порогов некоторые авторы туристической литературы советуют Капитану поменяться местом с Матросом и сесть вперёд\mdash так лучше видно воду, пороги собственно, да и Матросу не так страшно. Не знаю насколько это верно, потому что выгребать, исправляя курс байдарки, лучше всё\sdash таки сидя на корме, да и серьёзных порогов пока что проходить не доводилось, только множество коварных шивер на Лиди. У нас же жестоких порогов на маршруте не планировалось, так что посадка у всех была снова классическая. 

Со средствами сплава на этот раз получалось так: у меня\mdash <<Таймень>>  либо двушка, либо трёшка с ПВХ <<шкурой>>, у С.Ю.\mdash каркасно\sdash надувная байдарка <<Гарпун\sdash 2>>, а остальные 2 экипажа хотели взять напрокат тоже что\sdash нибудь каркасно\sdash надувное. После прошлогоднего сплава мне пришлось починить лопнувший второй шпангоут, ибо новый покупать\mdash 6000 руб. (!), выпрямить пару погнутых стрингеров и собственно всё\mdash можно идти. <<Гарпун\sdash 2>> ждал своего часа целый год\mdash представляю, как хотелось быстрее опробовать его в деле! Насчёт других экипажей мы не волновались\mdash они заверили, что сами разберутся с прокатом необходимого снаряжения. 

И конечно снова не обошлось без Закона <<Кто\sdash то не сможет пойти>>. Владимир с женой выбыли из списка нашей команды буквально за пару дней до старта, когда уже были закуплены продукты и подготовка была фактически окончена. Но~что поделать\mdash такова Жизнь и Законы Сплава. Получалось, что пойдём без женщин. Моя жена тоже не смогла пойти со мной в этот раз\mdash снова не удалось совместить наши отпуска по времени$\ldots$~я был расстроен. Но зато ей удалось съездить на юг к морю\mdash позагорать и пропитаться солёным морским воздухом, что также немаловажно. А мне так хотелось, чтобы она тоже увидела собственными глазами эти песчаные косы и обрывы, пронеслась со мной по водным просторам$\ldots$ но что поделать, придётся снова надеяться, что мы наверстаем это в следующий раз. 
Таким образом, наша команда представляла собой следующее:
\begin{enumerate}
	\item <<Таймень\sdash 2>>: Я в качестве Адмирала эскадры, Геннадич\mdash мой Старпом и Медик команды.
	\item <<Гарпун\sdash 2>>: Сергей Юрьевич\mdash  Капитан байдарки, Паша\mdash наш Рыбак (потому что не только взял удочку, но и умеет ловить рыбу) и Матрос С.Ю.
	\item <<Шуя\sdash 2>>: Дима в качестве Капитана байдарки и его сын Ваня в роли Юнги.
\end{enumerate}

С командой мы определились, оставалось только дождаться дня отплытия, что само по себе непросто\mdash хочется же поскорее вырваться на просторы! Последние дни на работе перед отпуском были откровенной пыткой\mdash я почти каждый день мысленно перебирал всё снаряжение, соображал, чего бы мы могли забыть или не учесть, рассматривал картографический материал$\ldots$

\begin{center}
	\psvectorian[scale=0.4]{88} % Красивый вензелёк :)
\end{center}
