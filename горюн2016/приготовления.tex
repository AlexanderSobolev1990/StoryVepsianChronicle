\chapter{Приготовления} 
%\corner{64}
\vepsianrose

Необходимо было собрать снаряжение, а точнее переправить его из деревни в Москву. Хорошо, что всё необходимое у меня уже есть с прошлых сплавов\mdash  байдарка, палатка, штормовка, бахилы химзащиты вместо сапог, сапёрная лопатка, двуручная пила, котелки, костровые крючки, фляги, тент от дождя, гермы и разные другие полезные мелочи. В~этом году только прикупил надувной матрас и налобный фонарик, поскольку старый пал смертью храбрых, просто перестав работать на ровном месте. Хотел ещё купить складной полноразмерный стул с подлокотниками, но его размеры в сложенном состоянии меня как\sdash то совсем не впечатлили, поэтому решил довольствоваться, как и прежде, маленьким раскладным стульчиком.

Итак, комплект костей и шкура <<Таймени\sdash 2>> были переправлены из деревни в город и заботливо разложены по комнате под негодование жены насчёт всего этого <<хлама>>. Собственно, вот и вся подготовка. За~снаряжение в этом году я был спокоен, а прикупив ещё 3~(три!) пузырька клея и пачку заплаток для ПВХ шкуры байдарки на всякий случай, успокоился совсем. Пусть будут! Не хочу <<попасть>> с пробоинами так, как <<попал>> в прошлом году на Лиди. До сих пор помню свой шок от десяти пробоин в один день! %, бр\sdash р\sdash р

Геннадичу и Паше я дал свой список снаряжения, с~С.Ю. мы всё обговорили устно, а наш третий экипаж собирался самостоятельно, что несколько беспокоило меня, но я решил ничего не предпринимать\mdash всё, что они лишнего возьмут с собой, повезут на своей же байдарке. По снаряжению было требование, чтобы вещи на одного человека влезали в один гермомешок, который в байдарке будет в качестве сидения.

Параллельно со сбором снаряжения, прорабатывался вопрос заброски на маршрут. Первоначально планировали сделать как при забросе на Лидь в прошлом году\mdash добраться поездом до Череповца, там закупиться продуктами, сесть на Арханегельский поезд и рвануть уже не до~Заборья, а~до~Ефимовского, что чуть дальше. Там в~Ефимовском взять машину и~доехать до~Вожанского озера, на~берегу которого решили устроить стапель. 

Был и альтернативный вариант заброски через Волхов\mdash также на поезде, но с пересадкой на электричку(!), которая стоит на станции пересадки всего одну минуту(!!) и,~вдобавок, имеет неизвестно сколько вагонов(!!!) с~неизвестно~какой~наполненностью~местными~дачниками(!!!!). Такой вариант был чрезвычайно авантюрным и я бы согласился на него, если бы шёл как в прошлом году\mdash одной байдаркой. Но~в этом году всё иначе\mdash команда у~нас разношёрстная, есть ребенок, да~и, чего уж там, хотелось чего\sdash то более определенного. А~то вдруг отменят электричку или не сможем в неё впихнуться и застрянем в Волхове$\ldots$ а~это довольно далеко от стапеля.
 
Короче говоря, этот чрезвычайно заманчивый своей авантюрностью вариант я оставил про запас и мы, с подачи Владимира, который на тот момент ещё был в списках команды, стали искать микроавтобус для заброски. Немного с шиком, так сказать. В пересчете на одного человека заброска выходила не намного дороже, чем поездами, зато комфорта несравненно больше\mdash заберут прямо от дома и привезут к реке\mdash  красота! Выбрасываться с маршрута также решили микроавтобусом. Поиск машины мы С.Ю. взяли на себя.

Последний месяц перед отплытием проходил плохо\mdash мне всё казалось что я чего\sdash то не учёл, чего\sdash то забыл. Потом, когда мы закупили продукты и договорились с~водителем о~заброске, стало полегче. Но все же дни до стапеля были мучением\mdash я себе места не~мог найти\mdash хотел быстрее вырваться на природу, оторваться от привычной жизни. Сутками просматривал карты, спутниковые снимки маршрута$\ldots$~почему так? Может потому, что я не умею как следует отдыхать. Я имею в~виду отдыхать в~обычной жизни. Всё хочется мне чего\sdash то такого$\ldots$~дикого и первобытного\mdash  требующего полного отречения от привычного и перестройки мышления. Хочется полного, полноценного, полнейшего очищения сознания от~тяготящего и окружающего города. Неужели так не устраивает меня моя жизнь? Вроде взглянешь на неё\mdash всё хорошо. Но стоит только начать копаться$\ldots$~ останавливаешься и говоришь себе: 

\diagdash Хватит, надо жить дальше. 

И живёшь. Точнее\mdash существуешь от похода до похода. А заноза странствий сидит в мозгу и давит, давит, давит, требует своего:

\diagdash Рвани в кругосветку, да под парусом!$\ldots$

\diagdash Но$\ldots$\mdash пытаешься было возразить ты занозе.

\diagdash ПОД. ПА-РУ-СОМ!$\ldots$

Жалею сейчас, что книга <<Практика вольных путешествий>> \cite{Кротов}, не попала в мои руки раньше, скажем в институте. Сколько я бы мог совершить этих вольных путешествий во время летних каникул$\ldots$ А что\mdash сдал сессию и свободен как птица в небесах. Но не такой я человек, чтобы ездить зайцем в товарных поездах, бегать от~охранников и перемещаться по стране автостопом, неизвестно где ночуя (кто знаком с ранним творчеством Кротова\mdash поймёт о чём это). Это сейчас я бы мог спланировать что\sdash то подобное, уже имея за плечами опыт организации сплавов, но$\ldots$~момент, кажется, упущен. Это~всё хорошо в пору безрассудной студенческой молодости, увы.

Помню, в детстве моими любимыми томами в Советской Детской Энциклопедии~\cite{ДетскаяЭнциклопедия} были том~1 и том~5\mdash <<Земля>> и~<<Техника>> соответственно. В первом меня привлекали статьи о климате и погоде, о природе и картографии, о походах и великих путешественниках. Во пятом мне нравилось решительно всё\mdash всё, что было связано с~техникой\mdash машины и поезда, самолёты и вертолёты, корабли и подводные лодки, всевозможные станки, механизмы, аппараты. Так оно примерно и сложилось по жизни\mdash образование и профессия технические, а тянет меня постоянно на~природу, в глушь. И этот природный зов приходится удовлетворять такими вот, казалось бы с одной стороны, <<дикими>> сплавами по водным просторам нашей необъятной Родины.

Как бы там ни было, к заброске и путешествию мы подготовились основательно. Оставалось только мучительно пережить последнюю рабочую пятницу на работе. За сутки до моего отъезда вернулась с юга жена. Мы хорошенько позастольничали в выходные всей семьёй\mdash отметили не~только приезд загорелых отдыхающих с юга, но и мой предстоящий сплав. 

Конечно жалко, что на такой короткий срок удалось свидеться с женой после её возвращения с юга, но$\ldots$~заноза странствий страшная штука\mdash она зовёт в путь! Скорее! На~просторы! Я~ждал этого всю долгую осень, всю снежную зиму и зелёную весну! Ветер расправляет паруса, рука уже тянется к веслу, а взгляд затуманен в предвкушении безграничной пьянящей свободы!

И вот, наконец, свершилось то, <<о необходимости чего всё время говорили большевики>>\cite{ЛенинПСС}\mdash в ночь с~воскресенья 31~июля на понедельник 1~августа наша команда выехала из~Москвы навстречу приключениям!

\begin{center}
	\psvectorian[scale=0.4]{88} % Красивый вензелёк :)
\end{center}
