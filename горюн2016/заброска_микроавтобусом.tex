\chapter{Заброска микроавтобусом} 
%\corner{64}
\vspace{-1cm}
\vepsianrose

Водитель на микроавтобусе приехал вовремя, к 18:30, как и договаривались. Я, уже расхаживающий в нетерпении по квартире и облачённый в штормовку, рванул таскать вещи к лифту. Тёща и тесть, гостившие у нас, бросились помогать. Жена тоже включилась в процесс. 

%Уже внизу, у~машины, я~слегка недовольно повысил голос на жену\mdash она какого\sdash то лешего стала поднимать тяжеленные вещмешки со снаряжением! Она обиделась, что сильно подпортило мне настроение. Сам виноват, поскольку не пресёк попыток ненужного героизма\mdash я бы и сам всё прекрасно перетаскал, тем более такие тяжести\mdash в~армейских вещьмешках ехало всё продовольствие на всю нашу команду из 6~человек. 
%\newpage
%
%Как бы там ни было, вещи перетащили, погрузили и отъехали. Прощание вышло смазанным и нехорошим. От~этого сразу стало гадко на душе. На кой чёрт она все эти тюки бросилась таскать?\mdash конечно от большой любви ко мне, от желания помочь. А~я?
%
%\diagdash Брось! Что ты тяжести таскаешь?! 
%
%\diagdash Так много же$\ldots$
%
%\diagdash Так тяжело же!
%
%\diagdash Ну и неси сам!

%Тьфу ты чёрт! Но время вспять не повернуть$\ldots$ Микроавтобус уже несётся вдаль\mdash собирать остальную команду, а на душе у меня гадко и настроение испорчено. Отгоняю от себя плохие мысли тем, что я\mdash Адмирал Сплава, а Адмирал должен быть выше личных переживаний$\ldots$ но~куда же от них денешься? Тем временем встретились с~Геннадичем. У него из личных вещей был один гермомешок и палатка, что очень радовало, потому что я вспоминал горы вещей, наваленных в байдарку в прошлогоднем сплаве$\ldots$~тут такого явно уже не предвиделось.  

Спустя минут 15 вещи были загружены, и я, попрощавшись, уселся рядом с водителем спереди. Выезжаем! Микроавтобус понёсся вдаль\mdash собирать остальную команду. Какие\sdash то личные душевные переживания пришлось мигом отложить в сторону, поскольку теперь я\mdash Адмирал Сплава, а Адмирал должен быть выше личных переживаний, поскольку скоро под моим началом будет ещё 2 байдарки помимо моей собственной. Тем временем встретились с~Геннадичем. У~него из личных вещей был один гермомешок и палатка, что очень радовало, потому что я вспоминал горы вещей, наваленных в байдарку в прошлогоднем сплаве$\ldots$~тут такого явно уже не предвиделось.  

Далее должны были забрать С.Ю. и Пашу. Завернув во двор, мы почти сразу их увидели. Их байдарка была упакована в~не очень большой рюкзак, а~вещи в небольшие сумки. С.Ю. и Паша, как и обещали, взяли собой несколько удочек. Мы~быстро погрузили их снаряжение и помчались по~вечерней Москве, забирать Диму и~Ваню. Въехать во двор их сталинки оказалось легче, чем развернуться там и выехать обратно\mdash водителю пришлось делать это очень аккуратно и~в~несколько приёмов. У Димы и Вани оказалось три больших гермомешка и ещё немного сумок по мелочи. Я мысленно представлял, как это всё влезет, вернее, НЕ влезет в их надувную байдарку$\ldots$~ведь в~памяти у старшего поколения сидит ассоциация байдарки с <<Салютом>> или <<Тайменем>> советского производства, где~места гораздо больше, а тут всё\sdash таки КНБ и пространства в~ней$\ldots$~откровенно говоря катастрофически мало.

Задние ряды микроавтобуса все оказались завалены тюками со снаряжением. Я ходил около открытых задних дверей авто и недоумевал, откуда столько взялось вещей. С.Ю., увидев мою растерянность, пошучивал: <<Лишнее сожжём!>>. Я не разделял столь радикального подхода и~предлагал лучше что\sdash нибудь не брать$\ldots$~В итоге, конечно, всё взяли с собой.

Супруга Димы весело провожала нас и желала хорошо отдохнуть. В голове у меня промелькнуло\mdash как\sdash то она очень легко провожает мужа и ребёнка на такое, вообще говоря, рисковое мероприятие. Совершенно не волнуется и~не переживает, по крайней мере внешне. Позже я понял почему\mdash для неё ж это тоже отдых\mdash Ваня оказался бойким любознательным первоклашкой из разряда <<хочу всё знать>>. Но это даже хорошо\mdash наше путешествие оставит в его памяти наверняка очень яркий след, и он сможет во втором классе хвастаться, что ходил с мужиками в настоящий сплав, и рассказывать всякие небылицы. Я завидовал ему белой завистью\mdash в моём детстве таких приключений не~было$\ldots$~может потому\sdash то я это всё навёрстываю сейчас, да~так, что не могу остановиться? Мне так всегда хотелось уйти в такой отрыв на природу\mdash чтобы был костёр, палатка на берегу реки, глубокое ночное небо сверху и миллиарды звёзд, отчуждение от города, наконец. А это никогда не~получалось. Впрочем, я отвлёкся$\ldots$

Погрузившись, мы окончательно стартовали и взяли курс на Вожанское озеро. Заброска в ночь. Романтика дороги! По моим расчётам, на месте мы должны были быть к~5\thinspace\nbdash\thinspace 6 часам утра. Экипажи расселись в салоне, откинулись в~креслах и уже при выезде из Москвы послышался чей\sdash то раскатистый храп.

Я же расположился на переднем сиденье рядом с~водителем, разложив сбоку планшет (в~старом понимании этого слова!) с~картами (бумажными!!) и GPS (на~всякий случай!!!). Оказалось, что переднее сиденье у~этого микроавтобуса сдвоенное и назад не откидывается, поэтому поспать с комфортом у меня не выйдет. Ну ничего, переживу как\sdash нибудь. 

Выбирались из Москвы долго. Из\sdash за того, что я~сел спереди, потрепаться с командой не вышло. Команда же, в свою очередь, перезнакомилась, поболтала и вскоре задремала под звуки радио. У меня был припасён термос с~чаем\mdash я собирался как можно дольше не спать и на всякий случай поддерживать беседу с водителем, чтобы того вдруг не одолел сон. 

Вырвавшись из вечерней Москвы, объятой кошмарными пробками, мы долетели по Ленинградскому шоссе до Твери и свернули на Бежецк. Объездная дорога осталась при подъезде к Твери слева. Мы же проехали вдоль Волги, к бассейну которой относятся реки, по~которым нам предстоит сплавиться. Тверь почему\sdash то ничем не~запомнилась, кроме как мостом через Волгу. 

Дальше предстояло добраться до Бежецка. Дорога была нормальной, ехали быстро. Чай в термосе постепенно убавлялся, а за окном тёмная августовская ночь спускалась с небес. В~Бежецке пересекли реку Мологу. После Красного~Холма свернули налево\mdash на Устюжну через Молоково$\ldots$~как оказалось, это было большой ошибкой\mdash навигатор, который нас так повёл, не учитывал, видимо, что класс дороги тут на порядок ниже, чем через Весьегонск по трассе Р\sdash 84. Через Весьегонск хоть и дальше на несколько километров, но,~наверное,~по более менее нормальной трассе, думалось мне\mdash я этот момент упустил при подготовке заброски и~теперь проклинал себя на чём свет~стоит.

%\textit{(примечание от 2017 года\mdash дорога через Весьегонск на Устюжну и вовсе грунтовая в реальности, так что из двух зол мы выбрали меньшее, хоть и думали наоборот. В районе Весьегонска состояние дорог на 2017 год очень плохое.)}

\renewcommand*{\thefootnote}{\fnsymbol{footnote}}
В итоге, после Молоково, дорога стала не просто ужасной, а кошмарной\footnote{примечание от 2017 года\mdash дорога через Весьегонск на Устюжну ещё хуже\mdash она грунтово\sdash гравийная в реальности, так что из двух зол мы выбрали меньшее, хоть и думали наоборот. В районе Весьегонска состояние дорог на 2017 год очень плохое.}. Временами мы ехали всего лишь 20\thinspace\nbdash\thinspace 30 км/ч~(!!!)\mdash выбоины были такими, как будто тут только что отбомбилось всё чёртово Люфтваффе и~ВВС~США! И это, на минуточку, даже меньше чем в~500~километрах от Москвы. Я ехал и чертыхался, но~ничего поделать, понятно, было уже нельзя. Не возвращаться же обратно на дорогу через Весьегонск. Так и ехали потихонечку$\ldots$ 

Бригада моя, на удивление, даже не проснулась от дикой тряски и мирно посапывала в креслах. Меня тоже клонило в сон, но термос с чаем очень здорово выручал. Встречались участки со свежеположенным асфальтом. Кусками. Метров по 50\thinspace\nbdash\thinspace 100. Подумалось\mdash если проехали такие ухабы ужасные, то до чего должно было быть всё разбито на тех залатанных участках, если там соизволили положить новый асфальт$\ldots$ 
 
Где\sdash то на отрезке Тверь\thinspace\nobreakdash---\thinspace Красный\thinspace Холм я всё\sdash таки провалился ненадолго в сон. Потом, очнувшись и~ободрившись чаем, сверился с картой и расчетом времени. Получалось, что к стапелю прибудем очень рано\mdash часам к четырём. Но это даже хорошо\mdash будет время нормально собрать байдарки, не спеша всё погрузить и~не~гнаться за~рекордами в первый ходовой день, поскольку состояние своё после ночной заброски тоже надо оценивать правильно\mdash мы будем как минимум вялыми и сонными. 

Тем временем, мы пересекли железнодорожную ветку Пестово\thinspace\nobreakdash---\thinspace Сонково Савёловского хода Московской железной дороги и миновали Сандово. К 2000\sdash м годам пассажирское движение на этой ветке окончательно заглохло, существенно затруднив железнодорожный заброс сплавщиков в бассейн Верхней Волги. А~ведь ещё с детства сидят в голове строки из тома номер 5 Советской Детской Энциклопедии~\cite{ДетскаяЭнциклопедия} о~том, что железнодорожный транспорт самый дешёвый после водного$\ldots$ 

Заброшенная железнодорожная ветка с деревянными шпалами, псевдоасфальтовая дорога, очарование ночного небосвода и несметное количество звезд сверху\mdash всё в~совокупности это создавало некую, я бы сказал, романтику путешествия, заброса. Я сидел и умилялся пейзажам, проплывающим за~окном\mdash Луна красиво освещала луга, поля и леса нашей огромной Родины. 

Вскоре добрались до Устюжны. Временами снова клонило в сон и приходилось чаще доставать термос с чаем. Устюжна поразила запустением и старыми купеческими домами конца XIX\thinspace\nobreakdash---\thinspace начала XX\thinspace века. Сейчас всё безнадежно обветшало, регион находится в~беспросветном упадке. Постояли в гордом одиночестве на~красный свет единственного после Твери светофора на~главном перекрестке Устюжны. Впечатлило. 

Второй раз пересекли реку Мологу по автомобильному мосту и поехали дальше. А дальше навигатор снова повёл нас по кратчайшему пути, не учтя качество дорожного покрытия, и мы неожиданно оказались на грунтовке, свернув после Устюжны налево. Да\sdash да,  участок Устюжна\thinspace\nobreakdash---\thinspace Мочала оказался грунтовым! Вот это поворот! Снова пришлось тащиться 20\thinspace\nbdash\thinspace 30~км/ч, и это показалось мне просто вечностью. Зато увидел лис, зайцев, сову и ёжиков, зачем\sdash то выползающих на дорогу. 

Но вот, наконец, деревня Мочала и сразу после неё федеральная трасса А\sdash 114, которая после грунтовки показалась немецким автобаном, никак не меньше. Дальше заброска пошла гораздо веселее. Два раза пересекли сильно петляющую реку Кобожу, в Сазоново переехали через реку Песь, а потом, в Первомайском, и через реку Чагоду. Совсем скоро деревня Вожань\mdash наша конечная цель! Команда уже не спала, а пребывала в полудрёме после дикой тряски на~грунтовке.

Термос мой совсем опустел к тому моменту, как мы свернули с трассы к Вожанскому озеру и стали пробираться по довольно разбитой грунтовке. Наконец\sdash то мы в Вожанях! Ура\sdash а\sdash а\sdash а\sdash а! Рассвет был в полпятого, звёзды давно погасли. Скоро в дымке должно было взойти солнце. 

\renewcommand*{\thefootnote}{\fnsymbol{footnote}}
Мы въехали в деревню и были поражены огромным домом слева на отшибе с несколькими ветрогенераторами. Тут нет электричества? Деревня фактически мертва и~давно\mdash я~находил сведения, что с начала XXI~века здесь практически нет зарегистрированных жителей. Одна единственная улица\mdash это, фактически, дачный вариант. Постоянно, видимо, тут живёт только тот один большой дом с ветряком$\ldots$~а во времена расцвета Тихвинской водной системы в деревне проживало от 100 до 200 жителей. По~озеру проходили суда с грузами\mdash \textit{тихвинки}\footnote[1]{Тихвинка\thinspace(по названию р.\thinspace Тихвинки)\mdash парусное грузовое судно, использовавшееся в XIX\thinspace в. на реках, входящих в Тихвинскую водную систему, и Ладожском озере. В основном применялись для сообщения между Петербургом и Нижегородской ярмаркой. Длина судна до~45~метров, ширина до $8,5$ метров. Малые тихвинки также назывались \textit{соминками} (по названию р.\thinspace Соминки) \cite{МорскойСправочник}.}, а в этой деревне, быть может, они останавливались на ночлег$\ldots$ Здесь проходил Устюженский почтовый тракт, а когда трассу А\sdash 114 проложили в стороне для спрямления дороги, это, надо полагать, и определило окончательный упадок поселения.

Стали искать подъезд к воде, повернув на развилке направо. Доехали до края деревни и поняли, что дальше микроавтобус не проедет, да и бессмысленно, потому что подъезда к воде не видно. Кое\sdash как водителю удалось развернуться, и мы вернулись к развилке около дома с~ветряком. Чтобы снова не заехать в дебри, сначала пошли с С.Ю. в разведку по дороге через поле. 

Утренний холодок пробрал меня, сон мигом улетучился. Пришлось закутаться в штормовку. По грунтовой ухабистой дорожке вышли к замечательному месту на берегу озера: ровная площадка, вход в воду, камыши. Решили, что однозначно стапелиться будем тут. Вернувшись к~микроавтобусу, доехали до берега и стали выгружать наши вещмешки и байдарки. 

Всё! Этап заброски пройден! Который был точно час, я, к сожалению, не засёк, но по ощущениям было что\sdash то около пяти утра. Мы~расплатились с водителем, поблагодарили его, и он уехал. Впереди~нас ждал стапель!

\begin{center}
	\psvectorian[scale=0.4]{88} % Красивый вензелёк :)
\end{center}
