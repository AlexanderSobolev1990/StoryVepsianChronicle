%\chapter{Пятый дневной переход. 05.08.16} 
%\chapter{Дождь в наказание. 05.08.16} 
%\chapter{Людей совсем нет. 05.08.16} 
\chapter{Людей совсем нет} 
%\corner{64}
\vepsianrose

Проснувшись и~оценив стоянку в лучах утреннего солнца, дневать мне тут расхотелось окончательно и~бесповоротно\mdash стол стоял как раз на солнцепёке, а~деревья, которые могли бы дать тень, были далеко. К~тому же кровососы со старицы налетели уже с утра! Конечно, по~раскладке времени похода уже пора бы было сделать днёвку\mdash мы прошли три ходовых дня\mdash и все уже ждали этого окончательного решения, но я, заручившись поддержкой С.Ю., принял решение сворачиваться и уходить, несмотря на вчерашнее решение тут задневать. Будем искать место получше, а то комары здесь просто нереально лютые.

Итак, позавтракав кашей <<от Кэпа>>, эскадра встала на воду и ринулась вперед. Но я\sdash то помнил, что до моей прошлогодней стоянки за деревней Щепье места снова будут не очень. А это значит, что сегодня днёвки нам тоже, судя по~всему, не видать$\ldots$ 

\newpage
День снова выдался жарким и безоблачным. Мы часто останавливались на пологих песчаных берегах отдохнуть. В~тельняшке стало жарко и дальше после очередной остановки я уже пошел с голым торсом, лишь накинув на~плечи и шею шемаг. Даже Геннадич расстался со~своей штормовкой, чтобы выровнять загар. 

Берега были красивыми, но не очень пригодными для стоянки и тем более днёвки. Хотелось чтобы костер и~хозчасть были в тени сосен, но в то же время чтобы была и песчаная коса, где можно хорошо искупаться. Таких мест нам всё никак не попадалось. Песчаные косы оказывались поросшими сплошным кустарником и~редколесьем. Хорошего соснового леса без густого подлеска тоже к воде не~выходило$\ldots$ 

Мы прошли деревни Званец, Клавдино, Слудно и Щепье. Стало вечереть. И вот, наконец, причалили к моей старой стоянке. Координаты места\mdash \CoordsChagodoschaFifteenSecondDnevka. Тут я снова обнаружил свои старые костровые палки. Также, кто\sdash то сделал новый стол и скамейки прямо из толстенных сосен! Я был за то, чтобы задневать тут, но С.Ю. место не понравилось близостью к~деревне и обжитостью, выражавшейся в наличии деревни Щепье в двух километрах выше по течению. Пляжика тут тоже не было\mdash снова крутой подъем. Народ стал балагурить и требовать хороший пляж на днёвку. Времени, в принципе было ещё не много и я, на свою беду, согласился с командой. Впрочем, не повторять же точь\sdash в\sdash точь прошлогоднее путешествие? Надо и новые стояночные места разведывать. Мы решили пройти ещё немного в поисках Лучшей Стоянки. Я помнил с прошлого года, что, кажется, в километре\sdash двух ниже по течению кто\sdash то стоял лагерем на правом берегу.

\newpage
Мы отплыли и вскоре$\ldots$ правильно, начался дождь\mdash cначала мелкий моросящий\mdash словно в наказание туристам\sdash водникам, чересчур привередливо относящихся к поиску места для стоянки. Стали чаще причаливать к берегам для осмотра. После второй тщетной попытки найти стоянку, дождь усилился. Воистину\mdash <<Лучшая Стоянка была полчаса назад>>. Крупные капли дождя колошматили по панаме. Пришлось экстренно нацепить дождевик, предусмотрительно положенный тут же рядом, стиснуть зубы и плыть дальше. Мы ещё несколько раз причаливали для осмотра, но безрезультатно\mdash места никому не нравились. Дождь с ветром усилились, и~прекращаться не собирались. Уже не шла речь найти шикарное место для днёвки, а хотя бы просто переночевать и переждать непогоду\mdash как поразительно меняются наши запросы в~зависимости от обстоятельств!

Наконец, тучи затянули всё небо, и пошел сильный ливень. Мы совсем перестали грести и просто дрейфовали по~течению, закутавшись от дождя кто во что. Я размышлял что делать. Стоять лагерем на песчаной косе смысла мало, поскольку песок будет везде\mdash в палатке, в спальнике, одежде$\ldots$ стоять днёвкой, место для которой хотели вроде как найти, на высоком берегу\mdash намучаешься бегать вниз к реке за~водой. Этот клубок противоречий, осложнённый погодными условиями, предстояло как\sdash то разрешить. 

Дождь всё не переставал, а мы, дрейфуя, вышли из очередного поворота петляющей реки. Нашему взору открылся вид на~пару километров прямого русла, что в~сумме с весьма прохладным ветерком и крупным ливнем подействовало на меня угнетающе. Словно прочитав моё состояние, С.Ю. заприметил место на высоком правом берегу, и мы высадились. 

%Место было ни о чём. Но сверху лил дождь, с носа капало, времени было уже много и надо бы вставать лагерем$\ldots$ С.Ю. повернулся ко мне и сказал: <<Ну что, Адмирал, принимай решение!>>. Я колебался, но видя наше состояние и погодные условия, понял, что идти дальше бессмысленно и чревато. И решил остаться на ночёвку. Координаты экстренной стоянки\mdash \CoordsChagodoschaSixteenEmergencyNignt.

Место было ни о чём. Но сверху лил дождь, с носа капало, времени было уже много и надо бы вставать лагерем$\ldots$ С.Ю. повернулся ко мне: 

\diagdash Ну что, Адмирал, принимай решение! 

\diagdash Место ни о чём! Обрыв высоченный, еле забрались! Как лодки затаскивать и шмотки?\mdash парировал было я.

\diagdash Ёпрст-прст-прст!!!\mdash пошли толки в бригаде.

\diagdash Устали как черти и дождь, похоже, затяжной!\mdash звучало из трюмов$\ldots$

\diagdash Тут делов на 5 минут! Перекидаем по цепочке вещи на обрыв и дело с концом!\mdash посыпались рационализаторские предложения от команды.

Я колебался, но видя наше состояние и погодные условия, понял, что идти дальше бессмысленно и чревато. И решил остаться на ночёвку. Координаты экстренной стоянки\mdash \CoordsChagodoschaSixteenEmergencyNignt.

Мы организовали живой коридор для разгрузки байдарок и быстро сделали наверху песчаного обрыва гору из наших вещей и гермомешков, накрыв их полиэтиленовой пленкой от дождя. Потом, орудуя топорами, выровняли место от кустарника и кое\sdash как затащили байдарки на крутой подъём. Следующим верным решением было поставить тент, что мы и сделали первый раз за поход и вообще в~походной истории этого тента. Потом перетаскали под~него вещи. Фух, можно передохнуть! Я вновь почувствовал уверенность и контроль над ситуацией. Дождик чуть ослабел, но~не~прекращался. Все отправились на поиски дров. Вскоре мы притащили и распилили несколько, на удивление, сухих брёвен. С помощью газовой горелки разожгли костёр, что было непросто, когда вокруг всё мокрое, и стали готовить чай и ужин. Пока закипала вода, поставили палатки и накрыли их пленкой$\ldots$ В целом, всё как всегда, только под дождём. 

%Дима, встав на~краю обрыва, философски\sdash язвительно заметил, что место <<о\sdash о\sdash очень хорошее>>, как и хотели, а~главное\mdash <<людей совсем нет>>! И жестом руки показал на~противоположный берег вниз по течению. Ба! Там метрах в~300\thinspace\nobreakdash--\thinspace 400 стояли рыбаки. А мы и не заметили их сначала. Но что\sdash либо менять было поздно, да и не нужно\mdash на~кой сдались они нам, а мы им. В итоге мы доразвернули лагерь, приготовили ужин и, ютясь под~тентом, вспомнили про то, что трёхсолодовый давно почил, снадобье из агавы тоже$\ldots$ народ было загрустил, но тут Адмирал вытащил свою десантную флягу, полную 96\sdash го и~Старпом, приговаривая <<два к трём, два к трём>>, приступил к сложнейшим вычислениям пропорции, пользуясь моей кэпской кружкой, как мерной пробиркой. 

Дима, встав на~краю обрыва, философски\sdash язвительно заметил: 

\diagdash Место о\sdash о\sdash очень хорошее, как и хотели! А~главное \textit{людей совсем нет}!\mdash и жестом руки показал на~противоположный берег по диагонали вниз по течению Чагодощи. 

\diagdash А что такое?\mdash народ подтянулся к берегу.\mdash Чего там такое?

Ба! Там метрах в~300\thinspace\nobreakdash--\thinspace 400 стояли рыбаки. А мы и~не~заметили их сначала. Но что\sdash либо менять было поздно, поскольку лагерь был уже развёрнут, да и не нужно\mdash на~кой сдались они нам, а мы им?

\diagdash Это рыбаки, вон удочки,\mdash намётанным глазом заметил Паша.\mdash а сейчас в дождь какая рыбалка? Они скоро свернутся и уедут, вон, похоже, машина стоит в кустах.

\diagdash Да даже если не свернутся, я лодки у них не вижу,\mdash сказал я,\mdash и на наш берег им зачем?

В итоге мы доготовили ужин и, ютясь под~тентом, вспомнили про то, что трёхсолодовый давно почил, снадобье из агавы тоже$\ldots$ народ было загрустил, но тут Адмирал вытащил свою десантную флягу, полную 96\sdash го и~Старпом, приговаривая <<два к трём, два к трём>>, приступил к сложнейшим вычислениям пропорции, пользуясь моей железной кэпской кружкой, как мерной пробиркой. 

Дождь, тем временем, закончился, небо стало расчищаться. Просто фантастика! Ещё какой\sdash то час назад мне казалось, что гнуснее погоды быть не может, а сейчас я был даже счастлив! Костёр, вечерняя похлёбка, шикарное сало с чёрным хлебом и 96\sdash й мгновенно перевели эту стоянку из низшего разряда во вполне себе терпимый. Небосвод на~западе расчистился и заиграл закатными красками, а~розовое солнце подсветило облако необычной пирамидальной формы на~востоке, создав совершенно умопомрачительной красоты картину речного рая. 

Я~присел на плащ\sdash палатку на краю высокого обрыва, достал кисет с трубочкой и, отхлебнув чаю, подумал, что как же чертовски хорошо устроен мир$\ldots$ пока нет людей. Природа\mdash абсолютно прекрасна. Человек в~ней\mdash лишнее. Он только портит её и замусоривает. Не~венец он природы, а,~как говорит Геннадич, вирус.

\diagdash Да, Геннадич?\mdash желая ещё раз услышать подтверждение, спросил я.

\diagdash Вирус, вирус$\ldots$ никуда не деться от него!\mdash подтвердил мой Старпом.\mdash Всё засрали, пол-планеты уже$\ldots$\mdash полилась философия.

\diagdash Но почему именно вирус? А не, скажем, животное?\mdash не унимался я.\mdash Ведь вирус это не животное и не растение, а чёрт знает что такое$\ldots$

\diagdash Да как тебе сказать, вот проплывали сегодня пару стоянок\mdash намусорено, мать их ети!\mdash Геннадич распалился вдруг.\mdash И ведь то местные рыбачки грешат, местные, понимаешь!\mdash изрёк Старпом и смачно затянулся.\mdash Потом сюда ещё рыбачить придут другой раз, самим не противно?

Да, всё так. Психология временщика\mdash после нас хоть потоп. Нет понимания, что всё это останется нашим детям, внукам$\ldots$ а, впрочем, останется ли? Что ждёт нас в ближайшие годы, какие потрясения? К чему идёт всё наше <<развитие>>? Сложный вопрос, подобная философия так и~норовит проникнуть в сознание, особенно когда сух и~сыт. Мы~сидели и тихо подымливали с Геннадичем на краю обрыва, болтая о всевозможной ерунде$\ldots$ в~голову лезла ещё всякая разная философическая чепуха, но~пора было уже и~на~боковую\mdash вечерний марафон с развёртыванием лагеря в~тяжелых штормовых условиях нас подызмотал.

К этому времени палатки были давно поставлены и~немного окопаны от дождя. Команда накормлена и~переодета во~всё~сухое. Все собрались под тентом, пили чай со~сгущёнкой и печеньем. Рыбаки на противоположном берегу дожгли костёр и тоже утихомирились. На Совете Капитанов решили, что завтра специально начинаем искать днёвку ну~прямо с отплытия:

\diagdash Пусть даже приличное место окажется уже за~следующим поворотом, встаём!\mdash обозначил Паша, выражая командное желание устроить хорошую днёвку.

\diagdash Ну да,\mdash соглашался я,\mdash километраж мы опережаем уже примерно на полдня из\sdash за сегодняшего дождевого рекорда, поэтому почему бы и нет.

\diagdash Хотелось бы постоять денёк в красивом месте,\mdash заметили С.Ю. и Дима.\mdash Что там дальше по течению с~местами, Кэп? 

\diagdash Ниже обрыва в Загривье я не ходил, но до~Бывальцево вроде мест ещё будет предостаточно.\mdash поделился я с командой разведданными.

\diagdash А после? После Бывальцево?

\diagdash Там русло спрямится и высокие берега песчаные пропадут,\mdash отвечал я,\mdash Так что надо встать до него.

На том и порешили\mdash целенаправленно прямо с утра завтра ищем обалденное место под днёвку. Сегодня же денёк выдался под конец тяжёлый, но мы всё выдержали и даже, вообще\sdash то очень неплохо разместились. Спать разошлись опять поздно, гоняли чаи и дымили с Геннадичем. В ночь, как мне показалось, пошел ещё дождик, но я предусмотрительно накрыл также задний торец палатки полиэтиленовой пленкой от влаги и ветра. Хотя, про дождь вполне могло и присниться. Сон у меня в сплаве был ну просто богатырский\mdash я мгновенно отрубался на свежем холодном ночном воздухе, пропитанном запахами леса и разнотравья, а с утра вставал на удивление выспавшимся и обновлённым.%выспавшийся и обновлённый.

\begin{center}
	\psvectorian[scale=0.4]{88} % Красивый вензелёк :)
\end{center}
