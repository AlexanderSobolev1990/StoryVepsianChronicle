%\chapter{Пятый дневной переход. 05.08.16} 
%\chapter{Дождь в наказание. 05.08.16} 
%\chapter{Людей совсем нет. 05.08.16} 
\chapter{Людей совсем нет} 
%\corner{64}
\vepsianrose

Да\sdash да, именно пятый дневной переход. Проснувшись и оценив стоянку в лучах утреннего солнца, дневать мне тут совершенно расхотелось\mdash стол стоял как раз на солнцепёке, а деревья, которые могли бы дать тень, были далеко. К тому же комары со старицы налетели уже с утра! Конечно, по времени уже пора бы было сделать днёвку\mdash мы прошли три ходовых дня\mdash и все уже ждали этого окончательного решения, но я, заручившись поддержкой С.Ю., принял решение сворачиваться и уходить, несмотря на вчерашнее решение тут задневать. Будем искать место получше, а то комары здесь просто нереально лютые.

Итак, позавтракав кашей <<от Кэпа>>, эскадра встала на воду и ринулась вперед. Но я\sdash то помнил, что до моей прошлогодней стоянки за деревней Щепье места снова будут не очень. А это значит, что сегодня днёвки нам тоже, судя по всему, не видать$\ldots$ 

День снова выдался жарким и безоблачным. Мы часто останавливались на пологих песчаных берегах отдохнуть. В тельняшке стало жарко и дальше после очередной остановки я уже пошел с голым торсом, лишь накинув на плечи шемаг. Даже Геннадич расстался со штормовкой, чтобы выровнять загар. 

Берега были красивыми, но не очень пригодными для стоянки и тем более днёвки. Хотелось чтобы костер и хозчасть были в тени сосен, но в то же время чтобы была и песчаная коса, где можно хорошо искупаться. Таких мест нам всё никак не попадалось. Песчаные косы оказывались поросшими сплошным кустарником и редколесьем. Хорошего соснового леса без густого подлеска тоже к воде не выходило$\ldots$ 

Мы прошли деревни Званец, Клавдино, Слудно и Щепье. Стало вечереть. И вот, наконец, причалили к моей старой стоянке. Координаты места\mdash \CoordsChagodoschaFifteenSecondDnevka. Тут я снова обнаружил свои старые костровые палки. Также, кто\sdash то сделал новый стол и скамейки прямо из толстенных сосен! Я был за то, чтобы задневать тут, но С.Ю. место не понравилось близостью к деревне и обжитостью, выражавшейся в наличии деревни Щепье в двух километрах выше по течению. Пляжика тут тоже не было\mdash снова крутой подъем. Народ стал балагурить и требовать хороший пляж на днёвку. Времени, в принципе было ещё не много и я, на свою беду, согласился с командой. Впрочем, не повторять же точь\sdash в\sdash точь прошлогоднее путешествие? Надо и новые стояночные места разведывать. Мы решили пройти ещё немного в поисках Лучшей Стоянки. Я помнил с прошлого года, что, кажется, в километре\sdash двух ниже по течению кто\sdash то стоял лагерем на правом берегу.

Мы отплыли и вскоре$\ldots$ правильно, начался дождь\mdash cначала мелкий моросящий\mdash словно в наказание туристам\sdash водникам, чересчур привередливо относящихся к поиску места для стоянки$\ldots$ Стали чаще причаливать к берегам для осмотра. После второй тщетной попытки найти стоянку, дождь усилился. Воистину\mdash <<Лучшая Стоянка была полчаса назад>>. Крупные капли дождя колошматили по панаме. Пришлось экстренно нацепить дождевик, предусмотрительно положенный тут же рядом, стиснуть зубы и плыть дальше. Мы ещё несколько раз причаливали для осмотра, но безрезультатно\mdash места никому не нравились. Дождь с ветром усилились, и прекращаться не собирались. Уже не шла речь найти шикарное место для днёвки, а хотя бы просто переночевать и переждать непогоду\mdash как поразительно меняются наши запросы в зависимости от обстоятельств!

Наконец, тучи затянули всё небо, и пошел сильный ливень. Мы перестали грести и стали дрейфовать по течению. Я размышлял что делать. Стоять лагерем на песчаной косе смысла мало\mdash песок будет везде\mdash в палатке, в спальнике, одежде$\ldots$ стоять днёвкой на высоком берегу\mdash намучаешься бегать вниз к реке. Этот клубок противоречий, осложненный погодными условиями, предстояло как\sdash то разрешить. Дождь всё не переставал, а мы, дрейфуя, вышли из очередного поворота петляющей реки. Нашему взору открылся вид на пару километров прямого русла. Тут же, на высоком правом берегу, С.Ю. заприметил место, и мы высадились. 

Место было ни о чём. Но сверху лил дождь, с носа капало, времени было уже много и надо бы вставать лагерем$\ldots$ С.Ю. повернулся ко мне и сказал: <<Ну что, Адмирал, принимай решение!>>. Я колебался, но видя наше состояние и погодные условия, понял, что идти дальше бессмысленно и чревато. И решил остаться на ночёвку. Координаты экстренной стоянки\mdash \CoordsChagodoschaSixteenEmergencyNignt.

Мы организовали живой коридор для разгрузки байдарок и быстро сделали наверху песчаного обрыва гору из наших вещей и гермомешков, накрыв их полиэтиленовой пленкой от дождя. Потом, орудуя топором, выровняли место от кустарника и кое\sdash как затащили байдарки на крутой подъём. Следующим верным решением было поставить тент, что мы и сделали первый раз за поход и вообще в походной истории этого тента. Потом перетаскали под него вещи. Фух, можно передохнуть! Я вновь почувствовал уверенность и контроль над ситуацией. Дождик чуть ослабел, но не прекращался. Все отправились на поиски дров. Вскоре мы притащили и распилили несколько, на удивление, сухих брёвен. С помощью газовой горелки разожгли костёр, что было непросто, когда вокруг всё мокрое, и стали готовить чай и ужин. Пока закипала вода, поставили палатки и накрыли их пленкой$\ldots$ В целом, всё как всегда, только под дождём. 

Дима, встав на краю обрыва, философски\sdash язвительно заметил, что место <<о\sdash о\sdash очень хорошее>>, как и хотели, а главное\mdash <<людей совсем нет>>! И жестом руки показал на противоположный берег вниз по течению. Ба! Там метрах в 300\thinspace\nobreakdash--\thinspace 400 стояли рыбаки. А мы и не заметили их сначала. Но что\sdash либо менять было поздно, да и не нужно\mdash на кой сдались они нам, а мы им. В итоге мы доразвернули лагерь, приготовили ужин и, ютясь под тентом, вспомнили про то, что трёхсолодовый давно почил, снадобье из агавы тоже$\ldots$ народ было загрустил, но тут Адмирал вытащил свою десантную флягу, полную 96\sdash го и Старпом, приговаривая <<два к трём, два к трём>>, приступил к сложнейшим вычислениям пропорции, пользуюсь моей кэпской кружкой, как мерной пробиркой. 

Дождь, тем временем, закончился, небо стало расчищаться. Просто фантастика! Ещё какой\sdash то час назад мне казалось, что гнуснее погоды быть не может, а сейчас я был даже счастлив! Костёр, вечерняя похлёбка, шикарное сало с чёрным хлебом и 96\sdash й мгновенно перевели эту стоянку из низшего разряда во вполне себе терпимый. Небосвод на западе расчистился и заиграл закатными красками, а розовое солнце подсветило облако необычной формы на востоке, создав совершенно умопомрачительной красоты картину. Я присел на плащ\sdash палатку на краю высокого обрыва, достал кисет с трубочкой и, отхлебнув чаю, подумал, что как же чертовски хорошо устроен мир$\ldots$ пока нет людей. Природа\mdash абсолютно прекрасна. Человек в ней\mdash лишнее. Он только портит её и замусоривает. Не венец он природы, а, как говорит Геннадич, вирус. Мы сидели и тихо подымливали с ним на краю обрыва$\ldots$ в голову лезла ещё всякая разная философия, но пора было уже и на боковую. 

К этому времени палатки были поставлены и немного окопаны от дождя. Команда накормлена и переодета во всё сухое. Все сидели под тентом, пили чай со сгущёнкой и печеньем. Рыбаки на противоположном берегу дожгли костёр и тоже утихомирились. Я накрыл также задний торец палатки полиэтиленовой пленкой, покидал туда вещи и залёг спать. Денёк выдался под конец тяжёлый, но мы всё выдержали и даже, вообще\sdash то очень неплохо разместились. На Совете Капитанов решили, что завтра специально начинаем искать днёвку прямо с отплытия, пусть даже приличное место окажется уже за следующим поворотом\mdash километраж мы опережаем уже примерно на полдня. 

Спать разошлись опять поздно. В ночь, как мне показалось, пошел ещё дождик. Хотя, это вполне могло и присниться. Сон у меня в сплаве был ну просто богатырский\mdash я мгновенно отрубался на свежем холодном ночном воздухе, пропитанном запахами леса и разнотравья, а с утра вставал на удивление выспавшийся и обновлённый.

\begin{center}
	\psvectorian[scale=0.4]{88} % Красивый вензелёк :)
\end{center}
