%\chapter{В ожидании днёвки. 04.08.16} 
%\chapter{Кровососы. 04.08.16} 
\chapter{Кровососы} 
%\corner{64}
\vepsianrose

Утро выдалось просто отличным\mdash  солнечно, редкие белые облачка, лёгкий ветерок. На завтрак снова сварили шикарную кашу по моему рецепту. После традиционно обильной и сытной трапезы стали сворачивать лагерь. Видавшие виды кроссовки Геннадича, промокшие вчера, никак не желали сохнуть, и он решил с ними проститься. Организовал ритуал погребения старых кроссовок у сосны под минорную мелодию, сыгранную им на мажорной губной гармошке\mdash вышло комично. Дальше Геннадич сплавлялся в шлёпках. 

На воду встали нормально, не поздно.  Бодро рванули вперёд\mdash погода способствовала. Разгорался жаркий денёк! Кудрявые облака лениво ползли по небу, а солнце припекало нешуточно уже с утра. Геннадич плыл в одной распахнутой штормовке и к часам двум дня у него сгорел живот, став красным, а руки, прикрытые рукавами штормовки, остались белыми, незагоревшими.

\newpage
Мне было интересно, где мы сегодня сможем найти приличную стоянку, потому как мне в прошлом году этого сделать не удалось на этом переходе. Но это всё ближе к~вечеру, а пока что мы гребл\'{и}, часто перекладывая курс от~поворота к~повороту\mdash  река начала петлять и появились хорошие песчаные берега\mdash  расцвет так называемого <<золотого возраста>> реки.

Погода стояла изумительная\mdash крупные кучевые облачка, лёгкий ветерок и яркое солнышко. На одной из остановок я вытащил свою солнечную батарею, укрепил её на корме карабинами, зацепив их за люверсы кормового фальшборта\mdash за ходовой день эта батарея полностью заряжала небольшой аккумулятор, от которого ночью, в свою очередь, подзаряжался мобильный телефон. Такая вот походная <<зелёная>> энергетика.

На одном из золотистых песчаных бережков мы причалили перекусить на обед. У Геннадича окончательно сгорели грудь и живот, став красными, как у рака, потому что не были прикрыты распахнутой штормовкой. Я~же, зная о таком коварстве в сплаве, плыл в тельняшке, чтобы не обгореть, но шея сзади все\sdash таки коварно подвялилась и начинала нешуточно жечь, отчего я замотался в шемаг, надеясь облегчить немного участь шеи. Народ стал пошучивать насчёт этого предмета гардероба, что заставило меня замотаться им полностью на манер бедуинов в пустыне, залезть на огромный ствол дерева\sdash топляка, выброшенного на берег, и станцевать туземный танец вождя племени, подтвердив свои адмиральские полномочия$\ldots$

Уничтожили остатки чурчхелы, попили чаю и снова встали на воду. Ваня проявлял активность и даже временами грёб, помогая отцу, который снова удивлял нас, обгоняя эскадру и вырываясь вперёд. Вообще, мы старались идти если не кильватерной колонной, то хотя бы не удаляясь далеко друг от друга. А если такое и случалось, то~вырвавшийся вперёд экипаж, как правило, понимал своё положение и давал остальным догнать себя, временно дрейфуя по течению.

Сверившись с GPS, я убедился, что проплываем охотничий кордон на правом берегу. Тут в прошлом году мы лихо сели на мель посреди реки. В этом же году воды было больше и такого не случилось. Уже начинало вечереть и~километража оставалось совсем немного. Вскоре я заметил знакомую одинокую берёзу на левом берегу и снова свои костровые палки\mdash моя прошлогодняя стоянка. Причалили. Стоять тут в этом году не хотелось абсолютно, потому что слишком крутой подъем и мало места для нашей большой компании. Мне показалось, что берег даже немного обвалился с прошлого года, видимо после весеннего паводка, став ещё более крутым.

Байдарка С.Ю., тем временем, пока мы разминались у березы, прошла вперёд и разведала стоянку буквально в двухстах метрах ниже по течению. Вот так сюрприз! Как я её не заметил в прошлом\sdash то году? Подойдя спустя пару минут ближе, мы с Геннадичем поняли почему\mdash чтобы увидеть её с воды, надо обернуться и посмотреть назад. Единогласно решили остаться на разведанном новом месте. Экипаж С.Ю. и Паши, выступая в роли разведчиков, очень радовал. 

В наличии на стоянке был стол и скамьи из досок, а также реально большой запас уже наколотых и потемневших дров. Чуть поодаль в лесочке мы нашли обветшалые остатки шикарной стоянки рыбаков или охотников\mdash  большое слегка обвалившееся укрытие из брёвен, а рядом\mdash  навес и скамьи. Правда, туда было идти метров 100\thinspace\nobreakdash--\thinspace 150, поэтому решили оставаться на открытой поляне у стола, чтобы не таскать далеко вещи. Координаты места\mdash \CoordsChagodoschaSixteenNearKaban. 

Действуя уже привычно, достаточно быстро оборудовали лагерь, развели костёр из найденных дров. Впрок на дрова напилили сухостой, валявшийся тут же рядом. Неподалеку Паша нашёл старицу и отправился туда рыбачить, а я и Дима успели искупнуться в реке, преодолевая прибрежные водоросли, неприятно трущиеся об ноги и тело. Потом все занялись приготовлением ужина. В воздухе витало решение устроить тут днёвку. Что ж, очень даже может быть. Мы все немного устали за длинный дневной переход, так что нуждались в отдыхе. После вечерней трапезы я вытащил сигнал охотника, и мы с Ваней пальнули вверх красной ракетой в ознаменование адмиральского решения устроить тут завтра дневной привал.

%\newpage
Стемнело. Запасы снадобья из агавы и 96\sdash го ещё значительно поубавились, а после было выпито неимоверное количество душистого чая на тёмной торфяной воде. Появились надоедливые комары и какая\sdash то мошк\'{а}\mdash  видимо с той старицы$\ldots$ вот тебе раз! Просто адские \textit{кровососы}!!! Первый раз за поход мы были атакованы насекомыми с такой несвойственной для августа силой. Отпугивал их более менее только дым костра, поэтому мы постоянно норовили словить потоки дыма, отчего сильно слезились глаза. Дневать по соседству с такими комарами мне что\sdash то показалось не очень$\ldots$ По палаткам все расползлись довольно рано в напрасном предвкушении завтрашней днёвки. 

\begin{center}
	\psvectorian[scale=0.4]{88} % Красивый вензелёк :)
\end{center}
