%\chapter{Урочище Лукобор. 07.08.16} 
%\chapter{Потапыч. 07.08.16} 
\chapter{Потапыч} 
%\corner{64}
\vepsianrose

Проснулся достаточно рано, но вылезать из палатки совершенно не хотелось, и вскоре я снова задремал. Соизволил вылезти, когда народ тоже уже проснулся и начал шуметь. Стали готовить завтрак\mdash Геннадич проявил рвение и взял инициативу в свои руки. В итоге у него получилась обалденная каша, в которую также добавили орехи, изюм, курагу и другие сухофрукты\mdash как мы обычно делаем. Пока он кашеварил, я достал флаг Советского Союза и укрепил его на свежевырубленном импровизированном флагштоке. <<Неба утреннего стяг$\ldots$>> и снова аккорды гимна в исполнении Геннадича, одной рукой помешивающего кашу, а второй играющего на губной гармошке.

Потом вспомнили, что у нас есть блинная мука и~захотели сделать оладьи с яблоком. Приготовили тесто, как умели, и Дима начал готовить. Хотели сначала печь оладьи в~крышках котелков. Но крышки оказались слишком тонкими и всё сгорало. Усовершенствовали конструкцию\mdash стали печь в двух крышках, поместив между ними песок для~распределения тепла. Снова почему\sdash то не вышло. Блинчик снизу подгорал, а сверху по\sdash прежнему оставался абсолютно сырым. Дима достал свою газовую горелку, и~мы повторили всё то же самое, только теперь не~на~костре, а~на~горелке. Но результат вышел тот же\mdash сырой подгорелый блинчик. Тогда С.Ю. предложил просто запечь тесто в котелке. Мы поставили котелок на угли и вокруг тоже привалили углями. Пахло безумно вкусно. Все ходили вокруг котелка в нетерпении и норовили заглянуть что там получается$\ldots$ но увы, всё тоже сгорело, превратившись в~полусырую несъедобную массу. Короче, как мы поняли, блины в походе\mdash не мужских рук дело. А вот как теперь отчищать котелок? В нём напрочь пригорели остатки <<теста>>. Я сразу в шутку предложил этот котелок выкинуть к~чёртовой матери. Но С.Ю. сказал, что свой котелок в беде не бросит и грустно пошёл на берег отчищать его. Поели, короче говоря, оладушков.

Но, поскольку мы уже до того зарядились кашей, отсутствие оладушков нас не сильно огорчило. Так что днёвка начиналась почти отлично. По синему\sdash синему небу плыли частые кучевые облачка, день обещал быть жарким. После завтрака мы с Геннадичем решили сплавать из~интереса на противоположный берег, потому что Паша сказал, что видел там$\ldots$ медведя. Оказалось, Паша рано утром встал и пошел на рыбалку, и тут ему улыбнулась удача\mdash несколько крупных рыбёшек стали добычей. Поймав пару окушков, он и увидел медведя на том берегу. Не~очень приятное соседство на днёвке, однако. Надо разведать, что~там такое, и вот мы уже на пустой байдарке мчимся вдвоём на правый берег, на песчаную отмель чуть ниже по~течению, следуя указаниям Паши.

\newpage
Вылез из байдарки и сразу заметил следы, но не~медвежьи, а собачьи\mdash воображаемая линия проводилась между 1 и 4 когтём, пересекая подушечки лапы\mdash значит это собака, а не волк. Откуда тут собака? Ну да ладно, пошли с Геннадичем дальше и$\ldots$ наткнулись на свежие следы медвежонка. Отпечаток его лапы был меньше моего, у меня 44\sdash й размер ноги. Сразу стало как\sdash то не по себе. Сфотографировав свидетельства пребывания тут медведя, мы поспешили убраться с этого пляжа обратно в лагерь. 

Настроение у меня подпортилось. Мы встали на днёвку, да ещё место такое живописное, а тут медведь где\sdash то ходит неподалёку! Вот тебе раз! По прибытию в лагерь я сразу проверил на всякий случай на месте ли сигнальные ракеты. Глупо, правда, надеяться, что они как\sdash то помогут в случае нападения хищника. Хотя, с другой стороны, зачем ему нападать? Мы ему ничего не сделали$\ldots$ не считая того, что~встали тут на днёвку. Короче говоря, мысли роем носились в адмиральской голове, сталкивались там и думали, что предпринять в данной ситуации$\ldots$ вплоть до снятия со~стоянки. Но в итоге я решил оставаться, потому как медведь больше не показывался, погода была блеск, место\mdash супер, а~чувство тревоги понемногу ушло. 

Паше и С.Ю. улыбнулась рыболовная удача\mdash удалось за день наловить 6\thinspace\nobreakdash--\thinspace 7 крупных рыбёшек. Они решили их закоптить вечером на углях в специальном пакете. Я~успел вымыться в реке, перестирал все вещи и развесил их сушиться на верёвку, которую натянул между двумя соснами, росшими посреди лагеря. Потом решил пойти по песчаной косе далеко выше по~течению\mdash настолько она была красива. От лагеря отошёл наверно метров на~500 по~красивому песчаному берегу\sdash пляжу вдоль левого берега, делающего затяжной поворот, и~обратно приплыл самосплавом по~течению, вылез и~стал загорать. Часто в~нашем сплаве мы находили такие песчаные косы, на~которых совершенно не было следов\mdash ни~зверей, ни~человека. И~это просто фантастика, бальзам на~израненную душу городского жителя, который только и~делает, что каждый день сталкивается в~городе с~жутким столпотворением народа на улицах и~в~транспорте. Коса, по~которой я~шёл, была именно такой\mdash ни~единого постороннего следа\mdash

Красивейшее место мы выбрали! Людей нет, тишина просто звенит в ушах! Мирно плещется Чагодоща в своих золотистых песчаных берегах, дует лёгкий, едва уловимый, ветерочек, сверху греет приятное и ласковое августовское солнышко. Вокруг меня летают стрекозы\mdash показатель экологической чистоты местности$\ldots$ комаров нет, их сдувает лёгким ветерочком. Кайф! Я лежал на песочке, положив руки под голову и отдыхал\mdash душой и телом. Любимый мною сосновый лес на противоположном берегу дополнял картину\mdash мне хотелось просто распасться на молекулы и раствориться во всей этой красоте, окружавшей меня. А воздух какой?! Запахи соснового леса и разнотравья, приправленные влажностью реки\mdash это просто наслаждение! Вдоволь навалявшись на тёплом, нагретом солнышком песочке, я вернулся в лагерь. Дима сидел у~воды на стульчике и читал книгу, а Ваня играл рядом с~ним в~песочке. Идиллия! Паша и~С.Ю. продолжали рыбную ловлю, а Геннадич релаксировал в лагере.

День, тем временем, уже потихоньку клонился к~вечеру. Небо совершенно расчистилось от редких облачков, и~оранжевые краски закатного солнца залили нашу шикарную песчаную косу. Я~взял штатив, фотоаппарат, дистанционный пульт и пошёл на фотоохоту\mdash закатный свет известен своей мягкостью и кадры, сделанные в это время, получаются особенно хорошими. Так и случилось\mdash кадры вышли замечательными. Но оранжевое солнышко вскоре зашло за сосны\mdash пора было идти в лагерь. С.Ю.~и~Ваня, тем временем, уже почистили картошку в~реке. Ужин поспевал. Ваню обучали рубить дрова топором, но~у него пока не очень хорошо это получалось. С.Ю. нашёл в~лесу костянику\mdash добавили её в чай. Подоспела копчёная рыба от~Паши\mdash объедение! Наш рацион, и без того разнообразный, обогатился таким шикарным лакомством. К~копчёной рыбе у~нас был даже лимон\mdash просто барство! 

Я сидел на скамеечке, обильно закусывал с ломившегося от яств стола и блаженно выпускал вверх колечки из~трубочки. А вот у ребят кончилось курево$\ldots$ я им говорил до похода, что на природе расход в 2\thinspace\nobreakdash--\thinspace 3 раза больше, но~это все всегда рассчитать не могут\mdash их запасы сигарет кончились. Ребята стали стрелять мой рассыпной табачок и набивать его в кальку от конфет <<Коровка>>. Эти~адски смолящие калькой самокрутки мы так и~назвали\mdash <<Коровка>>. 

Вскоре после перекуса копчёной рыбой, мы навернули сытнейший ужин и продолжили опустошать запасы 96\sdash го, травя всякие байки и болтая о всевозможной чепухе. И тут\sdash то хозяин леса снова дал о себе знать\mdash на противоположном берегу послышался громкий хруст веток!!! Мы повскакивали от стола и вышли на берег. В зарослях определенно ходил кто\sdash то массивный\mdash ветки хрустели, кусты гнулись. Виновника не было видно. Честно, стало не по себе. Я~стоял в оцепенении и перебирал в голове варианты, что бы предпринять. В итоге я взял два топора и стал соударять их, извлекая громкий и мелодичный протяжный металлический звук из своего кованого топора. Хруст углубился в заросли и~удалился. Мы стояли и продолжали трапезу уже стоя\mdash не отвлекаться же от ужина, в самом деле? 

Но вскоре стало вообще не до смеха\mdash на том же бережку, куда мы с Геннадичем плавали с утра, показался \textit{мишка}. Видимо, испугавшись звона топора, он отошёл вглубь зарослей и вышел подальше от нас к воде. Не смотря даже в нашу сторону, он медленно подошёл к реке, постоял, помедлил, попил воды. В лагере нарастало напряжение. Я достал из штормовки два сигнала охотника и зарядил их. Стрелять или не стрелять вверх, вот в чём вопрос! Отпугнёт это или, наоборот, привлечёт зверя? С медведем я сталкивался 1 раз в Башкирии, но тогда я лично его не видел, да и нас было 50 человек, так что момент истины я тогда как\sdash то не прочувствовал. А сейчас, стоя с двумя сигналами охотника, готовый к стрельбе <<по\sdash македонски>> с двух рук, я~размышлял, что же всё\sdash таки делать. 

Мишка, вообще не обращая на нас никакого внимания, попил воды и стал переплывать реку на наш берег! Правда, не~в~нашу сторону, а по диагонали от нас\mdash ниже по~течению. Но это всё равно\mdash на наш берег!!! Я протрезвел буквально за~секунды! Ване, крутившемуся между нас, сказали не~бегать и он замолчал. А потом расплакался, видимо тоже, наконец, ощутив страх. Ваня запричитал, чтобы мишка не кушал его, а кушал нас, потому что мы\mdash старые по~его мнению. До\sdash о\sdash о\sdash брый мальчик! Ну,~так\sdash то да, заросшие щетиной мы кажемся старше, чем есть на~самом деле. Ваня, впрочем, быстро успокоился и ещё раз напомнил нам, что мы старые, и~нас кушать уже можно, а~его нельзя, чем разрядил обстановку. Я ещё немного постучал топорами друг об друга, издавая громкий протяжный звук. Мишка, в~свою очередь, ровно никак не реагировал на моё стучание топорами, спокойно, зараза, переплыл реку, вылез на берег и углубился в лес$\ldots$ вроде бы не в нашу сторону. Окружает, подумали мы.

Тут Дима схватил топор и вырубил две огромные рогатины, справедливо заметив, что наши предки как\sdash то же охотились на медведя без огнестрельного оружия$\ldots$ Никто~не~поддержал его энтузиазма. Я решил, в случае чего, уходить по реке на пустой байдарке, а потом вернуться за~вещами\mdash мишка не стал бы вечно околачиваться в~районе лагеря, а, скорее всего, поел бы что нашёл, да и побрёл прочь. Короче говоря, перепугались мы не слабо. Естественно, здоровый сытый зверь боится человека. Но~не~пойдешь же искать его и спрашивать, мол\mdash ты сытый и~здоровый или~нет? 

Потапыч ушёл, а тревога осталась. Кроме того, погода начала портиться, стал накрапывать дождик. С.Ю. сделал единственно верное дело\mdash унёс остатки сала и повесил пакет с ними на берёзу в стороне от лагеря, а остальные продукты сказал сложить около стола, а не в палатках как обычно. Потом, оглядев нашу порядком струхнувшую бригаду, он сокрушенно сказал, что мы зря боимся. Зверь боится нас гораздо сильнее, чем мы его и т.д. и т.п.\mdash что и так вертелось у меня в голове, но~нервишки подсказывали другое\mdash жги костёр. В~голове будто совещались здравый смысл и страх:

\diagdash Но ведь медведь боится нас сильнее$\ldots$  

\diagdash ЖГИ КОСТЁР! 

\diagdash А как же природный страх животного перед человеком? Ведь все животные боятся человека и огня$\ldots$  

\diagdash {\Large Ж\sdash Г\sdash И \enspace К\sdash О\sdash С\sdash Т\sdash Ё\sdash Р~!!!}

Дима с Ваней пошли спать, С.Ю. тоже улёгся пораньше, а мы с Пашей и Геннадичем остались жечь костёр и~добивать конфеты <<Коровка>>. Впрочем, Паша скоро тоже ушёл, напившись чаем, остались только мы с Геннадичем. Под шум редкого дождика жгли костёр и~болтали о разном, стараясь разогнать сон. В шуме ветра и дождя мне всё чудилась поступь хищника, блуждающего по лесу. 

Перевалило за полночь. Взбудораженное сознание рисовало медведя, сидящего в засаде и тихонечко ждущего, когда же эти негодные людишки наконец уснут. Стали болтать про печальную судьбу группы Дятлова, инопланетян, бактериологическое оружие массового поражения и прочие страшилки у костра. 

О~группе Дятлова слышал, пожалуй, каждый турист в нашей стране. Ребята~оказались неготовы к погодным условиям, а просчёт руководителя с местом ночёвки на склоне горы Холатчахль стал роковым в~той истории. Все~остальные бредовые версии об их гибели, конечно, являются спекуляцией на горе и результатом особенности человеческого сознания скорее поверить в нечто сверхъестественное, чем в банальную неготовность группы к~тому маршруту в сложных условиях. 

И вот теперь, тоже оказавшись в не очень приятной ситуации с этим медведем, аналогии с группой Дятлова так и лезли мне, как руководителю нашего сплава, во~взбудораженное сознание. Что если я был не прав, и надо было свернуть лагерь и уйти со стоянки? Вдруг медведю захочется подойти к лагерю поближе?! В здешних местах туристы не~слишком часто ходят чтобы медведи к ним привыкли, но всё же, кто знает что у этого медведя на уме?! Какого чёрта мы не снялись со стоянки? Выиграть в схватке с медведем современному человеку без~соответствующих навыков и без огнестрельного оружия\mdash шансы нулевые. Добежать вдвоём (потому что одному не~справиться!) до~байдарки, принести её к воде и отойти от~берега подальше\mdash на всё это просто не хватит времени при встрече с~потапычем. Кроме того, он великолепно не~только плавает и быстро бегает, но и лазает по деревьям\mdash на окрестных соснах тоже не~спастись. Кроме огнестрела особо весомых аргументов\sdash то и~нет, собственно. Но ружья у нас с~собой не имелось, так что мы продолжали сидеть у~костра и~заниматься болтологией для самоуспокоения.

Но, чем дольше мы сидели, тем больше я понимал, что это всё, конечно же, зря и надо бы спать. Но мы почему\sdash то всё сидели и~подкидывали дрова в костёр, когда он начинал угасать под редкими порывами ночного ветерка и очень некстати начавшимся под ночь дождичком. На~кой медведю в такую погоду шататься по лесу? Залёг, небось, уже в свою берлогу, или где он там ночует, и спит без задних лап, а~мы тут рефлексируем почём зря$\ldots$ 

Дождик помаленьку прекратился. К этому времени мы с Геннадичем выпили неимоверное количество чая и~порядком опустошили мой кисет. Болтали о всяком разном, вспоминали наш институт, студенческие годы, как <<учились понемногу чему\sdash нибудь и как\sdash нибудь>>, потом перемывали кости преподавателям, вспоминали однопоточников и,~естественно, немногочисленных однопоточниц$\ldots$

\newpage
\section*{<<Приём, приём!>>}
\addtocontents{toc}{\protect\setcounter{tocdepth}{0}}

$\ldots$Наверно каждый помнит гайдаевское <<Наваждение>>, историю про Шурика и Лиду, когда в институтской аудитории амфитеатром шёл приём экзамена по~электротехнике и легендарное: <<Диктую ответ на~1\sdash й~вопрос>>$\ldots$ Это~снимали в аудитории моего института, на кафедре физики. Антураж там был что надо! Старые парты, шкафы с~лабораторным оборудованием, рулонная доска крутящаяся и поражающая воображение вчерашних школьников, и, собственно, сам амфитеатр. Мне довелось не~только сдавать вступительные экзамены в этой аудитории, но и позже слушать в ней лекции по физике. Историческое, можно сказать, место! Немного, конечно, огорчало то, что образ профессора из <<Наваждения>> никак не стыковался с~реальными преподавателями. А позже эту аудиторию отремонтировали, напрочь убив современной отделкой и~мебелью весь колорит, антураж, саму былую атмосферу <<винтажности>> этого места, увы!

Учёба в институте, в целом, была намного лучше школьной\mdash было больше свободы, больше возможностей, и~при этом меньше контроля при кратно возросшей сложности и ответственности. Мне удалось поступить в~институт на~2~года раньше по возрасту, поскольку школу я окончил экстерном, и~поначалу эта разница в~возрасте между мной и~сокурсниками, одногруппниками, очень чувствовалась и угнетала. Выровнялось это лишь к~середине периода обучения, наверно, но никак не~раньше.

Проходные условия на факультет были не особо высоки, и поначалу все группы на потоке были большими, человек по~30, раздолбаев хватало. Впрочем, всё как по~<<Наваждению>>. Но, поскольку учиться было непросто, ряды поредели очень и очень сильно\mdash к концу обучения в~моей группе осталась только 1/3 первоначального состава\mdash среди моих знакомых в других институтах такого отсева не было ни у кого, но, должен признать, по большей части отсев был справедливым.

$\ldots$Лабораторные работы\mdash это <<песня>>. На них, традиционно, группу разбивали на бригады по несколько человек, чтобы каждому хватило приборов и т.д. Моя~первая бригада по физике определила мою дальнейшую дружбу с~Сергеем Лунёвым и Кириллом Рябковым. И хотя Кирилл (Кир или Киря, как мы его называем в нашей компании) позже отчислился и~мы ненадолго потеряли друг друга из~виду, впоследствии мы стали вместе ходить в~сплавы$\ldots$ но~я~забегаю вперёд. 

Итак, лабы по физике. Было~сложновато, поскольку школьные лабораторки были просто <<детским садом>>, и~этот разрыв между школой и институтом чувствовался мной очень жёстко. Но~как\sdash то справлялись. Серёга сразу на~1~курсе устроился на~подработку и учиться ему стало тяжелее обычного, я~даже не представляю как он вытянул. Поэтому в~основном подготовка к лабам в нашей бригаде держалась на мне, увы. Серёга однажды даже сказал мне, что если бы не я, то, скорее всего, вылетел бы он с~1\thinspace\nobreakdash--\thinspace 2\sdash го курса. Впрочем, я склонен думать, что он преувеличивает$\ldots$ 

\vspace{1.0cm}
$\ldots$Одно из ярких воспоминаний\mdash лабы моей бригады по~физике у товарища Гв\sdash го. Это был конец 2\sdash го курса уже. Наши неокрепшие умы, мой так точно, не~сразу поняли, что запах, источаемый им, это не~изысканный парфюм, а~просто\sdash напросто жёсткий вчерашний перегар:

\diagdash Ребят, ну у вас в подготовке тут$\ldots$ это вот что такое тут понаписано?\mdash странно растягивает Гв\sdash й.

\diagdash Так, ну здесь$\ldots$\mdash начинаю объяснять было я, но~понимаю уже, что у дяди тяжелейшая похмелуха, и все мои слова не ложатся на его мироощущение в текущий исторический момент.

\diagdash Так, незачёт, пш\sdash ш\sdash шли отсюда!\mdash это для студентов нашего вуза звучало как приговор, поскольку недопуск к этим, часто совершенно идиотским по содержанию и~техническому оснащению, лабам означал то, что их надо будет сдавать в конце семестра, в зачётную неделю перед сессией, когда и так пар из ушей. Соответственно, преподавателям на зачётной неделе тоже не до переделок этих лаб, а~без них нет допуска до экзамена$\ldots$ Сочетание этих факторов с неопытностью зелёных студентов приводило чаще всего к одному\mdash накапливанию долгов и отчислению. Поэтому чему лабы нас, как студентов, и научили, так это кромешной лжи, изворотливости, хитрости, безбожному подгону результатов ради допуска и сдачи, увы.

Нашу бригаду Гв\sdash й тогда не допустил до лабы и~удалился в свою каморочку в лаборатории. 

\diagdash Вот же пьянь!

\diagdash Шурик, а чё делать?!\mdash Серёга был, казалось, как~и~я, раздавлен ситуацией.

Откладывать на~конец семестра это всё\mdash последнее дело. Я понял, что Гв\sdash й сейчас, возможно, похмелится и~подобреет?! Подождали минут~10, пока он не~вышел к~нам снова с~а\sdash а\sdash абсолютно стеклянными глазами, и со 2 раза получили допуск к~выполнению, показав ровно те же самые расчёты, которые, что вдвойне обидно, были сделаны правильно. И такая штука случалась часто. Постепенно весь задор, запал, с которым я пришёл в институт, попросту пропал из\sdash за таких вот моментов. Исчезла какая\sdash то надежда, что <<чем дальше, тем интереснее>>, начались просто рутинные серые будни$\ldots$

\vspace{1.0cm}

$\ldots$Вспоминается незаурядный Леб\sdash в с кафедры физики. Сухенький старичок с огромными линзами в очках. Он,~конечно, был физик знатный, формулы из него лились сплошным потоком и устилали всю доску. Но,~увы, будучи хорошим лектором, семинарские занятия он вести попросту не умел\mdash это быстро осознала вся наша группа. Он~распылялся, уходил в сторону от темы, начинал что\sdash то несуразное объяснять вместо того, чтобы при разборе задач сказать, что вот так и так\mdash <<хорошо>> потому\sdash то и~потому\sdash то, а~вот эдак\mdash нет. Помнится, один раз мы с~ним схлестнулись в словесном поединке на целый семинар в обсуждении моего расчёта напряженности поля в металлах и диэлектриках\mdash он сыпал формулами, я пытался отвечать, так и не понимая чего он, собственно, хочет от меня. Длилась такая битва минут 30 точно, никто не хотел отступать$\ldots$ Пришлось мне переделывать расчёты, добывая сведения в~книжках, справочниках и у старшекурсников. Леб\sdash в так и~не смог нормально объяснить что было не так, я сам понял позднее. Ещё один гвоздь в крышку, как говорится$\ldots$

\vspace{1.0cm}

$\ldots$Радиоматериалы нам читал профессор Ар\sdash ев. Народ на потоке ошалевал от беспросветной отсталости даваемого материала\mdash весь мир давно ушёл в нанометры, а тут$\ldots$ был полный финиш. Полный. Как эту муть сдавать, что называется, <<без слёз и боли>>\mdash не знал никто. Впрочем, как\sdash то сдали. 

Вспоминается случай на лабораторках по этому предмету. Был опыт по пробою диэлектрика, в роли которого выступала многослойная промасленная трансформаторная бумага. В лабораторной методичке особо подчёркивалось, что \textit{нельзя, чтобы в зоне пробоя между слоями бумаги оказался воздух}. Ибо тогда при электрическом пробое бумага может воспламениться. Стоит ли говорить, что соблазн был чересчур велик??? Естественно, моя бригада устроила небольшой пожарчик, который, впрочем, был быстро мною потушен ветошью из старых синих лабораторных халатов под~улюлюканье бригады.

Профессор Ар\sdash ев очень плохо выглядел, он был уже в довольно преклонном возрасте. Вспоминается один случай с его лекции. Была, кажется, весна, а, значит, это был 2\sdash й семестр курса третьего, наверно. Слушать его радиоматериальскую муть никому не хотелось, понятно. И~вдруг~он, как чувствуя это, спросил:

\diagdash Ребята, ладно, а вот такой вопрос\mdash что лить: спирт в воду или наоборот, воду в спирт?

Народ мгновенно оживился! Люди у нас, конечно, на~потоке были всякие, но разведением спирта водой мало кто занимался в принципе по причине доступности всего остального. Тем не менее, разгорелась жаркая дискуссия:

\diagdash Да не, ну понятно, что$\ldots$\mdash начал было кто\sdash то.

\diagdash Не\sdash е\sdash ет, надо наоборот, потому что$\ldots$\mdash парировал другой, засыпая оппонента доводами.

\diagdash Да вы ващ\sdash щ\sdash ще не в теме! Слушайте сюда опытного человека!\mdash жарко спорили студенты между собой.

Ар\sdash ев слушал, слушал доводы противоборствующих сторон, потом он отчаялся, видимо, втолковать нам верный вариант и изрёк:

\diagdash Ребята, пейте лучше сразу водку!%$\ldots$

Аудитория грянула молодецким хохотом$\ldots$

\vspace{1.0cm}

$\ldots$Начкурса была одиозной личностью. Орала на поток она исключительно хорошо\mdash децибелы зашкаливали. Однако мне, прошедшему <<школу>> печально знаменитой Р., это было не в новинку и особо не трогало, поскольку никогда не касалось адресно меня лично\mdash мы с ней встречались лишь в конце семестров при подписании допуска к экзаменам. И всё. Но не все были как я, и именно к~таким <<не таким>> были обращены её вопли и праведный гнев, который, по~сложившейся традиции, слушали все, весь поток\mdash для профилактики, так скажем. Сложно сказать, насколько это действовало на отдельных раздолбайского плана личностей, скорее всего, что никак\mdash отсев на потоке был большой и постепенно таких студиозусов практически не осталось. А начкурса осталась и продолжала свою деятельность. Поговаривали, что вообще\sdash то она горой стоит за <<нормальных>> студентов, однако воочию я лично такого не наблюдал:

\diagdash Хороших студентов я не знаю, в том смысле, что они редкие гости у меня,\mdash распиналась она перед потоком.\mdash потому что им просто незачем ходить ко мне, а вот \textit{некоторые} (тут следовал её красноречивый взгляд в сторону какого\sdash нибудь претендента на вымещение злости) у меня из кабинета ну просто не вылазят со своими проблемами!!!

\diagdash Я$\ldots$\mdash хотел было что\sdash то возразить студиозус, но~начкурса набирала воздуха и, не давая тому опомниться, выдавала очередную тираду:

\diagdash Молчать!!! Вчера пропустил лекцию N!!! Да ты знаешь$\ldots$ \mdash брань нарастала, как цунами, окрашиваясь всё новыми виртуозными эпитетами и определениями наших умственных способностей, а децибелы её речи стремительно увеличивались, резонируя в историческом амфитеатре.

Огромные полупрозрачные тёмные очки наводили суровости на лицо, а короткую стрижку с кудрявящимися волосами дополняла ма\sdash а\sdash аленькая пластмассовая заколочка, зачем\sdash то прикреплённая сбоку надо лбом. Она, эта заколочка, была совсем ни к чему\mdash она ничего не держала, просто была прикреплена, словно говоря: <<Я\mdash женщина, если кто не понял!!!>>

%\diagdash Я\mdash женщина, если кто не понял!!! 

Она бдила нас на первых семестрах очень жёстко, что, как мне казалось, было абсолютно лишним. Те, кто хочет, и~так получат своё, а кто не хочет\mdash отсеется. Направила ли она <<на путь истинный>> хоть одного раздолбая? Едва~ли. Помогла ли тому, кто нуждался? Хотелось бы верить. Однако я убеждён\mdash всё хорошее можно нести и без ора и~без~крика$\ldots$

Однажды я вышел на сессию <<на пределе>>\mdash в~последний день зачётной недели, в последний час, можно сказать. И бегом бежал в её кабинет проставить печать:

\diagdash Успел!\mdash как отрезав, бросила она, когда я, постучав, зашёл в её каморочку.\mdash У кого?

\diagdash Метрология$\ldots$ у Кар\sdash ко.

Сдать даже зачёт, не говоря уже про экзамен, у Кар\sdash ко считалось не очень просто\mdash дядя считал метрологию не~иначе как царицей наук, а уж на лабораторки по этому предмету попортили кровушки знатно.

\diagdash Молодец!\mdash будто бы подобрев на мгновение, ответила она и молниеносно проставила штамп о допуске в~зачётке$\ldots$

\begin{center}
	\psvectorian[scale=0.4]{88} % Красивый вензелёк :)
\end{center}

\newpage
\section*{Шестнадцатая}
\addtocontents{toc}{\protect\setcounter{tocdepth}{0}}

На нашем потоке было 7 групп. Шестая носила обозначение <<16>> и имела специализацию на биомедицинские аппараты и системы. То была группа Геннадича. Помимо весьма необычной специализации для нашего факультета, особенностью данной группы было то, что она на процентов 70\thinspace\nobreakdash--\thinspace 80 состояла из девушек, в то время как в других группах потока было по одной\sdash двум девушкам максимум, а в каких\sdash то группах их не было и вовсе. Иначе говоря, 16\sdash я была просто малинником.

%\begin{center}
%	\psvectorian[scale=0.4]{88} % Красивый вензелёк :)
%\end{center}

\diagdash М-м-м, угоститься?

\diagdash Без проблем,\mdash протянул я Геннадичу кисет.\mdash Держи.

И он ловко скрутил очередную <<коровку>>. Я подкинул ещё дров в костёр, забрал свой кисет и потуже набил очередную трубочку, прикурив от костра\mdash моветон прикуривать от спичек или зажигалки, когда рядом горит костёр.

\diagdash 16\sdash я была класс$\ldots$\mdash мечтательно стал вспоминать былое я$\ldots$

\diagdash Да чё класс, Сань, не в моём все вкусе так то, а так всё хорошо.

\diagdash А вот N, например? 

\diagdash Она после техникума к нам пришла, она старше нас всех и потому всегда немного сторонилась$\ldots$\mdash Геннадич вдруг внезапно открыл мне неизвестную страницу былой истории, и это спустя почти $4,5$ года от выпуска!

Мне вдруг явственно и с горечью вспомнился 1~курс, когда всем было 17\thinspace\nobreakdash--\thinspace 18 лет, некоторым и более, а мне 16. Разница эта сотрётся только к выпуску, но не ранее, отчего существовала, особенно на 1 и 2 курсе незримая пропасть в гипотетических поползновениях в сторону прекрасной половины <<шестнадцатой>>. Быть может, оно и хорошо\mdash больше дум и времени было обращено, собственно, к учёбе, а~не~к$\ldots$ но по прошествии лет, думаю$\ldots$ зря, зазря ребята, пролетело то время! Всё равно от большинства предметов ничего не осталось в голове, стоило ли так упахиваться? Риторический вопрос.

На 3 курсе в конце весеннего семестра мы отмечали <<экватор>> учёбы. Под эту цель был снят домик загородом и заехали туда на выходные\mdash часть народа на электричке, часть на своих машинах (было и такое). Был, кажется, апрель. Снег ещё лежал в полях, и тут мы. Пиво отгружали ящиками$\ldots$ М. был ответственен за шашлык, заранее замаринованный ещё в пятницу. Не весь поток, естественно, был там, а наиболее дружные части 11\sdash й, 13\sdash й, 16\sdash й групп. Так сложилось, что именно таким составом мы часто отмечали сдачу экзаменов.

Помню, когда мы приехали, <<малинник>> был сильно озадачен\mdash посуды в доме не оказалось никакой! А народ хотел сварить картошечки, всё как полагается$\ldots$ Через полчаса Е. нашёл какую\sdash то, чистую, что не характерно, кастрюлю! Победа, казалось бы, но днище было пробито, такое впечатление, что шлицевой отвёрткой. 16\sdash я стала паниковать, что кулинарная часть под угрозой, так сказать, срыва. Но тут в дело включился я и достаточно быстро предложил законопатить узкую пробоину$\ldots$ деревянными зубочистками. Разбухнув от воды, они бы ещё лучше устранили течь, справедливо положил я. Так и сделали\mdash я и однокурсник, впоследствии получивший прозвище <<Вентилятор>>, проделали сию операцию и <<спасли>> мероприятие, сорвав немного внимания <<шестнадцатой>>.

Перепились, конечно, почти все на этом <<экваторе>>, но такова уж традиция. Наутро выжившие продолжили <<заряжаться>>, помню ещё даже остался целый не распакованный ящик пива$\ldots$ Ближе к середине дня я покинул сие мероприятие и в весьма абстрактном виде туманно добрался домой на перекладных. 

Впоследствии мы ещё соберемся в похожем формате позднее, уже после вручения дипломов, в ресторанчике около Измайловского парка, чтобы выйти из него в другую, совсем иную жизнь\mdash как обычно при подведении невидимой черты думалось: лучше ли она будет, счастливее ли? Тогда казалось, что да. А сейчас? Как обычно\mdash в чём\sdash то лучше, в чём\sdash то хуже, закон сохранения энергии никто не отменял. И~всё~же\mdash~хорошо, что во время обучения была 16\sdash я$\ldots$

\begin{center}
	\psvectorian[scale=0.4]{88} % Красивый вензелёк :)
\end{center}

$\ldots$Когда пошёл четвертый час ночи, мы c Геннадичем, не сговариваясь, подкинули в костёр три больших бревна, сделав подобие таёжного костра, и разбрелись спать по своим палаткам. Уснул, несмотря на поздний час, я очень неспокойно, сжимая рукоять топора в~руке$\ldots$ 

\begin{center}
	\psvectorian[scale=0.4]{88} % Красивый вензелёк :)
\end{center}
