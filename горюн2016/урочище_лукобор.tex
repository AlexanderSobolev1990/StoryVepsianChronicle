%\chapter{Урочище Лукобор. 07.08.16} 
\chapter{Потапыч. 07.08.16} 
\corner{64}

Проснулся достаточно рано, но вылезать из палатки совершенно не хотелось, и вскоре я снова задремал. Соизволил вылезти, когда народ тоже уже проснулся и начал шуметь. Стали готовить завтрак\mdash Геннадич проявил рвение и взял инициативу в свои руки. В итоге у него получилась обалденная каша, в которую также добавили орехи, изюм, курагу и другие сухофрукты\mdash как мы обычно делаем. Пока он кашеварил, я достал флаг Советского Союза и укрепил его на свежевырубленном импровизированном флагштоке. <<Неба утреннего стяг$\ldots$>> и снова аккорды гимна на губной гармошке в исполнении Геннадича. 

Потом вспомнили, что у нас есть блинная мука и захотели сделать оладьи с яблоком. Приготовили тесто, как умели, и Дима начал готовить. Хотели сначала печь оладьи в крышках котелков. Но крышки оказались слишком тонкими и всё сгорало. Усовершенствовали конструкцию\mdash стали печь в двух крышках, поместив между ними песок для распределения тепла. Снова почему\sdash то не вышло. Блинчик снизу подгорал, а сверху по\sdash прежнему оставался абсолютно сырым. Дима достал свою газовую горелку и мы повторили всё то же самое, только теперь не на костре, а на горелке. Но результат вышел тот же\mdash сырой подгорелый блинчик. Тогда С.Ю. предложил просто запечь тесто в котелке. Мы поставили котелок на угли и вокруг тоже привалили углями. Пахло безумно вкусно. Все ходили вокруг котелка в нетерпении$\ldots$ но увы, всё тоже сгорело, превратившись в полусырую несъедобную массу. Короче, как мы поняли, блины в походе\mdash не мужских рук дело. А вот как теперь отчищать котелок? В нём напрочь пригорели остатки <<теста>>. Я сразу в шутку предложил этот котелок выкинуть к чёртовой матери. Но С.Ю. сказал, что свой котелок в беде не бросит и пошёл на берег отчищать его. Поели, короче говоря, оладушков.

Но, поскольку мы уже до того зарядились кашей, отсутствие оладушков нас не сильно огорчило. Так что днёвка начиналась почти отлично. По синему\sdash синему небу плыли частые кучевые облачка, день обещал быть жарким. После завтрака мы с Геннадичем решили сплавать из интереса на противоположный берег, потому что Паша сказал, что видел там$\ldots$ медведя. Оказалось, Паша рано утром встал и пошел на рыбалку, и тут ему улыбнулась удача\mdash несколько крупных рыбёшек стали добычей. Поймав пару окушков, он и увидел медведя на том берегу. Не очень приятное соседство на днёвке, однако. Надо разведать, что там такое и вот мы уже на пустой байдарке мчимся вдвоём на правый берег, на песчаную отмель чуть ниже по течению. 

Вылез из байдарки и сразу заметил следы, но не медвежьи, а собаки\mdash воображаемая линия проводилась между 1 и 4 когтём, пересекая подушечки лапы\mdash значит это собака, а не волк. Откуда тут собака? Ну да ладно, пошли с Геннадичем дальше и$\ldots$ наткнулись на свежие следы медвежонка. Отпечаток его лапы был меньше моего, у меня 44\sdash й размер ноги. Сразу стало как\sdash то не по себе. Сфотографировав свидетельства пребывания тут медведя, мы поспешили убраться с этого пляжа обратно в лагерь. 

Настроение у меня подпортилось. Мы встали на днёвку, да ещё место такое живописное, а тут медведь где\sdash то ходит неподалёку! Вот тебе раз! По прибытию в лагерь я сразу проверил на всякий случай на месте ли сигнальные ракеты. Глупо, правда, надеяться, что они как\sdash то помогут в случае нападения хищника. Хотя, с другой стороны, зачем ему нападать? Мы ему ничего не сделали$\ldots$ не считая того, что встали тут на днёвку. Короче говоря, мысли роем носились в адмиральской голове, сталкивались там и думали, что предпринять в данной ситуации$\ldots$ вплоть до снятия со стоянки. Но в итоге я решил оставаться, потому как медведь больше не показывался, погода была блеск, место\mdash супер, а~чувство тревоги понемногу ушло. 

Паше и С.Ю. улыбнулась рыболовная удача\mdash удалось за день наловить 6\thinspace\nobreakdash--\thinspace 7 крупных рыбёшек. Паша решил их закоптить вечером на углях в специальном пакете. Я~успел вымыться в реке, перестирал все вещи и развесил их сушиться. Потом зашёл по песчаной косе далеко выше по течению и обратно приплыл самосплавом, вылез и стал загорать. Красивейшее место мы выбрали! Дима сидел у воды на стульчике и читал книгу, а Ваня играл рядом с~ним в~песочке. Идиллия! Паша и~С.Ю. продолжали рыбную ловлю, а Геннадич релаксировал в лагере.

День, тем временем, уже клонился к вечеру. Небо расчистилось, и оранжевые краски закатного солнца стали освещать песчаный берег. Я~взял штатив, фотоаппарат, дистанционный пульт и пошел на фотоохоту\mdash закатный свет известен своей мягкостью и кадры, сделанные в это время, получаются особенно хорошими. Так и случилось\mdash кадры вышли замечательные. Но оранжевое солнышко вскоре зашло за сосны\mdash пора было идти в лагерь. С.Ю. и Ваня, тем временем, уже почистили картошку в реке. Ужин поспевал. Ваню обучали рубить дрова топором, но у него пока не очень хорошо это получалось. С.Ю. нашёл в лесу костянику. Добавили её в чай. Подоспела копчёная рыба от Паши\mdash объедение! Наш рацион, и без того разнообразный, обогатился таким шикарным лакомством. К копчёной рыбе у~нас был даже лимон\mdash просто барство! 

Я сидел на скамеечке, обильно закусывал с ломившегося от яств стола и блаженно выпускал вверх колечки из трубочки. А вот у ребят кончилось курево$\ldots$ я им говорил до похода, что на природе расход в 2\thinspace\nobreakdash--\thinspace 3 раза больше, но это все всегда рассчитать не могут. Ребята стали стрелять мой рассыпной табачок и набивать его в кальку от конфет <<Коровка>>. Сии адски смолящие калькой самокрутки мы так и назвали\mdash <<Коровка>>. 

Вскоре после перекуса копчёной рыбой, мы навернули сытнейший ужин и продолжили опустошать запасы 96\sdash го, травя всякие байки и болтая о всевозможной чепухе. И тут\sdash то хозяин леса снова дал о себе знать\mdash на противоположном берегу послышался громкий хруст веток. Мы повскакивали от стола и вышли на берег. В зарослях определенно ходил кто\sdash то массивный\mdash ветки хрустели, кусты гнулись. Виновника не было видно. Честно, стало не по себе. Я стоял в оцепенении и перебирал в голове варианты, что бы предпринять. В итоге я взял два топора и стал соударять их, извлекая громкий и мелодичный протяжный металлический звук из своего кованого топора. Хруст углубился в заросли и удалился. Мы стояли и продолжали трапезу уже стоя\mdash не отвлекаться же от ужина, в самом деле? 

Но вскоре стало вообще не до смеха\mdash на том же бережку, куда мы с Геннадичем плавали с утра, показался мишка. Видимо, испугавшись звона топора, он отошёл вглубь зарослей и вышел подальше от нас к воде. Не смотря даже в нашу сторону, он медленно подошёл к реке, постоял, помедлил, попил воды. В лагере нарастало напряжение. Я достал из штормовки два сигнала охотника и зарядил их. Стрелять или не стрелять вверх, вот в чем вопрос! Отпугнёт это или наоборот, привлечёт зверя? С медведем я сталкивался 1 раз в Башкирии, но тогда я лично его не видел, да и нас было 50 человек, так что момент истины я тогда как\sdash то не прочувствовал. А сейчас, стоя с двумя сигналами охотника, готовый к стрельбе <<по\sdash македонски>> с двух рук, я размышлял, что же всё\sdash таки делать. 

Мишка, вообще не обращая на нас никакого внимания, попил воды и стал переплывать реку на наш берег! Правда, не в нашу сторону, а по диагонали от нас\mdash ниже по течению. Но это всё равно\mdash на наш берег! Хмель сошёл буквально за секунды! Ване, крутившемуся между нас, сказали не бегать и он замолчал. А потом расплакался, видимо тоже, наконец, ощутив страх. Ваня запричитал, чтобы мишка не кушал его, а кушал нас, потому что мы\mdash старые по его мнению. До\sdash о\sdash о\sdash брый мальчик! Ну, так\sdash то да, заросшие щетиной мы кажемся старше, чем есть на самом деле. Ваня, впрочем, быстро успокоился и ещё раз напомнил нам, что мы старые и нас кушать уже можно, а его нет, чем разрядил обстановку. Я ещё немного постучал топорами друг об друга, издавая громкий протяжный звук. Мишка же спокойно переплыл реку, вылез на берег и углубился в лес$\ldots$ вроде бы не в нашу сторону.

Тут Дима схватил топор и вырубил две огромные рогатины, справедливо заметив, что наши предки как\sdash то же охотились на медведя без огнестрельного оружия$\ldots$ Никто не поддержал его энтузиазма. Я решил, в случае чего, уходить по реке на пустой байдарке, а потом вернуться за вещами\mdash мишка не стал бы вечно околачиваться в районе лагеря. Короче говоря, перепугались мы не слабо. Естественно, здоровый сытый зверь боится человека. Но не пойдешь же искать его и спрашивать, мол\mdash ты сытый и здоровый или нет? 

Потапыч ушёл, а тревога осталась. Кроме того, погода начала портиться и стал накрапывать дождик. С.Ю. сделал единственно верное дело\mdash унёс остатки сала и повесил пакет с ними на берёзу в стороне от лагеря, а остальные продукты сказал сложить около стола, а не в палатках. Потом, оглядев нашу порядком струхнувшую бригаду, он сокрушенно сказал, что мы зря боимся. Зверь боится нас гораздо сильнее, чем мы его и т.д. и т.п.\mdash что и так вертелось у меня в голове, но нервишки подсказывали другое\mdash жги костёр. В голове будто совещались здравый смысл и страх:

\diagdash Но ведь медведь боится нас сильнее$\ldots$  

\diagdash ЖГИ КОСТЁР! 

\diagdash А как же природный страх животного перед человеком? Ведь все животные боятся человека и огня$\ldots$  

\diagdash Ж\sdash Г\sdash И \enspace К\sdash О\sdash С\sdash Т\sdash Ё\sdash Р!

Дима с Ваней пошли спать, С.Ю. тоже улёгся пораньше, а мы с Пашей и Геннадичем остались жечь костёр и добивать конфеты <<Коровка>>. Впрочем, Паша скоро тоже ушёл, напившись чаем, остались только мы с Геннадичем. Под шум редкого дождика жгли костёр и болтали о разном, стараясь разогнать сон. В шуме ветра и дождя мне всё чудилась поступь хищника, блуждающего по лесу. 

Перевалило за полночь. Взбудораженное сознание рисовало медведя, сидящего в засаде и тихонечко ждущего, когда же эти негодные людишки наконец уснут. Стали болтать про печальную судьбу группы Дятлова, инопланетян, бактериологическое оружие и прочие страшилки у костра. Но, чем дольше мы сидели, тем больше я понимал, что это всё, конечно же, зря и надо бы спать. Но мы почему\sdash то всё сидели и подкидывали дрова в костёр. Дождик помаленьку прекратился. К этому времени мы выпили неимоверное количество чая и порядком опустошили мой кисет. Болтали о всяком разном, вспоминали наш институт, как учились, потом перемывали кости преподавателям, вспоминали студенческие годы$\ldots$ однопоточников, однопоточниц$\ldots$ %потом углубились в школьные года$\ldots$

%\section*{Профессура}
\addtocontents{toc}{\protect\setcounter{tocdepth}{0}}

Математичка Р. влетала в собственный класс стремительно, как спецназ залетает в штурмуемое помещение. Стены класса были выкрашены в синий цвет, отчего это мрачное скорбное место становилось ещё холоднее и неуютнее. Ёжился на стуле при её появлении, наверное, каждый. На голове вечная химия, осветляющая тёмные волосы наверху, очки тёмные и полностью закрывающие глаза так, что практически невозможно было понять, куда она смотрит, полностью отсутствующая талия. Мужа у Р. не было, но было двое детей — взрослая дочь и старшеклассник сын.

- Садитесь!! - раздавалось её прокуренным, ненавистным голосом. Наверное, не было никакого другого учителя, которого ненавидели больше, чем её не только в нашем классе, но и в школе в целом! В арке 9-этажого дома, сквозь которую вела дорога к школе, кто-то из самых отчаянных голов в отместку за её ор, за унижения, за гнетущую атмосферу на уроках, за полную безжалостность, за всю ту боль, психологически уродовавшую малолетние сознания и ежедневно причиняемую нам, написал из баллончика синей краской шикарным курсивом шедевр, целое художество, под которым, я уверен, мысленно был готов подписаться каждый ученик Р. :

- «Р. - СУКА». Естественно, фамилия была написана полностью. Второе слово было написано смачно-размашисто.

Все знали, что она ходит на работу через эту арку каждый день. И другой дороги по асфальту нет. Надпись висела примерно полгода. Это низко и ужасно, но когда мы, ученики, шли через эту арку с утра и вечером, синяя надпись придавала сил - как от души отлегало. Ощущалось хоть некоторое отмщение. Словом, примерно полгода эта надпись была отдушиной. После того, как её закрасили, повторить подвиг никто не решился, хотя, вроде бы, автор так и не был найден.

- Света! К доске! - нарочито с наслаждением предстоящей казни вызывала она несчастную на эшафот. В тот день Р. Была особенно не в духе. Через несколько минут, когда Свету, особо не шарящую в алгебре, доводили чуть не до слёз, брань достигала апогея:

- Света, звезда… - тут Р. заминалась, подбирая приличное нематерное слово — звезда балета! Садись, 2!

Следующий несчастный выбирался по журналу и его ждали аналогичные эпитеты в случае провала.	

Но стоит признать, при всём при этом она часто, почти всегда, была справедлива. Если кто-то чего-то не знал — получал заслуженно свои двойки и тройки. Знал — без проблем ставила отлично. Она не снижала отметки из-за личного отношения… такое впечатление, что мы все были ей одинаково противны. 

Подача же материала у нее была институтская, как я понял позднее. Она требовала в расписании себе неуклонно спаренных уроков. На первом устраивалась лекция — мы писали конспекты. По сути, она, именно она, научила нас конспектировать, причём делала это профессионально — давала время записать, всё разжёвывала. Эти конспекты были, без иронии, шедевром изложения материала, как я понял позднее. На втором уроке был, на институтский манер, семинар — практическое занятие. Тут она оттягивалась по полной. Через призму лет абсолютно понятно за что — ведь материал разжёвывался ею досконально, а мы же часто много чего не понимали (но боялись спросить!), много тупили, не всегда всё усваивали с  первого раза. И даже со 2-го. И с 3-го. Кроме того, что мы были детьми, которым хочется гулять, прыгать, бегать, дурачиться и еще чёрте чем заниматься, многие из нас реально не понимали на кой нам сдалась эта математика на таком жёстком уровне. Многие откровенно не тянули и это нормально — не всем же быть лапласами, бесселями, колмогоровыми и т.д. и т. п. Но Р. считала иначе, она требовала с каждого, а кто не тянул — становился объектом едких подколок, унижений, ругательств.

В определенные дни казалось, что всё — если завтра школа не сгорит, на неё не упадет метеорит или не смоет цунами, то придётся каким-то самым нечеловеческим и ужасным образом расправиться с Р. чтобы избавиться от психологического гнёта. Но дальше синей надписи в арке ни у кого дело не зашло. Р. умела держать аудиторию — не хуже капо.

Нас, «ашников» она называла «профессурой» в пылу гнева и брани за очередной невыученный материал. Ругалась она виртуозно. Как ей удавалось не переходить при этом на мат — до сих пор не понимаю. Годы спустя, прочитав [географ глобус пропил], я был поражён – ведь роман был написан в 1995 году, а полностью издан только в 2003, а Р. называла нас так же, как и герой-учитель в книге — профессурой. Совпадение? Звучало это очень ёмко, хлёстко, издевательски обидно, с чуть протяжным <<c>>.

Май, конец года. Профессуре было поручено вымыть класс Р. Девочкам — отдраить окна, пацанам — вымести пол и помыть его. Хуже наказания и представить трудно. Каждый, я подчеркиваю, каждый был бы не прочь разбить эти ненавистные окна в этом классе, учинить расправу над каждой партой, каждым стулом, который был в часы уроков раскалён как геенна огненная под нашими ученическими задами. Но нас заставили мыть. Не было в мире бОльшего наслаждения, чем, улучив момент, незаметно помочиться в ведро с водой, которой потом помыли пол…

Ко мне Р. относилась как и ко всем — безразлично жестоко. Один раз я никак не мог понять куда в уравнении y=x2 девается, собственно, функция? y=f(x) — тут всё понятно, функция это f, а в уравнении y=x2 ? А что если уравнение сложнее, в несколько слагаемых? В голове, тем временем, совсем другое, не эти скучные функции, не области определения, не интервалы, отрезки, неравенства и многочлены. Всё это пригодится позднее… но тогда — тогда мы, никто из нас, я уверен, не понимал за что нам такая кара, где мы так провинились? Сейчас разбуди среди ночи и спроси формулу нахождения корней квадратного уравнения — отвечу. Но зачем, есть же справочники? Кто-то скажет, что всё обучение — ради нейронных связей в мозгу. Да, возможно, но не таким способом. 

Уже не помню по каким причинам, но, кажется, мы мыли её класс часто — то ли в какой-то год у нее не было классного руководства и, соответственно, собственных несчастных кто бы это делал, то ли ещё почему-то. Скорее потому что у нашей классной было 2 класса — 1-й мыл наш «родной» класс, а 2-й, то есть мы, профессура, отдувались на других фронтах. Конечно, эффективность такого подневольного труда была крайне низкой. 

Говорят, что школьные годы вспоминаются с теплотой — ничего подобного. Как по мне, невозможно вспоминать какое-то время только с теплотой или только с ненавистью — всегда речь идёт о конкретных моментах. Моменты у профессуры были большей частью негативные, даже, я бы сказал, жуткие. Может быть это моменты лишь моего восприятия? Но нет, синяя надпись говорила сама за себя — не только моего. С каким наслаждением я вздохнул после того, как покинул эти стены навсегда! И хотя впереди была полная неизвестность, «профессор» был если не счастлив, то по крайней мере рад. 

И хотя я много раз, уже в институте, впоследствии сказал мысленно спасибо Р. за вдолбленные в голову знания, за системный подход к преподаванию, чего многим учителям, как школьным, так и институтским, попросту никогда не удаётся добиться и близко, тогда кроме животного страха, ненависти, желания расправы она не вызывала ни-че-го, увы. Если так доставалось «профессуре», я просто не хотел и не хочу знать, как доставалось «отцам» и «зондеркомманде»[географ] — иногда, сидя на других уроках даже на другом этаже, даже за закрытой дверью, нам было слышно, как Р. орёт на очередных несчастных — слушала вся школа — дверь в свой класс она закрывала только на контрольных… 

Говорят, плохое забывается со временем. Нет уж! Забывается как раз хорошее, поскольку воспринимается как должное! А вот плохое, особенно очень плохое — въедается так, что никакими средствами не отмоешь, не выскребешь, не отчистишь. Любой из «профессуры» подтвердит...

%\section*{Джомолунгма}
\addtocontents{toc}{\protect\setcounter{tocdepth}{0}}

...У неё были густые длинные, чуть суховатые, натурально(?) рыжие волосы. С высоким ростом, нехудощавой комплекцией, широкими бёдрами, волнительной грудью, чуть узковато посаженными глазами, она, молодая географичка, казалась нам, 6-классникам, образцом совершенства. Словно для подчёркивания рыжины волос, она часто носила зелёный цвет. 

На уроках царил порядок — самые отъявленные негодяи стихали, откровенно любуясь её красотой. Конечно же она видела (не могла не видеть!) это и всё понимала, но вела себя в высшей степени достойно, как и подобает настоящему Учителю. Я не совру, если скажу, что почти все пацаны были в неё влюблены. Несмотря на то, что она была дочерью математички Р.

В противовес своей матери, она никогда не кричала, вела уроки тихо и спокойно, рассказывая нам о физической географии. Стоило каким нибудь раздолбаям в классе вдруг зашуметь, он лишь замолкала и пристально откровенно смотрела на них. Это работало ну просто безотказно — парни тут же смолкали под её взглядом, смущаясь, казалось бы такого желанного внимания своего идола, и сидели внимали. Порядок, как мне кажется сквозь года, держался исключительно на её внешних данных, её спокойствии и интересе к рассказываемому — страха не было — во время её уроков никто не вспоминал про её мать и уж точно никакая мысль, что Р. может прийти и наорать на нас или что географичка может нажаловаться на нас матери, не была причиной той чарующей атмосферы географических открытий, топокарт, гор и долин, морей и океанов, царившей на её уроках… Поразительно, но, будучи разительно другой, нежели чем своя мать, она не вызывала у нас никаких ассоциаций с Р., никогда ни у кого не возникало желания выместить на ней свою злобу к математичке, отыграться как-то, наоборот, она воспринималась нами как нечто кристально чистое и незамаранное в общей картине учительского болота.
 
Она учила нас пользоваться топокартой: читать легенду карты — условные знаки, определять азимут на местности, обходить препятствия по азимуту, вычислять расстояния. Обучала масштабу и прочей науке… Так моя любовь к топокартам и без того подпитываемая настоящим советским географическим атласом для учителей средней школы (с надписями главное управление геодезии и картографии при совете министров СССР), заиграла новыми красками — я осознал насколько география и картография интересные и полезные науки. Дополнительную пищу для восхищения давал том номер 1 Детской Советской Энциклопедии, где тоже был раздел про картографию и походы. Всё вместе это — да, именно это — зародило в самых потаённых уголках моей души чувство, что я должен когда-нибудь сам пойти в поход. Чтобы у меня была топокарта, компас и осознание того единения, того отчуждения от всех, которое испытывают топографы, географы, работая в «полях»
. 
Как помнишь всегда x1,2= b+-sqrt(b2-4ac) / 2a, так помнишь в любое время дня и ночи и высоту самой высокой горы — Джомолунгмы (Эверест). 8848 метров на уровнем моря. Но, как нетрудно догадаться, совсем по другим причинам… как поразительно разными методами, способами, можно добиться от детей знаний и как ужасно прочувствовать на себе весь спектр средств, применяемых иными <<учителями>> в работе… 

На географию, ясное дело, мы ходили как на праздник. Ни один школьный предмет не вспоминается с такой теплотой, как этот. Сквозь года очень сложно сказать, сколько географичке было лет, тем более в детстве все кажутся таким взрослыми… Вероятно ей было от 25 до 30. Она была самым молодым учителем в нашей школе. Вскоре на её руке появилось кольцо и она уволилась, а может быть ушла декрет — мы не знали. Курс экономической географии вместо нее читал нам другой учитель, чей «авторитет», манера подачи, внешние данные — короче всё было против нее. Тогда конечно вся пацанва была жутко расстроена уходом нашей географички, но сейчас, через года, ничего не остаётся как искренне за неё порадоваться. Работать в том коллективе учителей я бы не пожелал абсолютно никому. 

От экономической географии в голове не осталось ничего, а физическая до сих пор вспоминается не то чтобы даже с теплотой, а с каким то просветом, ярком лучом… поистине «лучом света» в «тёмном царстве» остальных предметов. А очарование топографической карты всегда вызывает в душе самые что ни на есть приятные, походные чувства — вот уже ощущаешь звуки леса и шум реки, запах полей и разнотравья лугов, проносишься по просекам и ниточкам лесных дорог, уносишься в глушь нашей территории, где нежданно в лесу тебя ждет избушка охотника….

Я мечтал стать Географом. Я им не стал...


\begin{center}
	\psvectorian[scale=0.4]{88} % Красивый вензелёк :)
\end{center}

$\ldots$Когда пошёл четвертый час ночи, мы c Геннадичем, не сговариваясь, подкинули в костёр три больших бревна, сделав подобие таёжного костра, и пошли спать. Уснул неспокойно, сжимая рукоять топора в руке$\ldots$ 

\begin{center}
	\psvectorian[scale=0.4]{88} % Красивый вензелёк :)
\end{center}
