\chapter{Заключение и выводы} 
%\corner{64}
\vepsianrose

Вот так и закончилось наше мероприятие\mdash я всё описал, что называется, от подъезда и до подъезда в прямом смысле. Мы вместе с женой затащили все вещи в квартиру, мне уже приготовили горячую ванну и бритву\mdash зарос щетиной я почти как кубинские революционеры. Отмывшись и побрившись, откушал завтрака и стал с энтузиазмом рассказывать про наше путешествие, а жена внимательно слушала и уверяла, что в следующем году обязательно пойдёт вместе со мной$\ldots$ я верил. Эмоции при рассказе про медведя просто зашкаливали!

Этот сплав принёс мне не только уверенности в~том, что я~могу организовывать такие мероприятия, но~и~сознание полноценной завершенности маршрута \textnumero57, хоть для этого и потребовалось два года. В~этом сплаве всё было гладко, мы прошли маршрут чётко по моему плану и безо всяких пробоин. Воды, как я уже писал, было достаточно. Экипажи, подобранные мной и С.Ю., оказались удачными. Все трудности, какие нам встречались, мы преодолели.

Выводы:
\begin{enumerate}
\item По маршруту:
	\begin{enumerate}
	\item[$-$] Интересный участок с высокими берегами и~узким руслом закончился вместе с рекой Горюн. Так~что, возможно, лучше стартовать все\sdash таки от~Ефимовского, а не от Вожаней, но тут опять же, не~угадаешь с уровнем воды. По сравнению с Лидью, Горюн более предсказуем и шивер по~нашей воде не~было замечено в принципе. Однако Лидь оставила в моём сознании более красочные воспоминания, нежели Горюн.
	\item[$-$] Чагода и Чагодоща, будучи более полноводными в~этом году, понравились больше, потому что не было опасности сесть на мель посреди реки.
	\item[$-$] Пороги на Чагодоще\mdash миф. Возможно, чтобы их хоть как\sdash то заметить, надо как раз попасть туда в~такое маловодное лето, как в 2015 году.	
	\end{enumerate}	
\item По заброске/выброске:
	\begin{enumerate}
	\item[$-$] Заброска микроавтобусом чрезвычайно удобна\mdash вас заберут прямо от подъезда и к нему же потом привезут. Если набирается хотя бы человек 6\thinspace\nobreakdash--\thinspace 8, то~это уже экономически оправданно\mdash получается не намного дороже, чем поездом, зато комфорта несравненно больше и быстрее. Нигде не придётся таскать тяжёлые рюкзаки, кроме как до машины и~от~машины. Поэтому, заброска/выброска поездами имеет право на жизнь в случае уж очень дальних забросов или если поблизости нет автодорог. Во всех остальных случаях, конечно же, автотранспорт лучше.
	\item[$-$] Водители микроавтобусов, которые ездят на дальние расстояния, как правило, уже опытные люди и~им~привычно всю ночь проводить за рулём. Но~я всё равно предпочёл как можно дольше не~спать и контролировать ситуацию. Лишним это мне не~показалось.
	\item[$-$] По выброске получилось, что мы раньше на~1~день снялись с маршрута, и пришлось заново договариваться с водителем. Это неизбежные трудности, которые надо учитывать при подготовке и обсуждении выброски с водителем.
	\end{enumerate}	
\item По экипажам:
	\begin{enumerate}
	\item[$-$] 6\nbdash 8 человек\mdash предельное количество народа, когда кончается порядок и начинается хаос. У нас был порядок, чему я несказанно рад. Все экипажи показали себя с лучшей стороны, преодолевая все встретившиеся нам трудности.
	\item[$-$] Ребенку одному в походе было скучновато. Лучше, если детей будет несколько. Тогда они организуют <<междусобойчик>>. Сколько радости принесет им поход сознанием того, что взрослые взяли их на~такое~мероприятие!
	\item[$-$] Опыт чисто мужского похода показывает, что женщин с собой всё\sdash таки надо брать. В их присутствии мужчины стараются не выражаться крепко и вообще ведут себя по\sdash другому. Дополняя сказанное про хаос от женщин в походе \cite{Квадригин}, стоит отметить, что, по моему мнению, это касается незамужних дам. Замужние, обычно, пытаются командовать только своими мужьями, а не всеми подряд. Тут уже наше мужское дело давать этому отпор.
	\end{enumerate}	
\item По сняряжению:
	\begin{enumerate}
		\item[$-$] Этот поход стал удачным подтверждением моей теории о том, что одному экипажу следует согласованно подготавливать список вещей. Геннадич умудрился все свои личные вещи впихнуть в одну герму, как и я. Ну, кроме, может быть фотографического штатива. Остальное снаряжение, которое было у нас в байдарке\mdash это групповое. Как то\mdash тент, топор, пила$\ldots$
		\item[$-$] Тент в сплаве нужен. Если в 2015 году его даже ни разу не вытаскивали из чехла, то это только от хорошей погоды. В этом году он нас здорово спасал. Вещь нужная, брать один на группу однозначно. Ставили его на моих вёслах от <<Тайменя>>, что оправдало себя.
		\item[$-$] Надувной матрас лучше изолирует от холода земли, нежели туристический коврик. Впрочем, Геннадич брал два туристических коврика и утверждал, что не мёрз. 2 коврика или надувной матрас\mdash выбор каждого. Мой выбор\mdash один коврик и~один надувной матрас. Днём коврик сворачивается трубочкой и~служит спинкой в байдарке.
		\item[$-$] Газовый резак для разведения костра\mdash брать однозначно! Это очень удобная штука, надо иметь по крайней мере одну такую горелку на~группу. В случае дождя и мокрых дров она выручает просто неимоверно. Даже не представляю, насколько затягивалось бы разведение костра в~этом дождливом сплаве, если бы я не обзавёлся такой~горелкой.
		\item[$-$] Газовая плитка. Один раз на ней мы приготовили плов и~несколько раз вскипятили воды на чай. Имеет право на жизнь, места и веса много не занимает. Но тащить их несколько на 6 человек бессмысленно при таком мероприятии, как наше. Вероятнее всего, будет правильно сказать, что это такой своеобразный запасной вариант или если точно знаешь, что дров на маршруте нет.
		\item[$-$] Панама оправдала себя лучше, чем кепки. Нос и~уши не обгорают на ярком солнце, будучи прикрыты полями панамы. Харизматичная фетровая шляпа С.Ю. была вне конкуренции.
		\item[$-$] Сплавляться при чувствительной и ещё незагорелой коже лучше в х/б с длинным рукавом. У меня на такой случай есть то-о-оненький камуфляжный костюм, но пойдет и обычная х/б рубашка. Стало жарко\mdash закатал рукава. Чувствуешь, что поджаривает солнышко\mdash раскатал и намочил водой. 
		\item[$-$] Более плотная курточка с капюшоном а\sdash ля штормовка\mdash однозначно должна быть. Только, не синтетическая. Вообще, судьба синтетических тканей в походе печальна. Обычно они заканчивают в костре. Исключение составляет разве что флис.
		\item[$-$] Сапоги. В этот раз в них был Ваня и это простительно\mdash они ехали у него на байдарке и никому не мешали. Остальные были либо в~шлепанцах, либо босиком, либо в химзащите. Моё предпочтение при прохладной воде, непогоде и~просто нежелании мочить ноги\mdash за химзащитой по~причине легкости, компактности и дешевизны.
	\end{enumerate}
\item Общие выводы:
	\begin{enumerate}
		\item[$-$] Сработал принцип 2/3 к 1/3, когда большинство команды к старту (2/3) знает друг друга, а~меньшинство, 1/3\mdash нет. Из этого получилась хорошая команда, способная к сплоченным, быстрым и организованным действиям.
		\item[$-$] Музыкальный инструмент в походе однозначно должен быть. В этот раз это была губная гармошка. Я был приятно удивлен, как столь, казалось бы, простой и небольшой инструмент, способен скрасить не только погодные условия, но и вообще принести много удовольствия. Хочу еще попробовать варган$\ldots$
		\item[$-$] Норма километража, запланированная на день, сильно зависит от уровня воды и скорости течения. А узнать заранее уровень воды в реке и силу течения невозможно. Поэтому получается что\sdash то вроде лотереи при планировании километража заранее. Выходит что, не планировать вовсе? Нет, планировать нужно, просто закладываться всегда не~на~пределе возможностей, а чуть меньше. Норму километража в день стоит наращивать постепенно, чтобы руки постепенно втягивались в греблю, и не возникло ненужных негативных чувств от~физического перенапряжения.
		\item[$-$] Всегда нужно иметь вариант досрочного антистапеля. ВСЕГДА.
		\item[$-$] Для отпугивания диких животных иметь с собой по крайней мере одну сигнальную ракетницу. Конечно, лучше иметь и полноценное охотничье ружьё с собой, но пока среди моих знакомых не нашлось человека с~ружьём.
		\item[$-$] Будильник Адмиралу ставить. Никуда не денешься от этого. Иначе проспишь  со 100\% вероятностью\mdash спится на свежем воздухе просто отменно.
	\end{enumerate}
\end{enumerate}	

На этом закачивается повествование о нашем водном приключении на реках древнего Тихвинского водного пути по рекам Горюн\sdash Чагода\sdash Чагодоща. Вся наша команда, безусловно, шикарно отдохнула на воде и вернулась загорелая и обновлённая. Мы испытали на себе и милость нашей природы, которая выражалась в хорошей погоде, упоительно красивых берегах и красивых стоянках, и хмурость дождей и ветра, напоминающую нам о том, что мы\mdash не хозяева природы, но составная часть её.

К концу нашего сплава у меня сложилось непроходящее чувство, что в следующий раз я обязательно попаду на реку Песь. Так ли это? Посмотрим.

\begin{center}
	\psvectorian[scale=0.4]{88} % Красивый вензелёк :)
\end{center}

\begin{center}
	\Large {КОНЕЦ}
\end{center}
\vspace{\fill}
\begin{flushright}
	\copyright~Соболев~А.А.~Москва~29.05.2017\\
	\textit{ред.~27.02.2018,~22.08.2022}
\end{flushright}