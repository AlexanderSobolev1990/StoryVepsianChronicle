\newpage
\section*{Спец}
\addtocontents{toc}{\protect\setcounter{tocdepth}{0}}

$\ldots$Это был уже 4 курс, вовсю начались спецпредметы. Один из таких спецпредметов нам читал профессор Б\sdash ов. Он был довольно грозным дядей, но объяснял материал не~особо хорошо. Пособия по этому предмету же выглядели словно древний манускрипт\mdash мало что было понятно. И нам приходилось как\sdash то выкручиваться, справляться. 

Один раз его лекция шла уж как\sdash то совсем вяло. Он то и дело ошибался, стирал пол\sdash доски, начинал заново. Обычно с ним такого никогда не бывало\mdash он всегда был грозен как скала. Но тут мы поняли\mdash это жесточайшее похмелье$\ldots$ очередная формула была записана как\sdash то не так:

\diagdash Ух, ё!\mdash стёр он последнюю строку чуть не рукавом своего зелёно\sdash коричневого пиджачка, осознав написанное.

В аудитории пошёл смешок\mdash это было уже не~первое исправление, а речь нашего лектора была несколько растяжной.

\diagdash Что, смеётесь над лектором?\mdash не то грозно, не~то отстранёно\sdash безразлично произнёс сказал он,\mdash Вчера на~кафедре диссертацию докторскую защитили, впервые за~N~лет! Да что вы в этом понимаете$\ldots$\mdash подумав о чём\sdash то о~своём, как\sdash то грустно добавил Б\sdash ов, махнув рукой.

Аудитория ликовала, подстроившись под его волну.

\diagdash Ладно, это лирика всё! Итак, кто скажет что тут надо исправить?\mdash продолжал он сыпать формулами.

\diagdash Издевается, ё-моё!\mdash подумал я,\mdash Знали бы что добавить и исправить, не ходили бы сюда!

Этот спецпредмет был одним из самых сложных, на лабораторках творился буквально страшный суд. И~тут со мной произошла неприятность\mdash я загремел в~инфекционную больничку почти на месяц или даже больше. И пропустил несколько лаб по этому предмету. Это~было катастрофой, серьёзно. Одно дело выполнять весь этот лабораторный кошмар вместе со своей бригадой и группой, и~совершенно другое\mdash в конце семестра в~одиночку$\ldots$ но~надо, значит надо. Я~подал себя в списки на переделку лаб, честно готовился. Надо было сделать, кажется, 3~лабы за~одну пару\mdash нереально. Провести все измерения, выполнить расчёты, построить графики, написать выводы и~т.д. как обычно. Естественно, я~был подготовлен\mdash у~меня был уже этот весь материал, я~представлял что и~как надо делать. Более того, у~меня уже были заготовки от~моей группы с~теми результатами, которые должны быть. Ну~а~что, иначе не~выжить. Б\sdash ов~появился в~лаборатории:

\diagdash Ну что, пришли, раздолбаи? 

\diagdash Пришли, да$\ldots$

\diagdash И чего хотите?\mdash вполоборота сев на стул, лениво полистал он материалы подготовки к лабам\mdash мои и~нескольких таких же несчастных, по тем или иным причинам тоже сдававших лабы в конце семестра.

\diagdash Да вот$\ldots$\mdash начал было кто\sdash то.

\diagdash Да знаю, знаю, что вы нифига не знаете! Так, давай, рассказывай, что тут должно и как быть, какие результаты, как будешь мерить.

Я начал отвечать, понимая, как откровенно не хочется Б\sdash ову нами заниматься, как ему это сейчас лениво.

\diagdash Достаточно,\mdash прервал он,\mdash рисуй такую\sdash то функцию, такое то сечение для такого\sdash то случая.

Рисунок покрыл бумагу, Б\sdash ов принял. Допрос длился долго, минут 30, не меньше. И это была только первая лаба, а там ещё две$\ldots$

\diagdash А почему пропустил\sdash то столько?\mdash спросил он меня.

\diagdash Да в инфекционке лежал.

\diagdash Ну, руки мыть надо, ёпрст, перед едой!\mdash в~своём неуклюжем грубом стиле, полагая, видимо, это верхом остроумия, попытался пошутить он, просматривая подготовленный мной материал на остальные две лабы,\mdash Всё, надоело, зачёт, иди отсюда,\mdash отрезал он и~расписался в~зачётке.

В голове у меня моментально возник закадровый голос Е.\thinspace Копеляна, объясняющий спасение радистки Кэт в~11\sdash й серии <<17 мгновений~весны>>, где Штирлиц вывозит её с младенцами в Швейцарию: <<Произошёл тот случай~$\ldots$~который бывает лишь раз в жизни$\ldots$>>



\begin{center}
	\psvectorian[scale=0.4]{88} % Красивый вензелёк :)
\end{center}