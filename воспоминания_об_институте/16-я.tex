\newpage
\section*{Шестнадцатая}
\addtocontents{toc}{\protect\setcounter{tocdepth}{0}}

На нашем потоке было 7 групп. Шестая носила обозначение <<16>> и имела специализацию на биомедицинские аппараты и системы. То была группа Геннадича. Помимо весьма необычной специализации для нашего факультета, особенностью данной группы было то, что она процентов на 70\thinspace\nobreakdash--\thinspace 80 состояла из девушек, в то время как в других группах потока было по одной\sdash двум девушкам максимум, а в каких\sdash то группах их не было и вовсе. Иначе говоря, 16\sdash я была просто малинником.

%\begin{center}
%	\psvectorian[scale=0.4]{88} % Красивый вензелёк :)
%\end{center}

\diagdash М-м-м, угоститься?

\diagdash Без проблем,\mdash протянул я Геннадичу кисет.\mdash Держи.

И он ловко скрутил очередную <<коровку>>. Я подкинул ещё дров в костёр, забрал свой кисет и потуже набил трубочку, прикурив от костра\mdash моветон прикуривать от спичек или зажигалки, когда рядом горит костёр.

\diagdash 16\sdash я была класс$\ldots$\mdash мечтательно стал вспоминать былое я$\ldots$

\diagdash Да чё класс, Сань, не в моём все были вкусе так то, а так всё хорошо.

\diagdash А вот N, например? 

\diagdash Она после техникума к нам пришла, она ведь старше нас всех года на три и потому всегда немного сторонилась$\ldots$\mdash Геннадич вдруг внезапно открыл мне неизвестную страницу былой истории, и это спустя почти $4,5$~года от выпуска$\ldots$

Мне вдруг явственно и с горечью вспомнился 1~курс, когда всем было 17\thinspace\nobreakdash--\thinspace 18 лет, некоторым и более, как выяснилось, а мне 16. Разница эта сотрётся полностью только к~выпуску, но не ранее, отчего существовала, особенно на 1 и 2 курсе незримая пропасть в гипотетических поползновениях в~сторону прекрасной половины <<шестнадцатой>>. Быть может, оно и хорошо\mdash больше дум и времени было обращено, собственно, к учёбе, а~не~к$\ldots$ но по прошествии лет, думаю$\ldots$ зря, зазря ребята, пролетело то время! Всё~равно от~большинства предметов ничего не осталось в голове, стоило ли так упахиваться? Риторический вопрос$\ldots$

На 3 курсе в конце весеннего семестра мы отмечали <<экватор>> учёбы. Под эту цель был снят домик загородом и заехали туда на выходные\mdash часть народа добиралась на~электричке, часть на своих машинах (было и такое). Был,~кажется,~апрель. Снег ещё лежал в полях, и тут мы. Пиво отгружали ящиками$\ldots$ М. был ответственен за шашлык, заранее замаринованный ещё в пятницу. Не весь поток, естественно, был на этом <<экваторе>>, а наиболее дружные части 11\sdash й, 13\sdash й, 16\sdash й групп. Так сложилось, что именно таким составом мы часто отмечали сдачу экзаменов.

Помню, когда мы приехали, <<малинник>> был сильно озадачен\mdash посуды в доме не оказалось никакой! А народ хотел сварить картошечки, всё как полагается$\ldots$ Через полчаса Е. нашёл какую\sdash то чистую, что не характерно, алюминиевую кастрюлю! Победа, казалось бы, но днище было пробито, такое впечатление, что шлицевой отвёрткой. 16\sdash я стала паниковать, что кулинарная часть под угрозой, так сказать, срыва. Но тут в дело включился я и достаточно быстро предложил законопатить узкую пробоину$\ldots$ деревянными зубочистками. Разбухнув от воды, они бы ещё лучше устранили течь, справедливо положил~я. Так и сделали\mdash я и однокурсник, впоследствии получивший прозвище <<Вентилятор>>, проделали сию операцию и <<спасли>> мероприятие, сорвав немного женского внимания <<шестнадцатой>>.

Перепились, конечно, почти все на этом <<экваторе>>, но такова уж традиция. Наутро выжившие продолжили <<заряжаться>>, помню ещё даже остался целый не распакованный ящик пива$\ldots$ Ближе к середине дня я покинул сие мероприятие и в весьма абстрактном виде туманно добрался домой на перекладных. 

Впоследствии мы ещё соберемся в похожем формате позднее, уже после вручения дипломов, в ресторанчике около Измайловского парка, чтобы выйти из него в другую, совсем иную жизнь\mdash как обычно при подведении невидимой черты думалось: лучше ли она будет, счастливее ли? Тогда казалось, что да. А сейчас? Как обычно\mdash в чём\sdash то лучше, в чём\sdash то хуже, закон сохранения энергии никто не отменял. И~всё~же\mdash хорошо, что во время обучения была 16\sdash я$\ldots$

\begin{center}
	\psvectorian[scale=0.4]{88} % Красивый вензелёк :)
\end{center}