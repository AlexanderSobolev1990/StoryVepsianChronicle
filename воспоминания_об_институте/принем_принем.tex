\newpage
\section*{<<Приём, приём!>>}
\addtocontents{toc}{\protect\setcounter{tocdepth}{0}}

$\ldots$Наверно каждый помнит гайдаевское <<Наваждение>>, историю про Шурика и Лиду, когда в институтской аудитории амфитеатром шёл приём экзамена по~электротехнике и легендарное: <<Диктую ответ на~1\sdash й~вопрос>>$\ldots$ Это~снимали в аудитории моего института, на кафедре физики. Антураж там был что надо! Старые парты, шкафы с~лабораторным оборудованием, рулонная доска крутящаяся и поражающая воображение вчерашних школьников, и, собственно, сам амфитеатр. Мне довелось не~только сдавать вступительные экзамены в этой аудитории, но и позже слушать в ней лекции по физике. Историческое, можно сказать, место! Немного, конечно, огорчало то, что образ профессора из <<Наваждения>> никак не стыковался с~реальными преподавателями. А позже эту аудиторию отремонтировали, напрочь убив современной отделкой и~мебелью весь колорит, антураж, саму былую атмосферу <<винтажности>> этого места, увы!

Учёба в институте, в целом, была намного лучше школьной\mdash было больше свободы, больше возможностей, и~при этом меньше контроля при кратно возросшей сложности и ответственности. Мне удалось поступить в~институт на~2~года раньше по возрасту, поскольку школу я окончил экстерном, и~поначалу эта разница в~возрасте между мной и~сокурсниками, одногруппниками, очень чувствовалась и угнетала. Выровнялось это лишь к~середине периода обучения, наверно, но никак не~раньше.

Проходные условия на факультет были не особо высоки, и поначалу все группы на потоке были большими, человек по~30, раздолбаев хватало. Впрочем, всё как по~<<Наваждению>>. Но, поскольку учиться было непросто, ряды поредели очень и очень сильно\mdash к концу обучения в~моей группе осталась только 1/3 первоначального состава\mdash среди моих знакомых в других институтах такого отсева не было ни у кого, но, должен признать, по большей части отсев был справедливым.

$\ldots$Лабораторные работы\mdash это <<песня>>. На них, традиционно, группу разбивали на бригады по несколько человек, чтобы каждому хватило приборов и т.д. Моя~первая бригада по физике определила мою дальнейшую дружбу с~Сергеем Лунёвым и Кириллом Рябковым. И хотя Кирилл (Кир или Киря, как мы его называем в нашей компании) позже отчислился и~мы ненадолго потеряли друг друга из~виду, впоследствии мы стали вместе ходить в~сплавы$\ldots$ но~я~забегаю вперёд. 

Итак, лабы по физике. Было~сложновато, поскольку школьные лабораторки были просто <<детским садом>>, и~этот разрыв между школой и институтом чувствовался мной очень жёстко. Но~как\sdash то справлялись. Серёга сразу на~1~курсе устроился на~подработку и учиться ему стало тяжелее обычного, я~даже не представляю как он вытянул. Поэтому в~основном подготовка к лабам в нашей бригаде держалась на мне, увы. Серёга однажды даже сказал мне, что если бы не я, то, скорее всего, вылетел бы он с~1\thinspace\nobreakdash--\thinspace 2\sdash го курса. Впрочем, я склонен думать, что он преувеличивает$\ldots$ 

\vspace{1.0cm}
$\ldots$Одно из ярких воспоминаний\mdash лабы моей бригады по~физике у товарища Гв\sdash го. Это был конец 2\sdash го курса уже. Наши неокрепшие умы, мой так точно, не~сразу поняли, что запах, источаемый им, это не~изысканный парфюм, а~просто\sdash напросто жёсткий вчерашний перегар:

\diagdash Ребят, ну у вас в подготовке тут$\ldots$ это вот что такое тут понаписано?\mdash странно растягивает Гв\sdash й.

\diagdash Так, ну здесь$\ldots$\mdash начинаю объяснять было я, но~понимаю уже, что у дяди тяжелейшая похмелуха, и все мои слова не ложатся на его мироощущение в текущий исторический момент.

\diagdash Так, незачёт, пш\sdash ш\sdash шли отсюда!\mdash это для студентов нашего вуза звучало как приговор, поскольку недопуск к этим, часто совершенно идиотским по содержанию и~техническому оснащению, лабам означал то, что их надо будет сдавать в конце семестра, в зачётную неделю перед сессией, когда и так пар из ушей. Соответственно, преподавателям на зачётной неделе тоже не до переделок этих лаб, а~без них нет допуска до экзамена$\ldots$ Сочетание этих факторов с неопытностью зелёных студентов приводило чаще всего к одному\mdash накапливанию долгов и отчислению. Поэтому чему лабы нас, как студентов, и научили, так это кромешной лжи, изворотливости, хитрости, безбожному подгону результатов ради допуска и сдачи, увы.

Нашу бригаду Гв\sdash й тогда не допустил до лабы и~удалился в свою каморочку в лаборатории. 

\diagdash Вот же пьянь!

\diagdash Шурик, а чё делать?!\mdash Серёга был, казалось, как~и~я, раздавлен ситуацией.

Откладывать на~конец семестра это всё\mdash последнее дело. Я понял, что Гв\sdash й сейчас, возможно, похмелится и~подобреет?! Подождали минут~10, пока он не~вышел к~нам снова с~а\sdash а\sdash абсолютно стеклянными глазами, и со 2 раза получили допуск к~выполнению, показав ровно те же самые расчёты, которые, что вдвойне обидно, были сделаны правильно. И такая штука случалась часто. Постепенно весь задор, запал, с которым я пришёл в институт, попросту пропал из\sdash за таких вот моментов. Исчезла какая\sdash то надежда, что <<чем дальше, тем интереснее>>, начались просто рутинные серые будни$\ldots$

\vspace{1.0cm}

$\ldots$Вспоминается незаурядный Леб\sdash в с кафедры физики. Сухенький старичок с огромными линзами в очках. Он,~конечно, был физик знатный, формулы из него лились сплошным потоком и устилали всю доску. Но,~увы, будучи хорошим лектором, семинарские занятия он вести попросту не умел\mdash это быстро осознала вся наша группа. Он~распылялся, уходил в сторону от темы, начинал что\sdash то несуразное объяснять вместо того, чтобы при разборе задач сказать, что вот так и так\mdash <<хорошо>> потому\sdash то и~потому\sdash то, а~вот эдак\mdash нет. Помнится, один раз мы с~ним схлестнулись в словесном поединке на целый семинар в обсуждении моего расчёта напряженности поля в металлах и диэлектриках\mdash он сыпал формулами, я пытался отвечать, так и не понимая чего он, собственно, хочет от меня. Длилась такая битва минут 30 точно, никто не хотел отступать$\ldots$ Пришлось мне переделывать расчёты, добывая сведения в~книжках, справочниках и у старшекурсников. Леб\sdash в так и~не смог нормально объяснить что было не так, я сам понял позднее. Ещё один гвоздь в крышку, как говорится$\ldots$

\vspace{1.0cm}

$\ldots$Радиоматериалы нам читал профессор Ар\sdash ев. Народ на потоке ошалевал от беспросветной отсталости даваемого материала\mdash весь мир давно ушёл в нанометры, а тут$\ldots$ был полный финиш. Полный. Как эту муть сдавать, что называется, <<без слёз и боли>>\mdash не знал никто. Впрочем, как\sdash то сдали. 

Вспоминается случай на лабораторках по этому предмету. Был опыт по пробою диэлектрика, в роли которого выступала многослойная промасленная трансформаторная бумага. В лабораторной методичке особо подчёркивалось, что \textit{нельзя, чтобы в зоне пробоя между слоями бумаги оказался воздух}. Ибо тогда при электрическом пробое бумага может воспламениться. Стоит ли говорить, что соблазн был чересчур велик??? Естественно, моя бригада устроила небольшой пожарчик, который, впрочем, был быстро мною потушен ветошью из старых синих лабораторных халатов под~улюлюканье бригады.

Профессор Ар\sdash ев очень плохо выглядел, он был уже в довольно преклонном возрасте. Вспоминается один случай с его лекции. Была, кажется, весна, а, значит, это был 2\sdash й семестр курса третьего, наверно. Слушать его радиоматериальскую муть никому не хотелось, понятно. И~вдруг~он, как чувствуя это, спросил:

\diagdash Ребята, ладно, а вот такой вопрос\mdash что лить: спирт в воду или наоборот, воду в спирт?

Народ мгновенно оживился! Люди у нас, конечно, на~потоке были всякие, но разведением спирта водой мало кто занимался в принципе по причине доступности всего остального. Тем не менее, разгорелась жаркая дискуссия:

\diagdash Да не, ну понятно, что$\ldots$\mdash начал было кто\sdash то.

\diagdash Не\sdash е\sdash ет, надо наоборот, потому что$\ldots$\mdash парировал другой, засыпая оппонента доводами.

\diagdash Да вы ващ\sdash щ\sdash ще не в теме! Слушайте сюда опытного человека!\mdash жарко спорили студенты между собой.

Ар\sdash ев слушал, слушал доводы противоборствующих сторон, потом он отчаялся, видимо, втолковать нам верный вариант и изрёк:

\diagdash Ребята, пейте лучше сразу водку!%$\ldots$

Аудитория грянула молодецким хохотом$\ldots$

\vspace{1.0cm}

$\ldots$Начкурса была одиозной личностью. Орала на поток она исключительно хорошо\mdash децибелы зашкаливали. Однако мне, прошедшему <<школу>> печально знаменитой Р., это было не в новинку и особо не трогало, поскольку никогда не касалось адресно меня лично\mdash мы с ней встречались лишь в конце семестров при подписании допуска к экзаменам. И всё. Но не все были как я, и именно к~таким <<не таким>> были обращены её вопли и праведный гнев, который, по~сложившейся традиции, слушали все, весь поток\mdash для профилактики, так скажем. Сложно сказать, насколько это действовало на отдельных раздолбайского плана личностей, скорее всего, что никак\mdash отсев на потоке был большой и постепенно таких студиозусов практически не осталось. А начкурса осталась и продолжала свою деятельность. Поговаривали, что вообще\sdash то она горой стоит за <<нормальных>> студентов, однако воочию я лично такого не наблюдал:

\diagdash Хороших студентов я не знаю, в том смысле, что они редкие гости у меня,\mdash распиналась она перед потоком.\mdash потому что им просто незачем ходить ко мне, а вот \textit{некоторые} (тут следовал её красноречивый взгляд в сторону какого\sdash нибудь претендента на вымещение злости) у меня из кабинета ну просто не вылазят со своими проблемами!!!

\diagdash Я$\ldots$\mdash хотел было что\sdash то возразить студиозус, но~начкурса набирала воздуха и, не давая тому опомниться, выдавала очередную тираду:

\diagdash Молчать!!! Вчера пропустил лекцию N!!! Да ты знаешь$\ldots$ \mdash брань нарастала, как цунами, окрашиваясь всё новыми виртуозными эпитетами и определениями наших умственных способностей, а децибелы её речи стремительно увеличивались, резонируя в историческом амфитеатре.

Огромные полупрозрачные тёмные очки наводили суровости на лицо, а короткую стрижку с кудрявящимися волосами дополняла ма\sdash а\sdash аленькая пластмассовая заколочка, зачем\sdash то прикреплённая сбоку надо лбом. Она, эта заколочка, была совсем ни к чему\mdash она ничего не держала, просто была прикреплена, словно говоря: <<Я\mdash женщина, если кто не понял!!!>>

%\diagdash Я\mdash женщина, если кто не понял!!! 

Она бдила нас на первых семестрах очень жёстко, что, как мне казалось, было абсолютно лишним. Те, кто хочет, и~так получат своё, а кто не хочет\mdash отсеется. Направила ли она <<на путь истинный>> хоть одного раздолбая? Едва~ли. Помогла ли тому, кто нуждался? Хотелось бы верить. Однако я убеждён\mdash всё хорошее можно нести и без ора и~без~крика$\ldots$

Однажды я вышел на сессию <<на пределе>>\mdash в~последний день зачётной недели, в последний час, можно сказать. И бегом бежал в её кабинет проставить печать:

\diagdash Успел!\mdash как отрезав, бросила она, когда я, постучав, зашёл в её каморочку.\mdash У кого?

\diagdash Метрология$\ldots$ у Кар\sdash ко.

Сдать даже зачёт, не говоря уже про экзамен, у Кар\sdash ко считалось не очень просто\mdash дядя считал метрологию не~иначе как царицей наук, а уж на лабораторки по этому предмету попортили кровушки знатно.

\diagdash Молодец!\mdash будто бы подобрев на мгновение, ответила она и молниеносно проставила штамп о допуске в~зачётке$\ldots$

\begin{center}
	\psvectorian[scale=0.4]{88} % Красивый вензелёк :)
\end{center}