\newpage
\section*{<<Приём, приём!>>}
\addtocontents{toc}{\protect\setcounter{tocdepth}{0}}

$\ldots$Наверно каждый помнит гайдаевское <<Наваждение>>, историю про Шурика и Лиду, когда в институтской аудитории амфитеатром шёл приём экзамена по~электротехнике и легендарное: <<Диктую ответ на~1~вопрос>>$\ldots$ Это~снимали в аудитории моего института, на кафедре физики. Антураж был что надо! Старые парты, шкафы с~лабораторным оборудованием, рулонная доска крутящаяся, поражающая воображение вчерашних школьников, и, собственно, сам амфитеатр. Мне довелось не~только сдавать вступительные экзамены в этой аудитории, но и позже слушать в ней лекции по физике. Историческое, можно сказать, место! Немного, конечно, огорчало то, что образ профессора из <<Наваждения>> никак не стыковался с~реальными преподавателями. А позже эту аудиторию отремонтировали, напрочь убив современной отделкой и~мебелью весь колорит, антураж, саму былую атмосферу <<винтажности>> этого места, увы!

Учёба в институте, в целом, была намного лучше школьной\mdash было больше свободы, больше возможностей, и~при этом меньше контроля при кратно возросшей сложности и ответственности. Мне удалось поступить в~институт на~2~года раньше по возрасту, поскольку школу я окончил экстерном, и~поначалу эта разница в~возрасте между мной и~сокурсниками, одногруппниками, очень чувствовалась и угнетала. Выровнялось это лишь к~середине периода обучения, наверно, но никак не~раньше.

Проходные условия на факультет были не особо высоки, и поначалу все группы на потоке были большими, человек по~30, раздолбаев хватало. Впрочем, всё как по~<<Наваждению>>. Но, поскольку учиться было непросто, ряды поредели очень и очень сильно\mdash к концу обучения в~моей группе осталась только 1/3 первоначального состава\mdash среди моих знакомых в других институтах такого отсева не было ни у кого, но, должен признать, по большей части отсев был справедливым.

$\ldots$Лабораторные работы\mdash это <<песня>>. На них, традиционно, группу разбивали на бригады по несколько человек, чтобы каждому хватило приборов и т.д. Моя первая бригада по физике определила мою дальнейшую дружбу с~Сергеем Лунёвым и Кириллом Рябковым. И хотя Кирилл (Кир или Киря, как мы его называем в нашей компании) позже отчислился и~мы ненадолго потеряли друг друга из~виду, впоследствии мы стали вместе ходить в~сплавы$\ldots$ но~я~забегаю вперёд. Итак, лабы по физике. Было сложновато, поскольку школьные лабораторки было просто детским садом, и этот разрыв между школьной партой и институтом чувствовался мной очень жёстко. Но~как\sdash то справлялись. Серёга сразу на 1~курсе устроился на~подработку и учиться ему стало тяжелее обычного, я~даже не представляю как он вытянул. Поэтому в~основном подготовка к лабам держалась на мне, увы. Серёга однажды даже сказал мне, что если бы не я, то, скорее всего, вылетел бы он с 1\thinspace\nobreakdash--\thinspace 2 курса$\ldots$ впрочем, я склонен думать, что он преувеличивает. 

И вот, особенно яркое воспоминание\mdash лабы по~физике у товарища Гв\sdash го. Наши неокрепшие умы, мой так точно, не~сразу поняли, что запах, источаемый им, это не~изысканный парфюм, а просто\sdash напросто жёсткий вчерашний перегар:

\diagdash Ребят, ну у вас в подготовке тут$\ldots$ это вот что такое тут понаписано?\mdash странно растягивает Гв\sdash й.

Я начинаю объяснять было, но понимаю уже, что у дяди тяжелейшая похмелуха, и все мои слова не ложатся на его мироощущение в текущий исторический момент.

\diagdash Так, незачёт, пш\sdash ш\sdash шли отсюда!\mdash это для студентов нашего вуза звучало как приговор, поскольку недопуск к этим, часто совершенно идиотским по содержанию и~техническому оснащению, лабам означал то, что их надо будет сдавать в конце семестра, в зачётную неделю перед сессией, когда и так пар из ушей. Соответственно, преподавателям на зачётной неделе тоже не до переделок этих лаб, а~без них нет допуска до экзамена$\ldots$ Сочетание этих факторов с неопытностью зелёных студентов приводило чаще всего к одному\mdash накапливанию долгов и отчислению. Поэтому чему лабы нас, как студентов и научили, так это кромешной лжи, изворотливости, хитрости, безбожному подгону результатов ради допуска, увы.

Нашу бригаду Гв\sdash й тогда не допустил до лабы и~удалился в свою каморочку в лаборатории. Вот же пьянь, подумали мы. Но выживает сильнейший, откладывать на~конец семестра это всё\mdash последнее дело. Я понял, что он сейчас, возможно, похмелится и подобреет?! Подождали минут 10 пока он не вышел к нам снова с а\sdash а\sdash абсолютно стеклянными глазами, и со 2 раза мы получили допуск к~выполнению, показав ровно те же самые расчёты, которые, что вдвойне обидно, были выполнены правильно$\ldots$

$\ldots$Вспоминается одиозный Леб\sdash в с кафедры физики. Сухенький старичок с огромными линзами в очках. Он, конечно, был физик знатный, формулы из него лились сплошным потоком и устилали всю доску. Но, увы, будучи хорошим лектором, семинарские занятия он вести попросту не умел\mdash это быстро осознала вся наша группа. Он распылялся, уходил в сторону от темы, начинал что\sdash то несуразное объяснять вместо того, чтобы при разборе материала сказать, что вот так и так\mdash <<хорошо>> потому\sdash то и потому\sdash то, а вот так\mdash нет. Помнится, один раз мы с~ним схлестнулись в словесном поединке на целый семинар в обсуждении моего расчёта напряженности поля в металлах и диэлектриках\mdash он сыпал формулами, я пытался отвечать, так и не понимая чего он, собственно, хочет от меня. Длилась такая битва минут 30 точно, никто не хотел отступать$\ldots$ Пришлось мне переделывать, добывая сведения в книжках, справочниках и у старшекурсников. Леб\sdash в так и не смог нормально объяснить что было не так, я сам понял позднее$\ldots$


\begin{center}
	\psvectorian[scale=0.4]{88} % Красивый вензелёк :)
\end{center}