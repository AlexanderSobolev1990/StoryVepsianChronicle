\newpage
\section*{<<Приём, приём!>>}
\addtocontents{toc}{\protect\setcounter{tocdepth}{0}}

$\ldots$Наверно каждый помнит гайдаевское <<Наваждение>>, историю про Шурика и Лиду, когда в институтской аудитории амфитеатром шёл приём экзамена по~электротехнике и легендарное: <<Диктую ответ на~1~вопрос>>$\ldots$ Это~снимали в аудитории моего института, на кафедре физики. Антураж был что надо! Старые парты, шкафы с~лабораторным оборудованием, рулонная доска крутящаяся и, собственно, сам амфитеатр. Мне довелось не~только сдавать вступительные экзамены в этой аудитории, но и позже слушать в ней лекции по физике. Историческое, можно сказать, место! Немного, конечно, огорчало то, что образ профессора из <<Наваждения>> никак не стыковался с~реальными преподавателями. А позже эту аудиторию отремонтировали, напрочь убив современной отделкой и~мебелью весь колорит, антураж, саму былую атмосферу <<винтажности>> этого места, увы.

Учёба в институте, в целом, была намного лучше школьной\mdash было больше свободы, больше возможностей, и~при этом меньше контроля при кратно возросшей сложности и ответственности. Мне удалось поступить в~институт на~2~года раньше по возрасту, поскольку школу я окончил экстерном, и~поначалу эта разница в возрасте между мной и~сокурсниками, одногруппниками, очень чувствовалась и угнетала. Выровнялось это лишь к~середине периода обучения, наверно, но никак не~раньше.

Проходные условия на факультет были не особо высоки, и поначалу все группы на потоке были большими, человек по~30, раздолбаев хватало. Впрочем, всё как по~<<Наваждению>>. Но, поскольку учиться было непросто, ряды поредели очень сильно\mdash к концу обучения в~моей группе осталась только 1/3 первоначального состава. 

гвоздовский, лебедев.


\begin{center}
	\psvectorian[scale=0.4]{88} % Красивый вензелёк :)
\end{center}