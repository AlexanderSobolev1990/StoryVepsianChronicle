\newpage
{
\addtocontents{toc}{\vspace{1\baselineskip}}
\thispagestyle{empty}
\addcontentsline{toc}{chapter}{Эпилог}
\section*{Эпилог}

Весь поход 2019 года был пронизан какой\sdash то незримой тоской осознания, что такое путешествие, в прежнем формате, больше не повторится. Кроме того, жутко хотелось сменить работу. На последней нашей стоянке, у~двух островков в русле Чагодощи, я вырезал ножом на находившейся там беседке с крышей и скамейками надпись:

{\centering\Large{\fontspec{NorseRus}{%
	В посл.\\
	раз\\
	2О19\\
}}}

Кто знал, что это окажется до того пророческим изречением? Но ощущение такое\mdash было. Потому что, во\sdash первых хотелось сменить работу, хотелось чего\sdash то нового, а во\sdash вторых\mdash семья моя пополнилась новым маленьким человеком, требующим внимания, заботы о себе каждую минуту и, естественно, жена больше не могла подумать и мысли, чтобы пойти куда\sdash то в глушь вместе с~маленьким ребёнком.

Но это ещё, как говорится, полдела. Наступил 2020~год и вокруг стала твориться сюрреалистическая дичь потипу сталкера и сюжетов из кино про постапокалипсис\mdash пандемия. 2020\thinspace\nobreakdash--\thinspace2021~года пролетели как в каком\sdash то кошмарном сне из\sdash за мировой обстановки, постоянного страха за здоровье, новых забот с маленьким ребёнком и новой работы, не доставлявшей ничего, кроме неудовлетворения. Вдобавок, в 2020~году ещё были введены временные ограничения на перемещение между областями\mdash в~Вологодскую область нас бы не пропустили. В последующие года походы тоже не состоялись\mdash то болезни, то проблемы со сбором команды, то изложенные выше причины, то печальные события 2022 года$\ldots$ Много~раз~клял~я~себя за ту вырезанную ножом надпись!

Но, быть может, нет худа без добра\mdash эта <<передышка>> дала время подумать, переосмыслить прошлые походы и жизнь в целом. Кроме того, я всегда теперь знаю, что есть на Земле моё \textit{место силы}, где всё будет так, как я хочу, где вокруг будут такие сосновые леса, как я хочу, где всё будет привычно и сладко\sdash приятно моему сердцу, где меандрирующие в~песчаных берегах реки и чистейший воздух сосновых боров всегда очистит моё сознание ото всего лишнего.

5 лет, 5 разных жизней прожито на реках Лидь, Горюн, Чагода, Чагодоща, Песь. Каждая оставила в~душе такие схожие и в то же время такие разные приятные воспоминания. Эти года, несмотря ни на что, навсегда засели в моём сердце\mdash камерная Лидь с таинственными зарослями лесов в верховьях, узкий Горюн со своими <<шлюзами>>, полноводная Чагода и прекрасная каменистая Песь, а~также величественная красавица Чагодоща. Я всегда буду вспоминать о сплавах в этих краях с теплотой.

Много всего произошло за эти 5 лет\mdash мы с женой постепенно встали, что называется, на ноги: обзавелись недвижимостью и движимостью, мы немало попутешествовали по родной стране, у нас появился ребёнок. И как невозможно любое время однозначно отнести к плохому или хорошему, так и эти года\mdash тоже. Всякое бывало, но в нас всегда находился огонёк, разгорающийся из искры (как у~Бога Огня~\cite{Территория}) зароненной в сердцах тогда, на~Башкирской реке Зилим в нашем первом с~женой сплаве.

Эта летопись подошла к концу, а будет ли другая? Ведь~нас ждёт ещё Карелия$\ldots$

\vspace{5mm}
\begin{flushright}
Соболев А.А., 2023 г.
\end{flushright}

}
