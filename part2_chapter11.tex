\chapter{Днёвка около урочища Лукобор. 07.08.16} 
\corner{64}

Проснулся достаточно рано, но вылезать из палатки совершенно не хотелось, и вскоре я снова задремал. Соизволил вылезти, когда народ тоже уже проснулся и начал шуметь. Стали готовить завтрак\mdash Саня проявил рвение и взял инициативу в свои руки. В итоге у него получилась обалденная каша, в которую также добавили орехи, изюм, курагу и другие сухофрукты\mdash как мы обычно делаем. Пока Саня кашеварил, я достал флаг Советского Союза и укрепил его на свежевырубленном импровизированном флагштоке. <<Неба утреннего стяг\ldots >> и снова аккорды гимна на губной гармошке в исполнении Сани. 

Потом вспомнили, что у нас есть блинная мука и захотели сделать оладьи с яблоком. Приготовили тесто, как умели, и Дима начал готовить. Хотели сначала печь оладьи в крышках котелков. Но крышки оказались слишком тонкими и всё сгорало. Усовершенствовали конструкцию\mdash стали печь в двух крышках, поместив между ними песок для распределения тепла. Снова почему\sdash то не вышло. Блинчик снизу подгорал, а сверху по\sdash прежнему оставался абсолютно сырым. Дима достал свою газовую горелку и мы повторили всё то же самое, только теперь не на костре, а на горелке. Но результат вышел тот же\mdash сырой подгорелый блинчик. Тогда С.Ю. предложил просто запечь тесто в котелке. Мы поставили котелок на угли и вокруг тоже привалили углями. Пахло безумно вкусно. Все ходили вокруг котелка в нетерпении\ldots  но увы, всё тоже сгорело, превратившись в полусырую несъедобную массу. Короче, как мы поняли, блины в походе\mdash не мужских рук дело. А вот как теперь отчищать котелок? В нём напрочь пригорели остатки <<теста>>. Я сразу предложил этот котелок выкинуть к чёртовой матери. Но С.Ю. сказал, что свой котелок в беде не бросит и пошел на берег отчищать его. Поели, короче говоря, оладушков.

Но, поскольку мы уже до того зарядились кашей, отсутствие оладушков нас не сильно огорчило. Так что днёвка начиналась почти отлично. По синему\sdash синему небу плыли частые кучевые облачка, день обещал быть жарким. После завтрака мы с Саней решили сплавать из интереса на противоположный берег, потому что Паша сказал, что видел там\ldots  медведя. Оказалось, Паша рано утром встал и пошел на рыбалку, и тут ему улыбнулась удача\mdash несколько крупных рыбёшек стали добычей. Поймав пару окушков, Паша и увидел медведя на том берегу. Не очень приятное соседство на днёвке, однако. Надо разведать, что там такое и вот мы уже на пустой байдарке мчимся вдвоём с Саней на правый берег, на песчаную отмель чуть ниже по течению. 

Вылез из байдарки и сразу заметил следы, но не медвежьи, а собаки\mdash воображаемая линия проводилась между 1 и 4 когтём, пересекая подушечки лапы\mdash значит это собака, а не волк. Откуда тут собака? Ну да ладно, пошли с Саней дальше и\ldots  наткнулись на свежие следы медвежонка. Отпечаток его лапы был меньше моего, у меня 44\sdash ый размер ноги. Сразу стало как\sdash то не по себе. Сфотографировав свидетельства пребывания тут медведя, мы поспешили убраться с этого пляжа обратно в лагерь. 

Настроение у меня подпортилось. Мы встали на днёвку, да ещё место такое живописное, а тут медведь где\sdash то ходит неподалёку! Вот тебе раз! По прибытию в лагерь я сразу проверил на всякий случай на месте ли сигнальные ракеты. Глупо, правда, надеяться, что они как\sdash то помогут в случае нападения хищника. Хотя, с другой стороны, зачем ему нападать? Мы ему ничего не сделали\ldots  не считая того, что встали тут на днёвку. Короче говоря, мысли роем носились в адмиральской голове, сталкивались там и думали, что предпринять в данной ситуации\ldots  вплоть до снятия со стоянки. Но в итоге я решил оставаться, потому как медведь больше не показывался, погода была блеск, место\mdash супер, а чувство тревоги понемногу ушло. 

Паше и С.Ю. улыбнулась рыболовная удача\mdash удалось за день наловить 6\sdash 7 крупных рыбёшек. Паша решил их закоптить вечером на углях в специальном пакете. Я успел вымыться в реке, перестирал все вещи и развесил их сушиться. Потом зашёл по песчаной косе далеко выше по течению и обратно приплыл самосплавом, вылез и стал загорать. Красивейшее место мы выбрали! Дима сидел у воды на стульчике и читал книгу, а Ваня играл рядом с ним в песочке. Идиллия! Паша и С.Ю. продолжали рыбную ловлю, а Саня релаксировал в лагере.

День, тем временем, уже клонился к вечеру. Небо расчистилось, и оранжевые краски закатного солнца стали освещать песчаный берег. Я взял штатив, фотоаппарат, дистанционный пульт и пошел на фотоохоту\mdash закатный свет известен своей мягкостью и кадры, сделанные в это время, получаются особенно хорошими. Так и случилось\mdash кадры вышли замечательные. Но оранжевое солнышко вскоре зашло за сосны\mdash пора было идти в лагерь. С.Ю. и Ваня, тем временем, уже почистили картошку в реке. Ужин поспевал. Ваню обучали рубить дрова топором, но у него пока не очень хорошо это получалось. С.Ю. нашёл в лесу костянику. Добавили её в чай. Подоспела копчёная рыба от Паши\mdash объедение! Наш рацион, и без того разнообразный, обогатился таким шикарным лакомством. К копчёной рыбе у нас был даже лимон\mdash просто барство! 

Я сидел на скамеечке, обильно закусывал с ломившегося от яств стола и блаженно выпускал вверх колечки из трубочки. А вот у ребят кончилось курево\ldots  я им говорил до похода, что на природе расход в 2\sdash 3 раза больше, но это все всегда рассчитать не могут. Ребята стали стрелять мой рассыпной табачок и набивать его в кальку от конфет <<Коровка>>. Сии самокрутки мы так и назвали\mdash <<Коровка>>. 

Вскоре после перекуса копчёной рыбой, мы навернули сытнейший ужин и продолжили опустошать запасы 96\sdash ого, травя всякие байки и болтая о всевозможной чепухе. И тут\sdash то хозяин леса снова дал о себе знать\mdash на противоположном берегу послышался громкий хруст веток. Мы повскакивали от стола и вышли на берег. В зарослях определенно ходил кто\sdash то массивный\mdash ветки хрустели, кусты гнулись. Виновника не было видно. Честно, стало не по себе. Я стоял в оцепенении и перебирал в голове варианты, что бы предпринять. В итоге я взял два топора и стал соударять их, извлекая громкий и мелодичный протяжный металлический звук из своего кованого топора. Хруст углубился в заросли и удалился. Мы стояли и продолжали трапезу уже стоя\mdash не отвлекаться же от ужина, в самом деле? 

Но вскоре стало вообще не до смеха\mdash на том же бережку, куда мы с Саней плавали с утра, показался мишка. Видимо, испугавшись звона топора, он отошёл вглубь зарослей и вышел подальше от нас к воде. Не смотря даже в нашу сторону, он медленно подошёл к реке, постоял, помедлил, попил воды. В лагере нарастало напряжение. Я достал из штормовки два сигнала охотника и зарядил их. Стрелять или не стрелять вверх, вот в чем вопрос! Отпугнёт это или наоборот, привлечёт зверя? С медведем я сталкивался 1 раз в Башкирии, но тогда я лично его не видел, да и нас было 50 человек, так что момент истины я тогда как\sdash то не прочувствовал. А сейчас, стоя с двумя сигналами охотника, готовый к стрельбе <<по\sdash македонски>> с двух рук, я размышлял, что же всё\sdash таки делать. 

Мишка, вообще не обращая на нас никакого внимания, попил воды и стал переплывать реку на наш берег! Правда, не в нашу сторону, а по диагонали от нас\mdash ниже по течению. Но это всё равно\mdash на наш берег! Я протрезвел буквально за секунды! Ване, крутившемуся между нас, сказал не бегать и он замолчал. А потом расплакался, видимо тоже, наконец, ощутив страх. Ваня запричитал, чтобы мишка не кушал его, а кушал нас, потому что мы\mdash старые по его мнению. До\sdash о\sdash о\sdash брый мальчик! Ну, так\sdash то да, заросшие щетиной мы кажемся старше, чем есть на самом деле. Ваня, впрочем, быстро успокоился и ещё раз напомнил нам, что мы старые и нас кушать уже можно, а его нет, чем разрядил обстановку. Я ещё немного постучал топорами друг об друга, издавая громкий протяжный звук. Мишка же спокойно переплыл реку, вылез на берег и углубился в лес\ldots  вроде бы не в нашу сторону.

Тут Дима схватил топор и вырубил две огромные рогатины, справедливо заметив, что наши предки как\sdash то же охотились на медведя без огнестрельного оружия\ldots  Никто не поддержал его энтузиазма. Я решил, в случае чего, уходить по реке на пустой байдарке, а потом вернуться за вещами\mdash мишка не стал бы вечно околачиваться в районе лагеря. Короче говоря, перепугались мы не слабо. Естественно, здоровый сытый зверь боится человека. Но не пойдешь же искать его и спрашивать, мол\mdash эй ты, да\sdash да, ты! Ты сытый и здоровый или нет? 

Медведь ушёл, а тревога осталась. Кроме того, погода начала портиться и стал накрапывать дождик. С.Ю. сделал единственно верное дело\mdash унёс остатки сала и повесил пакет с ними на берёзу в стороне от лагеря, а остальные продукты сказал сложить около стола, а не в палатках. Потом, оглядев нашу порядком струхнувшую бригаду, он сокрушенно сказал, что мы зря боимся. Зверь боится нас гораздо сильнее, чем мы его и т.д. и т.п.\mdash что и так вертелось у меня в голове, но нервишки подсказывали другое\mdash жги костёр. В голове будто совещались здравый смысл и страх:

\ndash  Но ведь медведь боится нас сильнее\ldots  

\ndash  ЖГИ КОСТЁР! 

\ndash  А как же природный страх животного перед человеком\ldots  

\ndash  Ж\sdash Г\sdash И   К\sdash О\sdash С\sdash Т\sdash Ё\sdash Р!

Дима с Ваней пошли спать, С.Ю. тоже улёгся пораньше, а мы с Пашей и Саней остались жечь костёр и добивать конфеты <<Коровка>>. Впрочем, Паша скоро тоже ушёл, напившись чаем, остались мы с Саней. Под шум редкого дождика жгли костёр и болтали о разном, стараясь разогнать сон. В шуме ветра и дождя мне всё чудилась поступь хищника, блуждающего по лесу. 

Перевалило за полночь. Взбудораженное сознание рисовало медведя, сидящего в засаде и тихонечко ждущего, когда же эти негодные людишки наконец уснут. Стали болтать про группу Дятлова, инопланетян, бактериологическое оружие и прочие страшилки у костра. Но, чем дольше мы сидели, тем больше я понимал, что это всё, конечно же, зря и надо бы спать. Но мы почему\sdash то всё сидели и подкидывали дрова в костёр. Дождик помаленьку прекратился. К этому времени мы выпили с Саней неимоверное количество чая и порядком опустошили мой кисет. Болтали о всяком разном, вспоминали наш институт, как учились, потом перемывали кости преподавателям, вспоминали студенческие годы\ldots  однопоточников, однопоточниц\ldots  хорошее было время, хоть и пролетело как\sdash то стремительно быстро.

Когда пошёл четвертый час ночи, мы, не сговариваясь, подкинули в костёр три больших бревна, сделав подобие таёжного костра, и пошли спать. Уснул неспокойно, сжимая рукоять топора в руке\ldots 

\begin{center}
	\psvectorian[scale=0.4]{88} % Красивый вензелёк :)
\end{center}
