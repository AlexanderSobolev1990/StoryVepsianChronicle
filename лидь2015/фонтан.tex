\chapter{Фонтан. 06.08.15}
\corner{64}

Ночь была достаточно холодной. Проснулся, потому что снова замёрз. Что за холодные ночи? Только же начало августа! Идёт дождь\ldots~вылезать из палатки неохота, да и рано ещё, пять утра. Накрылся с головой штормовкой и завалился спать дальше в надежде, что дождь вскоре закончится. Так и вышло. 

Вылез из палатки, вытащил продукты, начал готовить завтрак. Сделал замечательное блюдо\mdash сухое молоко, сгущёнка, орехи, изюм\ldots~а,~ну и конечно овсянка, чуть не забыл. Каша вышла\mdash супер. Ложка стоит! Бригада моя начинает бесноваться по поводу ранней побудки. Ну ничего, скоро днёвка\mdash отоспимся вволю, ребята! 

Погода резко улучшилась. От туч и утреннего свинцового неба не осталось и следа\mdash припекает яркое солнышко. Разложили~бахилы химзащиты просушиться на горячий песочек, а то холодными и мокрыми их нацеплять совсем нехорошо. Настроение снова боевое\mdash погода радует, заплатки на байдарке я проверил\mdash заклеились отлично, не подтекают. Вперёд\mdash Лидь ждёт! Достал свои карты и прикидки по километражу. На карте, когда «поднимал» её, через каждый километр русла реки я заранее проставил точки\mdash так можно быстро с хорошей точностью прикинуть расстояние без линейки, курвиметра и тому подобных ухищрений. Получалось, что сегодня мы должны были пройти остаток Лиди и встать на стоянку между устьем нашей коварной красавицы и городом Чагодой.

Только с утра оценил масштаб препятствия, неподалеку от которого вчера заночевали\mdash  разрушенный мост близ Тургоши. Завал капитальный, нужна разведка. В этот день встали на воду в 10:38. Нормально, учитывая опять кошмарное количество вещей к погрузке. 

Борясь с течением, аккуратно переправились к противоположному берегу для разведки завала. Причалили к низкому, заросшему густым кустарником песчаному берегу\mdash надо осмотреться, поскольку штурмовать сходу я не решился, помня вчерашние приключения. После разведки начали преодолевать остатки деревянного моста. Держась за брёвна, я в бахилах аккуратно прошёл и встал в середине завала на отмели. Лёня,~стоя на песчаном бережку, стал аккуратно сплавлять нашу ласточку на чалке, ориентируясь на меня, а Аня подруливала веслом, сидя в байдарке. Неплохая вышла командная координация. Пройдя~брёвна, мы снова залезли в байдарку и пошли дальше. Надо пройти остаток Лиди и впасть в Чагоду. Лопатить нам будь здоров\mdash по плану я наметил километров 35. Начали~очень бодро\mdash горланим на всю мощь молодецких глоток песни, громко переговариваемся, о чём\sdash то шутим, смеёмся.  

Напрасно я надеялся, что после того кошмара на перекатах, что был до этого, подобное больше не повторится. Снова влетаем на шкуродер, чтоб его! Мысленно опять перебираю все отчеты, что видел по Лиди\mdash ну ни одного упоминания перекатов и шивер ниже Заборья\ldots~воды в этом году очень мало. Как бы там ни было\mdash сначала получаем фильтрующие пробоины и идём дальше, терпимо. Воды под ногами не много. Одну\sdash две дырочки заработали где\sdash то, думается мне. 

Но вот, на очередной шивере, мы крупно зацепляем днищем и у моих ног, в районе перекладины кильсона, начинает бить самый настоящий\ldots~фонтан. Паника у экипажа! А перекат казался вполне безобидным! Затыкаю пробоину пальцем, бригада моя гребёт к берегу\ldots~вылезаем. Оцениваю, водя рукой по днищу, характер пробоины. Подло пробились. Нельзя в байдарке вставать на перекладины кильсона, на эти «лесенки». Это грубое нарушение. Они, эти лесенки, прогибаются вниз и предательской дугой начинают тереться об шкуру, которая, в свою очередь, прогибается вверх под давлением воды\ldots~выгнутая вверх шкура и прогнутая вниз перекладина кильсона начинают соприкасаться, тереться даже друг об друга, что повышает риск заработать в этом месте пробоину. Итог\mdash минимум 4 пробоины под кильсоном\ldots~и ещё одна дырочка от аниного сиденья\mdash шкура так сильно прогибается внутрь, что трётся об нижний брусок фанерного байдарочного сиденья. 

Место, где пристали к берегу\mdash никакое. Берега высокие, густая трава, сплошные заросли. Прикидываю\mdash выгружать байдарку, опять заклеиваться\mdash это процедура на 3\sdash 4 часа минимум. Так мы точно от намеченного графика прохождения маршрута отстанем безвозвратно, а я ещё полон уверенности, что мы обязательно пройдём маршрут до Лентьево\ldots~принимаю решение отрезать от своего туристического коврика небольшой кусок и, свернув его в несколько раз, подпихнуть враспор между кильсоном и шкурой, заткнув ту пробоину, из которой бьёт фонтан. 

Проделав эту операцию с куском коврика, вычерпав кружечкой воду и убедившись, что фильтрация терпимая, отчаливаем и идём дальше, не заклеиваясь. Отдаю Ане свою кружечку и она периодически, раз в 10\sdash 15 минут, отчерпывает натёкшую воду. Что бы я делал без этой кружечки? Ничего,~дотянем так до вечера, а завтра устроим днёвку и нормально, не торопясь, заклеимся. Потому что уж в этот день точно будет не до ремонта\mdash задача наверстать километраж\mdash мы~бодро налегаем на весла, выдыхаемся\ldots~хочется найти хорошее место для завтрашней днёвки.

Забавно, когда проходишь перекаты и шиверы\mdash почти не гребёшь, замираешь, даже не дышишь, можно сказать. Только если требуется интенсивный маневр, выдаешь инструкции экипажу и начинается гребля\ldots~вспоминается кино «Das Boot» про немецких подводников времен Второй Мировой\mdash как те уходили от эсминца и, затаившись, слушали, как тот ставит мины. Погружение~на пределе возможностей подлодки\mdash появляется фильтрация через заклёпки корпуса (вот откуда я стащил этот термин). Мины~начинают рваться и тут\mdash бах!\mdash заклёпки вылетают, начинается течь! Схожие ощущения были у меня на перекатах, когда мы пробились\ldots~

Из примечательного, кроме пробоин, в этот день\mdash прошли мост через Лидь железнодорожной ветки Чагода\nobreakdash--Подборовье. Хотел причалить, поскольку из отчётов знал, что там должна быть табличка с надписью «р.~Лидь»\mdash я хотел её сфотографировать. Но~берега оказались у моста какими\sdash то никакими\mdash трава высоченная и ни намёка на место, где можно нормально причалить. Не~захотелось вылезать, пошли дальше. Примерно в полвторого пересекли границу Ленинградской и Вологодской областей\mdash теперь наш путь лежит по вологодской земле. Берега Лиди становятся крутыми и высокими\mdash красота, да и только! Деревья нависают кронами над водой, чуть не смыкаясь сверху. Засматриваясь на эту красоту, не забываем вовремя отчерпать воду со дна байдарки кружечкой.

После примерно 4 часов вечера замечаем, что характер берегов меняется\mdash крутые откосы пропадают, берега становятся низкими, с обильным кустарником, начинается редколесье. По~этим признакам догадываюсь, что скоро устье Лиди. Сверяюсь~с~навигатором\mdash так и есть. Через полчаса, пройдя, к счастью без повреждений, в устье лёгкий, но протяженный перекат с полуметровыми стоячими валами, оказываемся в реке Чагоде. Так~заканчивается участок маршрута по Лиди, который принёс нам немало адреналина, пробоин и заплаток, а также незабываемую атмосферу камерности реки, ширина которой редко была больше 10\sdash 15 метров. Жаль, конечно, пробитую шкуру, горько за погнутые стрингеры. Но что делать\mdash такова плата за экстрим, полученный нами сполна на этой небольшой речушке! 

Далее наш путь лежит по реке Чагода. Сразу около устья Лиди на левом берегу стоит база отдыха и имеется небольшой хороший пляжик. Причаливаем отдохнуть. У меня затекли ноги\mdash немного неудобно сидеть\ldots~на днёвке придумаю как усаживаться покомфортнее. С базы выходит смотритель, мы болтаем минут 15. Снова пора в путь! Преодолеваем протяжённый перекат\mdash трясёт вполне ощутимо. Следующая точка по GPS\mdash остров на Чагоде, где, судя по отчетам, можно остановиться. Проходим тот остров около 17 часов, а он никакой. То ли из\sdash за того, что воды мало, то ли просто мы слишком требовательные, но высокие заросшие берега с неудобным спуском к воде нас не прельщают. Вдобавок, на островке растет лишь трава и ивняк. То есть дров нет, а мы хотим устроить днёвку. Поэтому проходим дальше в поисках Лучшего Места и в 18:38 мы находим приличный берег\mdash есть выход к воде, полянка освещается вечерним солнцем, красивые янтарные сосны вокруг\mdash то, что надо! Единственно, разведка показывает наличие дороги. Ничего, наверняка тут всего пара машин в сутки, подумалось мне. Смущает то, что ровной площадки под палатки нет. Аня предлагает сначала подкрепиться\mdash мы здорово выдохлись за сегодняшний переход в попытке наверстать километраж. Координаты места\mdash N~59\degree~9.428$^\prime$~ E~35\degree~14.166$^\prime$.

Перекусив и выпив шиповника из термосов, решаем встать чуть подальше в тридцати метрах\mdash там нам всем больше приглянулось. Сказано\sdash сделано и мы начинаем разгружать байдарку. Через час лагерь развёрнут, костёр разведён, Аня суетится с ужином, Лёня добывает дрова. Показываю Лёне как обдирать бересту с берёз и удивляюсь как человек может ни разу в жизни этим не заниматься. Неподалеку ребята находят остатки домика\mdash то ли избушки охотника, то ли бани. Там оказывается стиральная доска деревянная, какой\sdash то ворот и наличники с вырезанными сердечками, которые Лёня без тени сомнения пускает в огонь. Мне удаётся лишь спасти старинную стиральную доску\mdash зачем сжигать такое?! 

Много гребли в этот день. Достаю карту, прикидываю\mdash прошли около 32 километров. Очень неплохой результат. Судя~по~карте, мы вплотную подобрались к поселку Чагода. Так что, возможно, даже хорошо, что раскладка по километражу немного сдвинулась\mdash на следующем переходе будет дозаправка провизией. С Аней днём примерно прикинули чего надо докупить, поскольку в Череповце при заброске изначально решили брать провизию с тем расчетом, что сможем дозаправиться в Чагоде.

От брызг с вёсел у всех штаны мокрые насквозь. Забиваю~около костра третий кол, навязываю между ним и двумя костровыми колами верёвку\mdash будем сушить одежду. Бригада уже готовит ужин, а я замечаю неподалеку от палаток поляну лисичек. Вечером сделаем жарёху из лука, картошки и лисичек! Но уже, видимо, не сегодня\mdash устали как не знаю кто! Чтобы возобновить силы, иду мыться к реке и купаться нагишом, пока поспевает ужин. Расчищаю себе путь по дну от камней, вхожу в воду\mdash течение приличное\mdash красота! Потом стираю вещи, развешиваю их сушиться у костра, немного прихожу в себя после такого большого перехода\mdash шутка ли, 32 километра! Ещё учесть то, что пробились в очередной раз, да частенько останавливались\mdash результат не то что хороший, а просто отличный, но сил не осталось вообще. Так что днёвка будет как нельзя кстати\mdash отдохнём, выспимся, заклеим шкуру и надувные сиденья\mdash их тоже как\sdash то прохудили, к сожалению, не представляю как, правда. 

Под берегом где\sdash то в траве непрерывно кто\sdash то плещется и шелестит\mdash наверно выдра. Потому что рыба плещется по\sdash другому. Засыпать из\sdash за этого было немного некомфортно\mdash постоянные всплески, невольно заставляющие насторожиться. Вдобавок, на этой стоянке много муравьев и они по праву считают себя тут хозяевами\mdash лютуют страшно\mdash кусаются и заползают куда только можно. Хорошо, что вечером они достаточно рано идут спать!

\begin{center}
	\psvectorian[scale=0.4]{88} % Красивый вензелёк :)
\end{center}