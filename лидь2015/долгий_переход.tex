%\chapter{Долгий переход. 09.08.15}
\chapter{Долгий переход}
\corner{64}

Проспали!!! Дико проспали! Я поднялся около половины 11\sdash го наверно. Выспался, конечно, просто замечательно! Отдохнул~шикарно$\ldots$~но всё\sdash таки мы проспали! Теперь~не~пройти маршрут до конца в срок, подумалось мне. Ладно, не будем паниковать раньше времени. Бригада~моя тоже просыпается, быстро готовим завтрак, а после начинаем паковаться. Собираемся~почему\sdash то снова чертовски медленно. От~стоянки отчаливаем, когда уже за полдень. Новый антирекорд$\ldots$

Исходная цель на сегодня\mdash песчаные косы перед устьем Внины. Хотя понимаю, раз вышли так поздно\mdash добраться туда нам сегодня не светит. Совсем скоро проходим деревню Залозно. Судя по отчётам, после неё начинается <<золотой возраст>> реки\mdash один бережок обрывистый, второй песчаный покатый. Так и выходит. Чем дальше плывём, тем сильнее это, так называемое меандрирование, начинает проявляться. Красиво, очень красиво! 

Переход запомнился большим количеством островов в~русле и~обилием мелей. Приходилось резко перекладывать курс от поворота к повороту, чтобы не оказываться на~мели. Но порой ветер создавал рябь на~воде, и было трудно прочитать воду\mdash где мель, а~где нет. Так, несколько раз мы садились днищем на грунт, благо он песчаный и без камней. Вылезали, проводили байдарку на~глубину и~снова садились. Интересен у Чагодощи переход с мели на~глубину\mdash очень резкий. Вот~идёт достаточно протяжённая мель и~тут р\sdash р\sdash раз!\mdash глубина такая, что~веслом не достать! Визуально~это хорошо видно, если~нет ряби\mdash вода резко становится тёмной.

На обед встали на низком левом песчаном берегу в~районе сомнительного ориентира\mdash большого дуба на~правом берегу, о чём было указано в~каком\sdash то из~отчётов с~маршрута. Тот~ли это был дуб, доподлинно неизвестно, но вид был хорош. Перекусили бутербродами и чаем. На~одной из мелей нашли следы кабанов и кабанят$\ldots$~зрелище, не~сулящее ничего хорошего\mdash судя по следам, их было не~менее~4\thinspace--\thinspace 5. 

Отчаливаем, идём дальше. Характер растительности меняется\mdash становится больше лиственных деревьев, чаще встречаются дубы, даже клёны! От~хвойного леса не~осталось и следа, зато буйство кустарников и лиственных пород. После~высоких сосен и низкой травы такие берега мне совсем не по вкусу. И хвойного леса на горизонте не~видать\mdash придётся долго искать приемлемую стоянку, думается мне$\ldots$~и~я оказываюсь прав\mdash вплоть до самого вечера мне не~приглянулось ни~одно место для стоянки.

Следующая путевая точка, судя по GPS, это охотничий кордон Линино. Правда, он находится в отдалении от берега, да и неизвестно, есть ли там вообще что\sdash нибудь\mdash карта\sdash то старая. Около путевой точки кордона проходим остров с~левой стороны и садимся на мель, приходится стаскиваться. В правой протоке замечаем рыбака на надувной лодке, но~подплывать желания нет, потому~что река уже под 70\thinspace\nobreakdash--\thinspace 80 метров в ширину и мы идём у противоположного берега. Идём дальше. Скорее всего, он с кордона и был, этот рыбак. Рыба, стало быть, есть\mdash ловить надо уметь, а~рыбаки из нас не очень, мягко скажем. Ну~и~ладно, белорусская тушёнка уплетается просто на ура\mdash лучшей тушёнки в~жизни не~ел,~уверяю.

Вечереет, хочется найти нормальную стоянку. Плывём, поём по очереди песни. Правда очередь состоит в основном из меня и Ани. Переп\'{е}л, кажется, всё, что знал\mdash разное народно\sdash популярно\sdash туристически\sdash прикостровое\mdash короче всё, что~помню наизусть. Усталость наваливается неожиданно. Ищу стоянку так, чтобы её освещало закатное солнце$\ldots$~и всё тщетно, лопатим вёслами дальше. То низкий берег, то травы по грудь. И вот, выдаётся прямой участок реки. Я удачно замечаю остатки костровых палок на левом берегу и даже небольшой запасец дров. Отлично! Подъём, правда, крутой, но встречали и круче, ничего страшного.

Уже успели немного поругаться из\sdash за выбора места, пока совершали манёвр разворота от противоположного берега. Причаливаем, взбираюсь на обрыв$\ldots$~красота! Поляна~хоть и~травянистая, но~терпимо. А более всего подкупает огромная одинокая берёза\mdash просто красотища! Экипаж мой стоит, осматривается. Крутой подъём им явно не по вкусу. С Лёней идём в разведку по протоптанной тропинке вдоль берега. Она приводит к другой полянке\mdash гораздо меньшей и неудобной. Откуда тут тропинка? 
\newpage
Решаю вставать там, где причалили. Координаты\mdash \CoordsChagodoschaFifteenKaban. Дров на стоянке у~костровища чуть\sdash чуть есть, плюс с собой запасец везём небольшой\mdash я~как чувствовал, что пилить на дрова тут особо будет и~нечего. Вокруг поляны растут дубы, кустарник, встречаются молодые берёзы, сухостоя нет. Экипаж~потихоньку начинает перетаскивать~вещи. 

Ещё только поднявшись на берег, я заметил, что земля под ногами как\sdash то странно взрыта. Подумал, это около костровища только. Потом смотрю\mdash нет, ещё одно место такое же есть. Ну,~иногда под палатку выравнивают грунт\mdash я сам делал так на Белой. А походил вокруг по поляне\mdash взрыто всё! Прикинул\mdash место глухое, деревень тут никаких по карте нет и близко. Урочище Витимец, правда, где\sdash то в километре примерно на север. Стало быть, не следы это кладоискателей с металлодетекторами, а зверей, скорее всего кабанов. Вокруг как раз много дубов\mdash жёлуди. Вот~тебе~раз! Холодок пробегает по спине. Встречаться~с~кабанами совсем не хочется. Тем более ещё и с кабанятами\mdash страху добавляют следы, встреченные сегодня на песчаной отмели. Стараюсь отогнать нехорошие мысли и успокаиваю себя тем, что у меня есть сигнал охотника, а бабахает он громко\mdash достаточно, чтобы отпугнуть нормального зверя. Если зверь нормальный\mdash уйдёт, а от бешеного и ружьё не поможет в~неумелых~руках$\ldots$ 

Грозы ничто не предвещает, поэтому палатки ставим прямо под большой берёзой. Затаскиваем с Лёней байдарку на берег, разворачиваем лагерь. Аня обещает что\sdash то вкусное на ужин. Мы верим. А пока таскаем немного дровишек, что валяются в округе, занимаемся своими заботами. Вечер выдаётся непривычно холодным. Аня соображает нам тушёнку с рисом и кукурузой, видимо отчаявшись, что~Лёня на эту кукурузу наловит нам хариусов. Получается замечательное блюдо. Уминаем по два\sdash три десантных подкотельника этого плова с хлебом и, напившись чаем, валяемся у костра на ковриках. Вокруг поляны этой растёт огромное количество шиповника$\ldots$~Аня и так с собой везёт мешочек сушёного шиповника и заваривает его на манер чая, а тут свежак! Естественно, N\sdash ое количество красных соцветий немедленно собирается в отдельный пакетик впрок.

Холодно! Наверно, самый холодный вечер за наш сплав. У~костра сидеть и то прохладно, хотя мы распалили прилично! Лицу~жарко, а спина стынет$\ldots$~ поворачиваешься, прогреваешь спину, голова и лицо замерзают. Ну, как\sdash то так и вертимся весь вечер. Можно было бы поснимать астропейзажи, но облачно, да и уже что\sdash то тянет ко сну. С~согласия ребят на ночь ложусь в их палатке\mdash она более тёплая, а~к полуночи совсем ледянющий холод начался. По привычке ложусь как в своей\mdash треники, тельняшка, флисовая курточка, шапка$\ldots$~ в итоге мне становится жарко, но раздеваться не тянет\mdash к утру\sdash то боюсь замерзнуть. Засыпается~плохо, сон не идёт. Мысль о кабанах не даёт покоя, сигнал охотника лежит рядом. 

\begin{center}
	\psvectorian[scale=0.4]{88} % Красивый вензелёк :)
\end{center}