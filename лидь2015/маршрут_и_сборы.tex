\chapter{Маршрут и сборы}
\corner{64}

Для начала немного сухого формализма\mdash сплав наш проходил на границе Ленинградской и Вологодской областей по рекам Лидь, Ч\'{а}года, Чагод\'{о}ща бассейна Верхней Волги. Даты\mdash с 3 по 13 августа 2015 года. Стапель 4 числа под~железнодорожным мостом через Лидь у~станции З\'{а}борье железной дороги Вологда\thinspace---\thinspace Санкт\sdash Петербург. Заброска~от~Москвы поездом с пересадкой в Череповце. 7 ходовых дней, 2 днёвки. Вынужденный антистапель 12~августа около деревни Загр\'{и}вье (рядом более крупная деревня Дубровка) ввиду невыполнения запланированного маршрута на 50 километров до штатного антистапеля в~селе Л\'{е}нтьево Устюженского района Вологодской области.  Байдарка\mdash <<Таймень\sdash 3>> с ПВХ шкурой. Экипаж\mdash Аня, Лёня (её молодой человек) и я.

Теперь подробнее обо всём в деталях. Почему был выбран такой маршрут? Главный Идеолог мероприятия\mdash мой начальник\mdash искал небольшую камерную речку в~районе от Москвы до Санкт\sdash Петербурга с относительно удобными вариантами стапеля и антистапеля. Относительно чего? Относительно наших пониманий об удобстве. Понятно, что можно отслюнявить цветных фантиков и тебя на тракторе подбросят до реки, но зачем, когда до неё от станции 300~метров? Кроме того, он ранее уже ходил в этом районе в~сплавы на~реки Песь, Коб\'{о}жа, Мол\'{о}га и~проникся какой\sdash то особенной теплотой к~этому региону из\sdash за его заброшенности и~первозданности.

Не скрою, был изначальный план попасть на~реку Чагодощу. Эта идея была поддержана и развита. Рассуждали далее\mdash как попасть туда? С реки Песь? Она скорее всего мелеет к августу и, судя по отчетам в~интернете, на ней имеются многолетние завалы, и, кроме того, с постепенным исчезновением движения поездов по~Савёловской железнодорожной ветке, попасть на станцию Хвойная, близ которой протекает эта самая Песь, становится всё сложнее. Все найденные нами варианты заброски на~Песь требовали пересадки на автобус, скажем, в~Великом Новгороде или того хуже\mdash заброски поездом через Санкт-Петербург. Все эти факторы привели к~выбору другого притока Чагодощи\mdash реки Чагоды, в~которую, в~свою очередь, впадает Лидь. Река Лидь и стала нашим отправным пунктом, а удобный стапель в трёхстах метрах от железнодорожной станции Заборье подкупил нас окончательно. 

Судя по найденным отчетам, Лидь обещала быть красивой и именно камерной речкой, чего так жаждали наши~души. Таким образом, маршрут был определён\mdash начать с~реки Лидь у~станции Заборье, далее попасть в~Чагоду, потом Чагодощу, дозаправиться продуктами в~городе Чагода, что примерно на 1/3 маршрута, и далее до Лентьево с~экстренным сходом с~маршрута в районе Дубровки (эх, знать бы заранее, что именно так и придётся делать$\ldots$).
 
Чагода и Чагодоща, напротив, уже не такие камерные реки, как Лидь, но зато более полноводные и~достаточно широкие\mdash около посёлка Чагода река достигает примерно 50 метров в ширину, а далее местами и все 100. Из отчётов и~спутниковых снимков стал понятен примерный облик рек\mdash узкая в верховьях Лидь, в низовьях течёт в высоких берегах с густым лесом, грунты в основном песчаные, берега имеют песчаные обрывы, порой достаточно большой высоты\mdash около 5\thinspace--\thinspace 6 метров. По~Чагоде нам необходимо было пройти лишь небольшой участок\mdash характер реки тут обычный, не~слишком примечательный\mdash полноводная река с~частыми островками ближе к устью. Чагодоща же имеет свой характер\mdash расширяясь примерно с середины маршрута, берега её приобретают такой характер\mdash один обрывистый с сосновым лесом, а второй покатый с песчаным пляжиком и кустарниковой растительностью. Далее~за~поворотом берега меняются местами, и теперь высокий песчаный обрыв оказывается с другой стороны, а отмель располагается напротив\mdash так называемое меандрирование реки на языке гидрологов. На конец маршрута по Чагодоще предполагалось прохождение трёх порогов\mdash Горынь, Буг и Вяльская Гряда. В отчетах было сказано, что это скорее лёгкие шиверы и каких либо опасений не вызывают. 

Что ж, про реки всё стало более менее понятно, однако тень сомнения закралась в душу\mdash мало, очень мало отчетов про Лидь и фотографий оттуда! Неужели так мало людей ходит на неё? Неужели все променяли свои <<Таймени>> на цветные фантики и продались капиталистам на всякие турции и египты? Почти все найденные отчеты относились к~весенним сплавам по высокой воде, а чего ждать от верховьев Лиди в августе? Может там впору пешком по руслу идти, а мы на байдарке хотим. Хоть~похожий маршрут и имеется в литературе под номером 57 \cite{Рыжавский}, но~с~81\sdash го года с окончательным упадком Тихвинской водной системы всё могло очень сильно поменяться. Много,~много вопросов оставалось без ответа, но, как говорится, кто не рискует$\ldots$~Маршрут был утверждён окончательным и~бесповоротным решением.

Как только маршрут окончательно сложился в~умах страждущих приключений, началась подготовка картографического материала. Были раздобыты старые топокарты, распланирован примерный километраж по дням\mdash в какой день сколько необходимо проходить, подобраны примерные места стоянок и тому подобные вещи. В GPS\sdash навигатор были загружены путевые точки, составлена раскладка, куплены железнодорожные билеты на всю команду$\ldots$~в один конец. На каком\sdash то подсознательном уровне мы решили подстраховаться и~не~брать обратные билеты заранее, поскольку не были уверены не~только в правильности расчета километража по~дням, поскольку я делал это впервые в жизни, но~и~в~том, что~на маршруте всё будет гладко. В голове крутились строчки песни: <<Билет в один конец, весна\sdash а\sdash а>>$\ldots$
 
Потом во время пробного сплава выходного дня по~реке Киржач был утоплен старенький навигатор и~у~него отказал экран. Пришлось одолжить на работе новый, более навороченный, но вполне выполняющий свою функцию навигатор. Впредь будем более чутко относиться к нежной буржуйской технике. Оказалось, что новый навигатор показывал географические координаты, чертил на экране пройденный трек, вёл статистику пройденного пути, но,~несмотря на свою навороченность, не имел карты. Вообще никакой! Для того, чтобы иметь возможность привязаться к местности, я распечатал бумажные карты с координатной сеткой UTM (Universal Transverse Mercator), которая показалась мне удобнее формата градусы/минуты/секунды, а в навигаторе, соответственно, включил отображение координат в~формате~UTM. Вопрос~с~навигацией и привязкой к местности был~решён.

В полной мере осознал всю мудрость решения приобрести хозяйственную тележку, как только попытался поднять рюкзак с <<Таймень\sdash 3>>, двуручной пилой, складным стульчиком, бахилами химзащиты, надувными сиденьями и~двумя флягами 96\sdash го. Вроде~бы по\sdash отдельности немного веса, а вместе\mdash ого\sdash го! А~ещё палатка, гермомешок со спальником и личными вещами. Не отличаясь силой тяжелоатлета, нести такой рюкзак далее пары метров проблематично. Так что тележка очень и очень пригодилась. Хоть~её~трубки и треснули немного, свою задачу она выполнила до конца.

Печальную тоску навевали вещи, сложенные на диване, и детали байдарки, разбросанные по комнате\mdash КАК это всё впихнуть в рюкзак и гермомешок?! Собравшись с мыслями и повыкинув всего не то чтобы ненужного, но без чего могу обойтись, проведя многократную оптимизацию как состава вещей, так и способа, а также очерёдности их укладки, добился нужного результата\mdash всё помещается. Итого вышло: огромный рюкзак с байдаркой, герма на 60 литров с личными вещами, палатка, маленькая герма для судового журнала, документов и фотоаппарата. Ура, сборы окончены, можно идти!
\newpage
Оставалось несколько дней томительного ожидания дня отъезда, отпуска и отпускных соответственно. Всё это время я чуть ли не ежечасно созерцал топографический материал и пересматривал снаряжение на предмет поиска чего бы ещё выкинуть. Однако, в конце концов, я пришёл к выводу, что дальше выкидывать что\sdash либо, сокращая объём и массу снаряжения, просто невозможно и на этом успокоился.

\begin{center}
	\psvectorian[scale=0.4]{88} % Красивый вензелёк :)
\end{center}