\chapter{Кабаны и удача. 10.08.15}
\corner{64}

Просыпаюсь рано. Солнышко встало, нагрело палатку. Палатка у ребят не только тёплая, но~и~очень душная\mdash весь~в~поту. Вентиляционные окошки не~помогают. О сне уже речи нет, ужасная духотища. Вылезаю, вытаскиваю коврик, спальник. Досплю на улице. Снаружи\sdash то благодать! 

Лежу~себе\mdash хорошо, полусонное состояние. Вяло рассматриваю топографическую карту, прикидываю наши перспективы дойти до конца намеченного маршрута\ldots~тают они с каждым днём\mdash не~укладываемся мы в запланированный километраж. Много~времени потеряно на Лиди, проспали в один день, в~какой\sdash то раз просто долго собирались\ldots~ладно, думаю, сойти досрочно всегда успеем в Дубровке, посмотрим как сегодня день~выдастся. Рядом солнечная батарея лежит\mdash внешний аккумулятор заряжается. Включил~телефон, cвязи нет. А~у~ребят~вчера~ловилась\ldots 
\newpage
Лежу, солнышко припекает, лёгкий ветерок колышет листву берёзы\ldots~идиллия! Пора бы завтрак начать готовить. И~тут внезапно\ldots~хруст веток! Упала ветка где\sdash то, думаю. Но~тут ещё хруст, потом ещё и ещё! Настораживаюсь. Раздаётся ХРЮКАНЬЕ?! Ё\sdash моё! Кабаны~пришли!!! Сердце~уходит в пятки, ёжиков полные штаны! В мгновенье ока оказываюсь в палатке ребят, кричу Лёне, чтобы подал мою сумку от изголовья, вытаскиваю сигнал охотника. 

Дрожащими руками ввинчиваю патрон, а он никак, зараза, не вкручивается! Почему я с вечера не ввинтил?! Наконец~совладал с патроном, взвёл боёк, выставил руку из палатки и пальнул вверх красной ракетой\ldots~сидим в~палатке, слушаем, что будет дальше. Судорожно ввинчиваю новый патрон, снова стреляю. Ввинчиваю~ещё один, беру топор\ldots~всерьёз готовлюсь к худшему\mdash <<Это есть наш последний и решительный бой!>> 

Но вроде всё стихло. Осторожно вылезаю из палатки, осматриваюсь. Ничего не слышно подозрительного\ldots~Бригада моя, похоже, отделалась лёгким испугом. Вероятно, их~больше напугал я, что так беспардонно разбудил, чем мысль о возможной встрече с кабанами.

С опаской провели утро и готовили завтрак. Сигнал~охотника с ввинченной ракетой и~топор всё время были под рукой на всякий случай. Решили побыстрее убраться с этой стоянки, что и сделали в половину двенадцатого. Поздно опять встали на воду, до Лентьево, расчётной точки антистапеля, точно не добраться в срок, понял я в тот момент. Только встав на воду, выйдя на~середину реки и~почувствовав себя в~безопасности, снимаю сигнал охотника с~боевого взвода и~вывинчиваю патрон. Путешествие~наше~продолжается\ldots~
\newpage 
Полоса лиственных лесов и зарослей вскоре заканчивается, снова начинается любимый нами сосновый лес, ура! Наверно, в этот день нам встречались самые красивые берега. Просто~упоительно красивые обрывы и здоровенные песчаные отмели\mdash расцвет <<золотого возраста>> реки. Душа поёт от такой красоты! Река~становится заметно шире, мели по\sdash прежнему не редкость. Как~и~вчера, перекладываем курс от поворота к повороту, чтобы не~тереться дном байдарки об песок. 

Поднимается сильный встречный ветер, становится реально тяжело идти. Еле~догребаем до обеда\mdash выдыхаемся полностью. Причаливаем к низкому левому берегу на огроменный песчаный пляж\mdash место прекрасное. Противоположный берег, как и положено, высокий, красивый,~с~соснами. Вылезаю из байдарки, буквально падаю на песок от~усталости. Ветер дул такой, что если не грести, плывёшь назад против течения. Пришлось нам втроём, и Ане тоже, активно бороться с таким ветрилой. 

Расположившись на песчаной отмели, перекусываем консервами <<Печень трески>>. Мне мало. Силы просто куда\sdash то улетучились\mdash всё встречный ветер! Вскрываю банку тушёнки, помня, что у нас есть запас, и в одиночку уминаю её с хлебом. Лёня следует моему примеру и уничтожает ещё одну банку. Аня ругается, что хлеба мало купили в~Чагоде и хвалится, что выклянчила у скряги\sdash Адмирала на одну буханку больше. Я молчу\mdash она права. Решаем немного тут передохнуть после перекуса. Встречный ветер реально выматывает. Прошёлся по берегу\mdash впереди виден дом. По~памяти вспомнил карту и сообразил, что это начало деревни Званец. За полчаса примерно до этого привала прошли слева устье Внины\mdash неплохое место для стоянки на высоком левом бережку. 

Перекусив и отдохнув, встаём на воду. Ветер~стихает, что не может не радовать. Проходим деревню Званец, где~на~низком левом песчаном берегу находится огромный пляж и полно народа. Немного общаемся, замедляя ход. Дети~с интересом наблюдают за нашей байдаркой. Сразу~после пляжа встречаем двух мужиков в~гидрокостюмах на деревянной лодке под вёслами. Поначалу хотим обойти их, но решаемся потрепаться и~подплываем ближе. 

Спрашиваю у них что с рыбой в этом году и~говорю, что~мы ничего не наловили. Удивляются. У них улов приличный по нашим меркам. Вы~бы, говорят, на~спиннинг ловили или~сетью. Они\sdash то~сетью ловят. Ну,~думаю, нормально так! Мы подплываем, стыкуемся бортами, и~мужики по доброте душевной отдают нам небольшую щучку, окуней, подлещиков, причем совершенно даром. Вот~это~удача! Принимаю на руки это богатство и~сваливаю прямо на дно байдарки себе под ноги. Спрашивают сигарет. Извиняйте, не держим. Рассказываю им про~наши приключения\mdash пробоины на перекатах. <<Ну~и~ладно, за~разговор спасибо>>,\mdash говорят мужики, и мы расцепляемся бортами. На том и расходимся\mdash путь наш лежит дальше.

Проходим хорошо укреплённый огромными валунами и просто камнями берег. Сейчас уже не помню название населённого пункта, но кажется это было в Клавдино. В~любом случае, проходим мимо. Я~кусаю локти, что~мы гружёные прилично. Так~бы можно было камней набрать, баньку сделать\ldots~отгоняю эту мысль. Всё~равно столько камней не принять~на~борт.

Душу греет рыба под ногами! Щу\sdash у\sdash учка! Прелесть! Из~мелких рыбёшек сварим уху однозначно, а крупных зажарим на огне. Красота! Разнообразие! Не всё ж тушёнку жевать, хоть и качественную белорусскую. Остаток пути проходим в этот день, как мне кажется, на одном дыхании\mdash жадно ищу глазами стоянку\mdash хочется пораньше встать, раз маршрут сокращается. 

Примечаю на высоком, опять\sdash таки, левом бережку песчаный подъём\mdash там наверняка должна быть хорошая полянка. А мы идём под правым берегом. К этому времени я уже научил бригаду причаливать против течения и~мы, совершив интенсивный маневр разворота на 180\degree, причаливаем по всей науке. Координаты\mdash {N~59.18169\degree~E~36.11501\degree}. 

Вылезли аккуратно\mdash вход в воду глубокий. На берегу много битого стекла\mdash я~поранил палец на ноге, но терпимо. Взбираюсь наверх. Поляна классная! Есть~стол, три скамейки, костровище. Правда, немного замусорено. Аня~начинает возникать по~поводу мусора. С~Лёней проводим разведку по~берегу. Обнаруживаем протоку, там куча топляка, причём есть сухой. Лес сосновый, много сухих нижних веток. Дровами, считай, обеспечены. Стол~со~скамейками меня подкупает окончательно\mdash решаю вставать тут и мы разгружаемся. Продолжается капанье на мозг относительно мусора. Ещё даже не~перетаскав полностью вещи и не разбив лагерь, выкапываю сапёрной лопатой приличную яму. Аня, вооружившись перчатками, собирает мусор и битое стекло\mdash закапываем. Становится чище, ясное дело. Остатки мусора сжигаем в тут же разведённом костре. Бригада успокаивается. Ставим~палатки, затаскиваем на~берег байдарку. Словом, всё как обычно. 

Надо чистить рыбку. Аня отчищает подлещиков, окуни ждут своей очереди. Достав свой ножик, присоединяюсь к~Ане, заняв место у~столика. Чищу окуней и~принимаюсь за~щуку: два~надреза\mdash под хвостом и~у головы\mdash и~внутренности выходят на раз\sdash два! Ушица будет знатная, из~трёх сортов рыб. Покончив с~рыбой, отдаю её~Ане на~готовку, а~сам иду отмываться в реке от~чешуи. У~нас будет уха\sdash а\sdash а\sdash а! А~пока нарезаем последний хлебушек и~достаём~96\sdash й. Вскоре~уха уже почти сварена, палатки поставлены, все~в~ожидании~ужина. 

Готовлюсь удивить свою бригаду, как когда\sdash то удивил меня этим крёстный, ритуалом, можно сказать, последним штрихом в приготовлении ухи\mdash в неё добавляется немного водки и окунается горящее полено из костра. Вообще это делается, чтобы осветлить уху, если она варится из~непотрошёной рыбы и для некоторой толики дезинфекции. Кроме того, водка отбивает запах тины, который может появиться от некоторых видов рыб. Ту же функцию выполняет и полено, добавляя ещё, плюс ко всему, аромат костерка. Ну и просто красивый обряд, хе\sdash хе! Наконец садимся вкушать приготовленные яства\mdash получилось отменно, а жареная рыбка и вовсе бесподобна. В походе вообще любая еда бесподобна, а уж уха\ldots~

На этом месте, где встали, решаю завтра задневать. Есть~связь, мне прозванивается жена, сеанс связи с~<<большой землёй>>. Сообщаю, что планы наши немного корректируются\mdash сойдём на один день раньше с маршрута. Как я и предполагал ранее, до Лентьево в срок нам не дойти, потому что много времени потеряно на перекатах, на ремонт байдарки и в поздних выходах на воду. Впереди по плану еще около 50\sdash 60 километров\mdash это два ходовых дня на пределе возможностей\mdash по 25\sdash 30 километров в день. Ранее мы делали и 32 километра в день в этом сплаве, но выдыхались полностью и окончательно. А на антистапеле ещё разбирать байдарку, мыть шкуру, паковать снаряжение\ldots~
\newpage
То есть, если укладываться в первоначальный план\mdash надо отменять завтрашнюю днёвку, оба дня рано вставать, лопатить вёслами как лосям\ldots~что\sdash то как\sdash то это не входит в моё понимание об отдыхе. Посему, завтра никуда плыть не будем. Покупаемся, помоемся, отдохнём как следует\mdash место мы выбрали отличное. Сходить с маршрута будем в~Дубровке, до неё от нашей стоянки километров 10\sdash 12\mdash нормально. Как раз, не напрягаясь, в последний день пройдём этот остаток. Итог\mdash вместо того, чтобы оставшихся два дня лосячить вёслами до Лентьево, устраиваем в~один день днёвку, а~во~второй\mdash небольшой переход и~досрочный~антистапель. 

Не успев доесть ужин, оказываюсь в одиночестве\mdash бригада немного расстроена планами о досрочном сходе с~маршрута, хотя разговоры об этом уже велись, и~залегает в свою палатку. Ничего,~раньше\mdash значит раньше. Адмиральское~Решение принято, и поход\mdash не место для~демократии. 

Посидел у костра под звёздным небом, усугубил 96\sdash ым\mdash его осталось ещё огромное количество. Забил~трубочку потуже, красота! Звёзды отражаются в зеркальной глади~воды, вдалеке видна мачта связи в~деревне~Слудно, метров 100~высотой, не меньше. Наверху~у~неё~горят красные сигнальные огоньки. Закат~был очень красив, а~ночь ещё краше. 

На этой стоянке немного лютовали комары, но сейчас всё стихло. Тишь~да~гладь. Только воды Чагодощи степенно проплывают мимо меня, устремляясь дальше и~дальше\mdash к~Мологе. Понимаю,~что~поход заканчивается и лёгкая тоска зарождается где\sdash то в груди. В~голове проплывают яркие пережитые моменты нашего сплава, особенно на Лиди. 
\newpage 
Эх,~вот~бы пойти дальше по~Мологе, да~до~Весьегонска\ldots~а~может и вовсе до Волги\ldots~ чтобы об антистапеле вообще не думать\ldots~мечты, мечты\ldots

\begin{center}
	\psvectorian[scale=0.4]{88} % Красивый вензелёк :)
\end{center}