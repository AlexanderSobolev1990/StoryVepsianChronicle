\chapter{Досрочный антистапель. 12.08.15}
\corner{64}

Пройти сегодня нам осталось немного, порядка 12~километров. Проснулись не слишком рано, приготовили завтрак, не торопясь свернули лагерь, погрузили вещи, отчалили. Действия по сворачиванию лагеря и распихиванию вещей в байдарке уже стали слаженными, привычными и~автоматическими\ldots~даже жаль, что сплав заканчивается. Преодолели расстояние до окрестностей Дубровки часа за~три. Один раз надолго останавливались\mdash Аня заприметила на берегу какую\sdash то траву, которую непременно захотела нарвать с собой. 

Вскоре приходим на точку досрочного антистапеля. В~этом месте карты упорно показывают паромную переправу, а её и в помине нет. Да, подъезд к реке есть, колья от парома есть\ldots~даже какой\sdash то остов валяется полусгнивший в траве. Но~не~более. Тросов,~как в Мегрино, уже нет. Огромный~песчаный~пляж. Отсюда~до~перекрестка в~Дубровке около двух километров. Хорошо~бы мужик на <<газели>> не~запротивился проехать к берегу. Иду~в~разведку\mdash дорога представляет собой сухой зыбучий песок. Мало того, что если собираться и паковаться тут, то этот песок будет во всех вещах, но ещё и <<газель>> сюда может не~пройти. Или не выйти потом отсюда. Координаты места\mdash {N~59\degree~11.192$^\prime$~ E~36\degree~13.353$^\prime$}.

Принимаю решение поискать в пешем порядке лучшее место ниже по течению. Оставляю ребят отдыхать, а сам иду бодрым шагом по дороге. Не обманываюсь в правильности решения\mdash по пути встречаю ещё более разбитые участки дороги, чем у реки. Боязно, что если <<газель>> тут застрянет, придётся разгружаться, толкать. Этого нам совсем не нужно. Нахожу развилку и дорогу по лугу к реке. Тут трава, песка нет. То, что надо! Выхожу на берег\mdash красота! Небольшой песчаный подъёмчик, дальше луг и ровное место. Отлично. Антистапель будет тут\mdash {N~59\degree~11.551$^\prime$~ E~36\degree~13.951$^\prime$}. 

Связь~мобильная есть, я звоню ребятам и говорю, что~остаток маршрута\mdash примерно один километр\mdash проходите без~меня, самостоятельно. Потому что мне лениво тащиться обратно километр. Лёню оставляю и.о. Капитана. Там где я~стою\mdash островок посреди реки. Говорю~бригаде идти в~левой протоке и с~разворотом на~180\degree, как я учил, против течения причаливать к берегу. К концу сплава бригада манёвр этот усвоила, я надеюсь, основательно. 

Пока жду их, кусаю локти\mdash впереди такой красивый берег левый\mdash высоченный песчаный обрыв с сосновым бором наверху! (фото этого обрыва украшает обложку) Я~вернусь, обязательно вернусь сюда! Пройду~дальше\mdash до Лентьево. И~пороги возьму\mdash все три\mdash Горынь, Буг, да~Вяльскую Гряду! Но~не сегодня\ldots~не сегодня\ldots 
\newpage
Замечаю за поворотом бригаду, машу руками. Видят, машут в ответ. Течение в левой протоке сильное, опасаюсь за ребят. Но ничего, должны справиться. Дует~сильный попутный ветер. Бригада заходит в~протоку, чуть передыхает, начинает манёвр разворота и, встав против течения, заходит, ориентируясь на меня, к~месту антистапеля. Почти идеально\mdash  молодцы! Подтягиваю на~конечном этапе байдарку к берегу за~чалку, высаживаются. Всё. Сплав окончен. Да здравствует Новый~Сплав! 

А сейчас\mdash антистапель. К этому времени я~уже немного отдохнул, мы быстренько разгружаемся. Потом~выносим с Лёней байдарку на берег, начинаем разборку. С разборкой традиционная проблема\mdash после длительного плавания клинит замки стреляющих стрингеров\ldots~но это я учёл заранее\mdash у меня есть спецсредство из верёвочной петли, которая надевается на запястье и накидывается свободным концом на замок. Далее медленно, нежно расшатываю при помощи этой петли замки и открываю их. Остальная разборка идёт без проблем. Потом промываем байдарочные кости в реке и~сматываем их изоляцией. Вынимаю сломанный шпангоут, держу в~руках погнутые стрингеры\ldots~горько на душе, сердце кровью обливается. Починю потом, не беда. 

Пока мы с Лёней занимаемся байдаркой, Аня упаковывает вещи, а потом успевает помыть голову в~реке. Когда, наконец, вещи упакованы, моем с Лёней байдарочную шкуру и сушим её, разложив на траве. Пока~шкура сушится, мы перекусываем и немного отдыхаем. После, запаковываю шкуру и кости в огромный рюкзак, что~получается у~меня, как всегда, не с первой попытки, и~креплю его резиновыми жгутами к транспортной тележке. Ласточка~наша, сине\sdash жёлто\sdash серая, пробитая 15 раз, на~заслуженном отдыхе. Пронесла она нас 150 километров пути по просторам рек Лидь, Чагода, Чагодоща. 

Жарко, хоть и дует сильный ветер. Весь взмок пока паковались. Моюсь в~реке, одеваюсь цивильно\mdash в припасённую единственную чистую футболочку. Накидываю штормовку и вперёд, к~перекрёстку в Дубровке\mdash встречать <<газель>>. Тем~временем раздаётся звонок, что машина на месте, а~я~ещё на берегу. Ладно, что ж такого\mdash бегом марш\mdash забег полтора километра. С~непривычки тяжело, но~терпимо. На~дальние дистанции я~всегда нормально бегал\mdash главное повторять какую\sdash нибудь простую однообразную ерунду в голове, гася естественное желание перейти на~шаг. 

Оказавшись на перекрёстке, вижу, что машины нет. Созваниваюсь с водителем. А он, оказывается, в самой Дубровке, а не на шоссе. Через пару минут вижу синюю <<газельку>>\mdash это он. Сажусь и мы доезжаем до берега без вопросов. Моментально грузимся, делаем небольшой перерывчик. Я успеваю спуститься к реке и умыться ещё раз твоей чистейшей водой, Чагодоща. Я вернусь сюда обязательно, обещаю тебе!

Стартуем! Впереди примерно 150 километров до~Череповца. Время\mdash около 17 часов. Поезд на~Москву почти в 23 часа. Запас~огромный. Едем 60\sdash 70 км/ч\mdash больше не~позволяет дорога\mdash бетонка, покрытая местами асфальтом. Разбитая\ldots~жуть. Эта,~с~позволения сказать, трасса соединяет Бабаево, которое мы проезжали при~заброске, с~Лентьево, где должен был быть расчетный антистапель. И тут нас обгоняет другой микроавтобус на такой скорости, будто мы стоим на~месте. Вот это особенности местного вождения! 
\newpage
Добираемся до Лентьево. Перекрёсток и пост ДПС. Дорога~резко становится не просто приличной, а~хорошей. Замечаю тот самый автовокзал, до начальника которого дозвонилась Аня. Видно магазин, автобусы, такси, недостроенную церковь, автозаправку. Быстро пролетаем Лентьево и уносимся в сторону Череповца. Водитель Пётр интересуется, откуда мы знаем начальника автовокзала. Рассказываю историю с поиском машины\ldots~Пётр, в свою очередь, рассказывает, что часто возит что\sdash нибудь для этого самого начальника\mdash продукты, стройматериалы. И вот тот позвонил и попросил забрать туристов. Спрашиваю,~а~откуда Пётр сам? Отвечает, что из Устюжны. Прикидываю по карте, где это\ldots~да\sdash а\sdash а\sdash а\sdash а. Отдам\sdash ка я тебе все 4.5 в Череповце как ты и называл\mdash это нормально. Тебе ж мало того, что до~нас из Устюжны было 60 километров, потом до~Череповца около~150, обратно домой столько же. Да, так будет~правильно. 

Перекидываясь небольшими рассказиками о жизни, о сплавах, о рыбалке, подбираемся к Череповцу. Пейзажи~меняются\mdash от лесов не осталось и следа, идёт какая\sdash то лесотундра. Потом снова начинаются леса, далее снова лесотундра. 

Въезжаем в промзону Череповца. Стоит совершенно скверный запах\ldots~как тут живут люди? Размах, с которым строили когда\sdash то эту самую череповецкую промзону, поражает. Видны заброшенные здания, где едва теплится жизнь, а кое\sdash где раскручено на всю катушку\mdash дымят трубы, пахнет отвратительно! Пётр усмехается\mdash сейчас нормально! В 70\sdash ые не видно было дороги от дыма\ldots~а~сейчас и производство сильно упало, да и~фильтры современные. Такие масштабы экологического бедствия, прямо скажем, угнетают. И~едем по этому химическому ужасу достаточно долго\ldots~площади~колоссальные.

Вскоре добираемся до железнодорожного вокзала, где уже были при заброске, и выгружаемся. У нас из провизии осталась пачка сахара, макароны, что\sdash то ещё. Всё это презентовали нашему водителю, хотя он и отнекивался. Я~расплатился, мы попрощались, пожали руки, и он уехал. 

Перетащили вещи на вокзал в зал ожидания, расположились. Хе\sdash хе, теперь надо было распечатать посадочные талоны, то есть, по сути, материализовать электронные билеты в терминале. Пошли с Аней. Оказалось, это просто\mdash вводишь номер купленного электронного билета, автомат распечатывает бумажный. Всё прошло гладко\mdash автомат издал странные звуки и внизу, в окошечке, на манер автоматов с газировкой, появился бумажный билет. Распечатав, вернулись к Лёне в~зал ожидания. 

Ребята пошли купили чай и пирожки. Перекусили. Позвонили родным. Заскучал. Решил пройтись по вокзалу, купил магнитиков на всех\mdash родителям, тёще, себе конечно. Потом вспомнил, что уже покупал магнитики при заброске, но было поздно. Взяли с Лёней пенного в кафе. <<А можно ли пить тут?>>,\mdash вопрошаю у продавщицы. Тут у кафе, оказывается, можно. А в соседнем зале ожидания\mdash штраф 1.5 тысячи. Театр абсурда\mdash чем два зала отличаются друг от друга?! Безобразное лицемерие. Ладно, сидим тут. По~большей части молча. На стене висит картина, ещё видно советского художника\mdash <<Строительство череповецкого химического комбината>>\ldots~добавляет колорита. 

Возвращаемся к Ане. Ждать остаётся час\sdash полтора. Подзаряжаю телефон от внешнего аккумулятора. Это~замечает мужик по соседству и тоже просит подзарядиться. Я не отказываю\mdash ёмкости хватит. Вскоре~подают наш поезд. Кажется, у нас снова последний вагон. Идём к хвосту. Билет на байдарку брать не стали\mdash купили полноценное четвёртое место, чтобы без проблем разместить этот огромный рюкзак на полке, памятуя какая была с ним морока при заброске. Проводница, как и предыдущие, округляет слегка глаза при виде байдарки, но, увидев полноценный билет на неё, лишних вопросов не~возникает.
 
Грузимся в числе последних, протаскивая огромный рюкзак через весь плацкартный вагон и кладем его на~нижнюю боковую полку. Народ пошучивает: <<Там~труп?>> Или: <<Хороший был товарищ?>>. Как низко. Байдарка это полноценный член экипажа и едет на своей полке. Размещаемся, распихиваем вещи. Сами~будем спать на~трёх верхних полках. К счастью, соседняя боковушка пустует, и Лёня с Аней временно оккупируют её. Берём~чаю у проводницы,~перекусываем. Немного соскучился по~разговорам, донимаю соседей историями о сплаве. Укладываемся потихоньку спать. Ноги традиционно торчат в проход\ldots~любимый нами плацкарт.

Переписываюсь с женой, пока есть связь. Смотрю~фотографии, перекачанные из фотоаппарата в~телефон\ldots~уже немного грустно и уже немного хочется назад, на реку. Так и должно быть. Жаль, что не~прошли маршрут до конца. Но\ldots~должно же оставаться чувство некой незавершённости, иначе если всё сделать, завершить, пройти\mdash то что потом делать? 

Созвонился с братом заранее, договорился, что он встретит и поможет с~байдаркой. В~одиночку тягать её не сахар. Решаю ехать с байдаркой сразу в деревню. От~Ярославского вокзала до~Каланчёвской\mdash пешком, там до~Курского вокзала, и уже от~Курского\mdash электричкой до~деревни.~Неплохо. 

Погода в Череповце испортилась к нашему отъезду, а в Москве жара, судя по погодным сводкам. Плацкарт утихает к половине двенадцатого, стихаем и мы на своих верхних полках. С мыслью, что скоро увижу жену и родных, проваливаюсь в сон. 

Ехать нам 8 часов, поезд прибывает на Ярославский в 9:30, а там до дома уже рукой подать. Несколько раз просыпался ночью. Красочно встают перед глазами картины пережитого\mdash перекаты на Лиди, посёлок Чагода, семафор, огромные песчаные обрывы на~Чагодоще\ldots~вспоминаются пробоины, Капитанская Кружечка\ldots~как~вода заливалась в химзащиту\ldots~как~кабаны приходили к нам на~стоянку\ldots~нежданно полученная рыба\ldots~почти ежевечерняя стрельба из сигнала охотника\ldots~красота! Отдых удался на славу, это я могу сказать с полной уверенностью. 

Поезд, тем временем, стремительно отсчитывает километры, приближая нас к Москве, а сверху спускается темень и зажигаются звёзды. Туристы\sdash водники едут домой.     

\begin{center}
	\psvectorian[scale=0.4]{88} % Красивый вензелёк :)
\end{center}