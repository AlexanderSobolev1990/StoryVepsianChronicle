%\chapter{Возвращение. 13.08.15}
\chapter{Возвращение}
%\corner{64}
\vepsianrose

Проснулся рано и на удивление свежим. Странное явление для поезда. Плацкарт потихоньку начинал пробуждаться. Взяли снова чаю, немного перекусили. Поезд отмерял последние километры. Скоро Москва$\ldots$

Приехали точно по расписанию в 9:30. Брат порадовал пунктуальностью\mdash пришел вовремя, помог вытащить вещи из вагона. И вот, стоим мы на платформе. Ярославский вокзал. Тюки вещей. Утро. Москва. Жарко. Снимаю штормовку и подсовываю её под резинки, крепящие байдарку к тележке. Вышли мы последними снова, чтобы никому не~мешать. Побрели потихоньку к вокзалу от нашего последнего вагона. Ещё на стапеле заметил, что силовой узел тележки, на которой я везу байдарку, треснул. Как бы она не развалилась вообще. Тогда, даже вдвоём, тащить этот тюк будет крайне неудобно. Шагаем, тележка разваливаться, похоже, не собирается, и вот мы уже у входа в метро. 

Провожаем с братом мою бригаду\mdash им в подземку. Наблюдаем как два огромных рюкзака, из\sdash за которых не~видать ни широкой спины Лёни, ни, тем более, Ани, неспешно скрываются в подземном переходе. Потом и сами потихонечку выдвигаемся к Каланчёвской, где буквально на минуту опаздываем на электричку. Приходится ждать следующей полчаса, поскольку не хочу спускаться с таким тюком в метро. 

Берём мороженое в ларьке. В Москве стоит жара, не то что в Череповце! В последний день там ходили тучи к вечеру, и я уж думал начнёт накрапывать. Дожидаемся электричку. Выходить на следующей\mdash Курский вокзал. Два больших лестничных марша на~вокзале и мы около горьковских электричек. Времени около половины одиннадцатого, а~электричка моя в час дня только. Делать~нечего,~ждём. 

Хотел взять брату билет провожающего, чтоб он помог прямо в вагон занести байдарку. Пошёл в~кассы\mdash везде <<обед>>, работает лишь несколько окошек. Билет~провожающего можно взять только в~одном. Как~водится, в него очередь и каждый что\sdash то безумно медленно спрашивает, расспрашивает$\ldots$~одна касса и~очередь, классика. Ладно, купил брату обычный билет до~Серпа~и~Молота в автомате, а у меня же был~проездной. 

Стоим, болтаем, время тянется. Договорились, что он дождётся, когда подадут электричку, мы затащим в вагон байдарку, а потом он поедет по своим делам. Беру себе светлого и бутерброд. Где она, моя утренняя овсянка с~изюмом, курагой и~орешками\mdash такая, что ложка стоит? 

В~рассказах о сплаве и, конечно же, пробоинах, Капитанской Жарёхе из лисичек и прочих приключениях, время пролетает быстрее, подают электричку. Мы~затаскиваем байдарку и~герму в вагон, отправление через полчаса. Ещё немного болтаем, сидим, смеёмся. Электричка пока пустая. Я~прохожу по вагону и открываю все окна\mdash жарко! Брату~пора идти, остаёмся только мы~с~байдаркой. 

Дожидаюсь отправления\mdash в открытые окна начинает немного задувать\mdash не сильное спасение от жары. Томительные час сорок в духоте и я почти в деревне. Родные~радостно меня встречают! Как же я по всем соскучился, хотя прошло всего 10 дней. Я успел вернуться ко Дню Рождения своей Бабушки, как и обещал. Она была очень рада меня видеть, как и все остальные. Приятно, даже после такого, казалось бы, недолгого отсутствия, вернуться домой! Домой, где тебя ждут. 

Жена приезжает через день\mdash после работы, в пятницу вечером. К этому моменту одежда моя уже отстирана от кисловатого и едкого запаха дыма, а я хорошенько пропарился в баньке. Радость от встречи огроменная\mdash не можем друг на друга насмотреться\mdash оба соскучились! В~следующий раз обязательно постараемся пойти в сплав вместе. 

Остаётся совсем немного\mdash разобрать рюкзак с~байдаркой, чем я и занимаюсь. Шкура и кости байдарки по~отдельности затаскиваю на чердак. Выпрямляю стрингера, сломанный шпангоут отправляется в починку. 

\begin{center}
	\psvectorian[scale=0.4]{88} % Красивый вензелёк :)
\end{center}