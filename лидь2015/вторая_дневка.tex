\chapter{Вторая днёвка. 11.08.15}
\corner{64}

Сегодня будем тут целый день. Можно позволить себе поспать подольше. Выспавшись, валяюсь в палатке, делаю путевые заметки. Лежу, пишу спокойно, вдруг слышу голоса. Прислушиваюсь. Точно\mdash приближаются голоса, минимум двое. Беру на всякий случай топор, надеваю штормовку, вылезаю. Увидев меня, двое велосипедистов резко дают по тормозам так, что те издают протяжный скрип, и,~развернувшись, начинают удирать обратно. Чего~это~они? Соображаю, что я помятый спросонья, небритый неделю, на~голове вязаная шапка набекрень, штормовка на~тельняшку, в~руке топор. Я~бы~тоже развернулся и свалил… ладно, поспим~ещё~чуток. 

Лежу, ворочаюсь, сон больше не идёт. Снаружи слышно как дятлы завтракают, стоит стук. Заметки больше писать не хочется, вылезаю снова из палатки, собираюсь разводить огонь. Бригада моя тоже показывается. Аня перехватывает инициативу приготовления завтрака. Не~возражаю. Вместо моей традиционной овсянки нас ожидает, как я предполагаю, гречка. Не обманываюсь. Сдабриваем гречневую кашу сгущёнкой и моментально всё это уминаем. Лёня сидит у стола, молотя целлофановый пакет кулаком. Оказывается, в пакете овсяное печенье, куда потом Аня высыпает остатки орехов и изюма и сдабривает это всё остатками сгущёнки. Получается вариация на тему известного походного лакомства.

После завтрака бригада решила пойти на~противоположный пологий берег позагорать, поскольку у нас в лесочке тенёк. И~они пошли вброд на другой берег. Течение конечно сильное и сносит временами, но~ничего\mdash идут, благо мелко. Присматриваю временами за ними\mdash прошли, даже не заходя по пояс. Мелко. Только под самым нашим берегом поглубже и приходилось плыть. 

Пока бригада загорает на мели у противоположного берега, притаскиваю дров\mdash сухих нижних веток сосен, ради которых приходится залезать на высоту 2\sdash 3 метров, а потом и несколько больших сухих топляков с протоки ниже по~течению. Мелкие~ветки ломаю ногами, крупные распиливаю двуручной пилой, обернув резинку вокруг дерева и накинув на вторую ручку пилы. Мой~перевёрнутый складной стульчик выступает в роли подставки для бревна\mdash так удобнее пилить. Получается нормально, терпимо. Дров~у~нас теперь просто завались\mdash сегодня будем долго сидеть у костра\mdash днёвка же последняя, последний вечер, последняя ночёвка\mdash прощание со сплавом! 

Бригада, тем временем, возвращается. Решаем отступить от правил и приготовить обед\mdash во время сплавного дня мы идём без обеда, обходясь перекусом, но сейчас хочется что\sdash то сообразить посущественнее. Дрова~теперь есть, костёр постоянно немного дымит. Подкидываем пару поленьев и~приступаем к готовке.

После обеда мирно разбредаемся по лагерю\mdash я~приваливаюсь подремать к~перевернутой для сушки байдарочке. Лежу~в~полудрёме, продолжаю путевые заметки. Бригада,~отдохнув, делает зарядку! Вот это, я думаю, ничего себе. Соскучились по нагрузке! Ничего, завтра это исправим. День проходит в ленивом состоянии\mdash сказывается усталость. Причём непонятно, какая больше\mdash от несхоженности команды или физическая. Склоняюсь, что первый случай\mdash меня порой сильно удивляют их~действия, как~и~мои, в~свою очередь,~их. Физически же сна хватает восстановиться, причем иногда я сплю часов 5\mdash и~хватает, вот загадка! В~городе такое\mdash просто фантастика. Тут~же\mdash реально! На себе ощущаю. Ну, оно и понятно\mdash никаких городских забот, шума, гама, суеты\ldots~только ты, лес, река и~байдарка. Что может быть милее сердцу водника? Только~спутница~жизни.

Вспоминаю про каменистый бережок, который проходили вчера\ldots~снова приходится отгонять мысли об~упущенной возможности позаимствовать там камни для бани. Баня бы сейчас вышла просто царская\mdash дров тут напилить вообще легко. Топи себе и топи. Прошёлся выше по течению\mdash может камни найду. Тщетно. Песчаный берег и всё тут. Как ни ищи, камней нет. Это~вам не~Башкирия. Там бы мы уже были с баней\mdash вспоминаются Зилим и Белая. Ну да ладно, мне терпимо\mdash я и так моюсь в реке почти каждый день и получаю прекрасные ощущения от преодоления некой боязни холодной воды. Окунаешься\mdash прохладно, плывёшь\mdash красота! Свежо! Бодрит! А растирания полотенцем? О\sdash о\sdash о, это просто прелестно. Вернувшись в лагерь, осторожно купаюсь, борясь с течением. 

Время катится к вечеру, продолжаем уминать оставшиеся запасы провизии. Достаю прикупленное в~Чагоде ячменное и охлаждаю его в десантном котелке, зачерпнув туда воды. В котелок аккурат помещаются 2 банки\mdash просто тютелька в тютельку! Красиво жить не запретишь. Вспоминается Моргунов: <<Жить~хорошо, а хорошо жить ещё лучше!>>. Сидим, наслаждаемся жизнью и природой, безудержно тянет петь. Прокручиваю в голове все наши дни сплава. Как же хорошо, что мы выбрались на реку! Впечатлений огромное, просто неимоверное количество. Начинаю скучать по жене и дому\ldots~

Пора позаботиться о выброске. Достаю свои записи телефонов мужиков с микроавтобусами. Мой главный претендент, готовый за 3 тысячи отвезти нас в Череповец, оказывается занят, но наводит на другой контакт. Там~предлагают за 4.5, потому что не от Лентьево, а~от~Дубровки, которая на 50 километров дальше. Ну, 50 туда, 50 обратно, 100 километров просто так никто конечно не поедет, нормально и за 4.5, я считаю. Тут бригада начинает бесноваться из\sdash за цены. Набиваю трубочку, отхожу в сторонку успокоиться\mdash палатка, значит, за 15, коврики минимум по 5 каждый\ldots~не моё, конечно, дело считать средства бригады, но\ldots~ещё куча разного не самого дешёвого снаряжения, а на выброску торгуемся из\sdash за \textit{одной} тысячи? Этого я не понимаю даже больше, чем фильтр воды. Ладно, телефон  мужика, согласного за 4.5, у  меня есть, а~остальное\mdash хотят дешевле, пускай ищут. И~Аня~ищет. И~находит в интернете телефон аж начальника автовокзала в Лентьево! Тот объясняет, что у него только большие автобусы, но обещает помочь. Я со всего этого просто внутренне угораю. Причём бригада разводит какую\sdash то ненужную панику, суету\ldots~зачем? Боятся не уехать что ли? Вот умора.

Поначалу меня раздражает такая самодеятельность бригады, а потом думаю ладно, если уж так хочется. Дымлю трубочкой, расслабляюсь, ситуация даже начинает меня забавлять, жду чем дело кончится. Через~час\sdash полтора дело кончается тем, что Аня заполучает от~начальника автовокзала наводку на какого\sdash то водителя. Звонит, договаривается, сбивая до 3.5. Я не возражаю\mdash пускай. Машина\mdash <<Газель>> пассажирская. По телефону объясняю про наш огромный рюкзак с байдаркой\mdash говорят, что~влезет. Вопрос решён.

Солнце радует красивейшим закатом. Вообще в этом сплаве с погодой нам очень повезло\mdash только в один день было подобие дождика с утра, а остальные дни просто как по заказу были солнечными и ясными.

Ужинаем последними банками тушёнки, разогревая их прямо в костре. После, усевшись у~костра с~чаем, начинаются разговоры о~билетах на~поезд\ldots~ паника номер 2, или <<давайте купим билеты через интернет сейчас>> и <<покупать билеты в кассе\mdash прошлый век>>\ldots~да\sdash а\sdash а, зря на этой стоянке есть интернет, думается мне. Хотя,~я~и~сам сначала думал так сделать. Ладно, давайте купим. Правда, из интернета только EDGE, 3G нету. Сайт РЖД не~грузится нормально и там регистрация нужна\ldots~думаю: <<фух,~пронесло!>>. Но нет, сам же ляпнул, что у меня в~телефоне есть приложение какое\sdash то, через которое можно купить. Смысл какой? В~плацкарте нижних уже нет, остались верхние, да боковушки. Но~бригада упорствует. Ладно\sdash ладно. Покупаем через мобильное приложение с комиссией, бригада успокаивается. 

А мне неспокойно. Да\mdash оплата прошла, да\mdash деньги с банковской карточки списались, а билет\sdash то где? Эта~электронная хрень в виде номера? Не привык я к этому всему прогрессу тогда ещё. Да и комиссию содрали,~ЪУЪ! Забиваю~трубочку потуже, успокоюсь. Решаю задать ребятам~жару.

Плещу в подкотельник 96\sdash й. Сначала думаю даже не~разводить и добавить туда перца, соли, горчички и~т.д.\mdash как обычно это делают на посвящении. Но~думаю, ладно, не~буду уж. Плеснул туда же воды и~сделал, как говорится, <<сортировку>>. Не~удержался, горчичку всё же добавил. Вытащил из палатки весло, собрал, дал им в руки, показал, как держать вдвоём согласно ритуалу. Вкрутил патрон в~сигнал охотника, взвёл. Остается последнее\mdash Страшная Клятва Туриста\sdash Водника. И вот, теперь уже я, сам не~так давно прошедший через~это, посвящаю новых людей в~наше Братство. Произношу слова Клятвы, бригада повторяет за~мной. Потом ритуал с подкотельником и завершающий аккорд\mdash стрельба ракетой в ночное небо\ldots~   

Западёт им это в душу или нет\mdash неизвестно никому. Пойдут~ли они во второй раз на воду? Тоже неизвестно. Может~быть этот сплав вообще останется единственным в их жизни. Каждый~по\sdash своему воспримет всё действо, связанное со сплавом, причём надо дать впечатлениям время улечься в~голове. Как я уже писал, кто\sdash то после первого сплава и, как ему кажется, пережитого <<кошмара>>, с ужасом убегает прочь от этих <<сумасшедших>>, а~кто\sdash то, вступая в~Братство, уже никогда не выйдет из него\ldots~всё~зависит от~всего, так сказать. Но одно я знаю точно\mdash если отрава бродяжничества попала в кровь и прижилась\mdash это навсегда. Покоя больше не~обрести. При первой же возможности будет тянуть на~выход. На Природу. На Реку.
\newpage
Надеюсь, наш сплав не стал кошмаром для моей бригады. Конечно, где то я, как Адмирал похода, был твёрдым, но, по\sdash возможности, справедливым командиром. Старался, безусловно, не перегибать. Насколько это получилось\mdash судить не мне. Я~пытался, как мог, смягчить походный быт для ребят и передать тот опыт, который доступен мне, параллельно сам получая всё новые бесценные уроки. Попадали мы во всякие передряги, особенно в~первые три дня и теперь, в конце маршрута, я мог сказать, что~изо всего мы вышли малой кровью, получили максимум удовольствия и экстрима, насколько можно было его получить в таком сплаве. Об этом думалось мне, сидя у~костра в последний наш сплавной вечер. Добив в своей самодельной трубочке последний табачок, сжёг её и кисет в~костре, чтоб больше не было соблазна до следующего сплава. Так\sdash то. 

\begin{center}
	\psvectorian[scale=0.4]{88} % Красивый вензелёк :)
\end{center}