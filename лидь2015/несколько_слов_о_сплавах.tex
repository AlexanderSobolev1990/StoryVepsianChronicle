\chapter{Несколько слов о сплавах} 
\corner{64}

\vspace{-0.5cm}
\setlength{\epigraphwidth}{0.6\textwidth}
\epigraph{%Не нужен бродягам дом и уют \\
%	Нужны - Океан, Земля \\
%	Что звёзды Медведицы им поют \\
%	Не знаем ни ты, ни я \\
%	~...\\
	Вел\'{и}к Океан и Земля велик\'{а},\\
	Надо бы всё пройти,\\
	Большая Медведица издалека \\
	Желает тебе пути.}
	{
	\begin{flushright}
		\small{Станислав~Пожлаков,\\<<Баллада~о~детях~Большой~Медведицы>> из~х/ф~<<Идущие~за~горизонт>>,~1972}%\\(экранизация~повести~Олега~Куваева\\ <<Птица~капитана~Росса>>,~1968)
	\end{flushright}
	}

Неизвестно, от чего получаешь б\'{о}льшее удовольствие\mdash от подготовки к сплаву, от собственно самого сплава или от смакования воспоминаний о нём в кругу друзей, родственников и единомышленников. Каждая часть этого действа хороша по\sdash своему и от каждой остаётся определённый отпечаток, который уже никогда никуда не денется и будет греть душу туриста\sdash водника долгими зимними вечерами.

В широко известной в узких кругах книге <<На~байдарке>> \cite{Квадригин} есть строки о том, что водниками не становятся\mdash в водный туризм втягивают. И это действительно так. Правда, осознаётся это только после нескольких сплавов. Меня втянула в водный туризм тогда ещё будущая жена, предложив сходить в сплав, сама видимо не вполне осознавая, на что идёт. С тех пор в моей душе поселился огонёк странствий по водным просторам нашей необъятной Родины.

Первый сплав\mdash это всегда особенное действо. После него люди либо брезгливо убегают прочь от <<этих ненормальных>> с рюкзаками и вёслами, так и не поняв, зачем всё это надо, либо в их сердцах поселяется огонь странствий, а в мозгу заседает заноза, не дающая покоя ровно от предыдущего сплава до следующего. 

С первым сплавом мне и будущей жене повезло\mdash мы попали на Южный Урал, реку Зил\'{и}м. Сплавлялись около 10 дней, на катамаранах, вместе с огромной компанией в 50 человек, среди которых было 5 или даже больше инструкторов. Река Зилим летом спокойная, но мелковатая в верховьях. Туристы-водники, любящие бурлящую воду и~белую пену, брезгливо называют такие сплавы матрасными. Что ж, это в какой-то мере справедливо\mdash течение реки спокойное, грести особо не надо, лишь местами могут встретиться быстринки и подобия перекатов. Но, в конце концов, все виды отдыха хороши. Мне, жене, да и всем тем людям, с кем мы ходили, нравится именно такой вид сплава\mdash без риска для жизни, но со спокойным единением с необычайно красивой природой нашей страны, когда забываешь обо всех проблемах и заботах, тяготящих тебя в~обычной жизни\ldots
\newpage
Инструкторы были заняты на стоянках приготовлением еды на костре, заготовкой дров бензопилами, ну и~конечно травлей анекдотов. Фактически, нам, как новичкам, оставалось только поставить палатку, покидать в неё свои вещи, да наслаждаться жизнью, травить байки и ожидать пока поспеет вкуснейшая вечерняя похлёбка от штатного кока, да закипит чай с душистыми травами, собранными на~местных лугах. 

Нам с будущей женой всё было в новинку\mdash мы никогда не ходили в походы или сплавы. И хотя я много раз порывался выбраться в какой-нибудь поход, каждый раз меня что\sdash то останавливало. Причём перспектива походной жизни меня никогда не страшила\mdash наоборот, мне очень хотелось попасть в настоящий поход и приобщиться к дикому туризму, но всё почему-то не получалось. А тут мы удачно попали в большую команду, нам выдали палатку, снаряжение. Словом, мы упростили себе <<порог вхождения>> в дикий туризм тем, что вопросы, связанные с маршрутом, заброской и выброской, снаряжением решали не мы, а руководители.

В первом сплаве мы испытали на себе и каменистые отмели, когда острые камни противно трутся о дно катамарана, и быстрые перекаты, когда река ускоряется в скалах Южного Урала и надо активно подгребать веслом, выправляя курс. Было и жаркое палящее солнце, моментально высушивающее намокшую одежду, и ураганный ветер с дождём, порой настигающим совсем не вовремя\mdash на воде или когда ставишь палатку. Была~и~походная баня. Это изобретение заслуживает отдельного рассказа, а тот, кто её придумал, безусловно, достоин какой-нибудь премии.
\newpage
Баня в походе\mdash это Баня с Большой Буквы. Помыться, в принципе, можно и в реке, но разве это хоть в какой\sdash то мере сравнится с ощущениями от Бани? Распаренный, вдыхая аромат трав, подстеленных на каменистый берег в роли коврика, хлещешь себя берёзовым, дубовым, еловым, пихтовым веником\mdash любым каким смог сделать. Инструктор, который уже тебе товарищ и друг и с которым выпиваешь по чарке за ужином, обсуждая как классно вообще жить на Земле, плещет воды на каменную печь и по Бане расходится превосходный пар, обдавая тело волнами колкого тепла. Веники ходят по спинам, раскаляя их докрасна\ldots~После, Баня открывается и распаренные, будто заново родившиеся, счастливые туристы с визгами восторга бегом мчатся в речку\mdash окунуться с головой, остудиться и~проплыть пару метров, а потом обратно\mdash греться и снова париться. Особого шика добавляет то, что если во время этого действа идёт тёплый летний дождик\ldots  

Сама Баня состоит собственно из печки и палатки, которую ставят потом поверх протопленной печи. Печь~складывают из камней, коих имеется неограниченное количество на берегах рек Южного Урала, выбирая при этом такие, чтобы не трескались при нагреве. Как выбирать\mdash эту науку постигли только Инструкторы, ходящие на эти реки как к себе домой по несколько раз каждый год. Потом~ставится палатка\mdash вроде торговой, представляющая из себя куб или прямоугольный параллелепипед и сверху накрывается дополнительно полиэтиленовой пленкой, чтобы меньше уходило тепло и пар. Далее кто-нибудь снаряжается за травой или папоротником для банного пола и ветками для веников. Всё, потрясающие ощущения гарантированы!

После Бани чистые, обновленные туристы усаживаются к импровизированному столу вкушать отменный ужин от кока и пить вкуснейший чай с местными травами. Уже начинает темнеть и сытые, чистые, счастливые люди разбредаются кто куда\mdash часть попеть у костра, часть медитировать на берег реки, кто-то подальше от костра и людей смотреть или фотографировать звёзды, а кто по палаткам творить походную романтику. Уверен, что походная романтика создала немало счастливых семей, где Она знает, что Он может добыть дров, развести огонь и~вообще, что топор из рук не валится. Он, в свою очередь, знает, что Она может фактически из ничего приготовить вкуснятину к ужину и уж конечно оба они\mdash туристы\sdash водники, всегда готовые прийти на помощь друг другу и~поделиться последним, неважно, куском хлеба или сухим спальником. 

Стапель на Зилиме был в Толп\'{а}рово, а снимались с~маршрута в Именд\'{я}шево. Толпарово\mdash прекрасное место! Мост через реку, поросшие соснами склоны горы Карамал\'{ы}, сельские домики, сельхозтехника, огороженные пастбища, стога сена\ldots~Я смотрел на всю эту красоту и не мог поверить, что всё происходящее\mdash реальность. Мне казалось это какой\sdash то параллельный мир. Какая\sdash то вдруг Швейцария, внезапно. Но нет, это наша страна, это Южный Урал! Всё\sdash таки как велика, необъятна, красива наша Родина!

На маршруте мы побывали около небольшого интересного озера недалеко от берега реки. Вода в нём была прозрачной, небесно\sdash голубого цвета, и очень\sdash очень холодной. Погода в тот момент была прохладной и дождливой, но всё равно среди нашей команды нашлись смельчаки, окунувшиеся в это природное чудо. Потом мы побывали в~пещере Победа, для чего потребовалось карабкаться вверх в горы по крутым склонам\mdash вход в пещеру находился высоко наверху. Всё это, конечно же, оставило неизгладимые впечатления в наших сердцах! Попасть в такое приключение было настоящей удачей.

Так я изложил общие впечатления человека, втянутого по собственной воле в туристическое братство и в первом же сплаве произнёсшего страшную клятву туристов-водников. После первого сплава был и второй\mdash по Башкирской реке Белая (Агид\'{е}ль) компанией из 10 человек. Этот,~более камерный что ли, сплав на катамаранах также нам очень понравился. В перерыве между первым и вторым сплавом моя женщина, спутница\sdash красавица, окончательно и бесповоротно стала моей спутницей жизни, что ещё раз подтверждает вышесказанное про походную романтику. Так~что второй раз на воду мы встали уже будучи супругами.

Сплав по Белой, безусловно, отличался от~предыдущего\mdash хотя бы впятеро меньшим количеством народа. У нас также были переходы и днёвки, походные Бани и созерцание звёзд, дегустация мёда и медовухи с~местной пасеки, рыбалка и фотоохота. Ещё было посещение пещеры Сказка, расположенной на живописнейшем склоне горы около Акбул\'{а}тово, и К\'{а}повой пещеры (пещера Шульг\'{а}н\sdash Таш), в которой раньше жили доисторические люди. Впечатлений также хватало, мы набирались опыта\mdash благо есть у кого\mdash наши руководители на всех реках Южного Урала были как родные.

Эти два сплава заложили, можно сказать, основу будущим моим путешествиям. В них я понял, что даже такой не слишком подготовленный человек, как я, вполне справляется с походным бытом и что никаких особенных премудростей не требуется. Нужно лишь желание и~стремление идти вперед. Из этих сплавов я вынес очень хорошие уроки организации всего этого действа, приобрёл уверенность в силе человека противостоять природе и~непогоде. А~это ли не главное? Где\sdash то в глубине души загорелась мысль, что я, в принципе, мог бы и сам организовать такое мероприятие, как сплав.

\begin{center}
	\psvectorian[scale=0.4]{88} % Красивый вензелёк :)
\end{center}
