\chapter{Байдарка}
\corner{64}

После второго сплава был большой перерыв. Не~удавалось выкроить время, чтобы попасть на~Южный Урал, да и, откровенно говоря, хотелось чего\sdash то нового. Правда,~пока сам не понимал чего. Потом, вместе с~рассказами начальника о байдарочных сплавах, пришло осознание\mdash вот оно, это новое! Это~же~то, что надо! Байдарка! Отдельный корабль! Твой! Собственный! Никаких~инструкторов, всё делать самому! Тебе~никто не~указ! Ты\mdash бесстрашный рекоплаватель и~первооткрыватель! Ты\mdash Капитан корабля! Сердце моё застучало чаще, разум затуманился. 

После того, как туман развеялся, было много разговоров и рассказов начальника о том, что такое байдарочный сплав и с чем его едят. Вместе с разговорами пришло и второе осознание\mdash идём на реку! Рассуждали о том, на какую реку идти, как забрасываться и как сниматься с маршрута$\ldots$~а кроме того\mdash нужны были байдарки. Мой выбор пал на консервативную <<Таймень>>, поскольку нашлись ещё школьные знакомые-одноклассники, готовые отдать каркас, или как говорят водники, <<кости>>, ну а новую ПВХ оболочку, <<шкуру>>, купить не проблема. Не обошлось без влияния наших разговоров\mdash начальник рассказывал, что места в <<Таймени>> просто завались и это действительно оказалось так.

Вообще, тут стоит отметить, байдарочный поход в~данном случае оказался неизбежен, как крах мирового капитализма. Я~попал на работу в <<почтовый ящик>>, начальник, увлекающийся водным туризмом, сама атмосфера <<почтового ящика>>$\ldots$~всё это настолько тонко, что почти непередаваемо людям, которым никогда не~пришлось столкнуться с этим. А когда я поделился с~начальником рассказами о своих прошлых сплавах, то тут\sdash то, надо полагать, в его голове и сложилась картина вновь замаячившей возможности оказаться в привычной стихии. Постепенно мы оба свыклись с мыслью, что однозначно пойдём на реку.

Время летит быстро\mdash зима, весна; покупка опять\sdash таки консервативной палатки\sdash домика (типа <<Памирки>>), долгие переговоры о передаче байдарки в мои руки и, наконец, я стал счастливым обладателем костей для <<Таймени\sdash3>>. Лишними трёшечные кости точно не будут\mdash вдруг ещё пойдём втроём? Заполучил кильсон, позволяющий собрать из трёхместной байдарки двухместную, а также обзавёлся ПВХ шкурой для <<Таймени\sdash 2>>, поскольку мы планировали идти двумя двухместными байдарками, набрав ещё двоих людей в матросы. Начальник же приобрел каркасно\sdash надувную байдарку <<Гарпун\sdash 2>>, поскольку старые свои <<Таймени>> давно продал, да и хотел судно полегче. Теперь у нас было то, что понесёт нас на просторы рек! Моё сознание рисовало бесстрашных первооткрывателей времён Великих Географических открытий!

Ранняя весна. Переправил всё снаряжение постепенно в деревню, провёл модернизацию палатки, сделал оттяжки для неё. Настало время для главного\mdash собрал в первый раз байдарку. Это совершенно непередаваемое ощущение. Ещё каких\sdash то 40 минут назад ЭТО было запаковано в~огромный рюкзак, потом лежало на траве в виде свёрнутой шкуры и горы дюралевых трубок, а теперь всё собрано в красавицу\sdash байдарку совершенно нешуточных размеров! От такого творения рук человеческих захватывает дух! Очнуться помогают только возгласы жены о том, как эту тяжесть тащить к реке? Да, весит <<Таймень>> немало, но~это того стоит. Не думаю что по простоте конструкции, комфорту, вместимости, ходкости, красоте в конце концов, какая\sdash либо байдарка может сравниться с <<Таймень\sdash 2>>$\ldots$~Ух! 

Снаряжение было собрано к майским праздникам\mdash байдарка, палатка, а остальное есть с прошлых сплавов, надо только достать с чердака и антресолей. Более~того, обладая комплектом костей на двушку и трёшку, я~получил свободу выбора\mdash собирать, соответственно, двухместную или~трёхместную байдарку, что, понятно, неплохо. Параллельно с~этими заботами прорабатывался маршрут будущего сплава, велась оценка буквально всех мелочей, читались отчеты в~интернете\mdash конкретно по планируемым рекам для сплава, да и по другим тоже. Изучал и просто литературу по~туризму, выискивая то, что могло ускользнуть от~меня ранее. Шутка ли, ведь предстояло первое в моей жизни большое походное мероприятие, которое я планировал от~начала до конца сам и~сам брал на себя ответственность за~всё происходящее.
\newpage
Разгар весны, потеплело. 9 мая, 70 лет Великой Победы. В~этот день было решено спустить байдарку на воду и~опробовать собственно\mdash а~каково это\mdash плыть на~байдарке? Понятно, что отличается от~катамарана, но~всё~же\mdash как~это? Байдарка была доставлена на берег реки Киржач недалеко от железнодорожной станции Илейкино, собрана и спущена на воду. Место стапеля идеальное\mdash есть подъезд к реке, ровный покатый берег, небольшой песчаный пляжик. Минут 30\thinspace--\thinspace 40 сборки и ОНА готова! Впечатления~и~тот трепет, который охватил меня, когда я~в~первый раз сел в байдарку на~воде\mdash непередаваемы совершенно. И вот\mdash первый гребок, второй, третий\mdash я плыву! Нос байдарки рассекает воду, создавая волну, в~кильватере бурлят буруны. Вдвоём с отцом прошли не более километра и вернулись обратно\mdash это была только проба~пера. Первая тренировка на воде. Своеобразная прелюдия перед чем\sdash то б\'{о}льшим$\ldots$
 
Река Киржач по весне очень полноводная и течение весьма и весьма быстрое, но, несмотря на это, мы вдвоём без груза легко шли против течения. Правда, в одном месте при сужении русла течение ускорялось, и надо было пройти противотоком в своеобразное <<игольное ушко>>. Это~далось нам только раза с третьего\mdash настолько сильным было течение. Мне, понятное дело, очень понравилось, и~я~подумал, что сюда я~ещё обязательно вернусь. 

Лето, прекрасная пора, полным ходом идёт подготовка к~основному сплаву\mdash закупается белорусская тушёнка, сухие супы, пишется примерная раскладка продуктов, которые купим уже там, на~месте, чтобы не тащить всё из Москвы. К этому времени подготовлены, распечатаны и <<подняты>> топографические карты\mdash гордость Капитана (поднять карту\mdash выделить что\sdash либо на карте цветом и/или дополнительными условными знаками). Приобретён б/у старенький GPS\sdash навигатор, в него заложен маршрут и~путевые точки. Кроме того, приобретены армейские бахилы химической защиты\mdash вместо сапог\mdash дешевле в 10 раз, легче раза в 3 и места в миллион раз меньше занимают, поскольку их можно свернуть рулоном. 

Приходит осознание необходимости пойти перед большим сплавом в сплав выходного дня потренироваться. Несмотря на опыт прошлых двух походов, я всерьёз опасаюсь что\sdash то не учесть и забыть при подготовке к~большому сплаву, потому что на этот раз я действую самостоятельно. Следует также понимать, что маршрут предстоящего двухнедельного путешествия будет проходить порой по очень глухим местам и помощи, в случае чего, ждать будет неоткуда. Хоть и придаёт уверенности участие в мероприятии моего начальника\mdash опытного байдарочника, всё равно дополнительная тренировка лишней не будет. Жена с~радостью соглашается выбраться с байдаркой на~природу, и мы собираемся.

Выбор падает на ту же реку Киржач, где был первый спуск на воду. Подкупает близость к деревне, удобство стапеля и антистапеля, полноводность и относительная малолюдность\mdash уже далеко не майские праздники, а~середина лета\mdash народу на~реке заметно поубавилось. Собираем снаряжение, закупаемся провизией и вот мы на~стапеле в Илейкино! Заправски, уже в третий раз, собирая байдарку, с лёгкостью на душе мы пакуемся и отчаливаем. Впереди нас ждали 2 дня единения с природой$\ldots$

Сплав выходного дня удался, хоть и вымотал нас с~непривычки. Повезло с погодой\mdash дожди обошли стороной, выглянуло солнце. На пути встретился один обнос\mdash разрушенная плотина бывшей Финеевской ГЭС и один небольшой слив с перепадом сантиметров 10\thinspace\nobreakdash--\thinspace 15, добавивший немного экстрима. В остальном\mdash очень спокойный и~прекрасный маршрут для неспешного сплава. Байдарка приносит удовольствие, это понятно ещё с первого спуска на воду, а когда идёшь с любимым человеком\mdash это вдвойне приятно.

К этому времени было собрано абсолютно всё для основного, большого 10\sdash дневного сплава\mdash закуплена тушёнка, которую везти из Москвы, байдарка подготовлена и упакована, вещи тщательно отсортированы и погружены в гермомешок, куплены железнодорожные билеты, проработан до мелочей маршрут. Учтены все интересные места, предполагаемые места стоянок, вся доступная картографическая информация изучена досконально и~достигнуто состояние полной, не побоюсь этого слова, боевой готовности! Но гораздо  важнее уверенность и~психологическая готовность\mdash ведь я уже 2 раза был на воде в байдарке. И ещё 2 раза до этого сплавлялся на катамаране. А это все\sdash таки чего\sdash нибудь, да значит. Немного страшит должность и ответственность капитана корабля, но не мы первые, не мы последние! Огорчает то, что жена в этот раз не может пойти со мной, потому что не удалось совместить наши отпуска. Придется обитать в~палатке одному, но это мы обязательно наверстаем с~ней в~следующем сплаве. 

Экипажи подобраны\mdash люди относительно незнакомые, но внушающие доверие. С ними проведён инструктаж и~психологическая обработка за рюмкой чая о предстоящих трудностях и тонкостях водного похода. Зелёные, они с головой окунаются в преподнесённую мной атмосферу~сплава. Также, по моему решению, наша команда встречается вся вчетвером вечером после работы, и мы ещё раз всё обговариваем, поскольку я считаю, что знакомиться на стапеле\mdash не самый лучший вариант.

Как и в любом деле, в деле байдарочных сплавов тоже существуют свои законы. Один из них\mdash <<Кто\sdash нибудь из~команды не сможет пойти>>. Так и случается\mdash начальник, наш Адмирал и Идейный Вдохновитель, не может идти с~нами в силу непреодолимых причин. Из~четырёх человек остаётся трое. По плану было две байдарки\sdash двушки, а теперь чего делать? Спасает положение то, что у меня есть кости для трёшки. Покупается шкура трёшки и третье весло, и мы снова команда$\ldots$
 
Психологически это удар\mdash остаться без Адмирала за~несколько недель до сплава. Теперь тяжесть ответственности целиком и полностью ложится на мои плечи. Теперь~я и~Адмирал сплава, и Капитан своей байдарки, и руководитель похода. Для меня это оказалось полной неожиданностью, поскольку я~рассчитывал на то, что у нас будет более опытный и~бывалый Адмирал, нежели~я. Однако, это обстоятельство не опустило мне~руки, а наоборот, мобилизовало все~психологические и~организаторские ресурсы. Остальная~часть команды идёт на реку в первый раз и ещё толком не готова, но полна решимости. Изрекая извечное\mdash <<Не дрейфь, прорвёмся!>>, продолжаем подготовку к нашему мероприятию и мчимся в~неизвестность. 

\begin{center}
	\psvectorian[scale=0.4]{88} % Красивый вензелёк :)
\end{center}