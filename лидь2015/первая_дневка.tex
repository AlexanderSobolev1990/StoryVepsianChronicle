%\chapter{Первая днёвка. 07.08.15}
\chapter{Первая днёвка}
%\corner{64}
\vepsianrose

Проснулся снова рано\mdash только\sdash только был восход. Весь в холоднющем поту от безумного кошмара. Мне~очень редко снятся сны, а когда снятся, то редко что\sdash то хорошее. Вылез тихонечко из палатки, пошел пройтись. Забил~трубку,~задымил$\ldots$ ~надо прогуляться, отойти от~такого реалистичного сна.

Решаю идти вдоль дороги, у которой мы встали. Через~каких\sdash то пару сотен метров выхожу из лесочка и~вижу$\ldots$~ работу Закона Лучшей Стоянки \cite{Квадригин}, который гласит: <<Лучшая Стоянка\mdash в~двухстах метрах ниже по реке>>. Полянка\mdash ровная, муравьёв нет, словом\mdash красота! И~спуск к воде более приличный. Короче\mdash вчера уже были настолько уставшими, что пройти сюда эти 200 метров было просто нереально. Ладно,~у нас и там в~лесочке хорошо, перемещать лагерь я не хочу. Правда сейчас проснутся или~уже проснулись муравьи$\ldots$

\newpage
Гуляю долго. Постепенно ночной кошмар уходит, и~я~возвращаюсь в лагерь. Потихоньку начинаю заниматься завтраком. Вытаскиваю вещи, начинаю разводить костер, дров нет$\ldots$~ уже~около 9 утра, поэтому бужу ребят, что~непросто, и отсылаю за дровами. Лёня снова разграбляет старый домик лесника, как я его назвал, и мы топимся старыми досками. 

Сегодня днёвка\mdash плыть никуда не будем. Отдохнём, помоемся, просушимся, байдарку отремонтируем как~следует. После завтрака Лёня, взяв удочку, удаляется на берег реки ловить рыбу на кукурузу. Пожелав ему удачи, начинаю подготавливаться к заклейке пробоин. Муравьи этому активно мешают\mdash оказалось, что байдарку мы поставили на ночь прямо на муравьиную тропу! Ассистировать при операции будет Аня\mdash кипятить воду и наливать её в Капитанскую Кружечку. Вскоре заплатки вырезаны, клей нанесён, Аня приносит в пехотном котелке кипяток. Я~заклеиваю прорехи, проглаживая заплатки кружечкой. Муравьи больно кусаются и~в~своей слепой любознательности умудряются залезть даже в~пузырёк~с~клеем! 

Пробоин немного\mdash 5 штук, но прямо под перекладинами кильсона. Решаю, чтобы избежать такого в дальнейшем, проложить между шкурой и кильсоном, а также под место крепления штевня\footnote{Штевни\mdash части корпуса, которыми заканчивается набор судна в~носу и~корме.} к кильсону куски туристического коврика. С этой целью отрезаю от него ещё полосу 10\thinspace--\thinspace 12 сантиметров шириной и то же проделываю с~одним из ковриков бригады. 10 сантиметров коврика погоды для сна не сделают всё равно, зато теперь за днище мне гораздо спокойнее.

Всё\mdash заплатки поставлены, коврик подложен, время выждано пару часов\mdash можно проверять байдарку. Отрываю~Лёню от безрезультативного сидения с удочкой, и~мы~вдвоём на пустой байдарке проплываем немного, чтобы убедиться в отсутствии фильтрации через заплатки. На пустой байдарке вдвоём идти просто сказка\mdash рассекаем по течению и против него с удивительной лёгкостью. Всё нормально\mdash заклеились хорошо, заплаты не~подтекают. 

Причаливаем, вытаскиваем байдарку сушиться. Обнаруживаю сломанный второй шпангоут и вылетевшую из него шпильку$\ldots$~благо шкура держит и конструкция относительно устойчива. Что делать? Не~разбирать же каркас для ремонта\mdash очень не хочется возиться со стреляющими стрингерами\footnote[1]{Cтреляющий стрингер\mdash стрингер с подпружиненной частью (в~байдарках <<Таймень>>).}\mdash их непременно уже подклинило от плавания. В~итоге обильно заматываю разрыв шпангоута любимым ремонтным материалом всех времён и народов\mdash синей изолентой. Должно~держать, место не столь критичное. Плюс~магия синей изоленты должна работать, куда ж без неё? 

Теперь очередь на заклейку за~надувными сиденьями\mdash вытаскиваю из них надувную часть, локализую многочисленные пробоины и~тоже клеюсь. Как~мы умудрились их проколоть? На эту заклейку уходит ещё часа два. 

Потом заготавливаем дровишки на вечер, гоняем чаи с печеньками и сгущёнкой. Ребята кайфуют, отдыхают, а я заканчиваю ремонтные работы и иду купаться\mdash красота! Правда, камни острые при входе в реку\mdash стараюсь их~откидывать. Вообще камней прилично на этой стоянке\mdash можно попытаться сделать и баню. Но~уже перевалило за~{15\thinspace\nobreakdash--\thinspace 16} часов, раньше начинать надо было\mdash ведь для бани надо печь сложить и протопить, а значит дров прилично напилить. Кроме того, надо мастерить каркас и чем\sdash то его накрыть. Ну, положим, мы для этого приспособим внешние тенты наших палаток. Хватит ли их? Не думаю. Ладно, может ещё будут стоянки с камнями, решаю я, и~баню откладываем до лучших времён, ведь изначально, при~подготовке к этому походу, я её не планировал.  

Под берегом однозначно живёт выдра, которая достаточно часто что\sdash то там по своим делам делает и~плещется. На ночь надо всё съестное надо убирать подальше на всякий случай. Наступает вечер, начинаем готовить ужин. Вспоминаю про ту полянку с лисичками\mdash делаем жарёху! Прямо на фанерном сиденье от байдарки режу картошку, лук, лисички. Лёня продолжает разграблять сломанный домик лесника\mdash приносит на этот раз часть дверного косяка. Кладём её параллельно костру, получается что\sdash то вроде плиты. Жарёху помещаю в~крышки армейских котелков на манер сковородок и~ставлю на~огонь. Регулировать жар очень неудобно\mdash приходится то отодвигать импровизированные сковородки, то вновь приближать к огню. Но, тем не менее, получается довольно сносно. Однако впредь, решаю я, надо обзавестись нормальной маленькой чугунной сковородкой на~такой~случай. 

Спустя минут 20 моих попыток не спалить ужин мы получаем наивкуснейшую, правда всё\sdash таки немного подгорелую, картошку с лисичками и жареным луком.  Просто сказка какая\sdash то! Полдня уже играет моя походная музыкальная приставка, сидим у костра сытые и довольные. Янтарное закатное солнце играет на верхушках сосен. К~вечеру ветер совсем стихает и ничто не нарушает идиллию нашей днёвки.

За готовкой, ужином и разговорами у костра не~замечаем, как наступает тёмная августовская ночь. Подводя итоги дня у костра, стреляю вверх зелёной ракетой. Она~взлетает ввысь, озаряя окрестности своим свечением, а затем, догорая, падает в реку\mdash слышен тихий <<бульк>>. Звук выстрела разносится эхом по лесу и растворяется в~этих глухих, почти не тронутых человеком, сосновых борах Вологодского края$\ldots$

Традиционно поздно расползаемся по палаткам. На этот раз я догадываюсь обильно подложить подо всю поверхность дна палатки сосновый и еловый лапник\mdash получается очень мягкая пружинящая <<подушка>>. Так, я надеюсь, будет теплее~спать. 

\begin{center}
	\psvectorian[scale=0.4]{88} % Красивый вензелёк :)
\end{center}
