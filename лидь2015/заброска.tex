%\chapter{Заброска. 04.08.15}
\chapter{Заброска}
\corner{64}

Обычно я не высыпаюсь в поезде и с утра поднимаюсь как варёный рак. Но тут на удивление удалось выспаться. Может потому что на нижней полке? Тем временем проехали Вологду. Скоро уже и Череповец. Погодка стоит безоблачная, разгорается утро. Взяли чаю, перекусили, плацкарт понемногу начал оживать. Поезд прибыл в~Череповец точно по расписанию\mdash в 8:29 утра, конечная. Хорошо\mdash не~надо торопиться с высадкой, вытаскивая наши тяжеленные рюкзаки$\ldots$
 
Поезд останавливается, народ быстренько выходит. Мы не спешим, ждём, пока все выйдут и только потом, последними выходим со всеми своими рюкзаками и~баулами. Вокзал в Череповце достаточно красивый, старый, зелёно\sdash белого цвета. Очарование старых железнодорожных станций\mdash это что\sdash то особенное$\ldots$~стою и рассматриваю вокзал. Ребята подтягиваются со своими огроменными рюкзачищами. Сваливаем вещи горой на скамейку и~усаживаемся передохнуть\mdash баулы наши не из лёгких, а скоро ещё прибавится провизия. Как на стапеле всё это понесём до реки? 

Слегка приходим в себя после поезда. Надо идти закупаться продуктами. Аня~остаётся присматривать за~вещами, а~мы с~Лёней собираемся\mdash берём две сумки и~выходим с~вокзала. 

Первоначально я наметил продуктовые магазины с~противоположной стороны, к~северо\sdash востоку от вокзала, но~как попасть туда мы не~поняли\mdash перехода через пути нет, на~путях стоят грузовые поезда, надо идти до автомобильной эстакады, что виднеется вдоль ж/д путей$\ldots$~далековато. Решили искать магазины на стороне вокзала. Спросили у~местных дорогу и пошли через привокзальный сквер на~юго\sdash запад. Сразу за сквером вышли к~продуктовому на~Комсомольской 39. Разгул капитализма\mdash закупились всем, чем только можно\mdash макароны, гречка, рис, чай, сахар, печенье, сухие супы, сгущёнка в банках и пакетах, хлеб, соль, перец, да всего и не упомнишь. Ещё купили мороженое и пошли обратно к вокзалу. Операция по закупке прошла достаточно резво\mdash быстро вернулись, прикупили местную газету <<Речь>>, пару магнитиков с символикой Череповца и пошли к Ане. Она загорала и читала книжку, ожидая нас. Погода выдалась просто отличная\mdash солнышко, немного белых облачков в высоком синем августовском небе, лёгкий северный ветерок. Прохладно~и~не~жарко\mdash просто~замечательно! 

Наш архангельский поезд отправлялся в 12:08, в запасе было ещё пару часов. За это время надо было снова купить билет на байдарку на вторую часть пути, потому что оформить этот билет в Москве не удалось. Девочки в~кассе сначала отправили меня в терминал, мол распечатай там. Обрадовался! Думаю\mdash ничего себе, до чего дошёл прогресс, сейчас через терминал куплю багажный билет на~байдарку! Но не тут\sdash то было\mdash терминал оказался просто терминалом для распечатывания заранее купленных через интернет простых билетов, не багажных. Делать нечего, надо опять в кассу$\ldots$~терпеливо отстоял очередь в 3\sdash ю кассу. Выяснилось, что билеты на багаж оформляются только в~4\sdash й$\ldots$~где, так и раз эдак, я спрашиваю, это написано?! Нигде! Обычная, касса, как и остальные три! Никаких надписей, что только там можно оформить багаж. Прелести общения с системой. Та самая действительность, от которой я сбегаю на реку$\ldots$

Устроил у касс небольшой скандал чисто из~профилактических целей\mdash мне потрепали нервы\mdash получайте и вы, раз вам так лень повесить хоть какие\sdash то информационные листки или объявления, где было бы русским по белому написано как оформляется багаж. Причём мне уже были готовы оформить и~без очереди, и~прямо сейчас, и~даже не в той кассе, но Штирлиц прокололся, ляпнув, что у него поезд только в полдень$\ldots$ и~снова очередь, теперь уже в 4\sdash ую кассу. Передо мной женщина оформляла возврат какого\sdash то бешеного количества билетов\mdash штук эдак 30\mdash пришлось терпеливо ждать, благо времени вагон$\ldots$~а~если бы было впритык? Тогда бы снова скандал и~тэ~дэ и~тэ~пэ$\ldots$~как же в обыденной жизни раздражают такие моменты. Я~стоял в очереди, а~воображение рисовало мне реку$\ldots$~где нет этой окружающей действительности, где~нет правил, нет этих дурацких билетов и прочей требухи, где~не надо никому подчиняться, где можно всю цивилизацию послать на три буквы, где первобытная свобода ждёт$\ldots$~скоро, очень скоро всё это будет у~нас!
\newpage
Оплатив положенные рубли за провоз байдарки, пошёл к ребятам на улицу. Аня всё это время непрерывно курсировала от Лёни, сидящего снаружи у наших вещей на лавочке, ко мне, стоящему в очереди, подбадривая меня разной болтовнёй. Наконец, покончив с формальностями, мы вместе стали ожидать поезд и читать местную газетёнку, где писали, что первые две недели августа обещают быть жаркими в противовес не слишком\sdash то тёплому лету в этом году. Кутаясь в штормовку от ветра, очень хотелось верить в потепление. По вокзалу особо не бродили\mdash будет время потом, на выброске.
 
Поезд Архангельск\thinspace\nobreakdash---\thinspace Санкт\sdash Петербург номер 009С подошёл минут за 20\thinspace\nobreakdash--\thinspace 30 до отправления. Внимательно вслушивались поначалу в объявление по громкой связи, силясь понять, на какой путь прибывает наш поезд. Потом в~итоге определили это визуально, поскольку, кажется, диктор специально говорила так, чтобы никто ничего не~разобрал. Взяли вещи и пошли не спеша. Оказалось, что у нас последний вагон, да ещё последние места, красота! Проедемся в последнем вагоне. Проводницей оказалась молоденькая студентка какого\sdash то железнодорожного училища. Оказалось, их таких проводниц\sdash студенток на практике\mdash весь поезд почти. Малинник, короче$\ldots$~Округлив глаза на~нашу байдарку, она попыталась что\sdash то пролепетать, мол, не имеет права нас с таким грузом пропускать и вообще чего вы тут. Но не тут\sdash то было! Борзо и дерзко сунув под нос ей бумажку, то есть, пардон, не бумажку, а высочайшую квитанцию об оплате бешеной суммы в сто с чем\sdash то рублей на провоз байдарки, мы с победоносным видом стали грузиться в вагон. Велика Россия, а порядку в ней$\ldots$~Проводники и сами не знают своих правил, которые здравому смыслу вообще не поддаются. То ли денег хотят ещё сверху от наглости и жадности, то ли чего$\ldots$
\newpage
В любом случае, мы погрузились в поезд. Родимый плацкарт, да ещё последний вагон\mdash вообще шик, можно сказать! Ехать~нам до Заборья всего ничего. Четвёртое место в купейной части снова пустовало, что~не могло не радовать. Запихнуть байдарку полностью под~сиденье почему\sdash то не получилось\mdash другой поезд, немного другой размер подполочного пространства? Возможно. 

Поезд~не~фирменный, но все достаточно терпимо. Тем более что потерпеть\sdash то 3.5 часика. Нажимаешь ногой на педальку в уборной\mdash привет, рельсы! Здравствуй, XXI век\mdash век, где можно купить билет на такой поезд через интернет. Мы\sdash то,~собственно, привычные, но так ли должно быть в~XXI~веке?

Время ползло улиткой. В Бабаево остановка на полчаса. Большая станция, депо, поворотный круг. Паровоз~на~постаменте. Дети лазают по~паровозу, как~муравьишки. Мы~с~Лёней вышли, походили, погуляли. Сфотографировали паровоз. На~нём была табличка: <<Заводы им.\thinspace Готтвальда, Брно>>, то есть он оказался чешского производства. Скоро, совсем скоро будем в Заборье. Хочется быстрее туда попасть! Звоню родным, говорю, что у нас всё хорошо и пусть не волнуются. Проводница зовёт всех внутрь, отправляемся. До Заборья остаётся совсем капелька.

Пока едем, снимаю на фото и видео убегающие вдаль рельсы через стекло двери последнего вагона. Завораживающее своей бесконечностью зрелище! Насколько же велики и необжиты пространства нашей огромной страны. И как хорошо, что они медленно обживаются и ещё есть места, не испорченные человеческим присутствием. Сейчас мы стремительно несёмся навстречу именно таким закоулкам Ленинградской и Вологодской областей. В край верховых болот, сосновых лесов и петляющих среди них рек с~тёмной торфяной водой и золотыми песчаными берегами$\ldots$

Договариваемся с проводницей, не зная по наивности, что это вообще\sdash то её обязанность, чтобы она выставила красный флажок, если мы будем не успевать выгрузиться\mdash ведь в Заборье поезд стоит всего одну минуту. Та,~поломавшись для виду, соглашается, кокетливо выясняя о нашем сплаве. 

Народ, проходя мимо нашей байдарки к титану за~кипятком, воняя лапшой быстрого приготовления, косится на наши вещи и лица.  <<Ненормальные>>,\mdash читается в их глазах. Ах\sdash ха\sdash ха! Да, сейчас мы\mdash ненормальные. Мы~несёмся вперед, на просторы. Нас ждёт освобождение от~города, от~цивилизации, от~самих~себя.

И вот, Заборье! Заблаговременно вытащили с Лёней рюкзак с байдаркой в тамбур, сбоку приставили остальные вещи. Стоим в тамбуре, а Аня пока подтаскивает остальные нетяжелые сумки. Как успеть десантироваться за 1 минуту?! Загружались мы обыкновенно минут пять, а выгружались в Череповце из первого поезда ещё дольше, как мне показалось. Волнительно! Наконец~поезд замедляет ход и~останавливается. Проводница открывает дверь и~опускает подножку\mdash момент истины настал! Я мигом вылетаю на~низкую платформу, принимаю у Лёни рюкзак с~байдаркой на тележке, остальные тюки. Парень, который ехал на~боковушке рядом с~нами, помогает подавать вещи$\ldots$

15 секунд. Именно столько, как мне показалось, мы выгружались. Даже возможно меньше. Пока горит спичка, не~иначе. В \cite{Квадригин} также говорится об этом эффекте высадки. Как бы там ни было, все вещи действительно оказались на платформе в мгновенье~ока. Пока я доставал фотоаппарат, чтобы запечатлеть нас и окрестности, поезд снова начал набирать ход. Проводница укоризненно покачала нам на прощание головой, не понимая наших душевных порывов отказаться от всех благ цивилизации, и укатила дальше. Мы~остались одни. В~глуши. Посреди платформы в~неведомых краях на границе Вологодской и~Ленинградской областей. На~стыке двух миров\mdash обыденности и~ярких красок. <<Жребий брошен>>,\mdash мелькнуло у~меня в~голове. Это~было то, чего~я хотел ровно с окончания сплава по~Белой\mdash я~снова на~реке! С~этого момента пути назад нет, только~вперёд, только по~красавице~Лиди!

Платформа в Заборье низкая, асфальт обновляли наверно ещё при советской власти. Зато название станции на табличке сделано в новом серо\sdash красном стиле и на двух языках\mdash продублировано но английском: <<Zaborye>>. Лучше бы поезда обновили, но проще же таблички\mdash как всегда показушничество! 

Ехать в последнем вагоне было хорошо, но дальше идти к реке. Ничего, справимся, куда денемся. Я на мощнейшем эмоциональном подъёме! Сейчас я\mdash первопроходец, разведчик; надо найти подход к реке.  Взваливаю гермомешок с вещами на плечи, рюкзак с байдаркой везу на тележке впереди себя\mdash так мне показалось удобнее. Лёня помогает Ане надеть на спину рюкзачище, а потом и сам взваливает на спину свой тяжеленный 120\sdash литровый рюкзак и берёт две сумки с продуктами. Идём~вперёд, народу никого. Никто кроме нас не сошёл тут с поезда. Идём сначала вдоль станции, а потом по грунтовке метров триста. И~вот\mdash железнодорожный мост через Лидь, а рядом с ним отличное место для стапеля на низком левом берегу. 
\newpage
Сразу оцениваю реку. Мелкая. Дно каменистое, очень отдалённо напоминает берега Зилима, только все сплошь заросшие. Начинаем собирать байдарку. Рядом деревенская ребятня жжёт костер и купается в речке. Спрашиваю про клещей\mdash говорят редкость. Успокаиваемся, продолжаем сборы. Меня, тем не менее, какая\sdash то букашка больно кусает в бок, но вроде не клещ. Немедленно проводим химобработку верхней одежды$\ldots$

По железнодорожному мосту с оглушительным грохотом часто проходят грузовые составы. Вскоре наша байдарка почти собрана. Никак не хочет натягиваться фальшборт. Просто ну никак. Бьёмся с ним очень долго. Наконец, при помощи пассатижей и какой\sdash то матери, удаётся застегнуть последний замок среднего фальшборта. Всё\mdash теперь байдарка готова! Можно грузиться! 

Корабль наш уже спущен на воду, а мы с Лёней в~бахилах химзащиты, шлёпая по воде, таскаем и грузим вещи. Бригада моя распаковывает свои рюкзачищи, в них всё оказывается разложенным по гермомешочкам. Неплохо, учитывая то, что вещи надо распихивать под борта байдарки. Грузовой отсек занимает гермомешок с продуктами, палатки, ещё одна сумка с продуктами и всё остальное по мелочи. Места впритык. При этом моих личных вещей\mdash одна герма, на которой я сижу, да палатка. Мой девиз этого сплава\mdash <<Спарта!>>\mdash у меня с собой даже футболка одна$\ldots$~рюкзаки же ребят фактически некуда девать. Делать нечего, кладу их сверху в~грузовой отсек, накрываю полиэтиленовой плёнкой от дождя, перехватываю крест\sdash накрест резиновыми жгутами. Жить можно. 

Собирались мы неприлично долго. Я планировал встать на воду максимум в половину пятого. Какой~там$\ldots$~стартуем только в 6 вечера. Понимаю, что в любом случае должны пройти запланированные 7\thinspace\nobreakdash--\thinspace 8 километров, чтобы не выбиться из графика, иначе все мои прикидки по километражу собьются, да и заночевать лучше не в населённом пункте, понятно. Координаты стапеля\mdash \CoordsLidFifteenStapel. 

Отчаливаем, садимся, устраиваемся поудобнее. Готово! Я~отталкиваюсь от берега ногой и тоже занимаю капитанское место на корме. Сплав начинается! Время около 18:00. Стапель в Заборье пройден. Впереди нас ждёт свобода, ждут приключения, ждёт полный отрыв\mdash нас ждут 200 километров рек Лидь, Чагода, Чагодоща. 

Речка неширокая, местами очень мелко, течение у стапеля слабое, 1\thinspace\nobreakdash--\thinspace 2~км/ч. Включаю GPS\sdash навигатор, позаимствованный на работе. Оцениваю нашу скорость. Идём хорошо\mdash около 7\thinspace\nobreakdash--\thinspace 8~км/ч. Встречаются каменистые шиверы, но вроде бы идём без повреждений, хотя изредка и цепляем днищем на мелководье. Вода~под ногами\mdash это стекает с бахил химзащиты, когда залезаешь в байдарку, успокаиваю я себя. Мелкие участки заканчиваются вскоре после Заборья, и воды становится заметно больше\mdash хорошо! Я просто упиваюсь восторгом от происходящего\mdash мы на~реке! Мы выбрались из города, мы преодолели себя, свою неуверенность, свои страхи и теперь ничто не отделяет нас от~природы, которая примет нас, смоет с нас городскую грязь, обновит и придаст сил жить дальше$\ldots$

Управлять байдаркой\sdash трёшкой нам с Лёней не особо легко. Также, видимо при загрузке на стапеле я неравномерно распределил вещи и нас кренит влево, отчего мы постоянно поворачиваем. Это приходится исправлять усиленной греблей, потому как руль на байдарке\mdash вещь бесполезная. Аня~же счастливо фотографирует окрестности и привыкает к новой обстановке. 
\newpage
После двух часов сплава, около 20:00, мы~приближаемся к капитальнейшему завалу, позади которого виднеется вышка\sdash засидка. При подходе становится ясно, что справа есть широкий проход и там можно пройти  без проблем. Сразу за завалом резкий поворот влево и мигом вправо открывается вид на автомобильный мост через реку близ Гришкино\mdash ориентир на сегодняшний день, означающий, что километраж прошли. Что~ж, пора искать стоянку! Сразу за мостом, метрах в трёхстах, на левом берегу замечаю приличную полянку, около которой мы и~причаливаем. Дрова есть, место хорошее, хоть и близко к деревне. Скорая разведка показывает наличие проезда к стоянке. Но на сегодняшний день другой альтернативы, судя по карте и времени, у нас нет. Мы разгружаем байдарку и начинаем разбивать лагерь. Координаты первой ночёвки\mdash \CoordsLidFifteenGrishkino. 

Задача сейчас быстро поставить палатки, найти дров, развести огонь и приготовить ужин. А уже вечереет\mdash надо поторапливаться. Мудро взятая с собой по совету начальника двуручная пила творит чудеса, моментально распиливая даже толстые брёвна. Никакие ножовки не идут в~сравнение! Достаю~оргстекло и, используя его как растопку, развожу костёр. Ребята, тем временем, ставят палатку, располагаются. Недолго~совещаясь, на ужин готовим суп и~макароны с тушёнкой. Пока Лёня налаживает нехитрый быт, а Аня готовит ужин, я быстро ставлю свою палатку. Вскоре в результате совместных усилий у нас уже есть ужин, который немедленно и жадно поглощается. Устали. Вымотал~нас не переход, а стапель. Борт~байдарки, который никак не хотел натягиваться, да долгая погрузка. Плюс, конечно, сказывается ночь в поезде и дорога в целом. Но~рядом течёт река, а значит всё хорошо, значит ты\mdash там, где хотел оказаться. Ты\mdash на реке. 
\newpage
Первая стоянка\mdash всегда особенная. Это переход от~городской жизни к~походной. Да\sdash да, именно на~первой стоянке. Не на заброске, не на стапеле, не на первом переходе, а~именно на~первой стоянке. Первая стоянка, как своеобразный итог, результат заброски на маршрут, подводит черту под твоими переживаниями, тревогами и способна окончательно очистить тебя перед Священнодейством Сплава, способна смести в сторону надуманные барьеры, ненужные запреты повседневной городской жизни. Одним словом, после первой стоянки\mdash ты окончательно в \textit{сплаве}, в другой жизни, в другой вселенной, в другом состоянии души. Ты\mdash на реке. И река наша, Лидь, после мелкого участка в Заборье, радует полноводностью.

У костра развожу 96\sdash й в подкотельнике десантного котелка. Набиваю самодельную трубочку из корня винограда ирландским табачком, дымлю не торопясь. Красота! Лёня разливает$\ldots$~отдых начался! Закатное солнце необычайно красиво освещает верхушки сосен на нашем и противоположном берегу. Редкие машины проезжают по мосту неподалёку, нарушая идеальную, просто звенящую в ушах тишину. Птицы изредка подают голоса. Потрескивают сырые дрова в огне. Ветра нет, дым нашего костра поднимается почти вертикально\mdash хороший знак\mdash завтра должна быть хорошая погода. Сверху резко спускается звёздная августовская ночь. Мы лежим у~костра на ковриках и медитируем в оглушительной тишине тихвинских лесов$\ldots$~фантастика! И вроде просто\sdash то как всё\mdash взял сел на поезд и приехал сюда! Ещё куча всяких разнообразных мыслей лезет во взбудораженное сознание. Как здорово жить на Земле, нашей прекрасной планете, как хорошо, что есть такие удивительно красивые своей нетронутостью места, куда можно сбежать из города и никто тут тебе не указ!

Спать уходим около полуночи, покидав вещи в~палатки, а~снаряжение в тамбур палатки моих матросов. Повысил у костра Лёню из Матросов до~Штурмана, потому что не расстаётся с наручными часами, а Аню\mdash до~Впередсмотрящего, чтобы приободрить, так сказать, на~завтрашний день. Безусловно, я~был доволен началом похода. Мы преодолели заброску и стапель вполне успешно, не отстали от планового километража, заночевав, как я и~хотел, в районе Гришкино.

Сон подкрадывается незаметно, всё\sdash таки мы устали. Да и 96\sdash й делает своё дело, расслабляя мышцы, внезапно получившие такой натиск от весла. В первый день я мало фотографировал и не вёл записей, хотя и хотел. Впрочем, на~второй и третий день я тоже, к своему стыду, позабыл про судовой журнал\mdash было не до того$\ldots$

\begin{center}
	\psvectorian[scale=0.4]{88} % Красивый вензелёк :)
\end{center}