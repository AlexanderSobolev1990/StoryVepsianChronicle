%%%%%%%%%%%%%%%%%%%%%%%%%%%%%%%%%%%%%%%%%%%%%%%%%%%%%%%%%%%%%%%%%%%%%%%%%%%%%%%%
%
% \file       main.tex
% \brief      Главный файл настроек TeX проекта
% \date       24.08.22 - создан
% \author     Соболев А.А.
% \details    Для сборки в TexStudio - выбрать компилятор xelatex в настройках проекта
%             Для сборки при помощи CMake - mkdir build && cd build && cmake .. && make	
%
%%%%%%%%%%%%%%%%%%%%%%%%%%%%%%%%%%%%%%%%%%%%%%%%%%%%%%%%%%%%%%%%%%%%%%%%%%%%%%%%
\documentclass[headings=chapterprefix,a5paper]{book}
\usepackage{tikz}
\usepackage{tocloft}
\usepackage{pgfornament}
\usepackage{amsmath}
\usepackage{gensymb} % Для знака градуса
\usepackage{cite}
\usepackage{pstricks}
\usepackage{psvectorian}
\usepackage{fontspec}
\usepackage[russian]{babel}

\newcommand\MyVarAuthorName{А.А.~Соболев}
\newcommand\MyVarBookName{Вепсская летопись}

\setmainfont[%
ItalicFont=NewCM10-Italic.otf,%
BoldFont=NewCM10-Bold.otf,%
BoldItalicFont=NewCM10-BoldItalic.otf,%
SmallCapsFeatures={Numbers=OldStyle}]{NewCM10-Regular.otf}

\setsansfont[%
ItalicFont=NewCMSans10-Oblique.otf,%
BoldFont=NewCMSans10-Bold.otf,%
BoldItalicFont=NewCMSans10-BoldOblique.otf,%
SmallCapsFeatures={Numbers=OldStyle}]{NewCMSans10-Regular.otf}

\setmonofont[ItalicFont=NewCMMono10-Italic.otf,%
BoldFont=NewCMMono10-Bold.otf,%
BoldItalicFont=NewCMMono10-BoldOblique.otf,%
SmallCapsFeatures={Numbers=OldStyle}]{NewCMMono10-Regular.otf}

%--------------------------------------
% эпиграф
\usepackage{epigraph}
\setlength{\epigraphwidth}{0.65\textwidth}
\renewcommand{\textflush}{flushleft} \renewcommand{\sourceflush}{flushleft}
\let\originalepigraph\epigraph 
\renewcommand\epigraph[2]{\originalepigraph{\textit{#1}}{\scriptsize{#2}}} %\textsc
%------------------------------------------------------------------------------------------------------------
%настройки для А4
%\usepackage[12pt]{extsizes}
%\usepackage[left=2.5cm,right=2.5cm,top=2.5cm,bottom=2.5cm]{geometry}
%\linespread{1.15} % по умолчанию 1.0
%------------------------------------------------------------------------------------------------------------
%настройки для А5
\usepackage[11pt]{extsizes}
\usepackage[left=1.5cm,right=1.5cm,top=2.0cm,bottom=2.0cm]{geometry}
\linespread{1.0} % по умолчанию 1.0
%------------------------------------------------------------------------------------------------------------
%настройки для А4
%\newcommand{\corner}[1]{%
%	\begin{tikzpicture}[remember picture, overlay]
%	\node[anchor=north east, shift={(-2.5cm,-5.3cm)}] at (current page.north east){%
%		\pgfornament[width=2.2cm]{#1}};
%	\end{tikzpicture}%
%}
%------------------------------------------------------------------------------------------------------------
%настройки для А5
\newcommand{\corner}[1]{%
	\begin{tikzpicture}[remember picture, overlay]
	\node[anchor=north east, shift={(-1.5cm,-4.2cm)}] at (current page.north east){%
		\pgfornament[width=2.2cm]{#1}};
	\end{tikzpicture}%
}
%------------------------------------------------------------------------------------------------------------
\usetikzlibrary{celtic}
\newcommand{\uzor}{%	
\begin{center}
	\begin{tikzpicture}[%
	scale=0.17,
	transform shape,
	celtic path/.style={
		draw,
		black!10!red,
%		double distance=0.6mm,
%		line width=0.375mm
		double=white,%black!10!red, 
		double distance=0.6mm,
		line width=0.3mm
	}
	]
	\CelticDrawPath{
		size={64,4},
		max steps=64
	}
	\end{tikzpicture}	
\end{center}
}

\newcommand{\greenline}{
\begin{center}
	\begin{tikzpicture}
		\fill[fill=black!30!green,] (0,0) rectangle (10.8,0.2);
	\end{tikzpicture}
\end{center}
}
%---------------------------------------------------------------------
\usepackage{indentfirst} % Первая строка главы - с красной строки
\setlength{\parindent}{1.0cm} % Отступ слева первой абзаца
\setlength{\parskip}{0.25cm} % Отступ между абзацами

% Заголовки сверху и снизу каждой страницы (Headers & footers)
\usepackage{fancyhdr}
\pagestyle{fancyplain}

% Сделаем печать названия Части на левой странице
\newcommand*\parttitle{}
\let\origpart\part
\renewcommand*{\part}[2][]{%
	\ifx\\#1\\% optional argument not present?
	\origpart{#2}%
	\renewcommand*\parttitle{Часть \thepart.\ #2}%
	\else
	\origpart[#1]{#2}%
	\renewcommand*\parttitle{#1}%
	\fi
}

\fancyhead[LE]{\fancyplain{}{\bfseries \parttitle}}
\fancyhead[CE]{\fancyplain{}{}}
\fancyhead[RE]{\fancyplain{}{}}

\fancyhead[LO]{\fancyplain{}{}}
\fancyhead[CO]{\fancyplain{}{}}
\fancyhead[RO]{\fancyplain{}{\bfseries \rightmark}}

\fancyfoot[LE]{\fancyplain{}{\bfseries \thepage}}
\fancyfoot[CE]{\fancyplain{}{}}
\fancyfoot[RE]{\fancyplain{}{\bfseries\scriptsize \MyVarBookName}}

\fancyfoot[LO]{\fancyplain{}{\bfseries\scriptsize \MyVarAuthorName }}
\fancyfoot[CO]{\fancyplain{}{}}
\fancyfoot[RO]{\fancyplain{}{\bfseries \thepage}}

\renewcommand{\footrulewidth}{0.4pt}
%\renewcommand{\chaptermark}[1]{\markright{Часть \thepart.\ \chaptername\ \thechapter.\ #1}{}}
\renewcommand{\chaptermark}[1]{\markright{\chaptername\ \thechapter.\ #1}{}}

\newcommand\blankpage{% Пустая страница
	\null
	\thispagestyle{empty}%
	\newpage
}

\newcommand{\sdash}{\nobreakdash-}  % Дефис неразрывный без пробелов до и после
\newcommand{\ndash}{\nobreakdash~--~}  % Короткое тире неразрывное с пробелами до и после
\newcommand{\mdash}{\nobreakdash~---~} % Длинное тире неразрывное с пробелами до и после

%%%%%%%%%%%%%%%%%%%%%%%%%%%%%%%%%%%%%
\usepackage{tocloft,calc}
%\usepackage[newparttoc]{titlesec}
%\usepackage{titletoc}
%\usepackage{lipsum}
%\usepackage{tocbasic}
%
%\titleformat{\part}[display]
%{\centering\Huge\bfseries}{\partname~\thepart}
%{1em}{\normalfont\bfseries}
%
%\titlecontents{part}[3pc]{\addvspace{3pc}\filcenter}
%{\bfseries\partname~\thecontentslabel\\*[.2pc]\large}
%{\bfseries\large}{}[\addvspace{1pc}]
%
%%\renewcommand\cftchapdotsep{\cftdotsep}
%\titlecontents{chapter}[0em]{}
%{\bfseries\chaptername~\thecontentslabel\hspace{2em}}
%{\bfseries}
%{\mdseries\titlerule*[0.75em]{.}\bfseries\contentspage}

%\renewcommand{\cftpartfont}{\hfill\Large\bfseries\centering}
\usepackage{xpatch}
\makeatletter
\patchcmd{\l@part}{#1}{{\underline{Часть #1}}\vspace{1.5em}}{}{}
\makeatother
\addtocontents{toc}{\cftpagenumbersoff{part}}

%\renewcommand{\cftpartpresnum}{\Large Часть }
%\setlength{\cftbeforepartskip}{2.5em}
%\AtBeginDocument{\addtolength\cftpartnumwidth{\widthof{\bfseries Часть }}}

\renewcommand{\cftchappresnum}{ ~~Глава }
\setlength{\cftbeforechapskip}{0.7em}
%\renewcommand\cftchapafterpnum{\vskip 0em} % for spacing after each entry
\AtBeginDocument{\addtolength\cftchapnumwidth{\widthof{\bfseries ~~Глава~~ }}}

\renewcommand\cftchapdotsep{\cftdotsep} % Добавление точечек к элементу chapter
%%%%%%%%%%%%%%%%%%%%%%%%%%%%%%%%%%%%%

\begin{document}

\clubpenalty=10000  % Это костыль против
\widowpenalty=10000 % "висячих" строк

\righthyphenmin=200 % Избавляемся
{\sloppy			% от переносов слов
%\begin{titlepage}
%\author{\MyVarAuthorName}
%\title{\MyVarBookName}
%\date{2006\\Москва\\Самиздат}
%\maketitle
%\end{titlepage}
\begin{titlepage}
	\newpage
	\begin{center}
		\huge \textbf \MyVarAuthorName
	\end{center}	
	\vspace{2.8cm}	
	
	%\greenline
	\uzor
	
	\begin{center}
		\Huge \textbf {ВЕПССКАЯ\\ЛЕТОПИСЬ}
	\end{center}	
	\vspace{0.0cm}

%	\greenline
	\uzor

	\begin{center}
		\small {СБОРНИК ПОВЕСТЕЙ О СПЛАВАХ НА БАЙДАРКЕ\\ПО РЕКАМ ГОРЮН, ЛИДЬ, ЧАГОДА, ЧАГОДОЩА, ПЕСЬ}
	\end{center}

%	\greenline
	\uzor
		
	\vspace{\fill}	
	\begin{center}
		\normalsize
		Москва \linebreak 
		2022 \linebreak
		Самиздат
	\end{center}	
\end{titlepage}
\chapter*{}

В сборник вошли повести по байдарочным сплавам за период с 2015 по 2019 годы. Каждая повесть представляет собой рассказ отчёт по одному из водных походов, проходивших на реках Чагодощенского края и оставивших неизгладимые впечатления в душе автора. В каждой части переплетены воедино сведения о подготовке к сплаву, о маршруте и его прохождении, а также личные переживания автора и множество сопутствующих отвлечённых рассуждений вместе с элементами автобиографии.

\vspace{\fill}
\begin{flushright}
	\copyright~Соболев~А.А.~2022
\end{flushright}
 % Аннотация
%\blankpage
%\include{introduce}
\blankpage
\tableofcontents
\blankpage
%
% ЧАСТЬ 1 - ЛИДЬ 2015
%
\include{part1}
\include{part1_annotation} % Аннотация
\blankpage
\include{part1_introduce}
\blankpage
\include{part1_chapter1} 
\include{part1_chapter2} 
%\include{part1_chapter3} 
%\include{part1_chapter4}
%\include{part1_chapter5}
%\chapter{Катастрофа. 05.08.15}
\corner{64}

Проснулся от того, что замёрз. Холод шёл от земли к спине. Никак не ожидал такого развития событий\mdash мы сплавлялись с~женой на Южном Урале в августе и отнюдь не мёрзли, хотя спали почти всегда на каменистых берегах рек, да и на земле тоже. И~спальник был тот же, и туристический коврик такой же. Правда,~там мы находились южнее градусов на 5 по широте. Неприятно, но что делать. Надо будет что\sdash то придумать, мёрзнуть так не годится. 

Встал, размялся, вышел по дороге в поле\mdash жидкий рассветный туман, солнышко только\sdash только встало. Красиво\ldots~ради таких моментов стоит жить. Стою в каком\sdash то оцепенении, встречаю рассветное солнце. Сосновые леса кругом, воздух чистейший, в дымке встаёт оранжевое солнце\ldots~картина заворожит любого. Я стою и взгляд мой спокоен, в голове наступает странный покой и порядок, мысли упорядочиваются и угасают\mdash ничто не тревожит, ничто не отвлекает на ненужное, лишнее. Тут~вообще ничего нет лишнего\mdash только человек и природа. Так~и~стою\ldots~долго стою, пропитываясь белым рассветным туманом.

Но довольно медитации, пора в лагерь! Натаскал дровишек, развёл огонь и стал готовить кашу на завтрак. Бригаду, естественно, тоже поднял ни свет ни заря\mdash полез за гермомешком с провизией, который лежал у них в палатке. Ничего, потом отоспимся\mdash надо на воду, быстрее! Жажда приключений! Нас~ждёт красавица Лидь! 

Паковались опять очень долго. Выясняются неприятные подробности, что мой экипаж тащит с собой бытовой фильтр\sdash кувшин для воды и абсолютно всю воду, которую мы употребляем, желает пропускать ещё и через фильтр, прежде чем кипятить. Более того, зубы, оказывается, моя бригада тоже чистит исключительно фильтрованной водой. Меня пробирает на крепкие выражения. Такого бреда я вообще не слышал. Чтобы~чистейшую воду, не прошедшую ещё ни одного крупного города или завода, воду, в которой плещется вечерами рыба, и лютуют выдры\mdash показатель чистоты реки\mdash и ещё её пропускать через фильтр?! Это~долго и, более того, не нужно\mdash кипячения более чем достаточно. Готовка еды и чая из\sdash за этого фильтрования сильно затягивается, что нервирует меня нешуточно. Коричневатый~оттенок воды\mdash это от торфа и песка, ничего страшного! Как не хватает в такие моменты нашего 4\sdash го человека, который не смог с нами пойти. Он\sdash то уж точно не позволил такой ереси сбыться. 

Потом показываются из рюкзака резиновые сапоги нашей Ани\ldots~для кого брались лёгкие и компактные бахилы химзащиты, выполняющие роль сапог? Зачем тащить эту резиновую тяжесть, занимающую к тому же много места? Таких непоняток встретится мне немало. Это\mdash плата за несхоженность и неопытность экипажа. Я оказался на реке во главе людей, которых видел фактически 4\sdash й раз к моменту стапеля. И хотя я целый вечер тогда ещё в Москве у себя на кухне распинался им про все тонкости и особенности водного похода, сплава, а потом ещё и написал инструкцию что брать с собой, всё равно вышло так, как вышло. Возможно, если бы у них был опыт турпоходов, было бы легче, но увы. Люди~оказались морально не готовы к таким, казалось бы, очень простым вещам, как вода из реки. А ещё и вещей понабрали\mdash мама дорогая! Надо было заранее встретиться перед сплавом и повыкидывать ненужного\ldots~вспоминаю при этом свой первый сплав сразу же. Не скажу, что я тащил много ненужных вещей, но они точно были. Не припомню досконально что именно, но должен признать, что тащил сандалии, хотя обошёлся в итоге и без них, тащил зачем\sdash то второй свитер, несколько футболок\ldots~наверняка ещё что\sdash то. Естественно, во второй сплав я брал минимум вещей, умудрённый прошлым опытом. Но в моём первом сплаве было одно большое НО. Тогда я мог брать с собой хоть бензопилу и слона на верёвочке\mdash катамаран обладает безумной грузоподъемностью\mdash там вещи лежат на настиле между баллонами и никому не мешают. Иное дело байдарка. Разве самому приятно сидеть всему обложенному тюками, притом что в «Таймени» вообще\sdash то прилично места? 

Ладно, как бы там ни было, на воду мы встали почти вовремя и бойко полопатили вёслами дальше. Я старался не показывать излишнего раздражения из\sdash за кошмарного количества вещей и фильтрования воды\mdash авось проблема рассосётся\mdash в конце концов, еды, а значит и объёма груза, будет становиться с каждым днём всё меньше и меньше; с фильтром решил так\mdash в свой наряд на камбуз я фильтром пользоваться не буду. На этом решил проблему закрыть, чтобы не портить ни себе, ни бригаде впечатлений от отдыха.

Экипаж же мой был в приподнятом настроении после Адмиральской Каши, и мы неслись навстречу приключениям, которые не заставили себя долго ждать. Но не этого ли мы так жаждали, не к этому ли так стремились? Нам~то и дело встречались завалы, но благо река за конец весны и лето уже расчистила себя путь, и обноситься нам не пришлось. Лишь~временами острые ветки и сучки на брёвнах, торчащих из воды, доставляли нам неудобства, да и то чисто психологические. Мне нравился характер реки и то, что мы один на один со стихией. Одни в беспросветной глуши на стыке двух областей, двух миров\mdash обыденного и параллельного, где ты свободен от всего\ldots~

Так, скользя по водной глади и увиливая от валунов, мы миновали ЛЭП близ Забелья. Затем впереди показалось раздвоение русла, и мы пошли в правой протоке\mdash там мне показалось глубже. Через несколько сот метров сужение русла и\ldots~зашумел водопадик! Времени отвлекаться на навигатор не было, да и у меня не осталось сомнений\mdash я вспомнил карту\mdash это остатки плотины бывшей местной ГЭС! Страх и ужас, обуявшие меня поначалу, быстро сменились задором и азартом пройти это препятствие. Причаливать~для осмотра не стали\mdash я выровнял курс байдарки и направил её аккурат на середину слива. Будь что будет, решил я, штурмуем сходу! Секунда и\ldots~мы преодолели этот перепад! У\sdash у\sdash ух! Быстро заложили крутой поворот влево и затем вправо, следуя изгибу реки и стараясь держаться середины русла. Прошли мимо неплохой стоянки на правом берегу. Но становиться на ночлег нам ещё очень рано\mdash в планах моих добраться сегодня до Тургоши и даже немного пройти ниже. В этот день прошли устья меленьких речушек\mdash Белой и Межницы\mdash все на правом берегу. Перекатики небольшие встречались нам по пути постоянно и днищем мы цепляли очень часто\ldots~река начала стремительно ускоряться и приближаться к железнодорожной ветке Подборовье\nobreakdash--Кабожа.

А потом внезапно начался АД. Череда шивер и перекатов с белой бурлящей пеной и огромными валунами в русле. Как~говорится, вот это поворот! Как хорошо, что шкура пролеена\mdash вся надежда на прочные леи, идущие вдоль стрингеров и под кильсоном. Хоть бы всё обошлось! Я ни разу не проходил серьёзных перекатов и шивер в своём сплавном опыте. Единственное правило, которое я знал, да которое и так понятно\mdash надо идти по тёмной воде, поскольку там глубже. В~первом нашем сплаве с женой была пара перекатов, но они легко проходились даже на катамаране. Во~втором сплаве я вообще таких приключений не припомню, а на Киржаче, где мы провели сплав выходного дня, там и вовсе только обнос плотины, да один лёгкий слив. Всё. Тут же начался просто кошмар. Временами мы налетали на камни, потому что Капитану с кормы трёшки достаточно плохо видно, что творится впереди. Плюс~в~первые дни сплава я ещё не догадался подкладывать гермомешок с вещами под себя на манер сиденья и сидел низко, на самом дне байдарки. Соответственно, спины экипажа сильнее закрывали мне обзор. Один раз на перекате байдарку поставило поперёк реки и прижало к валуну, а вода уже вот\sdash вот была готова перелиться через край фальшборта и тогда случилось бы страшное\ldots~быстро перенеся вес на другой борт и резко скомандовав то же самое сделать и экипажу, спасли положение, убрав губительный крен. Вылезли из байдарки, удерживая её от крена, провели на чалке, миновав опасное место, и пошли дальше. Зачерпнули с Лёней на перекате в химзащиту воды, когда оттаскивали байдарку от валуна. Пришлось снимать химзу, выливать, снова надевать\ldots~и вода на перекате не слишком тёплая, а прямо скажем холодная. 

Когда садишься в этих бахилах ОЗК в байдарку, как ни отряхивай, вода с них всё равно нальётся. Я полагал, что собранная на дне байдарки вода\mdash с бахил, но позже, анализируя этот день и эпизод на этом перекате, я пришел к выводу, что уже тут мы получили парочку фильтрующих пробоин. Через такую пробоину вода не хлещет фонтаном, а потихонечку сочится. В принципе не страшно, главное периодически отчерпывать её кружечкой.

О кружечке отдельная история. По совету начальника я прикупил простую белую эмалированную железную кружку. Любовно оплёл её ручку паракордом\mdash получилась красота. В~сплаве эта кружка\mdash то, из чего пить и чай, и что покрепче, и чем отчёрпывать воду со дна байдарки. Так вот, кружка предусмотрительно была положена в карман штормовки и я всегда был готов ей воспользоваться.

Бодро идём вперёд и\ldots~правильно! Снова череда перекатов. Снова какой\sdash то ад. Я уже понимаю, что воды в этом году мало и к августу Лидь совершенно неприлично обмелела местами. Неудачно входим в створ переката, если так можно выразиться, и нас начинает мотать из стороны в сторону. Налетаем на камень, байдарка кренится\ldots~камни трутся о шкуру, слышен металлический «Чпок!». У меня сердце уходит в пятки от этого самого «Чпока». «Надо было обноситься по берегу!»,\mdash промелькивает в голове. Думаю, погнулись кости где\sdash то\ldots~как выяснится позже\mdash лопнул второй шпангоут! С кормы мне казалось, что там глубже, а коварные камни почти сразу под водной гладью. Наверняка поэтому некоторые авторы туристической литературы советуют Капитану сидеть в байдарке спереди и «читать» воду\ldots~правда при этом Рулевой на корме тоже должен быть опытным, что, понятно, не наш вариант.

Ещё налетаем на валун\mdash совершенно озверевший поток носит нас по острым камням, слышен противный шелест по днищу и тут я замечаю, что воды под ногами прибавляется. Я~уже на взводе\mdash чертовски неумело проходим перекаты. Сплошной~шкуродёр! Управляется трёшка из ряда вон плохо, о чем я читал, но не верил до конца\mdash ведь «Таймень»\sdash двушка выправляется на раз\sdash два! С~трёшкой всё оказалось иначе. Мы с Лёней дико лосячим веслами, пытаясь исправить положение, но нет\mdash снова очередной перекат берёт своё, и мы получаем ощутимые толчки в днище, да такие, что сердце замирает. Почему мы не обнеслись по берегу? Вероятно, всё случилось молниеносно, да и картина на воде при подходе, в силу неопытности, не вызывала опасений.

Опыта прохождения таких препятствий, тем более на байдарке, у меня нет. Об экипаже моём и говорить не проходится\mdash они вообще в первый раз на реке. В голове промелькивают обрывки разговора со школьными товарищами, которые отдали мне кости\ldots~что\sdash то там про «отрицаловку», ага! Не сразу соображаем гасить скорость перед перекатами, а то и вообще проходить их с отрицательной скоростью относительно воды\mdash «на отрицаловке», но и это иногда не помогает, если требуется стремительный S\sdash образный манёвр\mdash чрезвычайно трудно быстро развернуть трёшку. Почему, никак не понимаю\mdash на двушке с женой всё было прекрасно. Мы легко выправляли байдарку на S\sdash образное прохождение препятствий вдвоём. Уже сейчас, после сплава, я понимаю, что Лёню, скорее всего, следовало бы посадить в байдарке на нос, а Аню посередине, а не наоборот. И тогда бы основная гребущая сила\mdash два мужика\mdash оказалась бы на носу и корме. Возможно, с таким распределением сил было бы легче выправлять байдарку. Но, как говорится, наш человек задним умом крепок. В~результате имели то, что имели. Это и называется накоплением опыта, на своих же, правда, ошибках.

Так вот, пройдя очередную шиверу, прилично чирканув днищем и получив снова несколько толчков под кильсон в районе грузового отсека, я обнаруживаю, что воды под ногами прибавляется уж как\sdash то больно стремительно. Не гребём временно, наблюдаю за водой под ногами\ldots~прибывает! ПРОБИЛИСЬ!!! Причём, видимо, очень капитально! Сотрясают воздух крепкие выражения. Достаю кружку, начинаю отчёрпываться. Вода начинает поступать с нешуточной скоростью, отчерпаться не получается! Пробоины не вижу, так бы хоть пальцем заткнул что~ли. Судорожно соображаю что делать. Впереди удачно замечаю песчаную косу за островком\mdash немедленно командую бригаде править туда. Сам продолжаю отчёрпываться кружечкой, чтобы хотя бы кильсон под ногами показался из воды. Добившись этого, бросаю отчерпываться и тоже активнейшим образом включаюсь в греблю. Хорошо, что заветная кружка была близко в кармане.

Влетаем на скорости носом и левым бортом на песчаную косу, вылезаем. Закатав рукава штормовки, обследую днище байдарки прямо на воде, не разгружаясь, чтобы оценить масштаб бедствия. Одна пробоина, ещё дырочка\ldots~вот ещё какая\sdash то шероховатость\ldots~ну, думаю, заткнём их чем\sdash нибудь и пойдём дальше, а в голове крутится мысль\mdash откуда тогда столько воды? Веду рукой дальше по днищу и тут\ldots~аккурат три пальца входят в пробоину! Шок! Это просто КАТАСТРОФА, подумалось мне, правда, в более крепких выражениях. Принимаю решение разгружаться\ldots~

Через минут 10, когда все вещи выгружены на песочек, переворачиваем с Лёней байдарку и\ldots~я замираю в ужасе. Леи~целы, но продраны капитально! Если бы их не было\mdash уже бы потопли к чертям. Далее сходу насчитываю минимум~6\sdash 7~дырок! Детальный осмотр выявляет 10 пробоин\ldots~ДЕСЯТЬ! Настроение подавленное. Надо клеиться. Аня, похоже, ещё не до конца поняла трагизм момента и счастливо фотографируется на фоне окружающей природы\ldots~

Мы в абсолютнейшей глуши\mdash одна из глухих частей маршрута. Стоим на песчаной косе маленького островка, рядом лежит пробитая байдарка и куча наших вещей. Мысли роем носятся в моей голове, пытаясь сложить картину\mdash как же так получилось?! Могли или не могли мы обнестись? Нет, не могли\mdash берега слишком заросшие. Могли лишь погасить скорость, но отчего\sdash то промедлили. И теперь мы имеем что имеем. Но разве не этого я хотел? Хотел~полнейшего отчуждения от цивилизации? На, получай! Хотел один на один со стихией? Получи, распишись! Ах\sdash ха\sdash ха! Но я не унываю. Наоборот, очнувшись и воспрянув духом, начинаю командовать\mdash надо ремонтироваться. 

Даю разнарядку Лёне добыть дров, чтобы развести костер, а сам тем временем мою днище от песочка\mdash пусть пока просушится\mdash и иду доставать заплатки с клеем. Хорошо, что мужик, у которого я брал шкуру байдарки, презентовал запасной пузырёк клея! Вспоминаю, что изначально я вообще отчего\sdash то не хотел брать с собой клей, и у меня по спине проходит ледяной холодок\ldots~даже не представляю, как бы мы выбирались из этой глуши, не окажись у нас нормального ремкомплекта. Я был в полнейшей уверенности, что пробоин эта река нам не сулит. Кроме~того, за три своих предыдущих сплава я ни разу не пробивался, что и сыграло злую шутку\mdash я был уверен, что и на этот раз всё будет хорошо.
 
Лёня, тем временем, приносит дров, для чего ему приходится вброд пересекать протоку между нашим островком и, собственно, берегом реки. Разводим кое\sdash как на песчаной косе костерок, греем воду в пехотном котелке. Нам нужен кипяток, чтобы прогреть место заплатки. Технология ремонта ПВХ шкуры проста\mdash наносится клей на шкуру, прикладывается заплатка, отводится заплатка, выжидается пара минут, далее заплатка снова прикладывается, плотно прижимается и сверху ставится та самая знаменитая железная кружечка, в которую налит кипяток. Этой кружечкой, как утюжком, проглаживается место заплатки для ускорения полимеризации клея. Большинство пробоин небольшие, заклеиваем их достаточно быстро. Долго ждать кипятка\mdash сильный ветер задувает маленький костерок, плюс сначала зачем\sdash то греем полный котелок воды, и лишь потом сообразим греть на донышке\mdash на~1\sdash 2~кружечки\mdash так быстрее закипает\ldots~

Опасения вызывают две большие пробоины\mdash как раз те, в которые входят по 3 пальца. Виноват я в них, конечно, более всех. Пробоины эти от мест, где к шкуре прилегали колёса тележки для транспортировки упакованной байдарки\mdash я положил эту тележку на самое дно грузового отсека, не учтя того, что сверху лежит гораздо больше вещей, чем когда мы с женой ходили по Киржачу. Тогда я тоже положил тележку в грузовой отсек. Ладно,~теперь учтём! Тележку привязываю резиновыми жгутами на корму байдарки прямо сверху. 

Заклеили все прорехи, выждали время. Аня вовремя подсуетилась с бутербродами\mdash перекусили, передохнули, успокоились. У Ани был термос с заваренным шиповником, и он пришёлся как нельзя кстати. Потом спустили байдарку на воду для проверки. Оказалось, что одна заплатка\mdash как раз от колеса тележки\mdash фильтрует! Решаю на эти две большие пробоины поставить заплатки ещё и изнутри шкуры, а также промазать края заплаток клеем. Сказано\mdash сделано и ещё через полчаса всё готово. Проверка на воде\mdash фильтрации нет, можно идти. В общей сложности потеряли на заклейку 4 часа. Да\sdash а\sdash а, выбиваемся из графика! 

В этот день больше серьёзных перекатов не встретилось, дошли до разрушенного деревянного моста близ заброшенного ж/д остановочного пункта Тургошь. Место на правом берегу мне сразу приглянулось, и мы встали на ночёвку, тем более что времени уже было к восьми вечера. Координаты стоянки\mdash N~59\degree~19.660\textprime~ E~35\degree~10.407\textprime. Позже, в Москве, по треку навигатора я точно посмотрел, что, несмотря на ремонтную остановку в 4 часа, мы прошли за этот ходовой день 24 километра. Очень~солидно! Но~я~планировал больше, и теперь все мои прикидки по километражу и прохождению маршрута сбились.
 
Место на правом берегу, выбранное нами под лагерь, было несколько обжитым и замусоренным. Разбили лагерь. Разведением~костра занялся я, Лёня\mdash палаткой, а Аня ужином\mdash будут опять макароны с тушёнкой\mdash нет сил соображать что\sdash то поинтереснее. Тем временем подчистили стоянку, сожгли мусор в костре. Быстро стемнело. Я развёл 96\sdash й, и мы сели вкушать приготовленные яства уже в глубоких сумерках.

Около Тургоши располагается пионерлагерь, и музыка оттуда не стихала до 11 вечера, дискотека. Заманчиво было бы сходить, но далековато, да и устали сильно. Сидим у костра, я прокручиваю в голове пережитое. Вот это выдался денёк! 10 пробоин\mdash не шутки! Для меня\sdash то это вновь, а ребята вообще первый раз на реке\mdash вот, думаю, выпало им на первый сплав испытаний. Но мы всё преодолели, сохранили боевой настрой и даже почти наверстали километраж благодаря сильному течению. Горько за пробитую шкуру, конечно. Пусть это будет нашей жертвой богам леса и реки.

Август\mdash пора метеоров. Жадно вглядываемся в безоблачное ночное небо, силясь увидеть хоть один. Красота природы безгранична, думается мне. А человек\mdash не царь её, но составная часть. Ещё точнее, он\mdash варвар, несущий разрушения во имя выживания, а больше вопреки, особенно в последнее время. Так,~лёжа на коврике у костра под звёздным небом, размышляя о далёких галактиках и попыхивая трубочкой, мне становится необычайно хорошо от осознания происходящего с нами. Я~радуюсь тому, что прошли плотину сегодня, что не потопили байдарку на перекате, что не утонули, получив пробоину; просто упиваюсь восторгом от нашего похода\mdash наслаждаюсь природой, потрескиванием дров, ароматом костерка\ldots~и ещё более жду неизвестности, которая поджидает нас на маршруте.

Увидели\sdash таки с Лёней один метеор в полночном небе. Или~это следствие 96\sdash го? Всё может быть. Вокруг нас стоит тихая звёздная ночь, чуть слышно плещется Лидь у разрушенного моста, а мы сидим у костра, задрав головы к бриллиантовому небу. Красота! Звёзды яркие\sdash яркие. Где\sdash то там, далеко над нами, скорее всего, есть иные цивилизации\mdash не может ведь быть так, что мы одиноки во Вселенной\ldots~они даже могут слать нам какие\sdash то свои сигналы или позывные, но\ldots~Земля~ещё не готова к этому. Мы~не живём в гармонии\mdash ни с природой, ни друг с другом в планетарных масштабах\ldots~и всё в нашем посткапиталистическом мире подчинено одному\mdash деньгам. Какие~уж тут иные цивилизации и всё вот это вот\mdash в своей бы разобраться для начала. Чтобы не хотелось от неё убегать в неведомые дали в слепом желании забыться\ldots~Так называемая Эра Разобщённого Мира (по~И.Ефремову), чтоб её.

Усугубив, спать завалились глубоко за полночь, и я моментально провалился в глубокий и крепкий молодецкий сон, опять напрочь забыв про судовой журнал.

\begin{center}
	\psvectorian[scale=0.4]{88} % Красивый вензелёк :)
\end{center}
%\include{part1_chapter7}
%\include{part1_chapter8}
%\include{part1_chapter9}
%\include{part1_chapter10}
%\include{part1_chapter11}
%\include{part1_chapter12}
%\include{part1_chapter13}
%\include{part1_chapter14}
%\include{part1_chapter15}
%
% ЧАСТЬ 2 - ГОРЮН 2016
%
%\include{part2}
%\include{part2_annotation} % Аннотация
%\blankpage
%\include{part2_introduce}
%\blankpage
%\include{part2_chapter1} 
%\include{part2_chapter2} 
%\include{part2_chapter3} 
%\include{part2_chapter4} 
%\include{part2_chapter5} 
%\include{part2_chapter6} 
%\include{part2_chapter7} 
%\include{part2_chapter8} 
%\include{part2_chapter9}
%\chapter{Поиски днёвки. 06.08.16} 
\corner{64}

Ночка была не самая тёплая, но, тем не менее, я не замёрз. Новый матрас надёжно изолировал от прохладной земли\mdash всё\sdash таки уже август, хоть и намного теплее, чем в прошлом году, когда я шёл без матраса. Я выполз из палатки не слишком рано, народ тоже потихонечку начал раскачиваться. У меня отчего\sdash то возникло лёгкое ощущение неотвратимо приближающегося конца похода, хотя на самом деле, только\sdash только минула его половина. Видимо, это потому что в прошлом году я антистапелился в Загривье, мимо которого мы должны были сегодня проплыть. 
 
Итак, предстояло найти уникальное место для днёвки. Чтобы и песчаная коса была для купания, и лес красивый сосновый, и ровная полянка для палаток, и чтобы без травы высокой, и чтобы столик был желательно и чтобы и чтобы и чтобы\ldots  

После завтрака мы неторопливо и аккуратно спустили байдарки на воду с высокого обрыва. Потом привычно погрузились и отчалили. В скорректированном плане на сегодня мы должны были пройти Загривье\mdash место моего прошлогоднего антистапеля\mdash и идти дальше, но не далее Бывальцево, поскольку после него характер берегов Чагодощи снова поменяется\mdash песчаные пляжи пропадут. Погода радовала\mdash небо снова расчистилось, ярко светило солнышко. Белые кучевые облака медленно бороздили высокую августовскую синь. Жарко. Я снова сидел в байдарке, лишь прикрыв плечи шемагом, а голову панамой, постоянно смачивая её водой. 

Около деревни Загривье мы остановились на широченном песчаном пляже\mdash там же, откуда я в прошлом году пошёл на разведку выброски. Тут, судя по старой карте ГШ, должен был быть паром. Остатков парома я также не нашёл, как ни искал. Решили помыться в реке и передохнуть. Течение в этом месте было довольно приличным, так что купались аккуратно. Все намылились и с громким <<плюх>>, оставляя на поверхности воды мыльную пену, заныривали на глубину промыть свои тела. Прохладная водичка приятно бодрила! Покончив с водными процедурами, немного перекусили. На берег приехали местные на машинах. Я стал расспрашивать одного дедка о порогах перед Лентьево. Как оказалось, он слышал о них, но никогда сам лично там не бывал\ldots  что ж, посмотрим скоро сами, что это за пороги такие. 

Отчалив от пляжа, прошли протоками около Загривья и повернули вслед за руслом направо, оставляя слева высоченный песчаный обрыв с беседкой наверху. Фотографии этого места вышли просто умопомрачительными. Трепет охватил меня\mdash дальше снова первопроход\mdash если маршрут до этого места был мне знаком, то дальше мы снова идём в неизвестность! Ура\sdash а\sdash а!

Мы с Саней замешкались на песчаной косе, делая фотографии, и сильно отстали от бригады. Вышли на широченный плёс, увидели нашу флотилию далеко впереди, лопатящую веслами. Налетел встречный ветер, стало тяжело идти. Расстояние между нами и авангардом флотилии стремительно увеличивалось. Надо было исправить ситуацию. Зарядил сигнал охотника зелёной ракетой и пальнул вперёд под углом градусов 45, чтобы бригада услышала выстрел, заметила ракету и подождала нас. Они, к счастью, догадались и подождали. Дальше пошли более менее стройной кильватерной колонной.

Песчаные пляжи и обрывы становились всё краше и краше, но, увы, они скоро должны были закончиться близ деревни Бывальцево, о чём я знал из отчётов и спутниковых снимков. Так что место на днёвку надо было в любом случае найти на этом промежутке до Бывальцево. Плыли медленно и даже, я бы сказал, лениво. Два раза причаливали оценить место\mdash нам всё казалось не очень. То не было пляжика, то место для лагеря слишком далеко от воды, то ещё что\sdash нибудь. Привередничали.

И вот, пройдя примерно половину пути от Загривья до Бывальцево, мне показалось, что на левом берегу сразу за прибрежными кустами должна быть хорошая поляна без травы. Мы высадились на песчаный берег и взобрались по невысокому подъему наверх. Стоянка была отличная! Поляна ровная, стол, скамейки, куча места для палаток, кое\sdash какие дровишки валялись, мусора не было. И огромная песчаная коса есть, как и хотели. Решили, что, возможно, это лучшая стоянка из найденных нами в этом сплаве. Координаты N~59\degree~09.8869\textprime~E~36\degree~16.8261\textprime. На правом берегу чуть поодаль в глубине леса находилось урочище Лукобор, как оно было обозначено на карте ГШ. Что там раньше находилось, мне найти не удалось. Возможно какая\sdash то деревня, потому что место тут и вправду очень красивое\mdash река делает затяжной S\sdash образный поворот и имеются отличные песчаные косы.  Ясное дело, мы остались на днёвку.

Делали всё лениво, потому что времени было вагон\mdash на стоянку пришли часов в 14 или 15. Как обычно, перетаскали вещи, байдарки затащили и перевернули вверх днищем, развернули лагерь. Над столом натянули тент, памятуя о часто настигавших нас дождях в последние дни. Костёр сделали чуть в стороне, чтобы не подпалить тент. Место под палатки расчищали от безумного количества шишек\mdash видимо тут кормятся б\'{е}лки, но самих пушистых непосед мы не видели. 

Прошел лёгкий дождичек и появилась радуга. Природа радовала нас, одним словом. Паша и С.Ю., тем временем, отправились рыбачить ниже по течению. Дима и Саня обсуждали рок\sdash группы и музыку вообще, а Ваня счастливо участвовал в ничегонеделании. Я тоже отдыхал, слоняясь по лагерю с фотоаппаратом. К середине похода мышцы уже окончательно перестали ныть, привыкнув к физнагрузке, а кожа на руках огрубела от весла, потому что я грёб по привычке без перчаток. 

С днёвкой всё сложилось как надо\mdash отличнейшее место в полнейшей глуши\ldots  связи и той нет. Я включил телефон и попытался отправить жене смску, но сообщение не проходило. Встал на пенёк, связь лучше не стала. В итоге, походив по берегу и поискав связь, я бросил это бесперспективное дело\mdash до ближайшего населенного пункта\mdash Бывальцево\mdash  километров 8\sdash 10 по прямой, да и не факт, что там есть вышка сотовой связи.

Вечером заготовили дров\mdash притащили ещё брёвен и распилили их. Дима вызвался приготовить ужин, а остальные стали ему помогать. Наш шеф\sdash повар порадовал нас кус\sdash кусом с тушёнкой, салатом с тунцом(!) и традиционным супом. Вот это да! Настоящее походное пиршество! Приготовив ужин, сидели застольничали допоздна, благодарили нашего шеф\sdash повара. Стемнело и в ход даже уже пошли фонари. Саня наигрывал на губной гармошке. Я блаженно сидел на стульчике, прислонившись к дереву, и пускал вверх колечки дыма из трубки. Пожалуй, в этот вечер я, возможно впервые за этот сплав, по\sdash настоящему прочувствовал атмосферу уединения и отречённости от цивилизации. Эти сладкие минуты! Дорожу ими. Собственно, ради этих мгновений и затевается всё вот это вот. Они проносят, как будто тайком, нелегально, в душу спокойствие, удовлетворение, сознание дикости, первобытности, естества. Именно эти минуты очищают сознание и ради них мы идём в походы, сплавы\ldots 

Спать расползлись очень поздно\mdash всё равно завтра выспимся! Ведь завтра долгожданная днёвка! Каждый из команды наверно уже много раз тихонько проклинал меня и С.Ю., как командиров, что мы никак не остаёмся на днёвку. Но от длительного ожидания место, выбранное сейчас нами, становилось только лучше в сознании притомленного туриста\sdash водника.

\begin{center}
	\psvectorian[scale=0.4]{88} % Красивый вензелёк :)
\end{center}

%\include{part2_chapter11}
%\include{part2_chapter12}
%\include{part2_chapter13}
%\include{part2_chapter14}
%
% ЧАСТЬ 3 - ЛИДЬ 2017
%
%\part{Лидь 2017 или Возвращение}
\setcounter{chapter}{0}

%\include{part3_annotation} % Аннотация
%\blankpage
%\chapter*{}
\begin{flushright}
\vspace{2.0cm}
\textit{ \large {
Отчуждению от привычной жизни посвящается.\\
\vspace{\fill}
Все имена, названия, события,\\
время и место действия вымышлены,\\
а совпадения случайны\\
(т.е. автор снимает с себя любую ответственность\\ 
за личное восприятие повести читателем).
}}
\end{flushright}



%\blankpage
%\include{part3_chapter1}
%\include{part3_chapter2}
%\include{part3_chapter3}
%\chapter{Радогощь, стапель и Лидское озеро} 
\corner{64}

«Буханка» медленно ползла по трассе. 80 км/ч для неё был просто предел. Стрелка на спидометре прыгала с разбросом ±20 км/ч из-за дикой тряски. Всё-таки предназначение этого пепелаца – далеко не шоссе… якобы его разрабатывали как катафалк для ядерной войны. Так ли это, точно не знаю, но суровость и простота – убийственные. Вдобавок, почему-то не выключалась печка. Из воздуховода дуло горяченным воздухом прямо мне в ноги и закрыть или как-то отключить это было нельзя. Водитель тоже ехал и, видимо, терпел, ведя машину в сапогах. Я же максимально вжался в дверь, отстранив ноги от патрубка, откуда дул горячий воздух, и мысленно завидовал бригаде, разместившейся в салоне. Открыл стекло двери ручкой, устроился поудобнее и начал писать, преодолевая тряску, путевые заметки, чтобы отвлечься. 
До Ефимовского добрались нормально, а дальше поехали по заметно ухудшившийся, но пока ещё асфальтовой дороге. Вскоре пересекли Тихвинский канал и дальше началась гравийная дорога, кстати, весьма и весьма приличная. Но скорость по ней держать высокую невозможно, да и на «буханке» это нереально. Поэтому ехали не торопясь. Один раз остановились сходить освежиться. Дорога как-то затягивалась. Путевые заметки стало писать невозможно – очень сильная тряска. Бригада сзади тряслась на лавочках. Паша откинулся и лёг на скамейку, получая, несомненно, отличный массаж спины.
Проехали поворот на Пудрино – там оказалась разбитейшая грунтовка с огромными лужами и грязищей. Даже на «буханке» там было не проехать. Но мы об этом знали из Google-панорам и ехали дальше – на Радогощь. Пересекли мост через маленькую речушку Ундрегу, которая впадает в Лидское озеро. Речушка была очень узкой и заваленной, так что начинать сплав с неё мы, ожидаемо, не стали. Гравийная дорога при подъезде к Радогоще закончилась и началась укатанная грунтовка. Вскоре пересекли ещё один мосток через ручей, также впадающий в Лидское озеро, а потом показалась и Радогощь. 
Радогощь – это одно из самых южных поселений вепсов, если не самое южное. И названия местные все сплошь из вепсского языка. К слову сказать, вепсы относятся к финно-угорским народностям, и их алфавит состоит из латинских букв. В настоящее время больше всего вепсов проживает в Ленинградской области и в Карелии. Так что, побывав в этих местах, мы смогли легонько прикоснуться к истории этого древнего народа, корни у которого, наверняка скандинавские. 
В Радогоще при советской власти, по моим предположениям, располагался крупный совхоз и посреди современной деревни стоят два обшарпанных трехэтажных панельных многоквартирных дома (!!!). Упадок русского севера просто колоссальный. Тут наверняка вовсю кипела жизнь, а малые гидроэлектростанции на реке Лидь давали электричество окрестным деревням и весям. Теперь-то электричество есть и так, без этих давно разрушенных гидроэлектростанций времен плана ГОЭЛРО, от которых остались только разрушенные плотины, а уровень жизни и обжитости края падает из года в год… все бегут в города – оно и понятно, кому охота жить в таком всеми забытом захолустье, куда ведет лишь гравийно-грунтовая дорога? А самое главное – как и где тут можно заработать на жизнь…? Ладно, брюзжу как всегда. 
Мы повернули направо на центральном перекрёстке, проехали мимо этих самых трёхэтажных панелек, которые в контексте места своего расположения смотрелись просто сюрреалистично. Дальше начиналось самое интересное – надо было найти место для стапеля. Я достал навигатор и рассудил, что лучше вставать на воду около разрушенной ГЭС, но реальность оказалась иначе – грунтовая дорога становилась всё хуже и хуже. В итоге водитель отказался ехать дальше и съехал на ответвление вправо, где уже стояли две машины. Делать нечего, нужна разведка. Мы вылезли, прошлись. Действительно, к ГЭС не проехать, хоть до неё и оставалось метров 300-400. Пешком туда не пошли – жидкая грязища и лужи – а жаль. Увидели б своими глазами, что там осталось от плотины… но не пошли так не пошли. Должно оставаться что-то непокорённое – а то если всё покорить, что дальше делать? Отвлеклись на диких, нет, просто дичайших комаров! Они атаковали нешуточно! Такое обилие насекомых очень не радовало. Надо было что-то решить с местом выгрузки и поскорее залезть в вещмешки за репеллентами!
Юрий ушёл вправо по разбитому ответвлению дороги и вскоре вернулся. Там, по его словам, до реки близко. Решено. Подключив передний мост «буханки», он развернулся и задним ходом подъехал максимально близко к началу топкого разбитого участка, за что ему большое спасибо. Координаты N59°46'19.97" E34°52'29.86". Там мы выгрузили вещи из салона, и надели бахилы химзащиты ОЗК, а С.Ю. – полукомбез от костюма химзащиты Л-1, подготовившись, таким образом, к грязище. Заброска автотранспортом пройдена! Теперь надо перетаскать вещи на берег реки. Я расплатился с водителем – заброска от Чагоды до Радогощи обошлась нам в 5 тысяч. Юрий пожелал нам удачи и, немного погодя, уехал, а мы стали потихоньку таскать вещи к берегу, обильно набрызгавшись репеллентами.
Бахилы химзащиты нереально выручили – грязь была мокрой, топкой и ужасной. Преодолев самое топкое место, мы пробрались сквозь заросли и вышли на берег Лиди. Это было то, чего я ждал целый год – начинается этот дзен, это отчуждение, это приключение, этот отрыв!!! По-о-олный отрыв! Но времени размышлять о высоком не было – комары дико атаковали и надо было быстро перетаскать все наши тюки и начинать собирать байдарки. 
Снаряжение перенесли довольно быстро – каждый сходил к берегу и обратно к месту выгрузки всего раза по два-три. На берегу около нашего стапеля стояла на привязи чуть подтопленная деревянная самодельная лодка. Жена моментально облюбовала место в ней и сидела копалась зачем-то в телефоне. Все приходили в себя после гравийной дороги и начинали распаковку вещей и байдарок, параллельно сделав несколько снимков на фотоаппараты. 
Места на берегу было маловато и везде грязь, однако удалось всё же найти более менее чистый участок с травой, где мы и начали сборку своей трёшки вместе с женой и Олегом. С.Ю. и Паша собирались тут же, в пяти метрах, тоже на траве, чтобы зря не пачкать снаряжение на жирной влажной почве. Примерные координаты N59°46'15.94" E34°52'28.43".
Выяснили, что рацию, которую я отдал Олегу, он оставил в машине в Чагоде. Связи между лодками не будет… обидно, знаете ли, но что делать – всегда без этого ходили и сейчас обойдёмся. Свою рацию, ставшую бесполезной, выключил и убрал в гермомешок.
На сборку байдарки ушло около получаса, а С.Ю. справился немного раньше. Олег на лету схватывал принцип сборки «Таймени» и работа спорилась. Трудности, как всегда доставила лишь натяжка фальшбортов, но тут уж, традиционно, с помощью пассатижей и какой-то матери, удалось, спустя минут 10, закрыть все застёжки. Привязали чалку, спустили байдарку на воду. Стали таскать и грузить вещи. Комары при этом не дремали и атаковали в полную силу – от такого я уже давно отвык! Лоб и руки начали мгновенно распухать от укусов. Репелленты не помогали практически никак, лишь на несколько минут ослабляя натиск неутомимых насекомых, жаждущих свежей кровушки… С.Ю. с Пашей быстрее нас погрузились в байдарку, отчалили, не дожидаясь на то команды, и скрылись за поворотом, желая оторваться поскорее от берега в надежде спастись на воде от дичайшего полчища летающих кровососов.
Мы же возились с погрузкой – компактно укладывали вещмешки. Как всегда, это дело техники и спустя пару пять мне удалось более менее приблизить компоновку вещей в трюмах к идеальной	, и мы стали усаживаться в байдарку. 
И тут случилось… непоправимое. Правда, понял это я уже потом, вечером на стоянке. В запарке, в желании догнать ушедший экипаж и тоже оторваться от комаров, усаживаясь в лодку, я на доли секунды потерял свой GPS навигатор из виду. Усевшись, я решил, что он упал на дно байдарки и откатился назад, сбоку от моего сиденья. Отчалив, мы стали держать курс на левый поворот реки, стараясь увернуться от веток, нависающих над водой с правого берега, а я шарил рукой по днищу в тщетных попытках найти прибор. Но нет. Отплывая всё дальше и дальше, я понял, что, скорее всего, выронил навигатор за борт... но надежда ещё теплилась – надо было на стоянке основательно обшарить корму – мне казалось, что он непременно должен быть там. Тщетно. На следующей стоянке навигатор найти не удалось. Так коварная Лидь недовольно встречала путешественников, дерзнувших покорить её…
Пошли без навигации, а куда деваться. Маршрут и карту я помнил достаточно хорошо, а уж ниже Заборья я и вовсе уже ходил. Кроме того, как справедливо заметил С.Ю., с реки-то всё равно никуда не денемся. Так-то оно так, но вскоре Лидь впала в озеро Глубокое, из которого надо было найти выход. Припомнив карту, относительно без проблем нашли выход. Не так-то уж и нужен навигатор, хе-хе! Тем более, что где-то в герме был резервный комплект бумажных карт, да и GPS в телефонах никто не отменял… Но вскоре попали во второе озеро. Его я по карте не помнил и сначала принял уже за Лидское, но размеры явно были меньше. Оказалось, что это озеро Малое. И вот в нём-то мы изрядно попетляли в поисках выхода. Тут бы, конечно GPS пригодился. И трек похода теперь не запишешь… у-у-у-у-ух, горе-горюшко! Это ж надо было так утопить навигатор прямо в первый день! В первую же минуту сплава!!! Но что произошло – то произошло, и уже ничего не изменишь. Даже если бы вернулись и стали обшаривать дно у стапеля – оно очень топкое. Илистое. И мутное. Шансы найти маленький утопленный прибор стремились к нулю. Подмывало достать флягу… но мы шли дальше.
Вскоре вышли в Лидское озеро, встретив несколько рыбаков на лодках… рыба, стало быть, есть, раз местные ловят. Наши рыбаки встрепенулись и тоже хотели поскорее закинуть удочки на стоянке, о которой уже пошли некоторые толки среди экипажей – народ хотел пораньше встать на стоянку после тяжелой дороги. 
По Лидскому озеру стало трудно идти. Течения нет, вода стоячая. Хорошо, что хотя бы ветра нет. Однако пошел дождь. Пришлось нацепить дождевики. В мои планы не входило геройствовать в первый же день – я намеревался встать лагерем на первом острове на этом широченном озере. Но уровень воды в этом году был высок, как никогда, и остров оказался подтоплен. Пришлось идти дальше. По мере продвижения стало понятно, что характер берегов тут везде одинаков – они все подтоплены и поэтому встать лагерем на озере вряд ли получится – придётся искать выход в реку и попытать счастье там. 
Погодка не радовала – лил дождь, с носа капало. Бригада была не в самом лучшем расположении духа после бессонной ночи, а погодные условия как будто решили испытать нас на прочность. Экипаж С.Ю. и Паши то и дело подходил к правому берегу на разведку – нет ли там выхода из озера. Но тщетно. Я не разделял их энтузиазма, поскольку помнил по карте, что до выхода ещё лопатить и лопатить – почти до конца озера, а мы только-только проплыли почти опустевшую деревню Пудрино, что в самом его начале. Вскоре прошли и заброшенную деревню Буржаиху за лесистым мысом. Выход был явно дальше. С.Ю. и Паша вновь предприняли попытку найти исток реки из озера и отклонились от фарватера к правому берегу, а наш экипаж сбавил ход передохнуть. Мне наскучило слепое скитание по водному простору, и я достал телефон из гермокоробки, включил GPS. Привязавшись к местности, сориентировался – выход из озера должен был быть перед следующим мысом, куда мы и стали держать курс после небольшого перерыва в гребле. GPS в телефоне выключил и убрал аппарат в гермокоробку – утопить ещё и телефон было бы слишком.
У всех было желание поскорее встать на стоянку, обогреться и основательно поесть – уже около суток мы держимся на перекусах и из горячего только чай в термосах, который, кстати, уже закончился. Почти сразу на выходе из озера в Лидь, встретили стоянку с дымящимся костровищем. Вспомнил, что пока мы шли по озеру, мимо нас пролетела моторная лодка – видимо это те рыбаки тут стояли. Бригаде стоянка не понравилась, пошли дальше. В принципе, времени было ещё мало – можно плыть и плыть. Преодолели что-то вроде остатков моста или небольшой плотинки. Позже посмотрел по карте, что это остатки моста дороги к Платанихе. Попытались найти место сразу за бывшим мостом у правого берега, но неудачно. Попадались лишь рыбацкие стоянки, да все почему-то замусоренные. И ведь местные сами мусорят, не сплавщики, коих тут раз, два и обчёлся. Эх… культура! После нас, я клянусь, остаётся всегда только потушенный костёр, остатки напиленных дров и обожжённые банки из-под тушёнки, не более! Тот мусор, что горит, мы обязательно сжигаем. А тут… стеклотара, полиэтилен, обёртки какие-то, ПЭТ бутылки… срамота! 
Отчалили от места очередного просмотра берега. Начала чувствоваться усталость. Сколько точно мы прошли, я без навигатора сказать не мог, но по моим прикидкам выходило около 15 километров, что подтвердилось позже. На стоянку встали после заброшенной деревни Платанихи, от которой ничего не осталось. Координаты места N59°41'20.90" E34°53'38.90". Имелись напиленные дрова, поляна, много места для палаток и вполне приличный для такого уровня воды спуск к реке. Стали выгружаться. 
Дальше действовали как обычно – разгрузили байдарки и затащили их на берег, развернули тент от дождя. Сложив всё снаряжение и провизию под тентом, отправились за дровами и костровыми палками. Лес мокрый после дождей и сухостой промок, но мы упёртые! Нам удалось найти парочку брёвен посуше. 
Пока ставили лагерь и разводили костёр – комары просто заедали. Выяснилось, что Олег забыл репелленты в машине… осталось только наших с женой два баллончика. При таких кровососах это катастрофа! Немедленно провели санобработку верхней одежды всей команды. Правда, спасало это ненадолго. Больше помогал дым костра… комары как будто с цепи сорвались – полчища некормленых комаров атаковали нас непрерывно!
Над костром уже приветливо пошумливал котелок кипятка на макароны с тушёнкой, когда Паша вдруг так кстати предложил, поглаживая бороду: 
– А где, собственно, канистра?...
– У-у-у-у-у! Ы-ы-ы-ы-ы! Да-а-а-а! – завопила походная братия, ибо, возможно, только горючее сейчас было способно отвлечь нас от комаров. 
Олег немедленно извлёк из пакета пятилитровую канистру яблочного снадобья. Это то, что нам сейчас нужно! Надо расслабиться и снять лишнее напряжение в мышцах, которые пока ещё не слишком понимают, что весло  – это дней на десять, а не просто так! Переход по озеру тяжело нам дался. Отсутствие постоянной спутниковой навигации и плохая погода тоже радости не добавляли, поэтому яблочная уходила исключительно хорошо.
Паша, тем временем, уже забросил удочку и выловил маленькую рыбёшку. Клевало не очень, поэтому он оставил удочку заброшенной и тоже присоединился к нашей яблочной вакханалии, достав припасённый лимон из гермы с продуктами. Временами, когда колокольчики, привязанные к удочке, начинали позвякивать, Паша стремглав срывался к берегу в напрасной надежде выловить достойный экземпляр… Дождик временами то начинался, то прекращался, и макароны с тушёнкой мы вкушали сидя под тентом на деревянных чурбачках, напиленных кем-то тут бензопилой. 
Девизом этого похода можно назвать фразу: «Хрен знает, в какой-то герме!»
– Са-а-ань, где то-то? (Имеется в виду что-то из еды или из снаряжения). 
– Хрен знает, в какой-то герме! – неизменно был мой ответ. 
Я что-то настолько расслабился и пустил всю кухню на самотёк, видя, как народ и так нормально справляется со всем, что справедливо рассудил – рано или поздно сами освоятся. Но этого не происходило – народ тоже расслабился и снова и снова спрашивал на каждой стоянке: 
– Са-а-а-а-ань, имей совесть, где то-то? 
– Хрен знает, в какой-то герме! – и взрыв моего подогретого хохота… хотя я был явно не прав. Если гермы с продуктами ещё отличались по цвету, то всё остальное лежало в четырёх одинаковых армейских вещмешках, и понять где что было невозможно – надо обязательно залезать внутрь… но, тем не менее, выглядело это всё забавно. Пришло решение впредь повязывать разноцветные ленточки на вещмешки для их опознавания. Впрочем, спустя несколько дней народ освоился, и, не стесняясь, преспокойно рылся в гермах и мешках в поисках нужных вещей, чего я и добивался.
Олег расчехлил гитару. О-о-о, это было нечто! «Таймень» – это, пожалуй, единственная байдарка на постсоветском пространстве, а может и в мире, в которую без проблем сверху можно положить гитару. А уж если «Таймень-3», как у нас сейчас, то это вообще крейсер! Чтобы был понятен масштаб – в нашей байдарке ехала вся провизия на пятерых (!) человек на 10 (десять!!!) дней, включая тушёнку, а также групповое снаряжение – пила, топоры, котелки – пехотный и десантные, газовая горелка для разведения огня, складная сапёрная лопатка, тент от дождя, плёнка полиэтиленовая для бани, складные стульчики. И, само собой, личные вещи экипажа, а также палатки. И, при всём при этом, сверху в грузовом отсеке ещё оставалось место для гитары...
«Вот! Новый поворот! И мотор-р-р  р-р-ревё-ё-ёт…», – понеслось над поворотами Лиди. Сидя у костра и отбиваясь от комаров, полистывая сборник песен, Олег выбирал что-нибудь из него или из памяти, и исполнял, а мы подпевали, как умели. С уменьшением содержимого канистры, песнопения наши становились все веселее и веселее… гитара в походе – это однозначно не просто огромный плюс, а то, что однозначно должно быть. Вместе с гитаристом, разумеется. Сразу атмосфера становится другой и даже на комаров сразу наплевать… ну, почти. Так мы сидели на берегу этой тёмноводной, красивой и местами коварной речушки и пели… самозабвенно пели любимые песни…
Солнце уже давно зашло, красивого заката не было – мы стояли в окружении высокого леса и густого подлеска, да и облачно. Настали сумерки. Яблочная уходила на ура, притупляя комариные укусы и бессонную ночь за рулём. Мы с женой залегли спать в этот день рано, около 10 часов вечера, потому как очень хотелось нормально выспаться, а бригада утихомирилась чуть позже.

\begin{center}
	\psvectorian[scale=0.4]{88} % Красивый вензелёк :)
\end{center}

%\include{part3_chapter5}
%\include{part3_chapter6}
%\include{part3_chapter7}
%\include{part3_chapter8}
%\include{part3_chapter9}
%\include{part3_chapter10}
%
\bibliographystyle{plain}
\bibliography{bibliography}
}
\end{document}
