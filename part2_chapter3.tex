\chapter{Приготовления к сплаву} 
\corner{64}

Необходимо было собрать снаряжение, а точнее переправить его из деревни в Москву. Хорошо, что всё необходимое у меня уже есть с прошлых сплавов – байдарка, палатка, штормовка и прочий камуфляж, бахилы химзащиты вместо сапог, сапёрная лопатка, двуручная пила, пехотные и десантные котелки, костровые крючки, фляги, тент от дождя и разные другие полезные мелочи. В этом году только прикупил надувной матрас и налобный фонарик, поскольку старый пал смертью храбрых, просто перестав работать на ровном месте. Хотел ещё купить складной полноразмерный стул с подлокотниками, но его размеры в сложенном состоянии меня как\sdash то не впечатлили, поэтому решил довольствоваться, как и прежде, маленьким раскладным стульчиком.

Итак, комплект костей и шкура <<Таймени-2>> были переправлены из деревни в город и заботливо разложены по комнате под неодобрительные реплики жены насчет всего этого <<хлама>>. Собственно, вот и вся подготовка. За снаряжение в этом году я был спокоен, а прикупив ещё 3 (три!) пузырька клея и пачку заплаток для ПВХ шкуры байдарки на всякий случай, успокоился совсем. Пусть будут! Не хочу попасть с пробоинами так, как попал в прошлом году на Лиди. До сих пор помню свой шок от десяти пробоин в один день.

Сане и Паше я дал свой список снаряжения, с С.Ю. мы всё обговорили устно, а наш третий экипаж собирался самостоятельно, что несколько беспокоило меня, но я решил ничего не предпринимать\mdash всё, что они лишнего возьмут с собой, повезут на своей же байдарке. По снаряжению было требование, чтобы вещи на одного человека влезали в один гермомешок, который в байдарке будет в качестве сидения.

Параллельно со сбором снаряжения, прорабатывался вопрос заброски на маршрут. Первоначально планировали сделать как при забросе на Лидь в прошлом году\mdash добраться поездом до Череповца, там закупиться продуктами, сесть на Арханегельский поезд и рвануть уже не до Заборья, а до Ефимовского, что чуть дальше. Там в Ефимовском взять машину и доехать до Вожанского озера, на берегу которого решили устроить стапель. Был и альтернативный вариант заброски через Волхов\mdash также на поезде, но с пересадкой на электричку(!), которая стоит на станции пересадки всего одну минуту(!!) и, вдобавок, имеет неизвестно сколько вагонов(!!!) с неизвестно какой наполненностью дачниками(!!!!). Такой вариант был чрезвычайно авантюрным и я бы согласился на него, если бы шёл как в прошлом году\mdash одной байдаркой. Но в этом году всё иначе\mdash команда у нас разношёрстная, есть ребенок, да и, чего уж там, хотелось чего-то более определенного. А то с нашим этим РЖД <<любимым>> всякое бывает\mdash отменят электричку или не сможем в неё впихнуться и застрянем в Волхове\ldots а это довольно далеко от стапеля.
 
Короче говоря, этот чрезвычайно заманчивый своей авантюрностью вариант я оставил про запас и мы, с подачи Владимира, который на тот момент ещё был в списках команды, стали искать микроавтобус для заброски. Немного с шиком, так сказать. В пересчете на одного человека заброска выходила не намного дороже, чем поездами, зато комфорта несравненно больше\mdash заберут прямо от дома и привезут к реке – красота! Выбрасываться с маршрута также решили микроавтобусом. Поиск машины мы С.Ю. взяли на себя.

Я обзвонил множество контор, предлагающих аренду микроавтобусов. В итоге нашел подходящий вариант, и мы вместе с С.Ю. встретились с водителем в Новокосино, обговорили маршрут, внесли задаток. Кто же знал, что Владимир отколется от команды\ldots Оставалось лишь дождаться отпуска и вперёд – на родные просторы! 

Последний месяц перед отплытием проходил плохо – мне всё казалось что я чего-то не учёл, чего\sdash то забыл. Потом, когда мы закупили продукты и договорились с водителем о заброске, стало полегче. Но все же дни до стапеля были мучением\mdash я себе места не мог найти, хотел быстрее вырваться на природу, оторваться от привычной жизни. Сутками просматривал карты ГШ, спутниковые снимки маршрута\ldots почему так? Может от того, что я не умею как следует отдыхать. Я имею в виду отдыхать в обычной жизни. Всё хочется мне чего-то такого\ldots дикого и первобытного – требующего полного отречения от привычного и перестройки мышления. Хочется полного, полноценного, полнейшего очищения сознания от тяготящего.  Неужели так не устраивает меня моя жизнь? Вроде взглянешь на неё\mdash всё хорошо. Но стоит только начать копаться\ldots останавливаешься и говоришь себе: <<Да это всё бред, надо жить дальше>>. И живёшь. Точнее\mdash существуешь от похода до похода. А заноза странствий сидит в мозгу и давит, давит, требует своего! Рвануть бы как-нибудь в кругосветку, да под парусом\ldots Эх!

Жалею сейчас, что книга А. Кротова <<Практика вольных путешествий>>, не попала в мои руки раньше, скажем в институте. Сколько я бы мог совершить этих вольных путешествий во время летних каникул\ldots а что? Сдал сессию и свободен как птица в небесах. Но не такой я человек, чтобы ездить зайцем в товарных поездах, бегать от охранников и перемещаться по стране автостопом, неизвестно где ночуя (кто знаком с творчеством Кротова\mdash поймёт о чём это). Это сейчас я бы мог спланировать что-то подобное, уже имея за плечами опыт организации сплавов, но\ldots момент, кажется, упущен. Это всё хорошо в пору безрассудной студенческой молодости, увы.

Помню, в детстве моими любимыми томами в Советской Детской Энциклопедии были том 1 и том 5\mdash <<Земля>> и <<Техника>> соответственно. В первом меня привлекали статьи о климате и погоде, о природе и картографии, о походах и великих путешественниках. Во втором мне нравилось решительно всё\mdash всё, что было связано с техникой\mdash машины и поезда, самолёты и вертолёты, корабли и подводные лодки, всевозможные станки, механизмы, аппараты. Так оно и сложилось по жизни\mdash по профессии и работе связан я с техникой, а тянет меня постоянно на природу, причём каждый раз в какую\sdash нибудь несусветную глушь. И этот природный зов приходится удовлетворять такими вот, казалось бы, с одной стороны, <<дикими>> сплавами по водным просторам нашей необъятной Родины.

Как бы там ни было, к заброске и путешествию мы подготовились основательно, оставалось  только мучительно пережить последнюю рабочую пятницу на работе, а потом ещё более мучительно протомиться дома субботу с воскресеньем. Как раз за сутки до моего отъезда вернулась с юга жена вместе с тёщей и тестем. Мы хорошенько позастольничали\mdash приходили мои родители, и мы одновременно отметили не только приезд загорелых отдыхающих с юга, но и мой предстоящий сплав. 

Конечно, жалко, что на такой короткий срок удалось свидеться с женой после её возвращения с юга, но\ldots заноза странствий страшная штука\mdash она зовёт в путь! Скорее! На просторы! Я ждал этого всю долгую осень, всю снежную зиму и зелёную весну! Ветер расправляет паруса, рука уже тянется к веслу, а взгляд затуманен в предвкушении пьянящей свободы!

И вот, наконец, свершилось то, <<о необходимости чего всё время говорили большевики>> – в ночь с воскресенья 31 июля на понедельник 1 августа наша команда выехала из Москвы навстречу приключениям!

\begin{center}
	\psvectorian[scale=0.4]{88} % Красивый вензелёк :)
\end{center}
