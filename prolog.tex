{
\cleardoublepage
\phantomsection

\fancyhead[LE]{\fancyplain{}{}}
\fancyhead[RO]{\fancyplain{}{}}

\addcontentsline{toc}{chapter}{Пролог}
\addtocontents{toc}{\vspace{-4mm}}
\section*{Пролог}

Одиноки ли мы во Вселенной? Мне хочется верить, что нет. Должны, просто обязаны где\sdash то быть подобные нам существа\mdash в далёких\sdash далёких галактиках, мирах. Как человеку осознать, понять всё это, окружающий его мир? Как научиться жить в гармонии с ним? Об этом давно мечтают мыслители, писатели\sdash фантасты. В частности, в~детстве на меня неизгладимое впечатление произвёл роман <<Туманность Андромеды>>\cite{ТуманностьАндромеды}, несмотря на то, что это было тяжело давшееся к прочтению, наверно только раза с 3\sdash го, произведение. Но проникшись им, его идеями, я понял\mdash его нельзя отпускать от себя ни\sdash ко\sdash гда. Это то, к чему должен стремиться каждый человек каждый день, каждый час, каждую секунду своего бытия\mdash постоянно. 

Оказываясь в деревне, я часто и подолгу тёмными августовскими вечерами стоял закинув голову к~ярчайшему Млечному Пути и внимал к нему. Звёзды, метеоры, туманности$\ldots$ это совершенно непередаваемые минуты тишины, таинства, наслаждения красотой природы. И~каждый такой раз мне хотелось буквально раствориться во всей этой красоте окружающего мира\mdash этом звёздном небе, этом влажном ночном холодном воздухе, полном ароматов цветов, скошенной травы и хвойных лесов, окружавших меня$\ldots$ мне было безумно жаль, что не находится слов выразить переполняющие меня чувства. 

Слова стали складываться внезапно\mdash после похода 2015 года. Я всегда ощущал, что поход нужен мне, как нечто неотъемлемое, как часть меня, как продолжение моего естества; нужна свобода, пыль дорог и ветра свист, нужен дождь в лицо, нужна штормовка, нужен согревающий костёр, нужна непременно двускатная палатка$\ldots$ Осуществив в~2015~году свою мечту\mdash сходить небольшой группой в полностью автономный поход, сплав на байдарке, \mdash я~вернулся другим человеком и$\ldots$ плотно засел за путевые заметки. Никто мне этого не~советовал, не~понукал, это поистине было велением очищенной души. Слова будто сами выбивались на клавиатуре, дело спорилось\mdash первая версия была готова меньше чем за 2~недели и$\ldots$ положена <<в стол>> на долгих 7~лет! Какая\sdash то неуверенность, кажущаяся ненужность и незавершённость каждый раз препятствовала мне вернуться к этим заметкам, придать им форму настоящим образом. Не зря говорят, что всему своё время. Но как понять, что это время \textit{настало}? Должен зазвонить звонок или колокол? Мол\mdash та\sdash дам!\mdash вперё\sdash ё\sdash ёд, действуй?! Конечно же, как все мы не раз убеждались, ничего подобного не происходит. Это всё созревает внутренне. Идея должна, что называется, <<вылежаться>>.

\renewcommand*{\thefootnote}{\arabic{footnote}}
И вот, в 2022 году, овладев системой вёрстки текста \LaTeX{}\footnote[1]{https://www.latex-project.org}, я вдруг понял\mdash время настало! Я~оформил с~помощью этой разметки первую свою книгу и был сильно воодушевлён результатом. Новый процесс написания текста на \LaTeX{} в~совокупности с~использованием системы контроля версий git\footnote[2]{https://git-scm.com} и автоматизацией сборки CMake\sdash ом\footnote[3]{https://cmake.org}\mdash всё это сразу привело процесс в привычное русло разработчика, положительно сказавшись на конечном результате. Что~в~итоге получилось из всего этого\mdash судить Вам, дорогие читатели.

%\vspace{5mm}
\begin{flushright}
	\textit{Соболев А.А., 2023 г.}
	%\copyright~Соболев~А.А.,~Москва,~27.08.2015
\end{flushright}

}
\fancyhead[LE]{\fancyplain{}{\bfseries \parttitle}}
\fancyhead[RO]{\fancyplain{}{\bfseries \rightmark}}