\chapter{А не пошл\'{о} бы всё?} 
\corner{64}

\epigraph{%Не нужен бродягам дом и уют \\
%	Нужны - Океан, Земля \\
%	Что звёзды Медведицы им поют \\
%	Не знаем ни ты, ни я \\
%	~...\\
	~\\
	~\\	
	~\\	
	Вел\'{и}к Океан и Земля велик\'{а} \\
	Надо бы всё пройти \\
	Большая Медведица издалека \\
	Желает тебе пути}
	{
	\begin{flushright}
		\small{Станислав~Пожлаков,\\«Баллада~о~детях~Большой~Медведицы» из~х/ф~«Идущие~за~горизонт»,~1972}
	\end{flushright}
	}

Дома. Счастливо плещется 150 красного на дне классического гранёного стакана… а хотя… какого классического? Наличие 250 граммов объёма и отдалённо ребристых краёв не делают его «тем самым стаканом» из детства, из школьной столовой. Тут и грани скругленные, не цепкие, не острые, и вообще что за… ерунда? Стакан сделать под советский уже слабо? Вот тебе на! Брюзжу, брюзжу как всегда. Словно немой упрёк потомкам, проср… потерявшим технологию изготовления гранёного стакана с нормальными гранями, мой экземпляр стоит на столе и заставляет задуматься – а ещё заводы АЗЛК, ЗИЛ, Серп и Молот и тэ дэ и тэ пэ по всей стране…
– А не пошло бы всё к чёрту? – говорю я себе в очередной раз! – что-то всё безумно тошно, безумно утомительно и безумно пошло; противоестественно, невозможно и вообще… – с почти такими мыслями я аккурат каждый день вываливаюсь из электрички ровнёхонько в 19:17 и иду на чёртов автобус, ибо добавить себе ещё двухкилометровую прогулку от станции до дома вечером после работы – почему-то выше моих сил. Не потому что не могу пройти эти 2.5 километра, а потому что лень. Впрочем, нет, обманываю – иногда иду пешком – если есть насущная, животрепещущая необходимость заглянуть в магазин... но это редко – маршрутка до дома теперь моё всё. Тем более что климат у нас, вообще говоря, не сахар – «то дождь, то снег, то мошкара над нами…», хоть мы и не в тайге, а всего в каких-то 24 километрах от московского кремля, оплота и символа уже новой власти, победившей таких ненавистных молоткастых-серпастых и всё вот это вот. А дома что? Дома принять горизонтальное положение, растянуть свои кости и усталые мышцы. Это старость? Увы, нет – было бы простительно! Это всего-навсего только чёртовых 27 лет…
О чём это я? Ах да – не пошло бы всё к чёрту? Встал в 7, в 8 утра на электричку, чтобы к 9 быть на работе… в 8 вечера оказался дома, при условии, что улизнул с работы без двадцати шесть. Это ох…. охренеть, простите. А что, у нас так всё подмосковье живёт и хорошо ещё, если так. И вообще «москвичи зажрались» со своим «часиком» или, страшно сказать, «полчасиком» до «работки». Тьфу! Электричка вся занюханная, что, впрочем, не смущает скотское руководство российских железных дорог регулярно поднимать на неё цены, и давка в ней неимоверная, ибо и «понарожали» и «понаехали», потому что в Москве и зарплаты выше и всё вот это вот… ну, вы понимаете – «коммунальный рай без хлопот и забот». Как в такой ситуации сохранить здоровье и не только физическое? Отрастить панцирь? Скорлупу?
Словом, каждое утро как селёдочка в бочке, будь любезен, 35-40 минут постой в тамбуре («Хо-хо, в ваго-о-о-он захотел просочиться?!») и марш с улыбкой на лице зарабатывать, как можешь, на своё якобы светлое и якобы счастливое капиталистическое будущее. Зато типа «честно», говорят нам. Заработал – вот оно, не заработал… пш-ш-шёл на! Ага, конечно. И никакой пенсии по старости. А что, это теперь не Советский Союз, не «совок», не «эсесесерия» – никто тебе ничего не обещал. А кто говорит, что обещали – гони поганой метлой.
Но какое оно, это будущее, нах… нахрен светлое? Какое будущее? Тут на настоящее, порой, не хватает. Порой молишь свои ботинки: «Родимые, ещё сезончик, ну давайте же! Не время рассыпаться!» И это притом, что, в общем-то, ничего такого себе не позволяешь – и это не преувеличение, не взмах руки «зажравшегося москаля», не огульная жалоба на «плохую жизнь», а воистину печальное положение дел. А что, вся страна примерно так и живёт, да чего уж там – хорошо, если так. А ты что, уникальный типа? Ха-ха, как бы не так! При этом мысли о выживании от зарплаты до зарплаты все чащё начинают походить на навязчивые идеи околокриминального характера.
– Езжай на машине, делов-то! – скажут вам прожжённые оптимисты! – не мучайся в своей зачуханной электричке, раз так не нравится, и вперёд, в «светлое капиталистическое будущее» на личном авто… Ха! Три раза ХА! Таких советчиков сразу посылайте по известному адресу! Действительно, это занятие для мазохистов – езда по Москве и Подмосковью на машине в часы пик на работу и с работы. А цены на бензин, а парковки… печальное положение дел… на машине, ха-ха! Словом, чем дальше в лес, тем толще партизаны и всё вот это вот – хочется просто выть.
– Право, заврался! Разбухтелся! Разнылся! Пойди и заработай на нормальную жизнь, – кричат те, кто, как правило, либо сравнительно легко всё получил, либо ослеплён столичной и околостоличной жизнью на контрасте со своим родным условным Гадюкино, где всё такое родное и милое, но совсем нет работы, – ноет он, понимаешь, что денег нет и ботинки дырявые! – слышатся недовольные высказывания! Марш работать! Да-да, уже иду. Вот только пойму, куда пойтить-то, куда податься, и пойду, полечу просто. Как фанера над Пари… над Москвой. 
«Москва! Как много в этом звуке!». Да ничего в этом звуке. Ничего в нём не осталось. Ни-че-го. Город трансформировался во что-то не просто уже не родное, а во что-то местами целиком и полностью гадкое, и разросся совершенно неприлично, зачем-то построив вставные челюсти вроде безвкусных небоскрёбов «Сити» и поглотив, вдобавок, троицкий район. Так легко же поднять условный уровень жизни в Троицке, а заодно и цены на землю (предварительно скупив её), на недвижимость (предварительно тоже скупив её и понастроив новой), на всё остальное (ну, вы понимаете), просто объявив этот район Москвой. Так-то. Ловкость рук и никакого тут вам понимаешь! Не забалуешь! Кто прав, тот и лев! А ты давай, сопли свои вытри о рукав и вперёд – в занюханную электричку! Не нравится? Плати в 2 раза больше и шуруй за счастьем на электричке-экспрессе. И чебурек в ней купи с пивасиком у проходящих торговцев. В пути, теперь, понимаешь, и кормят и поют. Чтоб совсем не подох.
«Москва…» Пустота в этом звуке! В старых усадьбах, особняках – либо запустение, либо какие-то непонятные конторы и ночные клубы со всякими сами знаете кем «с пониженной социальной ответственностью»; исторические здания тоже никому, в общем-то, не нужны, а вся их вот эта вот табличка «охраняется государством» – ложь и фикция – государство непременно извлечёт из этого выгоду, уж поверьте, ибо нечего тут. Выжмет всё как из лимона и бросит. Или продаст. Или выжмет и продаст. Нужное – подчеркнуть. И «привет»! Это уже не «всенародное достояние», которое «охраняется государством», а чёрт знает что –  частная собственность. И мгновенно выясняется, что это всё можно не реставрировать, а снести и построить что-то новое, дорогущее, безумно, дичайше (!) безвкусное, стеклянно-бетонное, и снимать с этого сливки… 
У власти непойми кто вообще. Не то чекисты, не то легализованные бандиты. Или вообще и то и то в одном флаконе. Мафия, одним словом, хоть всё культурно и на бумаге «законно». И всё выходит на манер «дикого помещика», который извести мужика захотел. Да только тут страшнее – с виду «всё культурненько, всё пристойненько», а держат тебя за яй… за горло. И туго, туго сжимают. Ослабят чуть – счастлив! Рученьки целовать готов. Что-то «ура-патриотичное» покажут по ящику, покрутят по радио, побомбят в Сирии – и кто-то даже, страшно сказать, подумает, что в типа Великой стране живёт! А страна-то у нас какая! У-у-у-у-у! Природа, погляди какая! Ы-ы-ы-ы-ы! Дух захватывает! Народ-то у нас какой! У-у-у-у-у! Великий! Люди, правда, отдельно взятые, почему-то часто на поверку – дерьмо редчайшее. И чем ближе хоть к какой-то власти и деньгам, тем… сами понимаете. Так и живём…
И среди всего этого, отягощенного непростым прошлым, попробуй-ка найди своё место… место посреди дикого посткоммунистического, недо- или пере- капиталистического, вообще не пойми какого мира. И можно бы, кстати, уже в этом месте попросить выдать «парабеллум» с одним патроном. Потому как не понятно вообще ничего. Дикое совершенно соединение коммунизма, социализма, национализма, религиозности и капитализма, царизма, непременно имперских амбиций и чудовищных, ужасных контрастов, а также воровства. 
– Совсем бред! Что не нравится-то! – слышим мы снова вопли тех, у кого всё хорошо или тех, кто думает, что у него всё хорошо. Тот, кто не пользуется общественным транспортом каждый день, не спускается в метро, не ездит нашими поездами и электричками, кто чувствует себя если не «хозяином жизни», то, по крайней мере, «удавшимся человеком». У которого всё «типа» удалось. Или тот, кто в «розовых очках», что многократно гнуснее.
Грустно всё это, товарищи. Тут хочется выйти на балкон, забить туго трубку, дымить и долго смотреть в закат. Но это всё, конечно, не то. И сбегаю я… сбегаю в очередной раз ото всего этого. Не могу и не хочу видеть я этой действительности. Не – хо – чу! Изменить что-то – не в силах, терпеть – терплю, но тошно. Зомбоящик смотреть – одно расстройство, радиопередачи скатились с упадком аудитории, СВ и КВ вещание благополучно закрылись, УКВ – сплошной шлак, а на интернет чекистская власть пока смотрит-смотрит, понять до конца не может ничерта что это такое, но прикрыть грозится – уже и торренты прикрывали и тэ дэ и тэ пэ. Проблем, чёрт подери, нет других. Дороги сделайте нормальные в стране! Нет, они обеспокоены, что народ музыку и фильмы в интернете «нелегально» скачает… да наши фильмы современные и смотреть-то за деньги не станешь… не все конечно, но всё же. Тьфу!
Я понимаю, что наивен донельзя, но почему нельзя жить как в романах Ефремова – Эра Мирового Воссоединения?! Нельзя… человеческая сущность, увы, не такая… сделав ставку на создание нового человека, нового общества, коммунисты и проиграли. Да, царя конечно в расход и старые порядки туда же, но люди-то, общество-то – всё с треском не переключить мгновенно, не перестроить за один миг; это всё надо ломать и строить заново – нудно, мучительно долго и, вообще говоря, непонятно как. Неужели этого никто не понимал? Странно, очень странно – вроде неглупые люди были, по крайней мере, не те, кто с винтовками. Но вся эта шумиха, коммуны и прочее, быстро относительно закончилось – получилось совсем не так, как мечтали эти наивные мечтатели с наганами. Их самих слегка пустили в расход и стали строить что-то другое, что было более реально, что могло защищаться, как-то развиваться и существовать, повергая в страх остальной мир кажущейся дикостью и непременно имперскими амбициями...
И всё бы ничего, как мне кажется. Не могло так быть вечно – всё равно бы встали на околокапиталистический путь, как это с успехом делает Китай сейчас. Но нет, терпеливо работать, менять всё потихоньку, учитывая недостатки – не наш стиль. Наш стиль – здесь и сейчас, наш стиль – 300% «навара» сегодня, а завтра трава не расти! Всё или ничего! И опять великие потрясения. «До основанья, а затем…». И что затем? Затем НАТОвские сволочи под масками ягнят ногами открывали все двери в стране. Всё производство, каким бы отстающим оно ни было – к чёрту. Зато баксы, шмаксы, джинсы, кока-кола, «заграница нам поможет» и прочее, прочее, прочее, так ярко пришедшееся на пору «детей 90-ых», к которым себя причисляю… А теперь, спустя ещё четверть века, вроде как Запад нам снова если не кровный, то точно какой-то враг и всё такое прочее… Где «свои», где «чужие»? Где политрук, который скажет об этом? Все шиворот-навыворот.
Но ведь и что-то хорошее было в Союзе? Не могло ведь не быть, иначе совсем петля или «парабеллум». Да, было. Почему было, оно и есть, продолжает быть. Только не для всех и не даром. Да и раньше не для всех и не даром, подсказывают тут всякие «либерасты» со всех сторон – того и гляди укусят, растерзают за свои сучьи свободы. Да, тоже не для всех и не даром, но по сравнению с теперешним «не для всех и не даром», тогдашнее – просто детский лепет. И опять не во всём! Нет универсального критерия, чтобы рассудить. На машину раньше фигушки накопишь, а накопишь – ещё фигушки купишь. Сейчас – работай, копи, покупай, бери в кредит… Лучше? Кажется, что да. По крайней мере обилие авто на улицах явно говорит само за себя – лучше. В сытые перерывы между кризисами накопить на загранпоездку и поехать – вот просто так, на недельку – реально? Реально. Привет, Союз! Такого не было точно! А ещё эти чекистские штучки перед выездом, ох-ё, в страны, страшно подумать, даже соцлагеря. Вот с жильём было проще, подсказывают заклятые сталинисты. Потолки под три метра и всё в таком духе. Но тут тоже как посмотреть… а ещё молоко по 36 копеек и т.д и т.п. 
Если так всё прелестно было, что ж так быстро всё свергли? Надоело всем, накопился огромный вакуум, стало быть. А что надоело? Чтобы понять – надо жить тогда… а всё остальное, что сейчас говорят по этому поводу – полуправда в лучшем случае. Нам уже не понять, не осознать – мы не прочувствовали. Вакуум… надо было чем-то его заполнить. Чем? Ясное дело, всем западным. Слепо, просто слепо всем западным, без разбору. Просто потому что оно другое. Красивое, другое… западное, одним словом. И это и есть наша самая большая трагедия – ломать, не реформируя старое. Ломать и не глядя заимствовать. А вакуум, надо полагать, был в самом, казалось бы, простом – продовольствии и товарах народного потребления. В одежде, обуви, каких-то предметах быта… чтобы и красиво было и не стоять за этим в очереди с талоном в руке… но это опять же – взгляд на прошлое от непрочувствовавшего – и поэтому никак не претендует ни на истину, ни на полноту охвата.
Тут многие снова начнут возмущаться – как же так, поднадоел уже со своим нытьём! Ты определись-ка, за кого ты – за красных или за белых? А то ни то и ни то, похоже, не нравится! И вообще, это уже о сплавах или где? Нет-нет, постойте, я ещё пожужжу! Чтобы понятнее было от чего я убегаю на реки. От окружающей реальности, какой бы замечательной ни хотели её представить нам!
А ещё и религиозный вопрос! Вот где последний гвоздь! За 70 лет вроде как все перекрасились, избавились от предрассудков, научный коммунизм, исторический материализм и всё вот это вот… ан не-е-е-е-ет! – снова в мгновение ока все перекрестились и впали в религиозность! И что самое плохое – с поддержкой на государственном уровне. Злость закипает внутри! Это полный провал, господа-товарищи коммунисты. А что делать, скажите вы, не поддерживать же всяких мерзотных лгбт-шников? – вот мы типа в противовес и поддерживаем духовность, религиозность – скажут вам! Чтобы всё чинно и благородно и вроде бы с благими намерениями. Врут конечно, твари. Там «скрепы духовные» и бронированные машины для «духовенства» и всё вот это вот. Тьфу! Нет никакой возможности смотреть на этот цирк!
Изменить ситуацию сверху – никто не хочет, потому что, похоже, не знает как. Да и не надо им этого – у них и так всё хорошо. Сверху знают лишь как удержаться на этом верху и продолжать всех доить. И доить как можно дольше. Снизу изменить это – нет больше такой возможности – как в 1917-ом – да и не переживёт страна такого в третий раз. Как пить дать не переживёт. И печаль-то момента в том, что изменить снизу – ха-ха! – уже нельзя – всех купили! Да-да… с потрохами просто! Купили капиталистическим укладом жизни, мнимой вседозволенностью, кредитами купили. Купили незаметной и заметной пропагандой, «лишь бы не было войны», купили «счастливым» будущим, купили круглогодичными бананами в магазинах (как дикарей, право), круглосуточными услугами и всё такое прочее. И сработало! Ну а кто бы отказался? Мало денег – заработай, говорят. Что-то не устраивает – в твоих силах изменить! Надорвёшься, правда, но что-то конечно изменишь. Своё положение относительно уровня земли, к примеру.
К чему я это всё? Больно оно надо рассуждать о всяком блядстве вокруг нас? Достало. Сильно достало. Но ведь есть же что-то хорошее, иначе совсем петля и «парабеллум»? И вообще, уже опять кричат тут, определяйся, за кого ты – за красных или за белых? А всё дело в том, что я не за красных или белых, а за то, чтобы сохранять всё лучшее, что у нас есть, при этом изобретая новое, нужное, перспективное, чтобы когда-нибудь жить хотя бы чуточку похоже на то, как писал Ефремов… делать все рационально, летать к далёким галактикам, исследовать космос… но, это так же несбыточно, как и Вторая Великая Октябрьская. И надо бы просто, как советуют некоторые, забить на эту всю несправедливость и попытаться улучшить что-то для своей семьи, для себя лично. Звучит просто, да бывает трудновато сделать… да и правильно ли это? Семья говорит, что да, правильно, естественно. Что-то внутри говорит, что не совсем – а как же ефремовская Эра… и всё вот это вот? Неужели проклятый капитализм в крови у человечества? И сбегаю я в леса Вологодчины… но правы они, эти некоторые, – чем понуро и угрюмо смотреть на весь этот ужас, на пустой кошелёк, ничерта не делать и охать, таки лучше начать с себя и благосостояния своей семьи. И никогда не получится так, как в романах Ефремова и статьях Советской Детской Энциклопедии. Но, чтобы начать что-то делать, улучшать – должно что-то случиться – «гром не грянет – мужик не перекрестится»… 
И случилось – порезали зарплату. Работа, и так не пыльная, стала вообще незаметной. Тематика стала загнивать с катастрофической скоростью. И что самое печальное, все, конечно, понимают, что происходит, но сделать ничего нельзя, как обычно. А почему? «Да потому что мы – м**ки», – говорит один достопочтенный человек. В галактическом масштабе – перспектива потери Родиной технологии того, что мы делаем на работе. Ну и хорошо – значит не надо! А все сказки про «затянуть потуже пояса» – мимо. Увы, ребята, мы не в 20 веке – хватит. Это мы уже проходили. Не надо – значит не надо. Мухляж с трудовыми договорами, непонятные выкрутасы кадровиков, урезание зарплаты и отсутствие реальной работы – это край. Осталось отгулять отпуск, чтобы при итоговом расчёте его не «забыли» учесть и… будь что будет. В любом случае, после протухающего НИИ должно быть только лучше… ибо что может быть хуже постсоветского «почтового ящика» со всеми отягчающими? Очень возможно, что я ошибаюсь, но искренне надеюсь, что нет.
А пока отпуска не отгуляны – единственная альтернатива – забыться… капитально забыться. Отстраниться. Сбежать. Освободиться. Сорваться. Уйти в параллельные миры. Миры путешествий по диким, дичайшим просто закоулкам нашей Страны. Где первозданная Природа и полная Свобода… где мы плывём на своих байдарках. В этой параллельной реальности ощущаешь – насколько мало надо человеку, чтобы радоваться жизни… эти бескрайние леса Вологодчины – светлые сосновые боры, верховые болота, буйство кустарников и мхов никого не оставят равнодушным; они очистят душу, откроют прекрасное в каждом сердце, однажды соприкоснувшимся со всем этим. 
Почему так полюбился мне этот край – бассейн Верхней Волги, а конкретно район рек Лидь, Чагода, Чагодоща, Кобожа, Песь, Молога? С.Ю. открыл мне этот заброшенный край с природой, очень похожей на нашу, родную Владимирскую, района реки Киржач. Только реки там, на Севере, интереснее – полноводные, местами порожистые, и населённых пунктов почти нет. Зато есть чувство бесконечной Свободы и глубокой отрешенности от остального мира. Это всё овладевает сознанием настолько, что становится навязчивой идеей оказаться там и прочувствовать это вновь и вновь. И что удивительно – не надоедает! Из года в год всё по-разному и в каждый год открываешь для себя что-то новое в этом забытом всеми крае… 
Так, сидя перед монитором, разглядывая топографическую карту и думая обо всём этом, хлопнув злосчастные 150, нажал клавишу Enter и маршрутные точки вместе с путевой нитью сплава стали перекачиваться в GPS-навигатор. Идём на реку! Alea jacta est!

\begin{center}
	\psvectorian[scale=0.4]{88} % Красивый вензелёк :)
\end{center}
